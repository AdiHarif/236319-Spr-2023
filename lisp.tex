\def\CPL{\E|C|\xspace}

%%%%%%%%%%%%%%%%%%%%%%%%%%%%%%%%%%%%%%%%%%%%%%%%%%%%%%%%%%%%%%%%%%%%%%%%%%%%%%%
% Code for collecting code exerpts into a separate library and kernel files
%%%%%%%%%%%%%%%%%%%%%%%%%%%%%%%%%%%%%%%%%%%%%%%%%%%%%%%%%%%%%%%%%%%%%%%%%%%%%%%
\newcounter{kernel}
\newcounter{library}
\setcounter{library}0

\newread \tempFile % A temporary reading stream
\newwrite \kernelFile % A stream to save kernel functions
\newwrite \libraryFile % A stream to save library functions
\immediate \openout \kernelFile=\jobname.kernel.lisp
\immediate \openout \libraryFile=\jobname.library.lisp

% An environment for writing kernel functions
\newenvironment{KERNEL}{%
  \stepcounter{kernel}
  \def\fileName{\jobname.kernel.\arabic{kernel}.lisp}%
  \global\let\savedifeof=\ifeof
  \def\ifeof##1{\global\let\ifeof=\savedifeof\iftrue}%
  \csname filecontents*\endcsname{\fileName}%
}{%
  \csname endfilecontents*\endcsname%
  \pagebreak[3]%
  \LTR
  \lstinputlisting[language=Mini,style=display,backgroundcolor=\color{olive!10}]{\fileName}%
  \endLTR
  \pagebreak[3]%
  \openin\tempFile=\fileName
  \begingroup\endlinechar=-1
  \loop\unless\ifeof \tempFile
  \read\tempFile to\fileline % Read one line and store it into \fileline
  \immediate\write\kernelFile {\unexpanded\expandafter{\fileline}}
  \repeat
  \immediate\write \kernelFile {¢}
  \immediate\write \kernelFile {\unexpanded\expandafter{\pagebreak}[3]}
  \immediate\write \kernelFile {¢}
  \endgroup
  \closein \tempFile
}

% An environment for writing library functions
\newenvironment{LIBRARY}{%
  \stepcounter{library}
  \def\fileName{\jobname.library.\arabic{library}.lisp}%
  \global\let\savedifeof=\ifeof
  \def\ifeof##1{\global\let\ifeof=\savedifeof\iftrue}%
  \csname filecontents*\endcsname{\fileName}%
}{%
  \csname endfilecontents*\endcsname%
  \pagebreak[3]%
  \LTR
  \lstinputlisting[language=Mini,style=display,backgroundcolor=\color{orange!20}]{\fileName}%
  \endLTR
  \pagebreak[3]%
  \newread \tempFile % open the file to read from
  \openin \tempFile=\fileName
  \begingroup\endlinechar=-1
  \loop\unless\ifeof \tempFile
  \read\tempFile to\fileline % Read one line and store it into \fileline
  \immediate\write \libraryFile
  {\unexpanded\expandafter{\fileline}} % print the content to copy.txt
  \repeat
  \endgroup
  \closein \tempFile
}%

§ מבוא

שפת LISP (כקיצור של \E|LIst ProceSsing|), הייתה, יחד עם שפת Fortran (קיצור של
\LR{FORmula TRANslation}), אחת משתי שפות התכנות הראשונות. LISP פותחה ב-MIT על
ידי \E|John McCarthy| עוד בשנת 1956. אולם מלכתחילה, מלכתחילה, לא נתפסה LISP
על ידי ממציאה כשפת תכנות, אלא כתחשיב מתמטי המיועד לכתיבת אלגוריתמים,
הדומה לתחשיב ה-$λ$ \E|(lambda calculus)|. רק לאחר מכן, מומש התחשיב כשפת תכנות.

בתחשיב של \E|McCarthy|, כל תכנית היא מה שנקרא \textbf{ביטוי~\E|S|}, והרצת
התכנית קרויה "\ע|שיערוך|" (\E|evaluation|) של ביטוי~\E|S|. ניתן לחשוב על
ביטוי~\E|S| כעץ בינארי מלא, כלומר עץ שדרגת כל צומת בו היא 0 או 2. העלים של עץ
בינארי זה, כלומר הצמתים שדרגתם 0, נושאים תגיות. לצמתים שדרגתם 2 לעומת זאת אין
תגיות. לדוגמה, בעץ הבא, יש שני צמתים פנימיים, ושלושה עלים מתוייגים, שניים מהם
בתגית~$a$, והשלישי בתגית~$b$.
\begin{quote}
  \center
  \Forest{s tree [{},cons
      [$a$,atom],[{},cons [$b$,atom] [$a$,atom]]] }
\end{quote}

שיערוך של ביטוי~\E|S| נתון כולל שני צעדים מושגיים:
\אבגד
✦ \ע|מציאת משמעות| בעבור תגיות המופיעות בעליו של הביטוי. למספר תגיות ישנה
משמעות קבועה תמיד. בעבור התגיות האחרות נעזר תהליך השיערוך בטבלת עזר המעניקה
משמעות לתגיות.
✦ \ע|חישוב| ערכו של הביטוי בהתאם לפרשנות. גם חישוב זה יכול להיכשל, ואזי גם
השיערוך נכשל. תוצאת השיערוך היא תוצאת החישוב שנעשה בצעד זה.
===
אם השיערוך אינו נכשל, תוצאת השיערוך היא ביטוי~\E|S|, שהוא, במרבית המקרים, שונה
מהביטוי הנתון. במילים אחרות, שיערוך הוא טרנספורמציה (העשוייה להיכשל) המעתיקה
ביטויי~\E|S| אל ביטויי~\E|S|. פעולת השיערוך אינה מבצעת רק את הטרנספורמציה הזו,
שכן במסגרת השיערוך יתכנו עדכונים לטבלת העזר הנותנת משמעות לתגיות.

לא רק תכניות הן ביטויי~\E|S|. כל הנתונים בשפת ליספ גם הם ביטויי~\E|S|. תכנית
בשפת ליספ היא פונקציה שיכולה לקבל כארגומנטים ביטויי~\E|S| ומחזירה ביטוי~\E|S|,
וכאמור, כל פונקציה כזו, היא בעצמה ביטוי~\E|S|. שפת ליספ ניחנה לכן בתכונה הידועה
בשם \ע|הומואייקוניות| (\E|homoiconicity|) לפיה ניתן מתוך השפה לבצע מניפולציה על
תכניות הכתובות בשפה כאילו היו הן נתונים. שפות אחרות שהן הומואייקוניות כוללות את
פרולוג, \LR{Wolfram Mathematica}, \E|Snobol| ו-\E|Julia|. לעומת זאת, שפות
כמו~\CPL, פסקל, ו-\E|Java|, כמו מרבית שפות התכנות אינן הומואייקוניות.

בנוסף לתחשיב McCarthy הציג גם אלגוריתם אוניברסלי, \E|eval|, המסוגל להריץ תכניות
בליספ, כלומר לבצע את השיערוך. ליספ הפכה מתחשיב מתמטי לשפת תכנות כאשר האלגוריתם
הזה מומש בשפת מכונה. מאוחר יותר, נמצא מי שכתב מימוש של \E|eval| בשפת ליספ עצמה.

אחת המטרות החשובות של סיכום זה היא הדגמה כיצד ניתן לממש את אלגוריתם השיערוך של
שפת ליספ באמצעות שפת ליספ עצמה.

שפת ליספ שימשה בין השאר לפיתוח \E|Macsyma|, התכנית הראשונה לחישוב מתמטי
סימבולי, הכולל נגזרות אינטגרלים, ופתרון סימבולי של משוואות דיפרנציאליות. דוגמה
לחישוב סימבולי ב-\E|Macsyma| היא חישוב האינטגרל המסויים \[
  ∫₀⁴ e^{√{ay}}dy=-2
\] עבור~$a>0$, בעזרת ההחלפה~$y=z²/a$:
ראשית נודיע ל-\E|Macsyma| שהערך של a חיובי,
\begin{LISP}
(%i1) assume(a > 0)$
\end{LISP}
ואחר כך, נבקש מ-\E|Macsyma| לחשב את האינטגרל הלא מסויים:
\begin{LISP}
(%i2) 'integrate (%e**sqrt(a*y), y, 0, 4);
 4
 /
 [ sqrt(a) sqrt(y) (%o2) I %e dy
 ]
 /
 0
\end{LISP}
ורק אחר נבצע את ההחלפה:
\begin{LISP}
(%i3) changevar (%, y-z²/a, z, y);
 0
 /
 [ abs(z)
 2 I z %e dz
 ]
 /
- 2 sqrt(a)
(%o3)-----------------------------
 a
\end{LISP}

שפת ליספ שימשה גם לפיתוחה של \E|ELIZA|, תכנית המחשב הראשונה שהייתה מסוגלת לנהל
שיח עם משתמש בדומה לבוט. ELIZA ניסתה לדמות בשיחה את תפקיד הפסיכיאטר או
הפסיכותראפיסט ודו-שיח עמה נראה כך

\begin{quote}
  \begin{minipage}{0.8\textwidth}
  \begin{LTR}
    \scriptsize
    \begin{verbatim}
ELIZA: Please tell me what’s been bothering you.
ME: Oh, hello, ELIZA. It’s been a long time.
ELIZA: Hi. What seems to be your problem?
ME: Lately, I’ve been feeling a little depressed.
ELIZA: And how do you feel right now?
ME: I feel sad, but relieved to be writing.
ELIZA: Of what does feeling sad remind you?
\end{verbatim}
  \end{LTR}
\end{minipage}
\end{quote}
שפת ליספ שימשה גם לפיתוח \E|Grammarly|, התוכנה המפורסמת לתיקוני שגיאות כתיב
ודקדוק.

במהלך השנים, נולדו ניבים רבים לליספ שהרחיבו בהרבה את ההגדרה המקורית. בשנת 1984
אוחדו רבים מהניבים הללו לכדי הגדרה של שפה אחת הידועה בשם \E|Common Lisp|. ניבים
נפוצים אחרים כוללים את Emacs Lisp (המשמשת כשפת סקריפטים בעבור עורך הקבצים
(\E|text editor|) הידוע בשם \E|Emacs|), AutoLisp (המשמשת כשפת סקריפטים בתכנת
\E|AutoCad|), \E|Scheme|, \E|Clojure|, ואפילו שפת התכנות \E|Dylan|, אשר למרות
הדקדוק השונה בתכלית שלה, היא מבוססת על ליספ.

%
%בהינתן שפת תכנות אוניברסלית~$𝓛$ יש טעם לשאול באיזו שפה כתוב המהדר או המפרש
%של~$𝓛\). אך, כיוון ש-\(𝓛$ היא אוניברסלית, הרי אם אפשר לכתוב את המהדר
%(לחילופין, המפרש) של~$𝓛$ בשפה '~$𝓛$ הרי גם ניתן לכתוב את המהדר (לחילופין,
%המפרש) בשפה~$𝓛$ עצמה. מתברר שמקובל מאוד לכתוב את המהדר של השפה בשפה עצמה. כך
%למשל, המהדר של שְׂפַת \סי כתוב בִּשְׂפַת \סי. המהדר של שְׂפַת \גאוה כתוב בִּשְׂפַת \גאוה,
%וכו'. כמובן שהדבר מעורר קושי: כי אם המהדר עבור שפה מסוימת כתוב באותה שפה, הרי
%כיצד הודר המהדר? התשובה הפשוטה היא שהמהדר הידר את עצמו, וכך הם בדרך כלל פני
%הדברים. אלא, שהדבר מוביל לרקורסיה אינסופית.
%
%חז"ל-הבחינו בבעיה דומה של רקורסיה אינסופית מעין זו בסוגיא התלמודית הידועה בשם
%"צבת בצבת עשוייה". בגמרא, במסכת פסחים דף נ"ד עמוד א' נאמר:
%
%\יניב{צבתא בצבתא מתעבדא וצבתא קמייתא מאן עבד הא לאי בריה בידי שמים}
%
%ובתרגום לעברית: "הצבת אינה נעשית אלא בצבת אחרת. וראשונה מי עשאה? על כרחך מאליה
%נעשית בידי שמים.". כלומר, צבת שהיא מכשיר לאחיזת מטילי ברזל לשם ליבונם באש
%ועיבודם, עשוייה אף היא ברזל, ואף היא מיוצרת בצבת אחרת שקדמה לה. כיצד אם כן
%נוצרה הצבת הראשונה? הפתרון המוצע על ידי הגמרא הוא שהצבת הראשונה נבראה בערב שבת
%הראשון, בזמן "בין השמשות", שעה שאלוהים סיים לברוא את כל הדברים האחרים, והתכונן
%לשבות ממלאכתו לקראת ירידת השבת.
%
%פתרון ניסי שכזה אינו בא בחשבון עבור שפות תכנות. מסכת פסחים מציגה גם דרך אחרת
%שבה יוצרה הצבת הראשונה (על ידי דפוס נחושת). לעומת זאת, בשפות תכנות, ניתן לבצע
%תהליך של \E|bootstrapping| שבאמצעותו ניתן לפתח מהדר המסוגל להדר את עצמו.
%התהליך דומה מאוד למה שהיה עושה נַפָּח עני שברשותו ברזל, אך לא כסף לרכישת צבת. נפח
%כזה היה משתמש בכבשנו באופן איטרטיבי, כאשר בכל פעם הוא היה משתמש בגוש הברזל
%הדומה ביותר לצבת שיש ברשותו, כדי ליצר קירוב טוב יותר לצבת.
%
%§§ מדד לאלגנטיות של שפה
%אבן בוחן מרתקת לאלגנטיות של שְׂפַת תכנות היא אורך המהדר (או המפרש) של השפה, כאשר
%הוא כתוב בשפה עצמה. שהרי ככל שהשפה מורכבת ומתוחכמת יותר, מחד קל יותר לכתוב את
%המהדר, אך מאידך המהדר לעסוק בכל המורכבות והעושר הזה. לחילופין, שפה שהיא פשוטה
%ביחס (כמו שְׂפַת ה-\שי{batch} של \קד{DOS}) שהוזכרה מעלה, היא קלה אולי להידור, אבל
%הפשטות של השפה מהווה אבן נגף בבואנו להשתמש בה כדי לכתוב מהדר. הנה מספר אורכים
%אופייני של מהדר לשפה הכתוב בשפה עצמה:
%\begin{enumerate}
% • מהדר לשפת \סי הכתוב בשפה עצמה, דורש כמה עשרות אלפי שורות. המהדר \שי{gcc}
% מתפרס על פני כשבעה מיליון שורות קוד.
% • המהדר הראשון לשפת פסקל, אשר נכתב בִּשְׂפַת \פסקל דרש כשבעת אלפים ומאתיים שורות.
% • משערך לשפת ליספ הכתוב בליספ, הידוע גם כפונקציה האוניברסלית \קד{eval} דורש
% כמאה שורות.
% • לעומת זאת, מפרש בסיסי לשפת פרולוג הכתוב בפרולוג יכול להכתב בשורה אחת בלבד,
% ומפרש מתוחכם, המאפשר למשל מעקב אחרי החישוב, לא ידרוש בדרך כלל יותר מעשר
% שורות.
%\end{enumerate}
%
%\begin{editing}
% §§ הגדרת שפות להגדרת שפה פורמלית, באמצעות עצמן
% מהי שפה פורמלית, כל שְׂפַת תכנות היא שפה פורמלית. ישנן שפות פורמליות שאינן שפות
% תכנות. שפות תכנות אינן מכניזם נוח להגדרת שפה פורמלית. מכניזמים להגדרת שפות
% פורמליות. מתברר שגם מכניזמים אלו הם שפה פורמלית.
%
% בדרך כלל, קל הרבה יותר להגדיר שפה פורמלית, מאשר לכתוב מהדר של השפה בעזרת
% עצמה. הנה דוגמאות.
% \begin{itemize}
% ✦ הגדרת BNF בעזרת עצמו
% ✦ הגדרת EBNF בעזרת עצמו
% ✦ ביטוי רגולרי המגדיר מהו ביטוי רגולרי חוקי
% \end{itemize}
% קל להגדיר את משפחת ה-BNF באמצעות ביטוי רגולרי.
% קל להשתמש ב-BNF כדי להגדיר מהו ביטוי רגוליר.
% אבל, ניתן להגדיר ביטוי רגולרי באמצעות ביטוי רגולרי? לא! רקורסיה.
% טבלת סיכום, הגדרה הדדית.
%
% מסיבה זו, השפות BNF וְ-EBNF הן אלגנטיות יותר מביטויים רגולריים.
%
%\end{editing}

§ מיני-ליספ
הדיון כאן איננו עוסק בליספ כשפת תכנות, וגם לא באף אחד מהניבים הנפוצים או הפחות
נפוצים שלה, אלא מתרכז אך ורק בהצגת גלעין \E|(core)| מינימלי ביותר של השפה, שהוא
בעצמו ניב של ליספ. אנו נכנה ניב זה בשם \ע|מיני-ליספ|. למעט הבדלים זעירים, תכנית
מיני-ליספ היא, בדרך כלל, תכנית חוקית של \E|Common Lisp|. אבל, בדרך כלל, תכנית
של \E|Common Lisp| אינה תכנית חוקית של מיני-ליספ.

בניגוד לניבים האחרים של ליספ, שפת מיני-ליספ אינה תומכת במספרים ובפעולות
אריתמטיות, והיא מזניחה את ענין היעילות כמעט לחלוטין---לא נעשה כל מאמץ להבטיח כי
המימוש של מיני-ליספ יהיה יעיל. בנוסף, רוב הניבים של ליספ, יודעים להדר, באופן
חלקי או מלא, תכניות ליספ לשפת מכונה. גם הידור כזה אינו נחוץ במיני-ליספ.

על אף המינימליות של שפת מיני-ליספ, השפה היא מה שקרוי \E|Turing complete|. לא
נגדיר מונח זה בדיוק כאן. אבל, משמעות הטענה היא שניתן לכתוב במיני-ליספ כל תכנית
שאפשר לכתוב בשפות תכנות פחות סגפניות ממנה, כמו פסקל ו-\E|C| למשל. אנחנו
נראה בהמשך שאפשר להוסיף למיני-ליספ תמיכה במספרים טבעיים. באופן דומה ניתן
להרחיבה כך שתתמוך במספרים שליליים, רציונליים, ממשיים, מרוכבים, וגם מחרוזות,
וברבות מהתכונות של ניבים אחרים של ליספ. מטבע הדברים, הרחבות אלו לא תהיה יעילות
במיוחד.

הביצוע של תכניות בשפת מיני-ליספ אינו יעיל. גם תכנות בשפה רחוק מלהיות יעיל:
בהעדר, למשל, תמיכה בפעולת החיבור או בפעולות קלט פלט. לעומת זאת, שפת מיני-ליספ
יעילה מאוד ללימוד: בזכות המינימליות של השפה ניתן ללמוד אותה על בוריה בקלות.
כדי להבין את השפה כולה, כל מה שנדרש להכיר הוא את הדקדוק הפשוט מאוד של
ביטויי~\E|S|, את שמונת הפונקציות האטומיות של השפה, את שמונה הפונקציות המוגדרות
מראש, ואת אלגוריתם השיערוך. את כל אלו נעשה כאן.

אכן, שפת מיני-ליספ אינה משמשת בדרך כלל לתכנות. היא פותחה על ידי מחבר מסמך זה
לצורך הוראה והדגמת העקרונות היסודיים של שפת ליספ, פרדיגמת התכנות הפונקציונלית,
ואלגוריתם השיערוך. למרבה הצער, אין עדיין מימוש מדוייק של שפת מיני-ליספ, כלומר,
מימוש שאינו מסתמך על \E|Common Lisp|, והעדר מימוש זה אינו מפתיע במיוחד.

כאמור, ישנם שלושה מרכיבים עיקריים למיני-ליספ:
\begin{description}
  ✦ [פונקציות אטומיות] אילו הן פונקציות פשוטות אשר הן "אקסיומטיות", כלומר,
  הגלעין מניח שהן קיימות וכי הן מצייתות למפרט מוגדר היטב. אולם, האופן שבו
  ממומשות הפונקציות האטומיות אינו חלק ממיני-ליספ.

  לעיתים קוראים לפונקציות האטומיות גם \ע|פונקציות פרימיטיביות|, במובן זה שהן
  הבסיסיות ביותר. אנחנו נעדיף את המינוח "האטומי", המדגיש את העובדה שהן בלתי
  ניתנות ל-"חלוקה", ואם נעיין בקרבי המימוש שלהן, לא נמצא שם פונקציות אחרות של
  מיני-ליספ, אלא מבנים אחרים, ככל הנראה מימוש בשפת~\CPL.

  ✦ [פונקציות מוגדרות מראש] בנוסף לפונקציות האטומיות, מציעה מיני-ליספ לנוחות
  המשתמש בשפה מספר פונקציות נוספות, אשר אותן מממש הגלעין באמצעות קריאה
  לאחת או יותר מהפונקציות האטומיות.

  בבדיקה השוואתית של שפות תכנות, נבחין בין פונקציות מוגדרות מראש ובין
  \ע|פונקציות ספרייה|. ספרייה של פונקציות היא אוסף של פונקציות המאורגנות יחד
  והממוממשות באמצעות האטומים של השפה. המשמשות למטרה קרובה או דומה. המשתמש
  בשפה רשאי, אך אינו חייב להשתמש בספרייה, ולכן פונקציות הספרייה הן אופציונליות
  בעבורו. לעומתן, קבוצת הפונקציות המוגדרות מראש בשפה היא חלק מהגדרת שפת התכנות,
  והמשתמש אינו יכול לבחור אם להשתמש בה אם לאו.

  בשפת התכנות~\CPL אין פונקציות מוגדרות מראש. גם פונקציה בסיסית כגון printf
  המשמשת להוצאת פלט ממוממשת כחלק מספרייה. ישנן סיפריות רבות לשפת~\CPL, אך
  הספרייה אשר מכילה את הפונקציה printf היא מיוחדת בכך שהיא קרויה \ע|הספרייה
  הסטנדרטית| (בה' הידיעה), או \E|libc|. מקובל לכלול את \E|libc| כחלק
  מתכניות~\CPL. אולם, ניתן לכתוב תכניות~\CPL גם מבלי להשתמש בספרייה זו, וישנן
  תכניות שימושיות שאינן משתמשות כלל בסיפריות של השפה.

  המקבילה בפסקל לפונקציה printf בשפת~\CPL, היא הפרוצדורה הידועה בשם
  \E|WriteLn|. פרוצדורה זו מוגדרת מראש בשפה. לא ניתן לכתוב תכנית פסקל אשר אינה
  כוללת את \E|WriteLn|, אם כי המתכנת אינו חייב להשתמש בפרוצדורה זו, והוא אף
  יכול להשתמש בשם \E|WriteLn| לצרכיו, ובכך להסתיר את הפרוצדורה המקורית.

  ✦[אלגוריתם השיערוך] שפת מיני-ליספ, כמו בניבים אחרים של ליספ, מכילה פונקציה
  מיוחדת, \E|eval| שמה, אשר מקבלת ביטוי~\E|S| ומשערכת אותו. מסיבות טכניות
  (עליהן נעמוד בקצרה בהמשך) הפונקציה eval נחשבת אף היא פונקציה אטומית, אולם, את
  רובה ככולה ניתן לממש כפונקצית ספרייה.
\end{description}

§§ הפונקציות האטומיות
כל הפונקציות האטומיות של מיני-ליספ הן גם פונקציות אטומיות של מרבית הניבים
החשובים של ליספ, ובפרט של \E|Common Lisp|, אם כי יתכנו הבדלים קלים במשמעות של
הפונקציות האטומיות בין מיני-ליספ ובין הניבים השונים.

בניבים אחרים של ליספ, יש בדרך כלל מספר רב של פונקציות אטומיות נוספות. לעומת
זאת, יש במיני-ליספ שמונה פונקציות אטומיות בלבד (למעט \E|eval|):
\ציינן
✦ \ע|שלוש פונקציות מבניות|: \E|car|, \E|cdr| ו-\E|cons|, אשר מאפשרות ליצור ביטוי
\E|S|, ולפרק אותו לחלקיו.

✦ \ע|שתי פונקציות לוגיות|: \E|atom|, \E|eq| ו-\E|cond|, המאפשרות לבדוק את תוכנו של
ביטוי~\E|S|.

✦ \ע|שתי פונקציות נוספות|: הפונקציה \E|set| המאפשרת לתת שמות לביטויי~\E|S|,
והפונקציה \E|error| המסייעת בטיפול במקרים שבהם החישוב נתקל בשגיאה.
===

הערכים בהם מטפלת שפת ליספ הם ביטויי~\E|S|, ונדרשות במיני-ליספ שמונה פונקציות
אטומיות כדי לבצע את כל המניפולציות הנחוצות של ערכים אלו, וביניהן פעולה המקבילה
להצבה ופעולה המקבילה להדפסת שגיאה. בהשוואה לאלו, התמיכה של שפת פסקל בטיפוס של
מספרים שלמים משתמשת ב-14 פריטים שונים: 5 אופרטורים אריתמטיים בינאריים, 2
אופרטורים אריתמטיים אונאריים, 6 אופרטורים של השוואה, וכן סימן מיוחד, \E|:=|,
אשר אינו נחשב לאופרטור בשפת פסקל, לציון פעולת ההצבה. (שפת פסקל אינ מציעה תמיכה
יחודית לתמיכה בשגיאות).

\פנה|טבלה:אטומיות| שבהמשך מתארת את הפונקציות האטומיות במדוייק. מיני-ליספ מעלה
על נס את המינימליות, ואכן למרות שהטבלה מביאה מפרט מלא של הפונקציות הללו ואאת כל
מה שנדרש כדי להשתמש בהן, היא אינה משתרעת על פני יותר ממחצית העמוד. (יש עדיין
צורך בכמה הגדרות וסימונים כדי לקרוא את הטבלה ולהבין את משמעות 8 הפונקציות בה,
אולם ניכר כי תיאורן קצר.)

אנו נתאר ראשית את הפונקציות המבניות, ואחר כך את הפונקציות הלוגיות \E|atom|,
ו-\E|eq| יחד עם הפונקציה המוגדרת מראש \E|null|. המשך הדיון יוביל לתיאורה של
הפונקציה \E|set|, ואחריה \E|cond|. הצורך בפונקציה \E|error| יגיע כאשר נממש את
אלגוריתם השיערוך במיני-ליספ.

§§ פונקציות מוגדרות מראש

מיני-ליספ מכילה שמונה פונקציות מוגדרות מראש, כלומר פונקציות הכתובות
בשפת מיני-ליספ תוך שימוש בפונקציות האטומיות, ואשר נטענות תמיד יחד עם
מיני-ליספ.

\ציינן
✦ \ע|קבועים| (פונקציות ללא פרמטרים):
\begin{itemize}
  ✦ \E|t| (המציין את הערך הבוליאני של אמת).
  ✦ \E|nil| (המציין את הערך הבוליאני של שקר).
\end{itemize}
✦ \ע|פונקציה לוגית|: \E|null| (פונקציה חד-מקומיות הבודקת אם ביטוי הוא \E|nil|).
✦ \ע| פונקציות המסייעות בהגדרת פונקציות|:
\E|quote|, \E|defun|, \E|ndefun|, \E|lambda| ו-\E|nlambda|.
\begin{itemize}
  ✦ הפונקציה defun משמשת להגדרת פונקציות חדשות.
  ✦ הפונקציה quote משמשת למניעת השיערוך של ביטויי~\E|S|.
  ✦ בפונקציות \E|ndefun|, \E|lambda| ו-\E|nlambda|, נדון בהמשך.
\end{itemize}
===

מלבד ndefun ו-nlambda ניתן למצוא, לעיתים בשינויים קלים, את הפונקציות המוגדרות
מראש של מיני-ליספ גם ב-\E|Common lisp| ובניבים אחרים של השפה.
\פנה|טבלה:מראש| שבהמשך מתארת את הפונקציות המוגדרות מראש, והיא כוללת, בנוסף
לתיאור הפונקציות הללו, גם את המימוש המלא שלהן במיני-ליספ, ובכל זאת, טבלה זו
קצרה אף יותר מ\פנה|טבלה:אטומיות| המתארת את הפונקציות האטומיות.

בהמשך נראה כי מימושן של הפונקציות המוגדרות מראש נעשה באמצעות קשירה
(\E|binding|) של שמן למה שקרוי "ביטוי \E|lambda|" או "ביטוי \E|nlambda|",
באמצעות הפונקציה \E|set| (ישירות או בעקיפין). במונח binding אנו מתכוונים לקישור
בין שם לבין המשויים אותו מציין השם. נניח ש-$f$ היא פונקציה מוגדרת מראש. אזי~$f$
היא ביטוי S שהוא מהצורה המיוחדת של ביטוי \E|lambda| או מהצורה המיוחדת של
ביטוי \E|nlambda|. מתן שם למשויים~$f$ הוא הקישור בין אטום של מיני-ליספ (שהוא
השם) למשויים~$f$.

קישור של שם למשויים קיים בכל שפות התכנות: כאשר אנו מגדירים בשפת תכנות כמו פסקל
משתנה אשר שמו הוא \T|i|, אנו נותנים שם לתא בזכרון שמכיל את ערכו של המשתנה.
המשויים הוא תא בזיכרון שמכיל את ערכו של המשתנה, ו-\T|i| הוא השם שאנו קושרים לתא
זה. המונחים קישור והגדרה הם זהים: פונקציות מוגדרות מראש בשפת מיני-ליספ הן
פונקציות אשר הקישור בין שמן ובין הגוף שלהן נעשה עוד טרם ריצת התכנית. כאשר אנו
אומרים שהסימן \T|+| מציין את פעולת החיבור בשפת פסקל, אנו בעצם אומרים שיש קישור
בין הסימן ובין הפעולה. גם הסימן \T|+| וגם \T|i| הם שמות של משוימים. הסימן~\T|+|
מציין משויים שהוא פעולה, ואילו \T|i| מציין משויים שהוא משתנה. הבדל אחד, אך לא
הבדל מהותי, בין~\T|+| ובין~\T|i| הוא שלמתכנת (בשפת פסקל לפחות) אין אפשרות
להשתמש בסימן~\T|+| כדי לציין משויימים שהוא יצר.

הדיון כאן יתאר תחילה את הקבועים, כלומר הפונקציות האפס-מקומיות, \E|t| ו-\E|nil|,
ולאחר את הפונקציה החד-מקומית \E|null|. אחר כך נעבור לתאר את \E|quote|, את
\E|defun| ואת \E|lambda|. הפונקציות \E|ndefun| ושל \E|nlambda| דורשות מעט יותר
תשומת לב, והן תוצגנה אחרונות.

§§ הפונקציה eval והסביבה

בנוסף לפונקציות האטומיות ולפונקציות המוגדרות מראש, שפת מיני-ליספ תומכת, כמו
מרבית הניבים של ליספ, בפונקציה \E|eval| אשר מקבלת ביטוי~\E|S| ומשערכת אותו.
הפונקציה eval גם היא נחשבת, מסיבות טכניות, לפונקציה אטומית. אנחנו נדגים כיצד
ניתן לממש אותה באמצעות מימוש פונקציה דומה \E|evaluate| כפונקצית ספרייה.

ישנו הבדל קטן בין \E|eval|, הפונקציה הפרימיטיבית, ובין \E|evaluate|, אותה נממש
כפונקציית ספרייה. שתי הפונקציות מקבלות ביטוי-S ומשערכות אותו, אלא שהפונקציה
\E|evaluate| מקבלת פרמטר נוסף, המתאר את הקישורים שנעשו בתכנית עד כה.
לאוסף זה של ההגדרות, קוראים \ע|סביבה|. הסביבה ממומשת בשפת ליספ באמצעות
מבנה נתונים הקרוי \E|a-list|. הפונקציה eval משתמשת אף היא בסביבה, אלא שבתוך
המימוש של מיני-ליספ, אין צורך להעביר את הסביבה כפרמטר: רשימת ה-\E|a-list| נגישה
לפונקציה \E|eval| כמין משתנה גלובלי, אותו אין צורך להעביר כפרמטר מפורש, וניתן
לשנותו מכל נקודה בתכנית. העדר הצורך להעביר את ה-\E|a-list| כפרמטר ל-eval הוא זה
אשר מבחין בינה ובין \E|evaluate| אשר אין לה גישה לערכים גלובליים, ועל כן היא
נדרשת לקבל את הסביבה, המיוצגת באמצעות-\E|a-list| כפרמטר.

המימוש של האלגוריתם שמאחורי \E|eval| (כלומר המימוש של הפונקציה \E|evaluate|)
נעשה כולו במיני-ליספ, תוך שימוש בפונקציות אטומיות או פונקציות מוגדרות מראש, אשר
כאמור גם הן ממומשות באמצעות הפונקציות האטומיות. מימוש זה הוא פשוט וקצר יחסית
ועולה לכדי כמאה שורות בלבד.

לבד מהמימוש של eval שפת מיני-ליספ מעניינת שכן היא מציגה בתמציתיות את מרבית
הרעיונות החשובים של ליספ, כגון עיבוד רקורסיבי של רשימות, קישור של שמות לערכים,
העברת פרמטרים, הסתכלות על תכנית כמבנה נתונים, ועוד.

§§ עצי שיערוך
לא רק תכניות הם ביטויי~\E|S|. כל הערכים בשפת ליספ הם ביטויי~\E|S|
\E|(S-Expressions)|. המונח~S-Expression בא לעולם כקיצור למונח \E|symbolic
expression| (ביטוי סימבולי). המילה "סימבולי" מדגישה את העובדה כי אין מדובר
בהכרח בביטוי מתמטי כגון~$13+41/2$, שבו המשמעות של כל הסימנים ידועה ומוכרת לכל,
אלא בביטוי \E|כללי| שיכול להכיל סמלים שמשמעותם דורשת הגדרה מפורשת, כגון
\begin{equation}
  (\amalg⊚✠) ⊘ (\Re≀⅁)
\end{equation}
פרשנות הביטוי הזה תלוייה במשמעות הסמלים המופיעים בו: האם הם אופרטורים בינאריים
(דו-מקומיים), אונאריים (חד מקומיים), או שמא הם אופרנדים, כלומר נולאריים
(אפס-מקומיים), הקריים גם אופרנדים.

אם האופרנדים בביטוי הם~$\amalg$,~$✠$,~$\Re$ ו-$⅁$, והאופרטורים
(הבינאריים) הם~$⊚$,~$⊘$ ו-$≀$, אזי עץ השיערוך הוא:
\begin{quote}
  \center
  \begin{forest}
    s tree [$⊘$,[$⊚$ [$\amalg$] [$✠$]] [$≀$[$\Re$][$⅁$]]]
  \end{forest}
\end{quote}
אולם, יתכן גם כי האופרנדים הם הסימנים~$⅁$ ו-$✠$, כלומר אלו הם אופרטורים
נולאריים, ואילו~$\amalg$,~$⊚$,~$✠$ ו-$\Re$ הם אופרטורים אונאריים (ממבנה
הסוגריים בביטוי עולה כי~$⊘$ הינו אופרטור בינארי). במקרה זה, עץ השיערוך הוא
\begin{quote}
  \center
  \begin{forest}
    s tree [$⊘$,[$\amalg$ [$⊚$ [$\amalg$]]]
          [$\Re$ [$≀$ [$⅁$]]]]
  \end{forest}
\end{quote}
בין כך, ובין כך, כיוון שלא רק המקומיות של הסמלים המופיעים בביטוי אינה ידועה,
אלא גם משמעותם אינה ידועה, לא נוכל לשערך את הביטוי.

נשים לב לכך שכל ביטוי בשפת פסקל (למשל) ניתן להצגה כעץ שיערוך, וכי כל עץ שיערוך
ניתן להצגה כרשימה של רשימות. אולם, לא כל רשימה של רשימות ניתנת להצגה כעץ
שיערוך. בפרט, אם הפריט הראשון ברשימה הוא בעצמו רשימה, כמו למשל כאן,
\begin{LISP}
((car '(f g)) x)
\end{LISP}
אזי לא ניתן להציג את הרשימה כולה כעץ שיערוך. הסיבה לכך היא שהטופולוגיה של עצי
השיערוך בה עשינו שימוש מניחה שבכל צומת פנימי של העץ, ישנה תגית, המציינת את
הפונקציה שיש להפעיל בצומת זה.

§§ האינטרפרטר של ליספ
אנו רואים כי המונח שיערוך כולל בתוכו שני מרכיבים עיקריים: ראשית, איתור המשמעות
של הסמלים המופיעים בביטוי, ושנית, חישוב הביטוי בהתאם למשמעות זו.

בדרך כלל מרכיב החישוב של השיערוך מחשב ביטוי חדש, אך לעיתים השיערוך מייצר משמעות
בעבור סמלים שלא הייתה להם משמעות טרם השיערוך: השיערוך אינו לקוח בלבד
בלבד של הגדרות הסמלים, הוא גם עשוי לייצר הגדרות חדשות כאלו.

בשפות תכנות כדוגמת~\CPL ופסקל, העוברות \ע|הידור| \E|(compilation)| שני מרכיבי
השיערוך נפרדים. מתן המשמעות לסמלים ואיתור המשמעות של סמלים נעשה על ידי
ה\ע|מהדר| \E|(compiler)|. החישוב עצמו נעשה בזמן ריצת התכנית. בליספ, כמו בשפות
אחרות שבהן עיבוד התכנית בשפה נעשה באמצעות אינטרפרטציה \E|(interpretation)| של
תכניות, שני השלבים של השיערוך נעשים על ידי האינטרפרטר \E|(interpreter)| אשר
קורא תכניות ומשערך אותן בזו אחר זו. האינטרפרטציה כוללת לכן לא רק את תהליך החישוב,
אלא גם את התהליכים הנלווים של מתן משמעות לסמלים, ואיתור משמעות זו.

\begin{minipage}\linewidth
  \footnotesize
  \begin{mdframed}[backgroundcolor=Lavender!20]
    האינטרפרטר של ליספ, כמו האינטרפרטר של פרולוג, \E|bash| ושפות תכנות אחרות
    העוברות אינטרפרטציה, עובד במחזורים, בשיטה הידועה בשם
    \LR{\textbf Read \textbf Evaluate \textbf Print \textbf Loop}
    או בראשי תיבות \E|REPL|:
    \ספרר
    ✦ \E|READ|: האינטרפרטר קורא את סדרת תווים מהקלט, ומנסה להציג את סדרה זו
    כמבנה בשפת התכנות. במקרה של ליספ, האינטרפרטר קורא סדרת תווים, ומנסה להציג
    אות כביטוי~\E|S|. אם לא ניתן לעשות כן, האינטרפרטר מדפיס הודעת שגיאה וחוזר
    לתחילת הלולאה של \E|REPL|, כלומר חוזר לקרוא קלט חדש.
    ✦ \E|EVALUATE|: האינטרפרטר משערך את ערכו של הביטוי שקרא.
    ✦ \E|PRINT|: אם השיערוך של הקלט מצליח, אז האינטרפרטר מדפיס את תוצאת
    השיערוך. אחרת, האינטרפרטר ידפיס הודעת שגיאה מתאימה.
    ✦\E|LOOP|: האינטרפרטר חוזר לצעד הראשון, לקריאת הקלט הבא.
===
  \end{mdframed}
\end{minipage}

האינטרפרטר של ליספ מכיל לכן שלושה מרכיבים עיקריים:
\begin{enumerate}
  ✦ ה-Reader אשר קורא קלט שהוא סדרה של תווים, ובונה את מבנה הנתונים המתאים
  לביטוי ה-S המתאים לתווים אלו.
  ✦ המשערך \E|(Evaluator)| אשר משערך ביטויי~\E|S| אלו.
  ✦ ה-Printer אשר מקבל ביטוי~\E|S| כמבנה נתונים ומתרגם אותו לסדרת תווים אשר
  מייצגת מבנה נתונים זה.
\end{enumerate}

כאן נניח שה-Reader וה-Printer נתונים והם דומים לאלו שבכל ניב של ליספ, ונתאר את
המימוש של המשערך של מיני-ליספ.

§ ביטויי~\E|S|
§§ ביטויי~\E|S| כשפה פורמלית

\newcommand\SX{\ensuremath{S_{\text{exp}}}}

בעבור מי שאין לו די בתיאור האינטואיבי של ביטויי \E|S|, תת פרק זה מוסיף הגדרות
מדוייקות מעט יותר: בהינתן אלפאבית, סופי או אינסופי,~$Γ$, נגדיר את~$\SX(Γ)$,
\textbf{קבוצת ביטויי ה-\E|S|, מעל האלפאבית~$Γ$}. אינטואיטיבית, ביטוי~\E|S| יכול
להיות אטומי, כלומר ביטוי S שאינו ניתן לפירוק לביטויי S אחרים, ואז הוא חייב
להיות סימן מתוך~$Γ$. ביטוי~\E|S| שאינו אטומי נקרא ביטוי מורכב, ואז הוא בהכרח
זוג סדור של שני ביטויי~\E|S| אחרים, שיכולים להיות אטומיים או מורכבים. הזוג
הסדור נכתב עטוף בזוג סוגריים כאשר שני ביטויי ה-S שבו מופרדים בסימן הנקודה.

כמה ביטויי~\E|S| מעל האלפאבית (הסופי)~$Γ=❴⌘a,⌘b,⌘c❵$ הם \[
  a,b,(a.b),(c.(b.a)),((a.b).(a.c))∈\SX❨❴a,b,c❵❩.
\] לעומת זאת,~$(a.b.c)$ אינו שייך ל~$\SX(Γ)$ משום שהוא שלשה סדורה ולא זוג סדור,
ואילו~$(a(b(c)))$ אינו שייך לקבוצה, משום שהוא אינו עונה על הדרישה שבין פריטים
ימצא סימן הנקודה.

ניתן לאפיין את הקבוצה~$\SX(Γ)$ באמצעות שני כללי היסק:

\begin{definition}[ביטויי~\E|S| מעל
    אלפאבית] בהנתן אלפאבית~$Γ$ אזי,~$\SX(Γ)$, קבוצת ביטויי ה-S מעל האלפאבי~$Γ$
  מוגדרת באמצעות שני בנאים
  \begin{enumerate}
    ✦ הבנאי הנולארי, המגדיר את הביטויים ה\ע|אטומיים|, כלומר, ביטויי~\E|S| אשר
    אינם ניתנים לפירוק לביטויי~\E|S| אחרים. בנאי זה מוגדר באמצעות כלל היסק שיש
    לו הנחה אחת בלבד,~$γ∈Γ$.
    \begin{equation*}
      \infer{γ∈\SX(Γ)}{γ∈Γ}.
    \end{equation*}
    לפי בנאי זה, כל מילה ב-$Γ$ היא ביטוי~\E|S| אטומי. הבנאי נקרא נולארי, משום
    שעל פי בנאי זה ניתן "ליצור" איברים של הקבוצה~$\SX(Γ)$ ללא "שימוש" באיברים
    אחרים של הקבוצה.
    ✦ הבנאי הבינארי, המתאר את המבנה של ביטויי~\E|S| \ע|מורכבים|:
    \begin{equation*}
      \infer{(s₁.s₂)∈\SX(Γ)}{s₁∈\SX(Γ) & s₂∈\SX(Γ)}.
    \end{equation*}
    לפי בנאי זה, אם~$s₁$ ו-$s₂$ הם ביטויי~\E|S| (שיכולים להיות אטומיים או
    מורכבים) אזי~$(s₁.s₂)$ הוא ביטוי~\E|S| (מורכב). בנאי זה נקרא בנאי
    בינארי, משום שבכלל ההיסק המתאר אותו
    יש שתי הנחות, וכל אחת מתייחסת לאיברים של~$\SX(Γ)$. במילים אחרות, על פי בנאי
    זה ניתן "ליצור" איבר חדש של הקבוצה~$\SX(Γ)$ מתוך "שימוש" ב-\ע|שני| איברים
    קיימים של הקבוצה.
  \end{enumerate}
\end{definition}

כאמור, נוח לייצג ביטויי~\E|S| כעצים בינאריים מלאים (כלומר, כאלו בהם לכל צומת
שאינו עלה יש בדיוק שני בנים) כאשר צומת פנימי של עץ כזה אינו נושא מידע, ואילו
עלה מכיל סימן מתוך~$Γ$. \פנה|איור:בינארי| מציג תיאור כעץ בינארי מלא של כמה
ביטויי~\E|S| מעל האלפאבית (האינסופי) \[
  Γ=❴a,b,c,d❵^*,
\] כלומר, קבוצת ביטויי~\E|S| שהאטומים שלהם הם מילים
הנבנות מהאלפאבית (הסופי)~$❴a,b,c,d$.

\newcommand{\TopAlign}[1]{\adjustbox{valign=t}{#1}}
\newcolumntype{T}{>{\collectcell{\TopAlign}}c<{\endcollectcell}}

\begin{figure}[htbp]
  \כיתוב|ביטויי~S והייצוג שלהם כעץ בינארי מלא|
  \תגית|איור:בינארי|
  \centering
  \begin{LTR}
    \rowcolors{2}{blue!10}{white}
    \begin{tabular}{*7T}%
      $(ab.cd)$                                                                             &
      $(abcd.ε)$                                                                            &
      $((ab.cd).ε)$                                                                         &
      $(ab.(c.d))$                                                                          &
      $((a.b).(c.d))$                                                                       &
      $(a.(b.(c.d)))$                                                                       &
      \multicolumn1c{$(((a.b).c).d)$} ⏎
      \scriptsize
      \Forest{s tree [{},cons[$ab$,atom][$cd$,atom]]}                                       &
      \scriptsize
      \Forest{s tree [{},cons[$abcd$,atom][$ε$,atom]]}                                      &
      \scriptsize
      \Forest{s tree [{},cons[\relax,cons[$ab$,atom][$cd$,atom]][$ε$,atom]]}                &
      \scriptsize
      \Forest{s tree [{},cons[$ab$,atom][{},cons[$c$,atom][$d$,atom]]]}                     &
      \scriptsize
      \Forest{s tree [{},cons[{},cons[$a$,atom][$b$,atom]][{},cons[$c$,atom][$d$,atom]]]}   &
      \scriptsize
      \Forest{s tree [{},cons [$b$,atom],[{},cons [$b$,atom] [{},cons[c,atom] [d,atom]]]] } &
      \scriptsize
      \Forest{s tree [{},cons [{}, cons [{}, cons
            [a,atom][b,atom]] [c,atom] ] [d,atom]] }
    \end{tabular}
  \end{LTR}
\end{figure}

ניתן להגדיר את~$\SX(Γ)$ גם באמצעות דקדוק חסר הקשר. כלומר, הקבוצה~$\SX(Γ)$ היא
שפה פורמלית מעל אלפאבית מורחב,~$Γ'$, המתקבל מהוספת סימן הנקודה ושני סימני
הסוגריים לאלפאבית המקורי~$Γ$. אנו מניחים, בלי הגבלת הכלליות, ששלושת הסימנים
הללו אינם מצויים ב-$Γ$.
\begin{equation}
  \begin{split}
    S &→(S.S)⏎ S &→A ⏎
    A &→ε⏎ A &→Aγ₁ ⏎
    A &→Aγ₂ ⏎
    ⋮ ⏎
    A &→Aγₙ ⏎
  \end{split}
\end{equation} כאשר~$Γ=❴γ₁,γ₂,…,γₙ❵$.
אבל, הגדרה זו באמצעות דקדוק סופי לא תוכל לפעול כאשר האלפאבית~$Γ$ הינו אינסופי.
כזה המצב למשל בשפת מיני-ליספ, שבה האלפאבית~$Γ$ שמעליו נבנים ביטויי ה-S הוא
אינסופי, ומוגדר כאוסף כל ה\ע|מילים| מעל~$Σ_{\text{Mini-Lisp}}$, האלפאבית המכיל
את כל ה\ע|תווים| שיכולים להופיע במה שקרוי בשפת ליספ \ע|אטום|:
\begin{equation}
  Γ_{\text{Mini-Lisp}}=Σ_{\text{Mini-Lisp}}^*,
\end{equation}
כאשר
\begin{equation}\label{alpahet:C}
  Σ_{\text{Mini-Lisp}}=
  Σ_{\text{upper}}∪
  Σ_{\text{digit}}∪
  Σ_{\text{other}}.
\end{equation}
בפרט~$Σ_{\text{Mini-Lisp}}$ מכיל 67 תווים:
\begin{enumerate}
  ✦ \ע|26 אותיות אנגליות גדולות| \[
    Σ_{\text{upper}}=❴⌘A,⌘B,⌘C,⌘D,⌘E,⌘F,⌘G,
    ⌘H,⌘I,⌘J,⌘K,⌘L,⌘M,⌘N,⌘O,⌘P,⌘Q,⌘R,⌘S,⌘T,⌘U,⌘V,⌘W,⌘X,⌘Y,⌘Z❵.
\] \relax
  ✦ \ע|10 ספרות| \[
    Σ_{\text{digit}}=❴⌘0,⌘1,⌘2,⌘3,⌘4,⌘5,⌘6,⌘7,⌘8,⌘9❵.
\] \relax
  ✦ \ע|11 סימנים מיוחדים| \[
    Σ_{\text{other}}=❴⌘|, ⌘&, ⌘?, ⌘!, ⌘:, ⌘+, ⌘*, ⌘/, ⌘-, ⌘<, ⌘>, ❵.
\] \relax
\end{enumerate}

הייצוג הרגיל של ביטוי~\E|S| במחשב הוא באמצעות עצים בינאריים: כך נעשה במימושים
הראשונים של ליספ. בייצוג זה, כל צומת פנימי של העץ הוא רשומה המכילה שני מצביעים,
שכל אחד מהם יכול להצביע או לרשומה אחרת מסוג צומת, או לאטום, כלומר לעלה של העץ.

§§ ביטויי~\E|S| בשפת ליספ וכתיב הרשימות

הניבים השונים של ליספ משתמשים בהגדרה דומה של ביטויי-\E|S|, ומוסיפות על ההגדרה
מונחים, כינויים, הרחבות וקיצורים שונים. במיני-ליספ נשתמש בשתי תוספות בלבד:
\אבגד
✦ כתיב הרשימות
✦ סימן ה-\E|quote| \E|(')|
===
תוספות אלו תוצגנה בהמשך.

מסיבות היסטוריות, לביטוי~\E|S| מורכב קוראים בשפת ליספ \E|dotted-pair|. לאיבר
הראשון ב-dotted-pair קוראים \E|car|, ולאיבר השני ב-dotted-pair קוראים \E|cdr|.
לזוג כולו קוראים לעיתים גם \ע|רשומת cons|.

לביטוי~\E|S| אטומי בשפת ליספ קוראים אטום \E|(atom)|. דוגמאות לאטומים הן \E|A|,
\E|12B|, \E|ZZZ| ו-\E|+|. מסיבות היסטוריות, האלפאבית של אטומים בליספ אינו מכיל
אותיות קטנות. מערכת ליספ מתרגמת את האותיות הקטנות לאותיות גדולות כשהן מופיעות
בתוך אטומים.

סדרת התווים המגדירה אטום יכולה להיות גם~$ε$, הסידרה הריקה. השם \E|nil| מציין את
האטום שסדרת התווים שלו היא~$ε$. כפי שנראה בהמשך, גם הכתיב \E|()| מציין את האטום
הזה. סדרת התווים המגדירה אטום היא חסרת משמעות בדרך כלל, אולם המתכנת בליספ יכול
להעניק לאטומים נבחרים משמעות. בנוסף, יש מספר אטומים שהמשמעות שלהם מוגדרת מראש
בשפה.

\ע|רשימה| \E|(list)| בליספ היא סדרה של פריטים העטופה בסוגריים. כל פריט ברשימה
יכול להיות אטום, \E|dotted-pair|, או רשימה בעצמו. ניתן לכתוב רווחים לפני ואחרי
כל אחד מהפריטים, אולם שני אטומים רצופים ברשימה חייבים להיות מופרדים בסימן רווח
אחד לפחות.

כך, \E|(a b c d)| היא הרשימה המכילה ארבעה פריטים: האטומים \E|a|, \E|b|, \E|c|
ו-\E|d|. באופן דומה, \E|((a.c)c)| היא רשימה המכילה שני פריטים, שהראשון בהם הוא
ביטויי~\E|S| מורכב, \E|(a.c)|, והשני הוא האטום \E|c|. הרשימה הריקה, זו שאינה
מכילה אף פריט, נכתבת כ-\E|()|. כל הרשימות, למעט הרשימה הריקה הן ביטוי~\E|S|
מורכב. הרשימה הריקה היא גם אטום, בפרט, האטום \E|nil| הוא גם שמה של הרשימה הריקה.

כל רשימה היא כתיב מקוצר לביטוי~\E|S|. הכתיב מוגדר באינדוקציה על אורך הרשימה:
הרשימה הריקה \E|()| שהיא גם כתיב אחר לאטום \E|nil|. רשימה שאינה ריקה, היא כתיב
מקוצר לביטוי~\E|S| מורכב, כלומר זוג של ביטויי~\E|S|. האיבר הראשון בזוג הוא
האיבר הראשון ברשימה. האיבר השני בזוג, הוא הייצוג של שארית הרשימה.

לפיכך, הרשימה \E|(a b c d)| היא כתיב אחר לביטוי~\E|S|
\begin{LISP}
(a.(b.(c.(d.nil))))
\end{LISP}

הפריטים ברשימה הם ביטויי~\E|S|, אבל כיוון שרשימה גם היא ביטוי~\E|S|, רשימה
יכולה להכיל בתוכה רשימות. למשל,
\begin{LISP}
  ((a b) c)
\end{LISP}
היא רשימה המכילה בתוכה שני פריטים, הראשון שבהם הוא רשימה בת שני אטומים, והשני
שבהם הוא אטום. רשימה זו ניתנת לתיאור באמצעות ביטוי~\E|S|,
\begin{LISP}
  ((a.(b.nil)).(c.nil))
\end{LISP}
\פנה|איור:רשימות| מציג רשימות אלו כעצים בינאריים מלאים.

\begin{figure}[!htbp]
  \caption{יצוג רשימות כעצים בינאריים}
  \label{איור:רשימות}
  \begin{LTR}
    \rowcolors{2}{orange!20}{white}
    \begin{tabular}{*5T}
      \lisp{()}                     &
      \T|(A)|                       &
      \T|(A B)|                     &
      \T|(A B C D)|                 &
      \multicolumn1c{\T|((A B) C)|}
 ⏎
      \Forest{%
        s tree [$ε$,atom]
      }                             &
      \Forest{s tree [{},cons [A,atom] [$ε$,atom]]
      }                             &
      \Forest{s tree [{},cons [A,atom] [\relax,cons[B,atom][$ε$,atom]]]
      }                             &
      \Forest{%
      s tree [{},cons [A,atom]
      [{},cons [B,atom]
      [{},cons [C,atom]
      [{},cons [D,atom]
      [$ε$,atom]
      ]
      ]
      ]
      ]
      }                             &
      \Forest{%
      s tree [{},cons
      [{},cons
      [B,atom]
      [{},cons[C,atom] [$ε$,atom] ]
      ]
      [{},cons
      [C,atom]
      [$ε$,atom]
      ]
      ]
      }
    \end{tabular}
  \end{LTR}
\end{figure}

המשמעות של המונח \ע|כתיב| היא שמרכיב ה-Reader של האינטרפרטר של ליספ יכול לקרוא
ביטויי~\E|S| הנתונים בכתיב הרשימות ושמרכיב ה-Printer של האינטרפרטר יתרגם
ביטוי~\E|S| לכתיב הרשימות כשהדבר אפשרי (לא כל ביטוי~\E|S| ניתן להיכתב בכתיב
הרשימות; למשל הביטוי \E|(a.b)| אינו ניתן להכתב כרשימה, אבל כל רשימה ניתנת
להיכתב כביטוי~\E|S|). בהעדר סיבה מיוחדת, מתכנתי ליספ נוהגים לכתוב ביטויי~\E|S|
בכתיב הרשימות בכל אימת שניתן לעשות זאת.

§§ שלוש פעולות היסוד על ביטויי~\E|S|, ושלושת ההיבטים שלהן

§§§ פונקציות מתמטיות מעל קבוצת ביטויי~\E|S|.
על ה\ע|חוג החילופי| \E|(abelian ring)|~$ℤ$ של המספרים השלמים \E|(the ring~$ℤ$
of integers)| מוגדרות שלוש פעולות: שתי הפעולות הבינאריות של הכפל והחילוק,
והפעולה האונארית של מציאת ההופכי החיבורי. באופן דומה, נגדיר שלוש פעולות על כל
קבוצה~$𝓢$ של ביטויי~\E|S|:
\begin{enumerate}
  ✦ הפעולה הבינארית~$p$ המצרפת ב-\E|dotted-pair| שני ביטויים של~$𝓢$,
  \begin{equation}
    p(s₁,s₂)=⌘(s₁⌘.s₂⌘),
  \end{equation}
  ✦ ושתי הפעולות האונאריות ההופכיות לה:
  \begin{enumerate}
    ✦ הפעולה~$p₁^{-1}$ המחזירה את המרכיב
    הראשון של ביטוי מורכב ב-$𝓢$
    \begin{equation}\label{eq:p1}
      p₁^{-1}(s)=\begin{cases}
        s₁ & ∃ s₂, s₁ ∙ s₂=⌘(s₁⌘.s₂⌘) ⏎
        ⊥  & \textit{otherwise},
      \end{cases}
    \end{equation}
    ✦ והפעולה~$p₂^{-1}$ המחזירה את המרכיב השני של ביטוי כזה,
    \begin{equation}\label{eq:p2}
      p₂^{-1}(s)=\begin{cases}
        s₂ & ∃ s₁, s₁ ∙ s₂=⌘(s₁⌘.s₂⌘) ⏎
        ⊥  & \textit{otherwise}.
      \end{cases}
    \end{equation}
  \end{enumerate}
\end{enumerate}
הסימןש~$⊥$ (אותו יש לבטא \E|bottom|) המופיע ב-\פנה|eq:p1| וב-\פנה|eq:p2|
מציין שערך הפונקציות~$p₁^{-1}$ ו-$p₂^{-1}$ אינו מוגדר כאשר הארגומנט שלהן הוא
איבר אטומי של~$𝓢$. נסכם ונאמר
\begin{equation}
  \begin{split}
    p&:𝓢×𝓢→𝓢⏎
    p₁^{-1}&:𝓢 ⇸𝓢⏎
    p₂^{-1}&:𝓢 ⇸𝓢.
  \end{split}
\end{equation}
כלומר,~$p$ היא פונקציה שהתחום הוא זוג של ביטויי~\E|S| והטווח שלה הוא ביטוי כזה.
הפונקציה היא פונקציה \ע|מלאה|, במובן זה \ע|שלכל ערך| בתחום, ממופה ערך בטווח.
לעומת זאת הפונקציות~$p₁^{-1}$ ו-$p₂^{-1}$ אשר התחום והטווח שלהן הוא~$𝓢$, הן
פונקציות חלקיות. שכן \ע|לא לכל ערך| של התחום מתאימות~$p₁^{-1}$ ו-$p₂^{-1}$ ערך
של הטווח. בפרט,~$p₁^{-1}$ ו-$p₂^{-1}$ אינן מוגדרות בעבור האטומים של~$𝓢$.

שפות מעטות בלבד, ובהן \E|Wofram Language| ו-\E|Newspeak| תומכות באופן ישיר
בייצוג של \ע|כל| מספר הלקוח מ-$ℤ$. עצם הייצוג של מספרים שאינן חסומים בגודל
הוא קל לתכנות, למשל באמצעות רשימה (שאינה חסומה בגודלה) של הספרות בייצוג עשרוני.
כמובן, במגבלות יצוג זה מוגבל על ידי הזיכרון הפיזי: גם עבור מחשב שבו הזיכרון
הכולל הוא~$2^{100}$ בתים, קל לתאר מספר ב-$ℤ$ שהוא גדול מכדי שיהיה ניתן לייצגו
במחשב זה.

ובכל זאת, מרבית שפות התכנות אינן תומכות במספרים שלמים במובן של החוג
החיבורי~$ℤ$. חוג זה מכיל אינסוף איברים, ולכן לא ניתן לייצג את כל המספרים
הלקוחים ממנו באף מבנה בעל גודל סופי. מתוך תשומת לב ליעילות התכניות הנכתבות בהן,
שפות תכנות נוטות לייצג מספרים שלמים באמצעות מילת מחשב \E|(word)| בת 32 או 64
ביטים, ולכן הן תומכות רק בקבוצה סופית של מספרים שלמים הלקוחים מ-$ℤ$.

כך לדוגמה, הטיפוס \E|\kk{long}| בשפת \Java הוא קירוב של הקבוצה~$ℤ$: בטיפוס זה מוצגים
מספרים שלמים, שליליים וחיוביים \E|(unsigned int)| בשיטת "המשלים לשתיים" באמצעות מילה בת
64 ביטים.
הערכים האפשריים של הטיפוס \E|\kk{long}| הם מספרים מהתחום
בטיפוס הזה הוא \begin{equation}
-2^{63},…,2^{63}-1=-9,223,372,036,854,775,808,…,9,223,372,036,854,775,807
\end{equation}
על אף היות התחום גדול מאוד, הוא עדיין סופי, ולכן הוא כאיין לעומת
הקבוצה האינסופית~$ℤ$.

יש בטיפוס \E|\kk{long}| לא מעט זוגות של ערכים חיוביים, שהסכום שלהם הוא שלילי.
הסיבה לכך היא שהחישובים ב-\Java בטיפוס \E|\kk{long}| הם חישובים
בקבוצה~$ℤ_{2^{64}}$, הלא היא חוג השאריות מודולו~$2^{64}$. ב-$ℤ_{2^{64}}$. בחוג
זה האבחנה בין מספרים "חיוביים" ו"שליליים" אינה מוגדרת, ויחס ה"סדר"
ב-$ℤ_{2^{64}}$ (ככל שיש כזה) אינו מקיים תכונה חשובה של יחס הסדר של השלמים:
שהעוקב של כל מספר גדול ממנו, או בניסוח אחר, שאין מעגלים ביחס העוקב.

אנו מקבלים לכן שלא זו בלבד שהטיפוס \E|\kk{long}| הוא קירוב של~$ℤ$, גם "\cc+",
הסימן לציון החיבור, וגם "\cc*", הסימן לציון הכפל \Java (כאשר הם
מופעלים על ערכים מטיפוס \E|\kk{long}|), אינם מציינים את פעולות החיבור והכפל
ב-$ℤ$, כי אם קירוב סופי שלהן. בדומה לכך, סימן המינוס, "\cc-", מציין את פעולת
ההופכי החיבורי בקבוצה~$ℤ_{2^{64}}$, אך פעולה זו היא קירוב בלבד של פעולת ההופכי
החיבורי שב-$ℤ$. מספר הערכים השונים של הטיפוס \kk{long} הוא זוגי, וכיוון
שהערך אפס נמנה עם ערכים אלו, קיים ערך בטיפוס ששונה מאפס שהוא הההופכי החיבורי של
עצמו,
תכונה שאינה מתקיימת ב-$ℤ$, בפרט מתקיים בטיפוס זה \begin{equation*}
-❨-2^{63}❩=-2^{63}.
\end{equation*}

בניגוד למרבית שפות התכנות, ו-\E|Common Lisp| בתוכן, מיני-ליספ אינה תומכת
במספרים כלל. כיוון שכך, אין לה צורך להשתמש בטיפוסים כגון \E|\kk{long}|. מנגד,
כיוון שמיני-ליספ תומכת בביטויי~\E|S| בלבד, יש לה צורך לתמוך בשלוש פעולות היסוד
על ביטויי~\E|S|, הלא הן~$p$,~$p₁^{-1}$, ו-$p₂^{-1}$.

האופרטורים של חיבור וכפל במרבית שפות תכנות מהווים קירוב בלבד של הפעולות על
השלמים. בניגוד לזאת, הפעולות בשפת ליספ הן מימוש מלא של הפונקציות
המתמטיות~$p$,~$p₁^{-1}$, ו-$p₂^{-1}$. כפי שנראה בתרגילים, ניתן להשתמש בשפת
ליספ כדי להגדיר יצוג מלא ולא מקורב של~$ℕ$ (המספרים הטבעיים),~$ℤ$ (השלמים),~$ℚ$(
הרציונליים),~$ℝ$ (הממשיים), ו-$ℂ$ (המרוכבים).

\פנה|טבלה:השוואה| משווה בין ביטויי~\E|S| ובין החוג~$ℤ$, חוג המספרים הטבעיים,
מבחינת התמיכה בהם בשפות תכנות שונות.

\begin{table}[!htbp]
  \rowcolors{2}{blue!10}{white}
  \begin{tabularx}\textwidth{r>{\small}X>{\small}X}
    \toprule
    \bf                                      &
    \bf \normalsize חוג המספרים השלמים~$ℤ$              &
    \bf \normalsize~$𝓢$, קבוצת ביטויי~\E|S|⏎
    \midrule
    פעולות יסודיות                           &
    כפל, חיבור, ומציאת ההופכי החיבורי        &
    $p$,~$p₁^{-1}$, ו-$p₂^{-1}$. ⏎
    תמיכה בשפות תכנות                        &
    פסקל,~\CPL, \Java, \E|Common Lisp|, וכו' &
    כל הניבים של ליספ ⏎
    טיב התמיכה                               &
    בדרך כלל קירוב סופי באמצעות~$ℤ_{2ⁿ}$, עבור \[
      n=3,4,5,6.
\] תמיכה מלאה בשפות
    תכנות פחות נפוצות, כגון \textsc{Mathematica} ו-\textsc{Newspeak}.
                                             &
    תמיכה מלאה במיני-ליספ כמו גם הניבים של ליספ ⏎
    סימון פעולות היסוד                       &
    בפסקל,~\CPL, \Java, ועוד:~\cc{+}, \texttt{*} (אופרטורים
    בינאריים), ו-\texttt{-}, אופרטור אונארי. &
    \E|cons|, \E|car|, \E|cdr| ⏎
    תמיכה בפעולות נוספות                     &
    \textbf{בפסקל:} אופרטורים בינאריים "\texttt{*}", "\texttt{/}" \texttt{*},
    \kk{mod}, \kk{div} ואופרטורים אונאריים \texttt{-}, \kk{succ}, ו-\kk{pred},
    ומספר פונקציות מוגדרות מראש.\hfill\newline
    \textbf{בשפות כגון~\CPL, ו-\E|\textsc{Go}|:} אופרטורים רבים
    נוספים, עליהן נוספת ספרייה עשירה         &
    \textbf{במיני-ליספ:} קומץ פונקציות מוגדרות מראש. \hfill\newline
    \textbf{בניבים אחרים של ליספ:} פונקציות אלמנטריות נוספות כגון, כגון
    \E|caar|, \E|cdar|, \E|cadr|, פונקציות מוגדרות מראש נוספות, וספרייה עשירה
    יותר ⏎
    פעולות השוואה
                                             &
    $n₁<n₂$,~$n₁=n₂$, וצירופים שלהם          &
    בדיקה אם שני אטומים הם זהים ⏎
    \bottomrule
  \end{tabularx}
  \כיתוב|תמיכת שפות תכנות בקבוצת ביטויי ה-S לעומת תמיכתן בחוג~$ℤ$|
  \תגית|טבלה:השוואה|
\end{table}

§§§ סימון פעולות היסוד באמצעות פונקציות של ליספ

ההיבט הראשון של שלוש פעולות היסוד על ביטויי~\E|S| הוא הייצוג שלהן כפונקציות
מתמטיות מעל~$𝓢$, ממש כשם שהפעולות האריתמטיות הן פונקציות מתמטיות המוגדרות על
החוג~$ℤ$. ההיבט השני של פעולות היסוד הוא ה\ע|סימון| של פעולות היסוד בשפת ליספ,
מה שקוראים גם הציון שלהן על ידי אטומים, ממש כשם שסימן הפלוס (\T|"+"|) משמש לציון
של פעולת החיבור בשפות תכנות אחרות.

המשמעות של שלושת האטומים \T|cons|, \T|car|, ו-\T|cdr| מוגדרת מראש במיני-ליספ
לציון פעולות אלו:

\begin{enumerate}
  ✦ האטום \T|cons| מציין את הפונקציה~$p$, שהיא פונקציה אטומית המקבלת שני
  ביטויי~\E|S|, ומחזירה ביטוי~\E|S| שהוא זוג בו האיבר הראשון הוא הארגומנט
  הראשון לפונקציה, ואילו האיבר השני של הזוג הוא הארגומנט השני לפונקציה.

  ✦ האטום \T|car| מציין את הפונקציה~$p₁^{-1}$, שממומשת כפונקציה אטומית של
  ליספ. פונקציה זו מקבלת כארגומנט ביטוי~\E|S| אחד. אם ביטוי זה הוא ביטוי מורכב,
  כלומר ביטוי שהוא \E|dotted-pair| הפונקציה מחזירה ביטוי~\E|S| שהוא האיבר
  הראשון בזוג ממנו הארגומנט בנוי.

  לעומת זאת, אם הארגומנט לפונקציה הוא אטום, הפונקציה נכשלת. כשלון זה דומה
  לכשלון של ניסיון לחלוקה באפס. ממש כשם שלא כל הפעולות האריתמטיות מוגדרות על כל
  המספרים, לא כל הפונקציות המבניות מוגדרות על כל ביטויי~ה-\E|S|.

  ✦ באופן דומה, האטום \T|cdr| מציין את הפונקציה~$p₂^{-1}$ שממומשת גם היא
  כפונקציה אטומית. פונקציה זו המקבלת כארגומנט ביטוי~\E|S| אחד, ואם ביטוי זה
  הוא ביטוי מורכב, הפונקציה מחזירה ביטוי~\E|S| שהוא האיבר השני בזוג ממנו
  הארגומנט בנוי. ממש כמו car הפונקציה cdr נכשלת אם היא מופעלת על אטום בודד
  או על הרשימה הריקה.
\end{enumerate}

נדגיש שוב את ההבדל בין שלושת האטומים ובין שלושת הפעולות אותן הם מציינים, ונזכיר
שהבדל זה אינו יחודי לשפת מיני-ליספ. בשפת~\CPL, יש הבדל בין הסימן \T|+| ובין
הקירוב לפעולת החיבור, אותו מציין סימן זה. בהמשך, נשתמש לא מעט באבחנה בין שם
הפונקציה, ובין גוף הפונקציה שאותו מציין השם.

§§§ פעולות היסוד על ביטויי~\E|S| כפעולות על רשימות
ההיבט השלישי של פעולות היסוד על ביטוי~\E|S| נוגע להסתכלות על ביטויי~\E|S|
כרשימות.

\begin{enumerate}
  ✦ אם נעביר לפונקציה \T|cons| ערך כלשהו~$x$, כלומר ביטוי~\E|S| שיכול להיות
  אטומי או מורכב,~$x$ וביטוי~\E|S| אחר שהוא רשימה~$ℓ$, אזי הפונקציה תוסיף את~$x$
  בתחילת הרשימה~$ℓ$: הפעלת cons על האטום a ועל הרשימה \E|((b~c)~d)| תחזיר את
  הרשימה \E|(a~(b~c)~d)|.

  ✦ אם הארגומנט לפונקציה \T|car| הוא רשימה~$ℓ$ שאיננה ריקה, אז הפונקציה מחזירה
  את הפריט הראשון ברשימה: הפעלת car על הרשימה בת שני איברים \E|((b~c)~d)| תחזיר
  את הפריט הראשון ברשימה זו, \E|(b c)|, שהוא בעצמו רשימה בת שני איברים.

  כאמור, אם הארגומנט ל-\E|car| הוא אטום, ואפילו יהא זה האטום \lisp{nil},
  כלומר הרשימה הריקה, הפונקציה נכשלת.

  ✦ באופן דומה, אם הארגומנט לפונקציה \T|cdr| הוא רשימה~$ℓ$ שאיננה ריקה, אז
  הפונקציה מחזירה את הרשימה המתקבלת מ-$ℓ$ אחרי שהסרנו ממנה את הפריט הראשון שבה:
  הפעלת cdr על הרשימה \E|((b~c)~d)| תחזיר \E|(d)|, רשימה בת איבר אחד. הפונקציה
  \E|cdr| גם היא נכשלת אם היא מופעלת על רשימה ריקה, כלומר על האטום \E|nil|, או על
  כל אטום אחר.
\end{enumerate}

תכנות ברשימות אופייני לשפות רבות אחרות מלבד ליספ. לפונקציה הדומה ל-cons קוראים
לעיתים \E|prepend|. כיוון ששלושת הפעולות הללו משתמשות ברשימה כמחסנית,
 אנו מוצאים מקרים בהם קוראים ל-\E|cons| גם בשם \E|push|. באופן
דומה, לפונקציה car קוראים בשפות תכנות אחרות בשמות כגון \E|first|, \E|head| (או
בקיצור \E|hd| ו-\E|top|), ולפונקציה cdr קוראים בשפות תכנות אחרות בשמות כגון
\E|rest|, \E|body| ו-\E|pop|.

§§ בדיקת התוכן של ביטויי~\E|S|

הקבוצה~$ℤ$ היא קבוצה סדורה היטב. ניתן לבדוק לגבי כל שני מספרים שלמים~$n₁,n₂∈ℤ$
מתקיים אחד מבין ההיגדים הבאים:~$n₁=n₂$,~$n₁<n₂$ או~$n₁>n₂$. אין יחס סדר דומה
לקבוצת ביטויי ה-S, אבל ניתן לבדוק את תוכנם באמצעות פונקציות מתאימות בליספ.

ליספ בדרך כלל תומכת בפונקציות לוגיות רבות המאפשרות לבדוק את תוכנו של
ביטויי~\E|S|. שפת מיני-ליספ מסתפקת בשלוש פונקציות כאלו בלבד: המשמעות של שלושת
האטומים \T|atom|, \T|null|, ו-\T|eq| מוגדרת מראש במיני-ליספ לציין פונקציות
המאפשרות בדיקות כאלו:

\begin{enumerate}
  ✦ האטום \T|atom| מציין פונקציה אטומית המקבלת ביטוי~\E|S| אחד ומחזירה את
  האטום \T|t| אם ארגומנט זה הוא אטום, ואחרת את האטום \T|nil|.
  בכתיב מתמטי,
  \begin{equation}
    \text{atom}:𝓢→\mathcal{B}
  \end{equation}
  כאשר
  \ציינן
  ✦~$𝓢$ מציינת את~$\SX(Σ_{\text{Mini-Lisp}^*})$, קבוצת ביטויי ה-S של מיני-ליספ,
  וכאשר,
  ✦~$\mathcal{B}=❴\text t, \text{nil}❵$ מציינת את
  הקבוצה המכילה את שני האטומים~t ו-\E|nil|.
===

  ✦ האטום \T|null| מציין את הפונקציה המוגדרת מראש המקבלת ביטויי~\E|S| אחד
  ומחזירה את האטום \T|t|. אם ארגומנט זה הוא האטום \T|nil| ואחרת את האטום \T|t|.
  בניסוח אחר, אם הארגומנט לפונקציה הוא רשימה, אז הפונקציה מחזירה \T|t| אם ורק
  אם הרשימה ריקה, ו-\T|nil| בכל מקרה אחר. נוכל לרשום על כן, מתמטית ניתן לכתוב
  פונקציה זו כך
  \begin{equation}
    \text{null}:𝓢→\mathcal{B}
  \end{equation}

  ✦ האטום \T|eq| מציין את הפונקציה האטומית המקבלת שני ביטויי~\E|S| ומחזירה את
  האטום \T|t| אם שני הארגומנטים לפונקציה הם אטומים, ושני האטומים הללו שווים.
  בכל מקרה אחר, הפונקציה מחזירה את האטום \T|nil|.
  מתמטית ניתן לכתוב פונקציה זו כך
  \begin{equation}
    \text{eq}:𝓢⨉𝓢→\mathcal{B}
  \end{equation}
  כלומר, eq היא פונקציה מזוגות של ביטויי~\E|S| שבשפת מיני-ליספ
  לקבוצה~$\mathcal{B}$.
\end{enumerate}

במובן מסויים, הקבוצה~$\mathcal{B}$, היא קבוצת הערכים הבוליאניים של ליספ. ניתן
לחשוב על האטום \T|nil| כמסמן את הערך של שלילה לוגית, כלומר \E|false|, ועל
האטום \T|t| כמסמן את ערך האמת, \E|true|. אולם, בליספ כל ביטוי~\E|S| שאיננו האטום
\T|nil| נחשב כערך אמת כלומר כ-\E|true|. כלומר האטום \T|t| אינו הביטוי היחיד
המייצג את \E|true|. במקרה של הצלחה, כל שלושת פונקציות ההשוואה יכולות היו לכן
להחזיר כל ערך אחר שאיננו \T|nil|. בכל זאת, ובכדי לשמור על עיקביות, פונקציות אלו
\ע|מחזירות| את האטום \T|t| כל אימת שההשוואה מצליחה.

פונקציות אחרות של ליספ ה\ע|מקבלות| ביטויי~\E|S| כפרמטר, וצריכות להתייחס אליו
כערך בוליאני של אמת או שקר, יתייחסו לכל ערך שאינו \T|nil| כ-\E|true|. פונקציה
חשובה כזו, והנמצאת גם בניב המינימלי של מיני-ליספ היא הפונקציה cond המדמה את
פקודת התנאי (\E|if command|).

§§ ביטויי~\E|S| כעצים מעל אלפאבית

ראינו שביטויי~\E|S| ניתנים להצגה כעץ בינארי שבו העלים, והעלים בלבד, מכילים
סמלים. אם ביטוי~\E|S| ניתן להכתב כרשימה של רשימות, אז ניתן להציג את הביטוי
הזה כ\ע|עץ|, שבו קיים סמל בכל צומת פנימי ובכל עלה. ומספר הבנים של כל צומת
פנימי יכול להיות כלשהו.

בהנתן אלפאבית~$Γ$, נגדיר את~$T(Γ)$, קבוצת העצים שבכל צומת שלהם ישנו איבר
של~$Γ$. עץ כזה יכול להיות עץ אטומי, ואז העץ מצטמצם לכדי עלה בודד, שחייב להכיל
איבר של~$Γ$. עץ ב-$T(Γ)$ יכול להיות גם עץ מורכב, ובמקרה זה הוא מורכב מצומת
פנימי שבו יש סימן מתוך~$Γ$ ומספר כלשהו של בנים שכולם עצים ב-\E|$T(S)$|. בניסוח
זה, אנו יכולים לחשוב על עץ אטומי כעץ נולארי, כלומר עץ "ערירי", אשר אין לו בנים.

נוכל להגדיר את הקבוצה~$T(Γ)$ לא כקבוצה של עצים, אלא כשפה פורמלית מעל אלפאבית
מורחב~$Γ'$, הכולל גם את זוג סימני הסוגריים ואת סימן הפסיק.
\begin{equation}
  Γ'=Γ∪❴⌘{(},⌘{.},⌘{)}❵.
\end{equation}

\begin{definition}[עצים מעל אלפאבית]
  בהנתן אלפאבית~$Γ$ אזי,~\E|$T(Σ)$|, קבוצת העצים מעל~$Σ$ מוגדרת באמצעות הבנאי
  ה-$n$ מקומי
  \begin{equation*}
    \infer{w(t₁,…,tₙ)∈T(Σ)}{w∈Σ^* & n≥0 & t₁∈T(Σ) & t₂∈T(Σ)&⋯& tₙ∈T(Σ)}
  \end{equation*}
\end{definition}

כמה איברים פשוטים של קבוצת העצים מעל האלפאבית בן שלוש האותיות~$Σ=❴a,b,c❵$ הם \[
  a,b(c),a(b,c), a(b(a)), a(a,b,c)∈T(❴a,b,c❵)
\] \cref{figure:tree} מדגים את הטופולוגיה של העץ המורכב יותר \[
  a(a(a,ab,abc),b(b,ab(c)),c(c(a(ab)))∈T(❴a,b,c❵)).
\] \begin{figure}[!htbp]
  \centering
  \forestset{%
    x tree/.style={%
        for tree={%
            math content,
            s sep'+=-3pt,
            fit=band,
          },
      },
  }
  \begin{forest}
    s tree [a
          [a,[a][ab][abc]]
          [b,[b][ab[c]]]
          [c,[c[a[ab]]]]
      ]
  \end{forest}
  \כיתוב|העץ~$a(a(a,ab,abc),b(b,ab(c)),c(c(a(ab))))$.|
  \label{figure:tree}
\end{figure}

אם ביטוי~\E|S| הוא רשימה של רשימות, אזי ניתן להציגו כעץ. התרגום מתבצע על ידי הפיכת
כל רשימה בביטוי לתת-עץ: האיבר הראשון ברשימה הוא שורש תת-העץ. שאר האיברים
ברשימה, הם הבנים של תת העץ. לדוגמה, בעבור הרשימה
\begin{LISP}
  (a b (car x) (+¢¢ b x))
\end{LISP}
יבנה כך העץ
\begin{LTR}
  \scriptsize
  \forestset{%
    x tree/.style={%
        font=\ttfamily,
        for tree={%
            s sep'+=-3pt,
            circle,
            fit=band,
          },
      },
  }
  \begin{forest}
    x tree [a,
        [a,[car,[x]]]
          [+, [b] [x]]
      ]
  \end{forest}
\end{LTR}

נשים לב לכך שהפונקציה \E|car|, כשהיא מופעלת על עץ מעל אלפאבית, היא מחזירה את
תוכנו של הצומת שבשורש העץ, ואילו הפונקציה \E|cdr| מחזירה את רשימת הבנים של
הצומת, שכל אחד מהם הוא עץ כזה בעצמו.

§§ עוד על עצי שיערוך

ניתן להסתכל על עץ מעל אלפאבית, שבו יש סמל בכל צומת, ושבו הדרגה אינה מוגבלת,
כעץ "שיערוך". נסתכל למשל על הביטוי הבא בשפת~\CPL
\begin{CPP}
  f(2) ? g(++a,--b,-sin(c)) : 10+h()
\end{CPP}
חישוב ביטוי כגון זה, דורש ראשית הבנה של המונחים שבו. לחישוב הכולל הבנת מונחים
ופיענוח משמעות הסמלים אנו קוראים שיערוך \E|(evaluation)|.
נתאר את השיערוך של הביטוי הזה על ידי הצגתו כעץ השיערוך המתואר ב\פנה|איור:עץ|.

\begin{figure}[!htbp]
  \caption{%
    עץ החישוב של הביטוי \protect\T|f(2) ? g(++a,--b,-sin(c)) : 10+h()|.
  }
  \תגית|איור:עץ|
  \centering
  \begin{LTR}
    \scriptsize
    \forestset{%
      x tree/.style={%
          for tree={%
              font=\ttfamily\scriptsize,
              s sep'+=3pt,
              l sep'+=3pt,
              fit=band,
            },
        },
    }
    \begin{forest}
      x tree [\E|:?|,
      [$f$ [$2$]]
      [$g$
      [\T|++|[$a$]]
      [\T|--|[$b$]]
      [$-$ [$\sin$ [$c$]]]
      ]
      [$+$[$10$][$h$]]
      ]
    \end{forest}
  \end{LTR}
\end{figure}

שיערוך ביטוי ניתן לתיאור רקורסיבי באמצעות העץ שמתאר אותו. בסיס הרקורסיה הוא
שיערוך של עלה, הנעשה באמצעות פיענוח משמעותו. בעץ שבאיור יש שישה עלים:
\אבגד
✦ משמעות העלים המסומנים ב-$2$ וב-$10$ היא המספרים~$2$ ו-$10$, שכן בשפת~\CPL
סדרות התווים \T|2| ו-\T|10| הן \ע|מילולונים|, כלומר סדרות אלו מציינות ערך הנקבע
באופן יחיד על ידי תוכן הסדרה, ללא תלות בטבלת סמלים כלשהי.
✦ משמעות העלים המסומנים ב-$a$, \E|$b$| וב-$c$ נעשית על ידי חיפוש השמות הללו
בטבלת הסמלים מתאימה, שכן בשפת~\CPL סדרות התווים \T|a|, \T|b| ו-\T|c| הם מזהים.
מזהים אלו מתייחסים, ככל הנראה, למשתנים אשר הוגדרו קודם.
✦ משמעות העלה המסומן ב-$h$ גם היא נעשית באמצעות חיפוש השם~$h$
בטבלת הסמלים, אלא שנדרש שהחיפוש בטבלה יקשור את השם לפונקציה.
אם על פי החיפוש, משמעות השם~$h$ היא משתנה, אזי השיערוך של הביטוי יכשל.
===

שיערוך של צומת פנימי מתחיל באופן דומה. ראשית יש לברר את משמעות הסמל אשר
נמצא בתוך הצומת. בדוגמה שלנו, ששת הסמלים \T|+|, \T|-|, \T|++|, \T|--|
ו-\T|?:| הם אופרטורים של שפת~\E|\CPL|. כלומר, המשמעות שלהם קבועה מראש בשפה,
ולכן אין לחפש את משמעותם בטבלת שמות. לעומת זאת, \T|f|, \T|g| ו-\T|sin| הם שמות
שהוגדרו על ידי מתכנת†{נזכר שהפונקציה~$\sin$
אינה בנויה בשפת~\E|\CPL|, אלא היא מוגדרת באחת מהסיפריות.} לאחר פענוח משמעותו של
צומת פנימי, יש לחשב את ערכי תתי העצים של הצומת, ולהעביר את תוצאות החישוב של תתי
העצים כארגומנטים לפונקציה.

ניתן להבין באופן מדוייק יותר את תהליך השיערוך באמצעות בדיקה של האופן שבו הוא
מתבצע בשפת ליספ.

§ שיערוך של ביטויי~\E|S|

§§ שיערוך של אטומים, הסביבה וה-\E|association list|.

השיערוך של ביטויי-\E|S| מוגדר רקורסיבית. בסיס הרקורסיה הוא בשיערוך של אטומים.
שיערוך של אטום אינו נדרש לבצע חישוב, אלא רק למצוא את משמעותו של האטום הנתון.
אטום בליספ הוא מזהה \E|(identifiier)|, וכמו בשפות תכנות אחרות, משמעותו של המזהה
נמצאת בטבלת סמלים \E|(symbol table)|. טבלת הסמלים היא מבנה נתונים הקושר בין
שמות ובין משמעותם, ומציאת המשמעות נעשית על ידי חיפוש בטבלה זו.

טבלת הסמלים בליספ בנוייה במבנה נתונים הידוע בליספ בשם \E|association list|
או בקיצור \E|a-list|. ה-\E|a-list| היא רשימה של פריטים אשר כל אחד מהם מבטא
קישור של שם לערך. כל פריט הוא dotted-pair אשר ה-car שלו הוא שם (המיוצג כאטום)
ואילו ה-cdr של הפריט הוא משמעותו של השם (שהיא~\E|S-Expression|, אטומי או מורכב.
אם משמעותו של האטום foo היא העץ~$ϕ$ אז הפריט הנשמר ב-\E|a-list|
הוא~\E|$(\text{foo}.ϕ)$|. הקישור בין השם והביטוי מתבטא בעובדה ששני אלו מחוברים
ברשומת \E|cons|.

ה-\E|a-list| ידועה גם בשם \ע|סביבה| (\E|\emph{environment}|). מושג הסביבה אינו
יחודי למיני-ליספ או לליספ. בכל שפת תכנות, הסביבה או ה-\E|environment| היא זו
אשר נותנת משמעות לשמות. כלומר, לדוגמה, נעיין בתוכנית הזו בשפת~\CPL:
\begin{CPP}
int main(int argc, char *[] argv, char **envp) {
  /* ¢$ℓ₀$¢ */ int i;
  for (i=1; i<argc;++i) {
    do {
      int e;
      /* ¢$ℓ₁$¢ */ if ((e=main(argc, argv,++envp)) !=0)
        return e;
    } while (*envp);
  }
  return i;
}
\end{CPP}
אף מבלי להכיר את שפת~\CPL על בוריה, ומבלי צורך לדעת את משמעותם של הפרמטרים
\E|argc|, \E|argv|, ו-\E|envp| לפונקציה \E|main|, ובוודאי מבלי צורך לפענח מה
עושה תכנית מוזרה זו (שאינה כוללת שימוש בספרייה הסטנדרטית של שפת~\CPL), ניתן
לזהות את הסביבה: בנקודה המסומנת~$ℓ₁$ בתכנית המתכנת יכול להשתמש בשישה שמות:
\lisp{main}, \lisp{argc}, \lisp{argv}, \lisp{envp}, \lisp{i}, ו-\lisp{e}, ולכל
אחד מאלו, ישנה משמעות משלו. הסביבה היא זו אשר מעניקה משמעות לשמות.
נשים לב לכך שהסביבה אינה זהה בכל התכנית. במיקום המסומן~$ℓ₀$ אין משמעות
לשמות \lisp{i} ו-\lisp{e}.

המונחים סביבה וטבלת סמלים הם מונחים דומים המציינים דברים דומים. המונח סביבה
משמש מעט יותר בכדי להסביר ולהבין את שפת התכנות. המונח טבלת סמלים משמש
כדי לתאר את המימוש של שפת התכנות.

בשפת מיני-ליספ, הסביבה המיוצגת ברשימה הקרוייה-\E|a-list| מנוהלת כמחסנית, כלומר
פריטים מתווספים ומוסרים ממנה רק בתחילתה. החיפוש אחר משמעות ברשימה נעשה סדרתית,
בסריקה המתחילה בתחילת הרשימה. חיפוש המשמעות של האטום foo נעצר בזוג הראשון שבו
מרכיב השם הוא \E|foo|. החיפוש נכשל אם לא נמצא זוג כזה. כל הפעולות על הסביבה
נעשות בזמן שיערוך התכנית.

מעיון חוזר בדוגמה למעלה ניתן להסיק כי הסביבה בנוייה גם בשפת~\CPL כמעין מחסנית
של מילונים. פתיחה של בלוק בסימן \texttt{❴} מוסיפה מילון חדש בראש מחסנית.  הגדרה
של שם מתווספת לסביבה, באמצעות עדכון המילון שבראש המחסנית.  הגדרות אלו מוסרות
מהסביבה עם סיום הבלוק, כלומר בסימן \texttt{❵}, אשר גורם לסילוק המילון שבראש המחסנית.

ישנם הבדלים טכניים לא מעטים בין הסביבה של~\CPL וזו של מיני-ליספ, אולם ישנו הבדל
מהותי אחד בין השתיים: פעולות על הסביבה במיני-ליספ נעשות בזמן שיערוך, הכולל בתוכו
מרכיב של חישוב. בניגוד לכך, הפעולות על הסביבה ב-\CPL, ובכלל הוספת שמות לסביבה,
וחיפוש שמות בתוכה, נעשים בזמן הידור, ועוד טרם להרצה.

כאשר האינטרפרטר של ליספ מתחיל את פעולתו, ה-\E|a-list| מכילה קישורים בעבור כמה
אטומים המוגדרים מראש בשפה. במיני-ליספ מספיק להניח כי ה-\E|a-list| מכילה קישורים
בעבור שני האטומים שבקבוצה~$𝔹$ בלבד, כלומר הרשימה תיתן משמעות ל-\E|t| ו-\E|nil|.
התוכן המינימלי של ה-\E|a-list| הוא
\begin{LISP}
(
  (t.t)
  (nil.nil)
)
\end{LISP}

הפריט הראשון ברשימה הוא dotted-pair הקובע שהמשמעות של האטום \T|t| היא אטום זה
עצמו. הפריט השני ברשימה קובע שמשמעותו של האטום \T|nil| היא אטום זה עצמו גם כן.

לפיכך, כאשר האינטרפרטר של ליספ יקרא את האטום \T|t| הוא ידפיס בתגובה \T|T|,
וכאשר הוא יקרא את האטום \T|nil| הוא ידפיס בתגובה \T|NIL|.
\begin{LISP}
> t
T
> nil
NIL
\end{LISP}

§§ הפונקציה set
הפונקציה \T|set| שבליספ היא פונקציה אטומית המקבלת שני פרמטרים, שהראשון בהם
הוא אטום, והשני הוא ביטוי~\E|S| כלשהו. הפונקציה מוסיפה ל-\E|a-list| פריט שהוא
dotted-pair המבטא קישור בין האטום לביטוי. במרבית הניבים של ליספ יש ואריאנטים
רבים של \T|set| אולם במיני-ליספ נסתפק ב-\T|set| בלבד.

אם נזין לאינטרפרטר של ליספ את הביטוי
\begin{LISP}
> (set foo (bar baz))
\end{LISP}
כבקשה לקשור את האטום \T|foo| לרשימה \T|(bar baz)|, הרי ניתקל בשגיאות, שכן
האינטרפרטר ינסה לשערך את שני הארגומנטים של הפונקציה \E|set| טרם שהוא יפעיל
פונקציה זו. אלא ששיערוך זה יכשל, שכן לאף אחד משלושת האטומים המופיעים בקריאה
ל-set אין משמעות מלכתחילה.

בכדי להתגבר על מכשלה זו יש להשתמש בפונקציה \T|quote|, פונקציה של ארגומנט אחד
אשר \ע|אינה| משערכת את הארגומנט שלה, אלא מחזירה אותו כמו שהוא.
כך למשל
\begin{LISP}
> (quote foo)
FOO
> (quote (bar baz))
(BAR BAZ)
\end{LISP}
קשירת האטום \T|foo| לרשימה \T|(bar baz)| יכול להעשות באמצעות
\begin{LISP}
> (set (quote foo) (quote (bar baz)))
(BAR BAZ)
> foo
(BAR BAZ)
\end{LISP}
נשים לב לכך שהשיערוך של הביטוי
\begin{LISP}
(set (quote foo) (quote (bar baz)))
\end{LISP}
אינו במטרה לחשב ערך כלשהו, אלא במטרה לקשור שם לערך. נזכר שבאופן כללי פעולת
השיערוך בליספ כוללת שלושה סוגים של פעולות: מציאת משמעות של שם, הגדרת משמעות של
שם, וחישוב שהוא הפעלת פונקציה על ביטוי~\E|S| או ביטויי~\E|S|. בכל זאת, כל
פונקציה בליספ חייבת להחזיר ערך.

במקרה של הפונקציה set הערך המוחזר הוא ערכו של הפרמטר השני שלה. בתגובה לקלט
\begin{quote}
\T|(set (quote foo) (quote (bar baz)))|
\end{quote}
האינטרפרטר ידפיס את הערך \T|(BAR BAZ)|:
\begin{LISP}
> (set (quote foo) (quote (bar baz)))
(BAR BAZ)
\end{LISP}

הפונקציה set מוסיפה זוג לתחילת ה-\E|a-list|: לכן, אם התוכן של רשימת
ה-\E|a-list| לפני הקריאה ל-set היה
\begin{LISP}
(
  (t.t)
  (nil.nil)
)
\end{LISP}

\pagebreak[3]
אזי אחרי הקריאה תוכן רשימה זו יהיה
\begin{LISP}
(
  (foo.(bar baz))
  (t.t)
  (nil.nil)
)
\end{LISP}

ויזואלית, נשרטט את רשימת ה-\E|a-list| לפני הקריאה כך,
\begin{LTR}
  \begin{tikzpicture}[list/.style={rectangle split, rectangle split parts=2,
          draw,minimum height=3ex, fill=blue!20,rectangle split horizontal}, >=stealth, start chain, node distance=3ex]
    \foreach \x/\y/\z in {%
        h/t/t,
        i/nil/nil
      } {%
        \node[on chain, list,font=\tt\scriptsize] (\x) {\y};
        \node[below=4 ex of \x.one,anchor=north west,align=left,font=\tt\scriptsize,color=red] (temp) {\z};
        \draw[->,bend left] (\x.one south) .. controls+(270:0.3) and+(120:0.6) .. (temp.north west);
      }

    \node[on chain,font=\tt\scriptsize] (j) {nil};
    \draw[*->] let \p1=(i.two), \p2=(j.center) in (\x1,\y2)--(j);

    \foreach \a/\b in {h/i} {%
        \draw[*->] let \p1=(\a.two), \p2=(\b.center) in (\x1,\y2)--(\b);
      }
    \node[above=of h] (A) {a-list};
    \draw[->] (A.south)--(h);
  \end{tikzpicture}
\end{LTR}
ואחרי השיערוך של הביטוי
{\setLTR
\begin{quote}
  \T|(set (quote foo) (quote (bar baz)))|
\end{quote}
}
תראה ה-\E|a-list| כך,
\begin{LTR}
  \begin{tikzpicture}[list/.style={rectangle split, rectangle split parts=2,
          draw,minimum height=3ex, fill=blue!20,rectangle split horizontal}, >=stealth, start chain, node distance=3ex]
    \foreach \x/\y/\z in {%
        g/foo/(bar baz),
        h/t/t,
        i/nil/nil
      } {%
        \node[on chain, list,font=\tt\scriptsize] (\x) {\y};
        \node[below=4 ex of \x.one,anchor=north west,align=left,font=\tt\scriptsize,color=red] (temp) {\z};
        \draw[->,bend left] (\x.one south) .. controls+(270:0.3) and+(120:0.6) .. (temp.north west);
      }

    \node[on chain,font=\tt\scriptsize] (j) {nil};
    \draw[*->] let \p1=(i.two), \p2=(j.center) in (\x1,\y2)--(j);

    \foreach \a/\b in {g/h,h/i} {%
        \draw[*->] let \p1=(\a.two), \p2=(\b.center) in (\x1,\y2)--(\b);
      }
    \node[above=of g] (A) {a-list};
    \draw[->] (A.south)--(g);
  \end{tikzpicture}
\end{LTR}
שפת מיני-ליספ יכולה לאתחל את ה-a-list באמצעות
\begin{LIBRARY}
(set (quote t) (quote t))
(set (quote nil) (quote nil))
\end{LIBRARY}
לא ניתן \ע|להסיר| פריטים מתוך ה-\E|a-list|. אבל, ניתן \ע|להסתיר| קישור באמצעות
הוספת פריט לרשימה, אשר יסתיר את הקישור הקודם. בפרט, אם נכתוב כעת
\begin{LISP}
> (set (quote foo) (quote ((baz))))
(BAZ)
\end{LISP}
הרי התוכן של ה-a-list יהיה
\begin{LISP}
(
  (foo.((baz)))
  (foo.(bar baz))
  (t.t)
  (nil.nil)
)
\end{LISP}
וויזואלית כך,
\begin{LTR}
  \begin{tikzpicture}[list/.style={rectangle split, rectangle split parts=2,
          draw,minimum height=3ex, fill=blue!20,rectangle split horizontal}, >=stealth, start chain, node distance=3ex]
    \foreach \x/\y/\z in {%
        f/foo/((baz)),
        g/foo/(bar baz),
        h/t/t,
        i/nil/nil
      } {%
        \node[on chain, list,font=\tt\scriptsize] (\x) {\y};
        \node[below=4 ex of \x.one,anchor=north west,align=left,font=\tt\scriptsize,color=red] (temp) {\z};
        \draw[->,bend left] (\x.one south) .. controls+(270:0.3) and+(120:0.6) .. (temp.north west);
      }

    \node[on chain,font=\tt\scriptsize] (j) {nil};
    \draw[*->] let \p1=(i.two), \p2=(j.center) in (\x1,\y2)--(j);

    \foreach \a/\b in {f/g, g/h,h/i} {%
        \draw[*->] let \p1=(\a.two), \p2=(\b.center) in (\x1,\y2)--(\b);
      }
    \node[above=of f] (A) {a-list};
    \draw[->] (A.south)--(f);
  \end{tikzpicture}
\end{LTR}
ואם ננסה לשערך את \T|foo| לאחר ההסתרה הזו, נקבל את הערך הנוצר מהקישור החדש
\begin{LISP}
> foo
((BAZ))
\end{LISP}
וזאת משום שהחיפוש ב-a-list מתחיל מתחילתה, ועוצר במקום הראשון שבו הוא מצליח.

§§ כתיב מקוצר לפונקציה quote
כתיב מקוצר לביטוי
\begin{quote}
  \setLTR \[
    \text{(quote~$s$)}
\] \end{quote}
המפעיל את הפונקציה quote על הביטוי~$s$, הוא \[
  's
\] כלומר סימן מרכאה בודד
לפני הביטוי~$s$. הטיפול בכתיב זה נעשה על ידי
ה-Reader וה-Printer של ליספ, והם אינם חלק מהמשערך של השפה.
הכתיב המקוצר פועל כאשר הביטוי~$s$ הוא אטום בודד, למשל,
\begin{LISP}
> 'foo
FOO
> (quote 'foo)
'FOO
\end{LISP}
וגם כאשר~$s$ הוא ביטוי המכיל בתוכו הפעלה של הפונקציה \E|quote|,
\begin{LISP}
> '(quote foo)
'FOO
\end{LISP}
וגם כאשר~$s$ הוא רשימה,
\begin{LISP}
> '(bar baz)
(BAR BAZ)
\end{LISP}
הקריאה ל-\E|set| בשימוש בכתיב המקוצר היא אכן קצרה יותר,
\begin{LISP}
> (set 'foo '(bar baz))
(BAR BAZ)
> foo
(BAR BAZ)
\end{LISP}
הכתיב המקוצר של הפונקציה quote יכול להיות מקונן. את הביטוי הזה
\begin{LISP}
(quote ((quote a) (quote b)))
\end{LISP}
ניתן לכתוב בקיצור כך
\begin{LISP}
'('a 'b)
\end{LISP}
המקרים שבהם יש צורך להשתמש בפונקציה quote מקוננת אינם נפוצים, אולם הם מופיעים
מדי פעם.

§§ מילולונים
ניזכר שבביטויי~\E|S| כל אטום הוא סדרת סימנים נטולת משמעות. בעבור מיני-ליספ,
מספיק להניח כי גם האטום 42 הוא סדרה (בת שתי אותיות) שאין לה משמעות משלה.
במיני-ליספ נוכל לתת לאטום זה משמעות, למשל על ידי
\begin{LISP}
(set '42 'answer)
\end{LISP}
במרבית המימושים של ליספ האטום 42 מייצג ערך שהוא מספר
שלם. אנו אומרים שהאטום 42 הוא \ע|מילולון| \E|(literal)|. מילולון הוא
אטום שמשמעותו קבועה והוא אינו מציין ביטוי~\E|S| אחר. משמעותו של מילולון
נקבעת על ידי פירוש "מילולי" של סדרת התווים שבו, בניגוד למשמעותו של סמל
הנקבעת באמצעות טבלת סמלים.

כאמור, אין מילולונים במיני-ליספ. ב-\E|Common Lisp| יש כמה סוגים של מילולונים,
הכוללים
\begin{enumerate}
  ✦ מספרים שלמים, כגון \T|-12|.
  ✦ מספרים רציונליים, כגון \T|3/7|.
  ✦ מספרים ממשיים כגון \T|3.1415926535897932384d0| ו-\T|6.02E+23|.
  ✦ מספרים מרוכבים כגון \T|#C(5-3)| אשר ערכו הוא~$5-3i$.
  ✦ מחרוזות כגון \T|"Hello, World"|
\end{enumerate}

התמיכה במילולונים נעשית על ידי הרחבת שיטת השיערוך של אטומים: אם האטום אותו יש
לשערך הוא סדרת תווים הנחזית להיות מספר או מילולון אחר, אז השיערוך של האטום נעשה
ללא היוועצות בטבלת הסמלים. תמיכה בפעולות אריתמטיות נעשית באמצעות הגדרה מראש של
קישור בין האטום~\T|+| לפעולת החיבור של מספרים, בין~\T|*| ובין פעולת הכפל,
וכו'.

בכמעט כל הניבים של ליספ, ניתן לכתוב ביטויים אריתמטיים בכתיב הרשימות, והשיערוך
שלהם יביא לתוצאה הצפוייה,

\begin{LISP}
> (+¢ ¢2 (* 3 5))
17
\end{LISP}

הרחבות אלו אינן נחוצות לשם הבנה של ליספ, שכן ניתן לקודד בתוך ביטויי~\E|S| את
מספרים, וניתן באמצעות מיני-ליספ לממש את כל הפונקציות האריתמטיות הפועלות על
מספרים. אבל, מיני-ליספ אינה מסוגלת לבדוק את תוכנה של סדרת הסימנים היוצרת אטום:
הפעולה היחידה המותרת על אטום במיני-ליספ היא השוואתו לאטום אחר: כיוון שכך שפת
מיני-ליספ אינה יכולה להפוך אטום כמו \T|43217| למילולון אשר משמעותו היא המספר
הטבעי \E|$43,217$|, דהיינו, ארבעים ושלושה אלפים מאתיים ושבע עשרה.

§§ ביטויי~$λ$
מתכנת בליספ יכול להוסיף פריטים לטבלת הסמלים באמצעות הפונקציות \E|set|
ו-\E|defun|.

הפונקציה defun מאפשרת להגדיר פונקציות חדשות. נגדיר לדוגמה פונקציה בשם mirror
המקבלת פרמטר x שהוא ביטוי~\E|S| מורכב, ומחזירה ביטוי מורכב אחר שבו ה-car וה-cdr
שב-x הוחלפו.
\begin{LISP}
(defun ; define a new function
  mirror; named mirror
  (x) ; which expects a single parameter, named x
  (cons (cdr x) (car x)); and whose body is this S-expression
)
\end{LISP}
אנו רואים שהפונקציה defun מקבלת שלושה פרמטרים: שם הפונקציה אותה יש להגדיר,
רשימת הפרמטרים ה\ע|פורמליים| לפונקציה, וגוף הפונקציה. הדוגמה גם מראה שהערות
בליספ מתחילות בסימן~";" (נקודה ופסיק) ונמשכות עד סוף השורה.

אחרי הגדרה זו של \E|mirror| באמצעות \E|defun|, נוכל להשתמש בפונקציה החדשה
שהגדרנו כך:
\begin{LISP}
> (mirror '(a.b))
(B.A)
\end{LISP}

מקובל לקרוא ל-\E|a-list| גם סביבה \E|(environment)|. הסביבה היא אוסף של קישורים
בין שמות למשמעותם, והיא זו שנותנת משמעות לשמות. ראינו כבר שהפונקציה \E|set|
מוסיפה קישור לסביבה. גם הפונקציה \E|defun| מוסיפה קישור לסביבה: שיערוך הביטוי
\begin{LISP}
(defun mirror (x) (cons (cdr x) (car x)))
\end{LISP}
מביא לקשירה בין השם mirror ובין מימוש הפונקציה. שיערוך הביטוי \T|(mirror
'(a.b))| משתמש בשם זה, למציאת מימוש הפונקציה.

המימוש של פונקציה כולל את ביטוי ה-S המהווה את גוף הפונקציה. בדוגמה שלנו ביטוי
זה הוא
\begin{LISP}
(cons (cdr x) (car x))
\end{LISP}
אבל, מלבד גוף הפונקציה המימוש כולל גם את רשימת שמות הפרמטרים הפורמליים שלה.
בפונקציה mirror יש פרמטר אחד אשר מיוצג ברשימה \T|(x)|. שמות הפרמטרים הפורמליים
קובעים כיצד יש לחשב את גוף הפונקציה: בכל מקום שבגוף הפונקציה מופיע שם של פרמטר
פורמלי. בזמן הקריאה לפונקציה וחישוב הגוף יש להחליף שם זה בערכו של הפרמטר
האקטואלי, כלומר, הערך המחושב.

הצירוף של גוף הפונקציה ושמות הפרמטרים אליה נקרא ביטוי-$λ$. ביטוי כזה
מייצג פונקציה אנונימית. ביטוי ה-$λ$ בעבור mirror הוא
\begin{LISP}
  (lambda (x) (cons (cdr x) (car x)))
\end{LISP}

באופן כללי ביטוי~$λ$ הוא רשימה בת שלושה פריטים בדיוק:
\ספרר
✦ האטום \T|lambda| המזהה את הרשימה כביטוי~$λ$.
✦ רשימה של אטומים, המייצגת את שמות הפרמטרים הפורמליים לפונקציה.
בביטוי שלמעלה, זו רשימה המכילה איבר אחד בלבד, האטום \E|x|, שהוא שמו של הפרמטר הפורמלי לפונקציה.
✦ גוף הפונקציה, כלומר הביטוי שהשיערוך שלו אחרי הקשירה בין הפרמטרים
הפורמליים לאקטואליים, יתן את ערכה של הפונקציה. בביטוי שלמעלה גוף הפונקציה הוא
\lisp{(cons (cdr x) (car x))}.
===
נשים לב ששם הפונקציה אינו חלק מביטוי ה-$λ$, שכן ביטוי זה מציין פונקציה
\ע|אנונימית|. הפונקציה defun יוצרת ביטוי-$λ$ וקושרת אותו לשם הפונקציה המוגדרת.
מסיבות של יעילות, מרבית המימושים של ליספ מנהלים את הקישורים בין שמות הפונקציות
ובין הגוף שלהן באמצעות מנגנונים יעודיים. במיני-ליספ הקישורים הללו מנוהלים
באמצעות ה-\E|a-list|:

אם תוכן ה-\E|a-list| הוא
\begin{LISP}
(
  (foo.(bar baz))
  (t.t)
  (nil.nil)
)
\end{LISP}
אז לאחר שיערוך הביטוי
\begin{LISP}
(defun
  mirror (x)
  (cons (cdr x) (car x))
)
\end{LISP}
תוכן ה-\E|a-list| יהיה
\begin{LISP}
(
  (mirror.
     (lambda (x)
        (cons (cdr x) (car x))))
  (foo.(bar baz))
  (t.t)
  (nil.nil)
)
\end{LISP}

ניתן במיני-ליספ לייצר ביטוי-$λ$ מבלי להשתמש ב-\E|defun|, בשימוש בפונקציה
\E|quote|,
\begin{LISP}
  '(lambda (x) (cons (cdr x) (car x)))
\end{LISP}
ולכן ניתן להגדיר את הפונקציה mirror תוך שימוש ב-set במקום ב-\E|defun|,
\begin{LISP}
(set mirror
  '(lambda (x)
      (cons (cdr x) (car x)))
)
\end{LISP}
לחילופין, ניתן להשתמש בפונקציה lambda המקבלת שני פרמטרים: הראשון שבהם הוא רשימה
של אטומים, המייצגים את שמות הפרמטרים הפורמליים לפונקציה, והשני הוא גוף הפונקציה
אשר אותו יש לשערך בקריאה לפונקציה, בסביבה הכוללת קישורים בין הפרמטרים הפורמליים
ובין האקטואליים. תוצאת השיערוך של קריאה לפונקציה lambda היא ביטוי~\E|S| שנראה
בדיוק כמו הקריאה לפונקציה.
\begin{LISP}
> (lambda (x) (cons (cdr x) (car x)))
(LAMBDA (X) (CONS (CDR X) (CAR X)))
\end{LISP}
ניתן לחקות את הפעולה של defun באמצעות set ושימוש בביטויי~$λ$.
\begin{LISP}
(set
  mirror
  (lambda (x) (cons (cdr x) (car x)))
)
\end{LISP}
לאחר הקישור בין השם mirror ובין ביטוי ה-$λ$ המתאר את מימוש הפונקציה \E|mirror|,
האטום mirror ישוערך לביטוי~$λ$ זה. בקריאה \T|(mirror '(a.b))| תוחלף המילה
mirror בביטוי זה, אשר מגדיר כיצד לבצע את הקישור בין הפרמטרים האקטואליים
והפורמליים.

ניתן גם לקרוא לפונקציה מבלי לנקוב בשמה, אלא תוך שימוש בביטוי המתאר את המימוש
שלה:
\pagebreak[3]
\begin{LISP}
> (
    '(lambda (x)
      (cons (cdr x) (car x)))
    '(a.b)
)
(B.A)
\end{LISP}
ניתן להשמיט את סימן ה-\E|quote|, ולנצל את העובדה שהאטום \T|lambda| אינו רק
הפתיח של ביטוי~$λ$ אלא גם פונקציה המחזירה ביטוי כזה,
\begin{LISP}
> (
    (lambda (x) (cons (cdr x) (car x)))
    '(a.b)
)
(B.A)
\end{LISP}

§§ שיערוך מותנה ורקורסיה

שיערוך מותנה פירושו שהשיערוך מתבצע כאשר תנאי מסויים מתקיים, והוא אינו מתבצע
כאשר התנאי אינו מתקיים. בליספ שיערוך מותנה מתבצע באמצעות הפונקציה \E|cond|
אשר מכלילה את פקודות ה-if וה-switch שיש בשפת-\E|\CPL|.

נשתמש ב-cond כדי להגדיר פונקציה zcar המחזירה את ה-car של הפרמטר אם
הוא ביטוי מורכב, ואת הפרמטר עצמו אם הוא אטום
\begin{LISP}
(defun zcar(x)
  (cond ((atom x) x) (t (car x)))
)
\end{LISP}
גוף הפונקציה zcar הוא הביטוי
\begin{LISP}
  (cond ((atom x) x) (t (car x)))
\end{LISP}
שהוא קריאה לפונקציה cond עם שני פרמטרים שכל אחד מהם הוא רשימה בת שני איברים:
\begin{LTR}
  \begin{itemize}
    ✦ \lisp{((atom x) x)}
    ✦ \lisp{(t car(x))}
  \end{itemize}
\end{LTR}
cond היא פונקציה רב מקומית המקבלת מספר כלשהו של פרמטרים שכל אחד מהם נקרא
\E|test-form|. כל \E|test-form| הוא רשימה בת שני פריטים שכל אחד מהם הוא ביטוי~\E|S|
אשר עשוי להיות משוערך במהלך פעולתה של \E|cond|. הפריט הראשון ב-\E|test-form|
נקרא test-condition והפריט השני נקרא \E|test-value|.

הפונקציה cond עוברת על ה-test-forms שקיבלה כפרמטרים לפי סדרם, ובכל אחד כזה היא
משערכת את ה-\E|test-condition|.
\אבגד
✦ אם תוצאת השיערוך היא \E|nil|, כלומר ה-test-condition אינו מתקיים, אזי cond
\ע|אינה| משערכת את ה-\E|test-value|, וממשיכה ל-test-form הבא.
✦ אם לעומת זאת תוצאת השיערוך אינה \E|nil|, כלומר ה-test-condition מתקיים, אזי
cond משערכת את ה-\E|test-value|, ומחזירה את תוצאת השיערוך הזו, \ע|מבלי|
להמשיך לשאר ה-\E|test-forms|.
✦ אם cond ממצה את רשימת ה-\E|test-forms| מבלי להיתקל באף \E|test-condition|
שמתקיים, הרי cond מחזירה את הערך \E|nil|.
===
על פי תיאור זה של הפונקציה \E|cond|, הביטוי \begin{LISP}
  (cond ((atom x) x) (t (car x)))
\end{LISP}
שבגוף הפונקציה zcar ישוערך לכן ל-x אם x הוא אטום, ול-\E|car| של \E|x|, אם~x
איננו אטום.

מקובל להשתמש תמיד באטום \E|t| דווקא כדי לציין את התנאי שמתקיים תמיד, כלומר חלק
ה-else של פקודת ה-\E|if|, אבל טכנית ניתן היה להגדיר את zcar תוך שימוש בביטוי
אחר שערכו אינו \E|nil|, למשל
\begin{LISP}
(defun zcar(x)
  (cond ((atom x) x) ('(any S expression) (car x)))
)
\end{LISP}
הנה דוגמה נוספת בה cond מופעלת עם מספר רב יותר של test-forms
\begin{LISP}
(defun is-atomic(name) ; determine whether name denotes a atomic function
  (cond ((eq name 'atom) t)
        ((eq name 'car) t)
        ((eq name 'cdr) t)
        ((eq name 'cond) t)
        ((eq name 'cons) t)
        ((eq name 'eq) t)
        ((eq name 'error) t)
        ((eq name 'eval) t)
        ((eq name 'set) t)
        (t nil)))
\end{LISP}
הפונקציה is-atomic משתמשת בסידרה של test-forms כדי לבדוק אם הפרמטר שלה הוא
אטום המציין פונקציה אטומית. אם תנאי זה מתקיים, הפונקציה מחזירה את האטום
\E|t|. (מסיבות טכניות אנו מניחים כאן שגם \T|eval| היא פונקציה אטומית.)
האטום \E|nil| יוחזר בכל מקרה שהפרמטר אינו אטום, או שהוא אטום שאינו מציין
פונקציה אטומית. לשם כך, משתמשת
is-atomic בפונקציה \T|eq|, שהיא פונקציה אטומית, המשמשת לבדיקה אם הפרמטרים
שלה הם שני אטומים השווים זה לזה.

כדאי לשים לב לכך שה-test-form האחרון שברשימה (כלומר הרשימה \T|(t nil)|) מיותר,
שהרי אם אף אחד מה-test-conditions שב-test-forms שקדמו לו אינו מתקיים, ממילא
cond תחזיר \E|nil|. בכל זאת, מקובל להוסיף את ה-test-form \T|(t nil)| אף אם אינו
נחוץ, כדי להדגיש שערך ברירת המחדל הוא \E|nil|.

דרך קצרה יותר להגדיר את הפונקציה is-atomic היא באמצעות בדיקה אם הפרמטר
שלה מצוי ברשימה המכילה את שמות כל הפונקציות האטומיות
\begin{KERNEL}
(defun is-atomic(name); determine whether name denotes an atomic function
  (exists name '(atom car cdr cond cons eq error eval set)))
\end{KERNEL}
במימוש זה, אטום הוא שמה של פונקציה אטומית אם הוא מצוי בתוך רשימה של האטומים
המציינות פונקציה אטומית. הפונקציה exists עצמה אינה מצוייה במיני-ליספ, אבל
קל לממש אותה ברקורסיה.
\begin{LISP}
(defun exists (x xs) ; determine whether atom x is in list xs
  (cond ; three case to consider
    ((eq xs nil) nil) ; (i) list of xs is exhausted
    ((eq x (car xs)) t) ; (ii) item x is first in xs
    (t (exists x (cdr xs))))) ; (iii) otherwise, search recursively for x on rest of xs
\end{LISP}
הפונקציה \T|exists| מקבלת אטום המסומן ב-\T|x| ורשימה של אטומים המסומנת ב־\T|xs|. 
על פי הגדרה זו, הפונקציה \T|exists| מקבלת אטום המסומן ב-\T|x| ורשימה של אטומים
המסומנת ב-\T|xs|, ומחזירה~t אם האטום x מצוי בין פריטי הרשימה-\T|xs| ו-\T|nil|
אחרת.

רקורסיה דורשת תמיד בדיקת תנאי, הנדרש כדי להבחין בין בסיס הרקורסיה שבו אין צורך
לבצע קריאות רקורסיביות נוספות, ובין המקרה הכללי, שבו יש צורך בהפעלה רקורסיבית.
במקרה של הפונקציה \T|exists| יש ל-\T|cond| שלושה \E|test-forms|:
\ספרר
✦ \T|((eq xs nil) t)| כלומר, אם רשימת ה-\T|xs| ריקה, ברור ש-\T|s| אינו מצוי
בתוכה ואז \T|exists| מחזירה \T|nil|. לעומת זאת, אם תנאי זה אינו מתקיים,
\T|exists| עוברת לבדוק את התנאי הבא:

✦ \T|((eq x (car xs)) t)| כלומר, הרקורסיה בוחנת את הפריט הראשון ברשימה \T|xs|,
על ידי חישוב הביטוי \T|(car xs)| (מובטח שחישוב \T|(car xs)| יצליח שכן
ה-test-form הקודם שבו נבדק אם הרשימה ריקה נכשל) ומשווה אותו ל-\T|x|. אם מתקיים
שיוויון, אזי החיפוש הצליח, ו-\T|exists| מחזירה \T|t| מבלי להידרש ל-test-form
הבא. אחרת, יש בדיקה של ה-test-form הבא:

✦ \T|(t (exists x (car xs)))|, כלומר, במקרה שהרשימה \T|xs| אינה ריקה, ו-\T|x|
אינו שווה לפריט הראשון שב-\T|xs|, הפונקציה \T|exists| קוראת לעצמה רקורסיבית
לחיפוש של \T|x| בשארית הרשימה אשר מתקבלת מחישוב \T|(cdr xs)|.
===
הבדיקה אם הרשימה ריקה נעשית באמצעות חישוב התנאי \T|(eq xs nil)|. השוואה של
אטום ל-nil היא נפוצה בחישובים רקורסיביים. מיני-ליספ מגדירה פונקציה הנקראת null
המיועדת לשם ביצוע השוואה זו:
\begin{LIBRARY}
(defun null(x) (eq x nil))
\end{LIBRARY}

ניזכר שהפונקציה eq היא פונקציה אטומית, במובן זה שלא ניתן לממש אותה באמצעות
פונקציות אחרות, שהרי מלבד פונקציה זו, אין אף פונקציה אטומית שמאפשרת לבדוק את
תוכנו של אטום. לעומת זאת, כיוון שהפונקציה null שבמיני-ליספ ניתנת להגדרה
באמצעות eq היא נחשבת לפונקצית מוגדרת מראש.

מימוש exists באמצעות null יראה כך
\begin{KERNEL}
(defun exists (x xs) ; determine whether atom x is in list xs
  (cond ; Three cases to consider:
    ((null xs) xs) ; (i) list of xs is exhausted
    ((eq x (car xs)) t) ; (ii) item x is first in xs
    (t (exists x (cdr xs))))) ; (iii) otherwise, recurse on rest of xs
\end{KERNEL}

כזכור, ה-a-list היא רשימה של dotted-pairs שכל אחד מהם הוא קישור בין שם לערך.
נגדיר פונקציה רקורסיבית, \E|lookup| אשר מקבלת שני פרמטרים: id שהוא אטום,
ו-a-list, שהיא רשימה במבנה של \E|a-list| כפי שהצגנו אותו, כלומר סדרה של
dotted-pairs אשר ה-car של כל אחד מהם הוא אטום, וה-cdr שלו הוא משמעותו של האטום:
\begin{KERNEL}
(defun lookup (id a-list) ; lookup id in an a-list
  (cond ; Three cases to consider:
    ((null a-list) ; (i) a-list was exhausted.
      (error 'unbound-variable id))
    ((eq id (car (car a-list))) ; (ii) found in first dotted-pair
      (car (cdr (car a-list)))) ; return value part of dotted pair
    (t (lookup id (cdr a-list))))) ; (iii) otherwise, recursive call on remainder of a-list
\end{KERNEL}
גם בפונקציה lookup אנו רואים קריאה רקורסיבית עם cond המתפצל לשלושה מקרים: במקרה
של רשימה ריקה הפונקציה נכשלת, במקרה שראש הרשימה מתאים לתנאי, הפונקציה מצליחה.
הפונקציה קוראת לעצמה רקורסיבית בכל מקרה אחר.

בהגדרת lookup נעשה שימוש בפונקציה אטומית נוספת, \E|error|. פונקציה זו
מדפיסה את כל הפרמטרים שלה וגורמת לאינטרפרטר לעבור לסבב הבא של לולאת ה-\E|REPL|.

נשתמש ב-error כדי להגדיר את הפונקציה bind שהיא פונקציה רקורסיבית נוספת, אשר
מקבלת שתי רשימות באורך שווה: רשימת שמות (\E|names|) ורשימת ערכים (\E|values|),
וכן רשימה שהיא במבנה של \E|a-list|. \E|bind| ומחזירה את ה-a-list
שקיבלה בתוספת קישורים בין השמות לערכים.

\begin{KERNEL}
(defun bind (names values a-list) ; bind names to values, and append to a-list
  (cond ((null names) ; no more names left
        (cond ((null values) a-list) ; no more values left, binding done-> return a-list
              (t (error 'missing-names)))) ; more values than names
        ((null values) ; names is not nil but values is, i.e., more names than values
          (error 'missing-values))
        (t ; both names and values are not empty
          (cons ; create new binding and prepend it to result of recursive call
            (cons (car names) (car values)) ; new dotted-pair defines single binding
            (bind (cdr names) (cdr values) a-list))))) ; recursive call
\end{KERNEL}

§§ שיערוך ביטוי מורכב וקריאה לפונקציה

תיארנו למעלה כיצד נעשה שיערוך של אטום. נותר לתאר כיצד מתבצע השיערוך של ביטוי
מורכב.

\minipage\textwidth
\newcommand\exception[1]{{\textcolor{red}{#1}}}
\begin{mdframed}[backgroundcolor=Lavender!20]
  \footnotesize
  בהינתן ביטוי מורכב \E|$s$|, אלגוריתם השיערוך של ליספ מנסה
  להציג אותו כרשימה בת~$n+1$ איברים \[
    s=(s₀\;\;s₁\;\;⋯\;\;sₙ).
\] \begin{enumerate}
    ✦ \exception{
      אם הביטוי אינו רשימה, כלומר, אם הביטוי הנתון הוא \E|dotted pair| (כמו
      \E|(a.b)| למשל) השיערוך נכשל.}
    ✦ אם הרשימה ריקה, תוצאת השיערוך היא האטום \T|nil|.
    ✦ אחרת, המשערך מסתכל על הרשימה כעל עץ שיערוך, כלומר כעל קריאה
    לפונקציה~$s₀$ המקבלת~$n$ ארגומנטים \E|$s₁$,…,$sₙ$|, והשיערוך של~$s$ מתבצע
    על פי הסתכלות זו:
    \begin{quote}
      \begin{enumerate}
        ✦ \ע|שיערוך רקורסיבי של~$s₀$, הפונקציה אותה יש להפעיל.|
        \begin{itemize}
          ✦ \exception{אם השיערוך של~$s₀$ נכשל, אזי גם השיערוך של~$s$ נכשל.}
          ✦ אחרת, שיערוך זה מחזיר ביטוי~\E|S| שנסמן~$s₀'$.
          ✦ \exception{השיערוך של~$s$ נכשל אם~$s₀'$ אינו פונקציה.}
          ✦ \exception{השיערוך של~$s$ נכשל גם אם~$s₀'$ הוא פונקציה, אך כזו שאינה מצפה ל-$n$ ארגומנטים בדיוק.}
        \end{itemize}
        ✦ \ע|שיערוך הארגומנטים.|
        \begin{itemize}
          ✦ המשערך ממשיך כעת לשערך רקורסיבית את הארגומנטים לפונקציה~$s₀'$,
          הלא הם הביטויים~\E|$s₁$,…,$sₙ$|. נסמן את תוצאות השיערוך הללו
          ב-\E|$s₁'$,…,$sₙ'$|.

          ✦ \exception{אם השיערוך הרקורסיבי של אחד מבין
            הביטויים~\E|$s₁$,…,~$sₙ$| נכשל, אז גם השיערוך של הביטוי~$s$ כולו
            נכשל.}
        \end{itemize}

        ✦ \ע|הפעלת הפונקציה על הארגומנטים.|

        \begin{itemize}
          ✦ תוצאת השיערוך של הביטוי~$s$ היא תוצאת הקריאה לפונקציה~$s₀'$
          על~$n$ הביטויים \E|$s₁'$,…,~$sₙ'$|.

          ✦ \exception{אם הפעלה זו נכשלת, השיערוך של הביטוי~$s$ כולו נכשל.}
        \end{itemize}
      \end{enumerate}
    \end{quote}
  \end{enumerate}
\end{mdframed}
\endminipage

עלינו עוד להדגים כיצד נעשית קריאה לפונקציה. לשם כך, נשתמש בפונקציה המוגדרת מראש
defun, המשמשת להגדרת פונקציות חדשות, כדי להגדיר את הפונקציה הרקורסיבית append
המוסיפה פריט בסופה של רשימה \begin{LISP}
(defun append(x xs) ; append item x to end of list of xs
  (cond ((null xs) ; no more xs, recursion base¢…¢
          (cons x nil)) ; ¢…¢ return a list containing x
        (t ; recursive call
          (cons
            (car xs) ; prepend first of xs to¢…¢
            (append x (cdr xs)))))) ; ¢…¢ result of recursive call on remaining xs
\end{LISP}

לאחר ביצוע הגדרה זו, כלומר לאחר שיערוך הביטוי מעלה, יווצר קישור בין השם append
ובין גוף הפונקציה \E|append|, כלומר ביטוי ה-$λ$
\begin{LISP}
(lambda (x xs)
  (cond ((null xs) (cons x nil))
        (t (cons
              (car xs)
              (append x (cdr xs))))))
\end{LISP}

וקישור זה יתווסף לתחילת ה-\E|a-list|. ויזואלית,
אם ה-a-list נראתה טרם ההגדרה כך,
\begin{LTR}
  \begin{tikzpicture}[list/.style={rectangle split, rectangle split parts=2,
          draw,minimum height=3ex, fill=blue!20,rectangle split horizontal}, >=stealth, start chain, node distance=3ex]
    \foreach \x/\y/\z in {%
        h/t/t,
        i/nil/nil
      } {%
        \node[on chain, list,font=\tt\scriptsize] (\x) {\y};
        \node[below=4 ex of \x.one,anchor=north west,align=left,font=\tt\scriptsize,color=red] (temp) {\z};
        \draw[->,bend left] (\x.one south) .. controls+(270:0.3) and+(120:0.6) .. (temp.north west);
      }

    \node[on chain] (j) {\huge$⋯$};
    \draw[*->] let \p1=(i.two), \p2=(j.center) in (\x1,\y2)--(j);

    \foreach \a/\b in {h/i} {%
        \draw[*->] let \p1=(\a.two), \p2=(\b.center) in (\x1,\y2)--(\b);
      }

    \node[above=of h] (A) {a-list};
    \draw[->] (A.south)--(h);
  \end{tikzpicture}
\end{LTR}
אזי, אחרי ההגדרה של append ה-a-list תראה כך
\begin{LTR}
  \begin{tikzpicture}[list/.style={rectangle split, rectangle split parts=2,
          draw,minimum height=3ex, fill=blue!20,rectangle split horizontal}, >=stealth, start chain, node distance=3ex]
    \foreach \x/\y/\z in {%
    g/append/{%
    (lambda⏎
    \quad (x xs)⏎
    \quad (cond ⏎
    \quad\quad((null xs) (x))⏎
    \quad\quad (t (cons (car x) ⏎
    \quad\quad\quad\quad\quad\quad(append x (cdr xs))))))},
    h/t/t,
    i/nil/nil
    } {%
    \node[on chain, list,font=\tt\scriptsize] (\x) {\y};
    \node[below=4 ex of \x.one,anchor=north west,align=left,font=\tt\scriptsize,color=red] (temp) {\z};
    \draw[->,bend left] (\x.one south) .. controls+(270:0.3) and+(120:0.6) .. (temp.north west);
    }

    \node[on chain] (j) {\huge$⋯$};
    \draw[*->] let \p1=(i.two), \p2=(j.center) in (\x1,\y2)--(j);

    \foreach \a/\b in {g/h, h/i} {%
        \draw[*->] let \p1=(\a.two), \p2=(\b.center) in (\x1,\y2)--(\b);
      }

    \node[above=of g] (A) {a-list};
    \draw[->] (A.south)--(g);
  \end{tikzpicture}
\end{LTR}
בקריאה ל-append כמו
\begin{LISP}
(append 'a '(b c))
\end{LISP}

פונקציית השיערוך תאתר בתוך ה-a-list את ביטוי ה-$λ$ הקשור לפונקציה, כלומר
את הביטוי
\begin{LISP}
(lambda
  (x xs)
  (cond ((null xs) (x))
        (t (cons
              (car x)
              (append x (cdr xs))))))
\end{LISP}
ותפרק אותו לשלושת
מרכיביו: האטום lambda המזהה את הביטוי, רשימת הפרמטרים הפורמליים \lisp{(x xs)}
והביטוי לחישוב,
\begin{LISP}
(cond ((null xs) (x))
      (t (cons
            (car x)
            (append x (cdr xs)))))
\end{LISP}
אותו נסמן לרגע ב-$ϕ$. הקריאה לפונקציה append מתבצעת על ידי שיערוך הפרמטרים
האקטואליים, כלומר שיערוך של הפרמטר \lisp{'a} לכדי הערך \lisp{A} ושיערוך של
הפרמטר \lisp{'(b c)} לכדי הערך \lisp{(B C)}, ולאחר מכן חישוב הביטוי~$ϕ$ בתנאים
שבהם הפרמטר x משתערך לאטום \lisp{a} והפרמטר xs משתערך לרשימה \lisp{(b c)}.

קריאה לפונקציה בפרוצדורת השיערוך בליספ נעשית באמצעות יצירת קישורים בין שמות
הפרמטרים הפורמליים וערכי הפרמטרים האקטואליים (פרמטרים אקטואליים נקראים גם
\ע|ארגומנטים|). במקרה של הפונקציה append מתווספים שני \E|dotted-pairs|
\lisp{(x.a)} ו-\lisp{(xs.(b c))} ל-\E|a-list|.
ויזואלית, ה-a-list תראה כך
\begin{LTR}
  \begin{tikzpicture}[list/.style={rectangle split, rectangle split parts=2,
          draw,minimum height=3ex, fill=blue!20,rectangle split horizontal}, >=stealth, start chain, node distance=2.5ex]
    \foreach \x/\y/\z in {%
    e/x/a,
    f/xs/(b c),
    g/append/{%
    (lambda⏎
    \quad (x xs)⏎
    \quad (cond⏎
    \quad\quad((null xs) (x))⏎
    \quad\quad (t (cons (car x) ⏎
    \quad\quad\quad\quad\quad\quad(append x (cdr xs))))))},
    h/t/t,
    i/nil/nil
    } {%
    \node[on chain, list,font=\tt\scriptsize] (\x) {\y};
    \node[below=4 ex of \x.one,anchor=north west,align=left,font=\tt\scriptsize,color=red] (temp) {\z};
    \draw[->,bend left] (\x.one south) .. controls+(270:0.3) and+(120:0.6) .. (temp.north west);
    }

    \node[on chain] (j) {\huge$⋯$};
    \draw[*->] let \p1=(i.two), \p2=(j.center) in (\x1,\y2)--(j);

    \foreach \a/\b in {e/f, f/g, g/h, h/i} {%
        \draw[*->] let \p1=(\a.two), \p2=(\b.center) in (\x1,\y2)--(\b);
      }

    \node[above=of e] (A) {a-list};
    \draw[->] (A.south)--(e);
  \end{tikzpicture}
\end{LTR}
לאחר הקריאה לפונקציה, יש לשחזר את ערכה של ה-a-list לערכה המקורי טרם הקריאה לה.

כאשר הפונקציה append קוראת לעצמה רקורסיבית בפעם הראשונה, יתבצע קישור נוסף בין
הפרמטרים האקטואליים והפורמליים, וגם זאת על ידי תוספת של שני \E|dotted-pairs| נוספים:
\lisp{(x.a)} ו-\lisp{(xs.(c))}.

לאחר ביצוע קישורים אלו, ה-a-list תראה כך
\begin{LTR}
  \begin{tikzpicture}[list/.style={rectangle split, rectangle split parts=2,
          draw,minimum height=3ex, fill=blue!20,rectangle split horizontal}, >=stealth, start chain, node distance=3ex]
    \foreach \x/\y/\z in {%
    c/x/a,
    d/xs/(c),
    e/x/a,
    f/xs/(b c),
    g/append/{%
    (lambda⏎
    \quad (x xs)⏎
    \quad (cond⏎
    \quad\quad((null xs) (x))⏎
    \quad\quad (t (cons (car x) ⏎
    \quad\quad\quad\quad\quad\quad(append x (cdr xs))))))},
    h/t/t,
    i/nil/nil
    } {%
    \node[on chain, list,font=\tt\scriptsize] (\x) {\y};
    \node[below=4 ex of \x.one,anchor=north west,align=left,font=\tt\scriptsize,color=red] (temp) {\z};
    \draw[->,bend left] (\x.one south) .. controls+(270:0.3) and+(120:0.6) .. (temp.north west);
    }

    \node[on chain] (j) {\huge$⋯$};
    \draw[*->] let \p1=(i.two), \p2=(j.center) in (\x1,\y2)--(j);

    \foreach \a/\b in {c/d, d/e, e/f, f/g, g/h, h/i} {%
        \draw[*->] let \p1=(\a.two), \p2=(\b.center) in (\x1,\y2)--(\b);
      }

    \node[above=of c] (A) {a-list};
    \draw[->] (A.south)--(c);
  \end{tikzpicture}
\end{LTR}
כלומר הקישורים החדשים יסתירו את הקישורים הקודמים.

בקריאה הרקורסיבית השניה של append לעצמה, יווצרו שני קישורים נוספים,
\lisp{(x.a)} ו-\lisp{(xs.())}, וה-a-list תראה כך
\begin{LTR}
  \usetikzlibrary{chains,arrows}
  \begin{tikzpicture}[list/.style={rectangle split, rectangle split parts=2,
          draw,minimum height=3ex, fill=blue!20,rectangle split horizontal}, >=stealth, start chain, node distance=3ex]

    \foreach \x/\y/\z in {%
    a/x/a,
    b/xs/(),
    c/x/a,
    d/xs/(c),
    e/x/a,
    f/xs/(b c),
    g/append/{%
    (lambda⏎
    \quad (x xs)⏎
    \quad (cond⏎
    \quad\quad((null xs) (x))⏎
    \quad\quad (t (cons (car x) ⏎
    \quad\quad\quad\quad\quad\quad(append x (cdr xs))))))},
    h/t/t,
    i/nil/nil
    } {%
    \node[on chain, list,font=\tt\scriptsize] (\x) {\y};
    \node[below=4 ex of \x.one,anchor=north west,align=left,font=\tt\scriptsize,color=red] (temp) {\z};
    \draw[->,bend left] (\x.one south) .. controls+(270:0.3) and+(120:0.6) .. (temp.north west);
    }

    \node[on chain] (j) {\huge$⋯$};
    \draw[*->] let \p1=(i.two), \p2=(j.center) in (\x1,\y2)--(j);

    \foreach \a/\b in {a/b, b/c, c/d, d/e, e/f, f/g, g/h, h/i} {%
        \draw[*->] let \p1=(\a.two), \p2=(\b.center) in (\x1,\y2)--(\b);
      }

    \node[above=of a] (A) {a-list};
    \draw[->] (A.south)--(a);
  \end{tikzpicture}
\end{LTR}
כאמור, בכל פעם שבה הפונקציה append חוזרת מקריאה רקורסיבית, ה-a-list ישוחזר,
ושני הקישורים בין הפרמטרים פורמליים לאקטואליים שנעשו למען הקריאה הרקורסיבית
יוסרו ממנו.

§§ שיערוך דחוי
אלגוריתם השיערוך כפי שתואר עד כה אינו מספיק כדי לשערך את העץ ב\פנה|איור:עץ|
כהלכה. הסיבה היא שהאופרטור הטרנארי \LR{\texttt{$·$?$·$:$·$}} אשר מצוי בשורש העץ, מחשב את
הארגומנט
הראשון שלו, ובהתאם לתוצאת החישוב, מחשב את הארגומנט השני או השלישי, אבל לא את
שניהם גם יחד. למימוש של האופרטור הטרנארי

בפרוצדורת השיערוך שתוארה למעלה, יש לחשב את כל הארגומנטים לפונקציה
(או אופרטור), טרם שמפעילים את הפונקציה עצמה.

אלגוריתם השיערוך גם אינו מתאים לפונקציות כגון quote ו-lambda אשר אינן משערכות
את הפרמטרים שלהן כלל. הפרוצדורה גם אינה מתאימה לפונקציות כגון cond בהן השיערוך
של חלקים מגוף הפונקציה תלוי בערכים המחושבים בחלקים אחרים של הגוף.

אנו מבדילים בין שני סוגים של סמנטיקות של שיערוך של פונקציה:
\begin{description}
  ✦ [eager] בקריאה לפונקציה כגון set שהסמנטיקה שלה היא \E|eager|, הארגומנטים
  לפונקציה משוערכים \ע|טרם| הקריאה לפונקציה, ורק ערכי השיערוך מועברים לפונקציה.
  הפונקציה אינה יכולה לדעת מה היו ערכי הארגומנטים לפני ששוערכו.

  פונקציות בעלות סמנטיקה שהיא eager מיוצגות על ידי ביטוי~$λ$ שהאטום הראשון שבו
  הוא \T|lambda|.

  ✦ [normal] לעומת זאת, בקריאה לפונקציה כגון quote שהסמנטיקה שלה היא
  \E|normal|, הארגומנטים לפונקציה \ע|אינם| משוערכים טרם הקריאה לפונקציה,
  והפונקציה יכולה לבחור אם לשערך את הארגומנטים.

  פונקציות בעלות סמנטיקה שהיא normal מיוצגות על ידי ביטוי~$λ$ שהאטום הראשון שבו
  הוא האטום \T|nlambda|.†{%
  שימוש זה ב-nlambda היה קיים במימושים הראשונים של ליספ, אולם ברוב המימושים המודרניים
  הוא בוטל מסיבות של יעילות. במקומו, נוסף מה שקוראים מקרו \E|(macro)|
  שהוא דומה אך לא זהה ל-\E|nlambda|.}

  כלומר, אנו מבחינים בין שני סוגים של ביטויי~\E|$λ$| בליספ: כאלו שמשערכים את
  הפרמטרים שלהם לפני הפעלתם והמסומנים על ידי האטום \T|lambda| וכאלו שאינם עושים
  זאת, והמסומנים על ידי האטום \T|nlambda|.
\end{description}

הסמנטיקה של \ע|כל הפונקציות| וכמעט כל האופרטורים בשפות כמו \E|\CPL|, היא
\E|eager|. כמפורט ב\פנה|טבלה:אטומיות| שבהמשך, הסמנטיקה של כל הפונקציות
האטומיות של מיני-ליספ היא \E|eager|, זאת מלבד הפונקציה \E|cond|, אשר הסמנטיקה
שלה היא \E|normal|. הסיבה לכך היא ש-cond מדמה במיני-ליספ את פקודת התנאי \E|(if
command)| של שפות אחרות. למעשה, cond מכלילה את האופרטור הטרנארי
\LR{\texttt{$·$?$·$:$·$}}, כך שיוכל לבדוק מספר תנאים, ולא תנאי אחד בלבד.

הפונקציה quote שבמיני-ליספ אינה אטומית שכן ניתן להגדיר אותה באמצעות ביטוי
ה-nlambda הפשוט הבא
\begin{LISP}
(nlambda (x) x)
\end{LISP}
לא ניתן להגדיר פונקציות שהן normal באמצעות \E|defun|, אך ניתן להוסיף את הפונקציה
quote לרשימת ה-a-list באמצעות set
\begin{LIBRARY}
(set 'quote
  '(nlambda (x) x))
\end{LIBRARY}
כפי שראינו מיני-ליספ מגדירה גם פונקציה lambda שהסמנטיקה שלה היא normal המאפשרת
ליצור ביטוי~$λ$ מבלי להשתמש ב-quote
\begin{LISP}
> (lambda (x) (cons (cdr x) (car x)))
(LAMBDA (X) (CONS (CDR X) (CAR X)))
\end{LISP}
נוכל לכן להגדיר את הפונקציה lambda באמצעות
\begin{LIBRARY}
(set 'lambda
  '(nlambda (parameters-list body) ('lambda parameters-list body)))
\end{LIBRARY}
כלומר, הפונקציה lambda היא פונקציה שמקבלת שני פרמטרים: parameters-list
ו-\E|body|. הפונקציה אינה משערכת פרמטרים אלו ומחזירה ביטוי lambda המוגדר על ידם
\begin{LISP}
  ('lambda parameters-list body)
\end{LISP}
גם הפונקציה nlambda מסייעת להגדרת ביטויי nlambda באופן דומה, וניתן להגדירה
באמצעות
\begin{LIBRARY}
(set 'nlambda
  '(nlambda (parameters-list body) ('nlambda parameters-list body)))
\end{LIBRARY}
כלומר, גם הפונקציה nlambda היא פונקציה שמקבלת שני פרמטרים שאינם משוערכים
(\E|parameters-list| ו-\E|body|) ומחזירה ביטוי nlambda המוגדר על ידי
שני הפרמטרים הללו
\begin{LISP}
  ('nlambda parameters-list body)
\end{LISP}
לאחר הגדרה זו של הפונקציה nlambda הכתיב
\begin{LISP}
  (nlambda x y)
\end{LISP}
זהה לכתיב
\begin{LISP}
  ('nlambda x y)
\end{LISP}
ללא תלות בערכיהם של x ו-\E|y|.

ניתן גם להגדיר את הפונקציה defun באמצעות ביטוי nlambda
\begin{LIBRARY}
(set 'defun
  '(nlambda (name parameters body)
    (set name (lambda parameters body))))
\end{LIBRARY}
כלומר, הפונקציה defun היא ביטוי nlambda אשר מקבל שלושה פרמטרים: \E|name|,
\E|parameters| ו-\E|body|, ואשר קושר (באמצעות \E|set|) את הפרמטר אל ביטוי
ה-lambda המוגדר על ידי שני הפרמטרים האחרים, והכל מבלי לשערך את הפרמטרים שלו.

ניתן גם להגדיר את הפונקציה ndefun המגדירה פונקציה שהסמנטיקה שלה היא normal
\begin{LIBRARY}
(set 'ndefun
  '(nlambda (name parameters body)
    (set name (nlambda parameters body))))
\end{LIBRARY}
גוף ההגדרה זהה לזה של defun אלא ש-ndefun יוצרת ביטוי nlambda במקום ביטוי
\E|lambda|.

לאחר שהגדרנו את הפונקציה ndefun נוכל להגדיר את quote באמצעותה
\begin{LISP}
(ndefun quote(x) x)
\end{LISP}
הפונקציות \E|quote|, \E|defun|, \E|ndefun|, \E|lambda| ו-\E|nlambda| אינן
משערכות את הפרמטרים שלהן אף פעם. כאשר יש צורך בשיערוך הפרמטרים ניתן לקרוא
לפונקציה \E|eval|. נגדיר לדוגמה את הפונקציה setq אשר מאפשרת לקשור ערך לאטום ללא
צורך לבצע quoting על שם האטום
\begin{LISP}
(ndefun setq(a value)
  (set a (eval value)))
\end{LISP}
השימוש בסמנטיקה שהיא normal אינו מתעורר רק בהגדרת פונקציות כמו quote ו-defun
המשמשות לביצוע הגדרות. נגדיר למשל את הפונקציה \T|?:| אשר מתנהגת באופן דומה
לאופרטור בשם זה בשפת~\CPL
\begin{LISP}
(ndefun ?:¢¢ (condition true-value false-value)
  (cond ((eval condition) (eval true-value))
        (t (eval false-value))))
\end{LISP}
כדוגמה נוספת, נגדיר פונקציות \lisp{||} ו-\lisp{&&} הדומות לאופרטורים בשמות אלו
בשפת~\CPL, כלומר חישוב של הפונקציות הבוליאניות של and ו-or בשיטה הידועה בשם
\E|short-circuit|: ראשית שיערוך של הפרמטר הראשון, ואם זה קובע את תוצאת הפונקציה
הבוליאנית, הימנעות משיערוך הפרמטר השני. אחרת, ערכה של הפונקציה הבוליאנית הוא תוצאת
שיערוך הפרמטר השני
\begin{LISP}
(ndefun &&¢¢ (x y)
  (cond ((eval x) (eval y))
        (t nil)))
(ndefun ||¢¢ (x y)
  (cond ((eval x) t)
        (t (eval y))))
\end{LISP}

נתקן את אלגוריתם השיערוך כך שיתמוך בפונקציות שהן \E|normal|, על ידי הוספת התנאי
הבא:
\begin{mdframed}[backgroundcolor=Lavender!20]
  \footnotesize
  בשיערוך הביטוי \[
    s=(s₀\;\;s₁\;\;⋯\;\;sₙ).
\] אם שיערוך הביטוי~$s₀$ מחזיר פונקציה שהיא
  normal כלומר ביטוי~$λ$ שהאטום הראשון שבו הוא nlambda (ולא \E|lambda|) אזי יש
  להעביר לפונקציה זו כארגומנטים את הביטויים~\E|$s₁$,…,$sₙ$| ולא את הערכים
  המשוערכים שלהם~\E|$s₁'$,…,$sₙ'$|.
\end{mdframed}

§ מימוש אלגוריתם השיערוך במיני-ליספ
§§ הפונקציות evaluate ו-apply

מתברר שאפשר לממש את אלגוריתם השיערוך גם בליספ עצמה. לשם כך נגדיר פונקציה
evaluate המקבלת ביטוי~\E|S|, וסביבה במבנה של \E|a-list|, ואשר משערכת את הביטוי
בהתאם לקישורים שבסביבה הנתונה. הפונקציה evaluate קוראת לעצמה ברקורסיה כדי לשערך
את מרכיבי הביטוי.

כאשר evaluate תזהה ביטוי~$λ$ היא תקרא לפונקצית עזר apply אשר גם אותה נגדיר
בהמשך, ואשר אחראית להפעיל ביטוי~$λ$ על הפרמטרים המועברים לו. אם ביטוי ה-$λ$ הוא
\E|eager|, אז apply תקרא ל-evaluate כדי לשערך את הפרמטרים.

מבנה הקריאות והרקורסיות ההדדיות שבמימוש של evaluate ופונקציות העזר שלה מתואר
ב\פנה|איור:שיערוך|.
\begin{figure}[!htb]
  \כיתוב|רקורסיות הדדיות בין הפונקציה evaluate ופונקציות העזר שלה|
  \תגית|איור:שיערוך|
  \forestset{%
    call tree/.style={%
        for tree={%
            font=\scriptsize,
            rounded rectangle,
            fill=olive!10,
            % draw,
            s sep'+=-4pt,
            fit=tight,
          },
      },
  }
  \centering
  \begin{forest}
    call tree [evaluate,name=evaluate
    [lookup,name=lookup] {%
    \draw[->] () [out=south west,in=north] .. controls+(250:2) and+(120:2) .. ();
    }
    [is-primitive [exists]
    {%
    \draw[->] () [out=south west,in=north] .. controls+(250:2) and+(120:2) .. ();
    }
    ]
    [evaluate-primitive
    [apply-primitive
    [evaluate-cond] {%
    \draw[->] (.-20) .. controls+(-60:0.9) and+(60:0.9) .. (.20);
    \draw[->] (.200) .. controls+(210:9) and+(120:6) .. (evaluate.120);
    }
    [apply-eager-primitive
    [apply-trivial-primitive]
    ]
    ]
    ]
    [apply, name=apply
    [apply-decomposed-lambda
    [bind] {%
    \draw[->] () [out=south west,in=north] .. controls+(250:2) and+(120:1) .. ();
    }
    [evaluate-list] {%
    \draw[->] (.210) .. controls+(250:2) and+(120:1) .. (.150);
    \draw[->] (.-20) .. controls+(-10:6) and+(60:3) .. (evaluate.60);
    }
    ]
    ] {}
    ]
    \draw[->] (.180) [out=south west,in=north west] .. controls+(south west:1) and+(north west:2) .. ();
    \path (current bounding box.south west)++(-13ex,0) coordinate (A);
    \path (current bounding box.north east)++(11ex,1ex) coordinate (B);
    \clip (A) rectangle (B);
  \end{forest}
\end{figure}

אנו רואים באיור גם ש-apply קוראת לפונקציה הרקורסיבית bind (שאותה כבר הגדרנו),
וזאת כדי לקשור בין הפרמטרים הפורמליים והאקטואליים. האיור מראה גם ש-evaluate
משתמשת בכמה פונקציות נוספות שהוצגו כבר: הפונקציה הרקורסיבית lookup כדי לשערך את
ערכו של אטום, ו-is-atomic הקוראת ל-exists הרקורסיבית כדי לזהות אטומים המזהים
פונקצית אטומיות.

בנוסף לכל הפונקציות הללו, evaluate נעזרת גם בפונקציה \E|evaluate-atomic|, אשר
מצידה משתמשת בכמה פונקציות עזר משלה, בהמשך כדי לממש את הפונקציות האטומיות שבהם
משתמשת מיני-ליספ. שיערוך של אטומים, יחד עם מימוש הפונקציות האטומיות, מהווה את
בסיס הרקורסיה של \E|evaluate|. השיערוך של כל ביטוי~\E|S|, מורכב ככל שיהיה
מתורגם על ידי evaluate לסדרה מתאימה של שיערוך של אטומים והפעלות של פונקציות
אטומיות.

הגוף של evaluate עצמה הוא קצר, והוא מבוסס על אבחנה בין שלוש אפשרויות שונות ביחס
לביטוי אותו היא נדרש לשערך.

\minipage\textwidth
\begin{KERNEL}
(defun evaluate(S-expression a-list) ; evaluate S-expression in the environment defined by a-list
  (cond ((atom S-expression) ; recursion base: lookup of atom in a-list
          (lookup S-expression a-list))
        ((is-atomic (car S-expression)) ; case of handling atomic functions
          (evaluate-atomic S-expression a-list))
        (t ; recursive step---lambda applied to parameters
          (apply (evaluate (car S-expression) a-list) ; find lambda expression
                  (cdr S-expression) ; find actual parameters
                  a-list))))
\end{KERNEL}
\endminipage

אם הביטוי ש-evaluate מקבלת הוא אטום, evaluate מפעילה את \E|lookup| כדי למצוא את
הערך הקשור אליו ברשימת ה-\E|a-list|. אם לעומת הביטוי הוא ביטוי מורכב, אזי
dotted-pair אשר ה-car שלו הוא פונקציה, וה-cdr שלו הוא רשימה של
ארגומנטים שיש להעביר לפונקציה. אם ה-car הוא שם של פונקציה אטומית, נדרש טיפול
מיוחד, ואז evaluate קוראת לפונקציה evaluate-atomic אשר מבצעת זאת.
בכל מקרה אחר, evaluate מוצאת את הפונקציה אותה יש להפעיל על ידי השיערוך
\begin{LISP}
(evaluate (car S-expression) a-list)
\end{LISP}
קריאה רקורסיבית זו תחזיר את ביטוי ה-$λ$ אותו יש להפעיל על הפרמטרים. evaluate
גם מחשבת את הפרמטרים באמצעות הקריאה-\T|(cdr S-expression)|, וקוראת לפונקציה
apply אשר מפעילה את ביטוי ה-$λ$ על הפרמטרים.

הפונקציה apply מצידה מפרקת את ביטוי ה-$λ$ לשלושת מרכיביו (תגית, רשימת פרמטרים
וביטוי לחישוב), ומעבירה מרכיבים אלו לפונקצית העזר apply-decompsed-lambda
\begin{KERNEL}
(defun apply(lambda-expression actuals a-list)
  (apply-decomposed-lambda
    (car lambda-expression) ; tag=lambda or nlambda
    (car (cdr lambda-expression)); list of formal parameters
    (car (cdr (cdr lambda-expression))); body
    actuals
    a-list))
\end{KERNEL}
הפונקציה apply-decompsed-lambda משערכת את גוף הפונקציה, בסביבה הכוללת קישור בין
הפרמטרים הפורמליים לאקטואליים. קישור זה נעשה באמצעות הפונקציה bind אותה כבר
הגדרנו.
\begin{KERNEL}
(defun apply-decompsed-lambda(tag formals body actuals a-list)
  (evaluate body
    (cond ((eq tag 'nlambda) (bind formals actuals a-list))
          ((eq tag 'lambda) (bind formals (evaluate-list actuals a-list) a-list))
          (t (error 'unkown-lambda tag))
)))
\end{KERNEL}

האבחנה בין הסוגים השונים של ביטוי ה-$λ$, נעשית באופן הבא בפונקציה
apply-decompsed-lambda
\begin{itemize}
  ✦
  אם התגית של ביטוי ה-$λ$ היא האטום nlambda הקישור נעשה
  ללא שיערוך הפרמטרים
  האקטואליים.
  ✦ אם לעומת זאת התגית היא האטום lambda הקישור בין הפרמטרים הפורמליים
  לאקטואליים נעשה לאחר שיערוך של הפרמטרים האקטואליים באמצעות הפונקציה
  \E|evalauate-list|. ✦
  אם התגית אינה אף אחד משני האטומים הללו השיערוך נכשל.
\end{itemize}

ההגדרה של הפונקציה evaluate-list המיועדת לשיערוך רשימת הפרמטרים האקטואליים היא
באמצעות הפעלה רקורסיבית של evaluate על כל אחד מאיברי הרשימה, ושרשור התוצאות
לכדי רשימה אחת.
\begin{KERNEL}
(defun evaluate-list(S-expressions a-list)
  (cond ((null S-expressions) nil) ; no more S-expressions to evaluate
    (t (cons
          (evaluate (car S-expressions) a-list) ; evaluate first S-Expression
          (evaluate-list (cdr S-expressions) a-list))))) ; recursive call on remainder
\end{KERNEL}

§§ מימוש הפונקציות האטומיות של מיני-ליספ בתוך הפונקציה evaluate
תיארנו את הפונקציה evaluate המממשת את מרבית הפעולות שבאלגוריתם השיערוך של ליספ.
נותר עוד לתאר את המימוש של הפונקציה evaluate-atomic אשר נועדה לטיפול ב-8
הפונקציות האטומיות של השפה.

נעיין ראשית ב\פנה|טבלה:אטומיות| המפרטת את כל שנדרש כדי \ע|להשתמש| בפונקציות
הללו.

\newcounter{magicrownumbers}
\newcommand\rownumber{\stepcounter{magicrownumbers}\arabic{magicrownumbers}}
\begin{table}[!hbt]
  \footnotesize
  \rowcolors{2}{blue!10}{white}
  \begin{tabularx}\textwidth{r>{\scriptsize\setLR}c>{\scriptsize}c
    >{\setRL\scriptsize\raggedleft\arraybackslash}X
    >{\scriptsize\setLR\raggedright\arraybackslash}X
    >{\scriptsize\setLR\raggedright\arraybackslash}X
    }

    \toprule
    \normalsize \bfseries #                          &
    \normalsize \bfseries {\text{/arity}}שם           &
    \normalsize \bfseries סמנטיקה                     &
    \normalsize \bfseries תמצית                       &
    \multicolumn1c{\normalsize \bfseries \RL{דוגמאות}} ⏎
    \midrule

    \rownumber                                        &
    atom/1                                            &
    eager                                             &
    בדיקה אם הפרמטר הוא אטום                        &
    \lisp{(atom nil)}~$⇒$ \lisp{T} \newline
    \lisp{(atom t)}~$⇒$ \lisp{T} \newline
    \lisp{(atom '(a a))}~$⇒$ \lisp{NIL} \newline
    \lisp{(atom 'a)}~$⇒$ \lisp{T} ⏎

    \rownumber                                        &
    car/1                                             &
    eager                                             &
    חילוץ האיבר הראשון ברשימה                         &
    \lisp{(car '(b.a))}~$⇒$ \lisp{B} \newline
    \lisp{(car '(b a))}~$⇒$ \lisp{B} \newline
    \lisp{(car '(a))}~$⇒$ \lisp{A} \newline
    \lisp{(car 'a)}~$⇒$ \text{✗} \newline
    \lisp{(car ())}~$⇒$ \text{✗} ⏎

    \rownumber                                        &
    cdr/1                                             &
    eager                                             &
    חילוץ שארית הרשימה, כלומר הרשימה ללא האיבר הראשון &
    \lisp{(cdr '(a.b))}~$⇒$ \lisp{b} \newline
    \lisp{(cdr '(a b))}~$⇒$ \lisp{(b)} \newline
    \lisp{(cdr '(b))}~$⇒$ \lisp{NIL} \newline
    \lisp{(cdr t)}~$⇒$ ✗ \newline
    \lisp{(cdr ())}~$⇒$ ✗ \newline
    \lisp{(cdr nil)}~$⇒$ ✗ ⏎

    \rownumber                                        &
    ($n≥0$) cond/n                                    &
    normal                                            &
    הכללה של פקודת \E|if|. &
    \lisp{(cond (t 'A))}~$⇒$ \lisp{A} \newline
    \lisp{(cond (nil 'A) (t 'B))}~$⇒$ \lisp{B} \newline
    \lisp{(cond (nil 'A) (t 'B) (t 'C))}~$⇒$ \lisp{B} \newline
    \lisp{(cond (nil 'A) (nil 'B) (nil 'C))}~$⇒$ \lisp{nil}\newline
    \lisp{(cond)}~$⇒$ \lisp{nil} ⏎

    \rownumber                                        &
    cons/2                                            &
    eager                                             &
    הוספת איבר בתחילת רשימה                           &
    \lisp{(cons 'a '(b c))}~$⇒$ \lisp{(A B C)} \newline
    \lisp{(cons 'b nil)}~$⇒$ \lisp{NIL} \newline
    \lisp{(cons 'a 'b)}~$⇒$ \lisp{(A.B)} ⏎

    \rownumber                                        &
    eq/2                                              &
    eager                                             &
    בדיקה אם שני הפרמטרים הם אטומים השווים זה לזה     &
    \lisp{(eq t t)}~$⇒$ \lisp{T} \newline
    \lisp{(eq t nil)}~$⇒$ \lisp{NIL} \newline
    \lisp{(eq nil nil)}~$⇒$ \lisp{T} \newline
    \lisp{(eq 'a 'a)}~$⇒$ \lisp{T} \newline
    \lisp{(eq '(a a) '(a a))}~$⇒$ \lisp{NIL} ⏎

    \rownumber                                        &
    ($n≥0$) error/n                                   &
    eager                                             &
    הדפסת כל הפרמטרים ועצירת ביצוע התכנית             &
    \lisp{(error)}~$⇒$ ✗ \newline
    \lisp{(error A)}~$⇒$ ✗ \newline
    \lisp{(error 'my-error 'message)}~$⇒$ ✗ ⏎

    \rownumber                                        &
    set/2                                             &
    eager                                             &
    יצירת קישור בין אטום ובין ביטוי~\E|S|. &
    \lisp{(set 'a '(b c))}~$⇒$ \lisp{(b c)}\newline
    \lisp{(set 'b nil)}~$⇒$ \lisp{NIL}
    \label{atomic:count}
 ⏎
    \bottomrule
  \end{tabularx}
  \כיתוב|הפונקציות האטומיות של מיני-ליספ|
  \תגית|טבלה:אטומיות|
\end{table}

מימוש הפונקציות האטומיות בתוך evaluate נדרש לקיים את המפרט שבטבלה. כמובן, לא
ניתן לממש את הפונקציות האטומיות עצמן. תפקידה של הפונקציה evaluate הוא לזהות
שהיא נדרשת לשערך פונקציה אטומית, ואז להעביר את המשימה לפונקציה האטומית המתאימה.
מרביתה של הפונקציה evaluate-atomic היא פעולות הכנה לקראת המטרה העיקרית שלנו:
\begin{quote}
  מימוש הפונקציות האטומיות בעבור פונקצית הספרייה evaluate וזאת בתוך הפונקציה
  \E|evaluate-atomic|.
\end{quote}

הצעד הראשון במימוש של evaluate-atomic הוא פירוק הביטוי אותו יש לשערך לשני
חלקים: שם הפונקציה הפרימיטיבית, והפרמטרים לפונקציה זו.
\begin{KERNEL}
(defun evaluate-atomic (S-expression a-list)
  (apply-atomic (car S-expression) (cdr S-expression) a-list))
\end{KERNEL}

הפונקציה evaluate-atomic צריכה לטפל בכל הפונקציות האטומיות, כמו גם בפונקציה
eval אשר אינה אטומית, אבל עדיין לא טופלה. אנו רואים \פנה|טבלה:אטומיות| ש-cond
היא הפונקציה היחידה מבין כל אלו שהסמנטיקה שלה היא \E|normal|.

\minipage\textwidth
במימוש של הפונקציה \E|apply-atomic| נפריד ראשית בין cond ובין כל שאר הפונקציות
האטומיות.
\begin{KERNEL}
(defun apply-atomic (atomic actuals a-list)
  (cond ((eq 'cond atomic) ; special case: cond has normal semantics
            (evaluate-cond actuals a-list)) ; don't evaluate actuals
        (t (apply-eager-atomic ; all other atomics are eager
              atomic
              (evaluate-list actuals a-list)
              a-list)))|)
\end{KERNEL}
\endminipage

במקרה ששם הפונקציה הפרימיטיבית cond נשתמש בפונקצית עזר, \E|evaluate-cond|. במקרה
שהשם הוא אחר, נמשיך עם הפונקציה \E|apply-eager-atomic|.

המימוש של evaluate-cond הוא ברקורסיה פשוטה על רשימת ה-test-forms
\begin{KERNEL}
(defun evaluate-cond(test-forms a-list)
  (cond ((null test-forms) nil) ; if no more test-forms, return nil
        ((evaluate (car (car test-forms)) a-list) ; evaluate condition part of the first test-form
        (evaluate (car (cdr (car test-forms)) a-list))) ; if true, evaluate value part (second part) of the first test-form
        (t (evaluate-cond (cdr test-forms) a-list)))) ; otherwise, recurse on rest of test-forms
\end{KERNEL}

כדי לטפל בשאר הפונקציות שב\פנה|טבלה:אטומיות| ובפונקציה eval נשים לב לכך שכל אלו
מלבד eval אינן נדרשות יותר ל-a-list וכולן, לבד מ-error מצפות לפרמטר אחד או שני
פרמטרים.

הפונקציה apply-eager-atomic מטפלת במקרים המיוחדים של \E|eval| ושל \E|error|
ומעבירה את המשך הטיפול ל-apply-trivial-atomic
\begin{KERNEL}
(defun apply-eager-atomic (atomic actuals a-list)
  (cond
    ((eq atomic 'error) (error actuals))
    ((eq atomic 'eval) (evaluate (car actuals) a-list))
    (t (apply-trivial-atomic
        atomic ; one of ¢\tt atom¢, car, cdr, cons, eq, set
        (car actuals); first actual parameter
        (car (cdr actuals)))))) ; second actual parameter, could be nil
\end{KERNEL}

כעת הפונקציה apply-trivial-atomic משתמשת ב-cond כדי לטפל בששת הפונקציות האטומיות שנותרו:
\begin{KERNEL}
(defun apply-trivial-atomic (atomic first second)
  (cond ((eq atomic 'atom) (atom first))
        ((eq atomic 'car) (car first))
        ((eq atomic 'cdr) (cdr first))
        ((eq atomic 'cons) (cons first second))
        ((eq atomic 'eq) (eq first second))
        ((eq atomic 'set) (set first second))
        (t (error 'something-went-wrong atomic))))
\end{KERNEL}

§§ הקוד המלא של evaluate ופונקציות העזר שלה
\תגית|סעיף:מימוש|

\immediate \closeout \kernelFile
\begin{LTR}
  \lstinputlisting[language=Mini,style=display,
    numbers=left,
    stepnumber=1,
    numbersep=2pt,
    xleftmargin=3ex,
    numberblanklines=false,
    numberstyle=\tiny\bf,
    backgroundcolor=\color{olive!10}
  ]{\jobname.kernel.lisp}
\end{LTR}

§§ הפונקציות המוגדרות מראש של מיני-ליספ

\begin{table}[!htbp]
  \footnotesize
  \rowcolors{2}{blue!10}{white}
  \begin{tabularx}\textwidth{>{\scriptsize}r>{\setLR\scriptsize}c>{\scriptsize}c
    >{\scriptsize\raggedleft\arraybackslash}X
    >{\setLR\scriptsize\raggedright\arraybackslash}X
    >{\setLR\scriptsize\raggedright\arraybackslash}X
    }
    \toprule
    \normalsize \bfseries #                                                &
    \normalsize \bfseries {\text{/arity}}שם                                 &
    \normalsize \bfseries סמנטיקה                                           &
    \normalsize \bfseries תמצית                                             &
    \multicolumn1c{\normalsize \bfseries \RL{הגדרה}}                        &
    \multicolumn1c{\normalsize \bfseries \RL{דוגמאות}} ⏎
    \midrule

    \rownumber                                                              &
    \E|defun/3|                                                             &
    normal                                                                  &
    יצירת קישור בין אטום ובין ביטוי~\E|S| שהוא פונקציה בסמנטיקה \E|eager|. &
    \lisp{(set 'defun }\newline
    \mbox\quad\lisp{(nlambda (n p b)} \newline
    \mbox\qquad \lisp{(set n (lambda p b))))}                               &
    \T|(defun f (x y) (y x))|\newline\quad
 ⏎

    \rownumber                                                              &
    lambda/2                                                                &
    normal                                                                  &
    יצירת פונקציה אנונימית שהסמנטיקה שלה היא \E|eager|. &
    \lisp{(ndefun lambda(p b)}\newline
    \mbox\quad\lisp{('lambda p b))}                                         &
    \lisp{((lambda (x) (car (cdr x)))}\newline
    \mbox\quad\lisp{'(a b)}~$⇒$ \lisp{B}
 ⏎

    \rownumber                                                              &
    nil/0                                                                   &
    n/a                                                                     &
    האטום הנקבע על ידי~$ε$, סדרת ריקה של תווים, והמציין את הרשימה הריקה. אטום
    זה מציין את עצמו, ונחשב גם לערך הבוליאני של שקר, \E|false|. &
    \lisp{(set 'nil 'nil)}                                                  &
    \lisp{nil}~$⇒$ \lisp{NIL} \newline
    \lisp{()}~$⇒$ \lisp{NIL} \newline
    \lisp{(eq nil t)}~$⇒$ \lisp{NIL} ⏎

    \rownumber                                                              &
    null/1                                                                  &
    eager                                                                   &
    בדיקה אם הארגומנט הוא האטום \lisp{nil}                                  &
    \lisp{(defun null (x)}\newline
    \mbox\quad\lisp{(eq x nil))}                                            &
    \lisp{(null t)}~$⇒$ \lisp{NIL} \newline
    \lisp{(null nil)}~$⇒$ \lisp{T}\newline
    \lisp{(null 'a)}~$⇒$ \lisp{NIL} \newline
    \lisp{(null '(a a))}~$⇒$ \lisp{NIL} ⏎
    \rownumber                                                              &
    ndefun/3                                                                &
    normal                                                                  &
    יצירת קישור בין אטום ובין ביטוי~\E|S| שהוא פונקציה בסמנטיקה \E|normal|. &
    \lisp{(set 'ndefun }\newline
    \mbox\quad\lisp{(nlambda (n p b)} \newline
    \mbox\qquad\lisp{(set n (nlambda p b))))}                               &
    \T|(ndefun f(x y) (y x))|\newline\quad~$⇒$ \T|F|⏎

    \rownumber                                                              &
    nlambda/2                                                               &
    normal                                                                  &
    יצירת פונקציה אנונימית שהסמנטיקה שלה היא \E|normal|. &
    \lisp{(ndefun nlambda(p b)}\newline\mbox\quad\lisp{('nlambda p b))}     &
    \T|(nlambda f (x y) (y x))|~$⇒$
    \newline\mbox\quad\T|(nlambda f (x y) (y x)|
 ⏎
    \rownumber                                                              &
    quote/1                                                                 &
    normal                                                                  &
    החזרת הארגומנט מבלי לשערך אותו. &
    \lisp{(ndefun quote (x) x)}                                             &
    \lisp{(quote a)}~$⇒$ \lisp{A} \newline
    \lisp{(quote (b c))}~$⇒$ \lisp{(B C)} \newline
    \lisp{'a}~$⇒$ \lisp{A} \newline
    \lisp{'(b c)}~$⇒$ \lisp{(B C)} \newline ⏎

    \rownumber                                                              &
    t/0                                                                     &
    n/a                                                                     &
    האטום \E|t|, המציין את \E|t|, כלומר את עצמו. נחשב גם לערך הבוליאני של אמת,
    \E|(true)|. &
    \lisp{(set 't 't)}                                                      &
    \lisp{t}~$⇒$ \lisp{T} \newline \lisp{(eq nil nil)}~$⇒$ \lisp{T} ⏎

    \bottomrule
  \end{tabularx}
  \כיתוב|הפונקציות המוגדרות מראש במיני-ליספ|
  \תגית|טבלה:מראש|
\end{table}

במהלך הדיון, הצגנו את המימוש של הפונקציות המוגדרות מראש באמצעות הפונקציות
האטומיות. הנה הקוד המלא של המימוש שהצגנו.
\immediate\closeout \libraryFile
\begin{LTR}
  \lstinputlisting[language=Mini,style=display,
    numbers=left,
    stepnumber=1,
    numbersep=2pt,
    xleftmargin=3ex,
    numberblanklines=false,
    numberstyle=\tiny\bf,
    backgroundcolor=\color{orange!20}
  ]{\jobname.library.lisp} \end{LTR}

§ תרגילים
\begin{enumerate}
  ✦ בדיקה מדוקדקת תגלה הבדל נוסף בין שני המימושים של הפונקציה \E|exists| חוץ
  מאשר ההבדל עליו מצביע הדיון. אתר הבדל זה, והסבר מדוע הוא אינו משנה את
  הסמנטיקה של המימוש.

  ✦ הפונקציה member היא פונקצית ספרייה סטנדרטית ברוב המימושים של ליספ. הפונקציה
  מקבלת אטום ורשימה, ומחזירה את זנב הרשימה הארוך ביותר שהאיבר הראשון שבו הוא
  האטום. אם האטום אינו מצוי ברשימה, הפונקציה מחזירה \E|nil|. ממש פונקציה זו
  במיני-ליספ, והסבר מדוע ניתן להחליף כל קריאה ל-exists בקריאה דומה
  ל-member עם אותם פרמטרים בדיוק.

  ✦ ממש במיני-ליספ את הפונקציה הרקורסיבית equal אשר משווה שני ביטויי~\E|S| אם
  הם זהים, כלומר אם הטופולוגיה שלהם כעצים בינאריים מלאים זהה, והאטומים המצויים
  בעלים זהים גם כן. הפונקציה מחזירה t במקרה של זהות, ו-nil אחרת.

  ✦ ב\פנה|איור:שיערוך| ישנה פונקציות אשר אינן יכולות לקרוא לעצמה רקורסיבית~(אם
  ישירות ואם בעקיפין). מיהן פונקציות אלו? מדוע אין צורך בהן ברקורסיה?

  ✦ חקור ביטויי~\E|S| שמקיימים את התכונה שהשיערוך שלהם מחזיר את הביטוי עצמו. האם
  יש ביטויים אטומיים כאלו? האם יש ביטויים מורכבים כאלו? (רמז: עיין בפונקציות
  המוגדרות מראש). כתוב פונקציה f המקיימת את התכונה הבאה: השיערוך של הביטוי
  \E|(f~$x$~$y$~$z$)| יחזיר את \E|(f~$x$~$y$~$z$)| עבור כל ערך של \E|$x$|,
  \E|$y$|, ו-\E|$z$|.

  ✦ היכן בודק המימוש של evaluate אם מספר הפרמטרים האקטואליים בהפעלה של
  ביטוי~$λ$ זהה למספר הפרמטרים הפורמליים שמוגדר בביטוי?

  ✦ המימוש של evaluate אינו בודק אם מספר הפרמטרים בפועל לפונקציות האטומיות
  הטריוויאליות תואם למספר הפרמטרים שהן אמורות לקבל. באילו מקרים העדר הבדיקה יביא
  לכך שמספר פרמטרים לא נכון לפונקציה טריביאלית לא ידווח כשגיאה? תקן את המימוש
  כך ששגיאה כזו תדווח.

  ✦ הסבר כיצד ניתן לתקן את evaluate כפי שתוארה כאן בכדי לתמוך בפונקציה אטומית
  \E|list| אשר מקבלת רשימה של ביטויי~\E|S|, ומחזירה את הרשימה של הביטויים לאחר
  שיערוכם.

  ✦ הסבר כיצד ניתן לתקן את evaluate כפי שתוארה כאן בכדי לתמוך בפונקציה אטומית
  \E|progn| אשר מקבלת רשימה של ביטויי~\E|S|, ומשערכת אותם לפי סדרם. תוצאת
  השיערוך היא תוצאת השיערוך של הפריט האחרון ברשימה, או nil במקרה שהרשימה ריקה.

  ✦ אילו שינויים יש לערוך במימוש של evaluate אם נשנה את מיני-ליספ כך שהפונקציות
  defun ו-quote תהיינה אטומיות, וכך שמיני-ליספ לא תתמוך בפונקציות המוגדרות
  מראש \E|lambda|, \E|nlambda| ו-\E|ndefun|.

  ✦ נגדיר \ע|יצוג אונארי| של המספרים הטבעיים באמצעות ביטוי-\E|S| באופן הבא:
  \begin{itemize}
    ✦ המספר 0 מיוצג על ידי \E|nil|,
    ✦ המספר 1 מיוצג על ידי הרשימה המכילה את \E|nil|, כלומר הרשימה \E|(nil)|,
    ✦ המספר 2 מיוצג על ידי הרשימה המכילה את המספר 1, כלומר \E|((nil))|,
    ✦ המספר 3 על ידי הרשימה המכילה את המספר 2, כלומר, \E|(((nil)))|,
    וכן הלאה.
  \end{itemize}
  הגדר פונקציה uadd המקבלת שתי רשימות המהוות יצוג אונארי של מספרים,
  ומחזירה רשימה שהיא יצוג אונארי של סכום המספרים. החיבור 1+2 יוצג על ידי
  הקריאה
  \begin{LISP}
> (uadd (nil) ((nil)))
(((NIL)))
\end{LISP}
  אשר, כפי שאנו רואים, מחזירה את הייצוג של המספר 3.
  כדאי להשתמש בהגדרות העזר הבאות:
  \begin{LISP}
(defun s(x) (cons x ())) ; the successor function
(defun p(x) (car x)) ; the predecessor function
; symbolic names for the first few numbers
(setq zero nil)
(setq one (s zero))
(setq two (s one))
(setq three (s two))
(setq four (s three))
(setq five ‘((((())))))
\end{LISP}
  בהגדרות אלו נקבל שהקריאה
  \lisp{(uadd three two)}
  תחזיר \lisp{((((()))))}.

  ✦ כתוב פונקציה בשם umult אשר מכפילה שני מספרים טבעיים בייצוג אונארי.

  ✦ \ע|בייצוג בינארי| על ידי רשימות, כל מספר מייוצג על ידי רשימת אטומים, כאשר
  כל אטום מייצג ספרה בינארית: את הספרה~$0$ מייצג האטום Z, ואת הספרה~$1$ מייצג
  את האטום O. המספר~$5$ מיוצג על ידי הרשימה \begin{LISP}
(O Z O)
\end{LISP} שכן הייצוג הבינארי של~$5$ הוא~$101$. הרשימות נכתבות ב-little-endian,
  כלומר, הביט המשמעותי פחות, מופיע ראשון. לדוגמה, המספר~$6$ מיוצג על ידי הרשימה
  \begin{LISP}
 (Z O O)
 \end{LISP}
  שכן הייצוג הבינארי של~$6$ הוא~$110$. אורך הרשימה תלוי בגודל המספר.
  לדוגמה, המספר~$3$ בכתיב בינארי מיוצג על ידי הרשימה
  \begin{LISP}
 (O O)
 \end{LISP} הייצוג הבינארי של~$3$ הוא~$11$. ניתן להוסיף אטומים של Z בסוף
  הרשימה. כל הרשימות הבאות הן יצוג של המספר~$3$.
  \begin{LISP}
(O O)
(O O Z)
(O O Z Z)
(O O Z Z Z)
\end{LISP}
  ובאופן דומה, כל הרשימות הבאות הן יצוג של המספר~$0$,
  \begin{LISP}
()
(Z)
(Z Z)
(Z Z Z)
\end{LISP}
  בתרגיל זה הנך נדרש לכתוב שתי פונקציות:
  \begin{itemize}
    ✦ כתוב פונקציה במיני-ליספ בשם bnormalize המקבלת מספר בייצוג בינארי כרשימה,
    ומחזירה את הייצוג הקצר ביותר שלו באותה דרך.
    ✦ כתוב פונקציה במיני-ליספ בשם badd המקבלת שתי רשימות המהוות יצוג בינארי
    של מספר טבעי לא שלילי, ומחזירות רשימה שהיא היצוג הבינארי של סכומם.
    הפונקציה צריכה להחזיר הרשימה הקצרה ביותר מבין הרשימות שמייצגות את הסכום,
    אבל היא אינה יכולה להניח שהמספרים אותם עליה לחבר מיוצגים כרשימה קצרה
    ביותר.
  \end{itemize}

  ✦ כתוב פונקציה בשם bmult אשר מכפילה שני מספרים טבעיים בייצוג בינארי.
  end{enumerate}

  ✦ השתמש ב-ndefun, ב-eval, וב-set כדי לממש במיני-ליספ את הפונקציה \E|setq| של
  \E|Common Lisp| אשר דומה לפונקציה set:
  \ספרר
  ✦ הפונקציה setq מקבלת שני פרמטרים, a ו-\E|e| והיא קושרת את e לאטום \E|a|
  ב-\E|a-list|.
  ✦ בניגוד ל-\E|set|, setq \ע|אינה משערכת| את הפרמטר~a טרם ביצוע הקישור.
  ✦ בדומה לפונקציות set ו-setq הפונקציה nsetq \ע|משערכת| את הפרמטר~e טרם ביצוע
  הקישור. שיערוך הביטוי \T|(nsetq a (car '((1 (2)) 3))| יקשור את האטום a
  לרשימה-\T|(1 (2))|.
  ✦ הערך אותה מחזירה setq הוא אותו הערך אותו מחזירה \E|set|.
===
  \begin{LISP}
> (setq a (car '((1 (2)) 3))
(1 (2))
> a
(1 (2))
\end{LISP}

  ✦ השתמש ב-setq וב-cond כדי לכתוב פונקציה בשם \E|nsetq| אשר דומה לפונקציה
  \E|setq|, ומשתמשת בה כדי לקיים את התכונות הבאות:
  \ספרר
  ✦ הפונקציה nsetq מקבלת שני פרמטרים, a ו-\E|e| של והיא קושרת את e לאטום \E|a|
  ב-\E|a-list|.
  ✦ בדומה ל-\E|setq|, הפונקציה nsetq \ע|אינה משערכת| את הפרמטר a טרם ביצוע
  הקישור.
  ✦ בדומה הן ל-set של מיני-ליספ והן ל-setq הפונקציה nsetq \ע|משערכת| את
  הפרמטר~e טרם ביצוע הקישור. שיערוך הביטוי \T|(nsetq a (car '((1 (2)) 3))|
  יקשור את האטום a לרשימה-\T|(1 (2))|.
  ✦ בניגוד לפונקציות set ו-setq, הפונקציה nsetq מחזירה את האטום nil ללא תלות
  בערכם של~a ושל \E|e|.
===
  \begin{LISP}
> (nsetq a (car '((1 (2)) 3))
NIL
> a
(1 (2))
> (nsetq b NIL)
NIL
> b
NIL
\end{LISP}
\end{enumerate}
