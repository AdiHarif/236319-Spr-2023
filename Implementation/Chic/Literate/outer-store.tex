
\documentclass[12pt]{article}

\usepackage[margin=1.5cm]{geometry}
\usepackage{newverbs,listings}
\usepackage[dvipsnames]{xcolor}

\newcounter{segment}
\setcounter{segment}{0}

%\immediate \openout \codeFile=\jobname.cc

\lstnewenvironment{code}[1][\relax]{\lstset{#1}}{}

\definecolor{eclipse-purple}{rgb}{0.50,0.00,0.33} % Keyowrds
\definecolor{eclipse-blue}{rgb}{0.16,0.00,1.00} % Doc 
\definecolor{eclipse-green}{rgb}{0.25,0.50,0.37} % Comments
\definecolor{eclipse-red}{rgb}{0.6,0,0} % for strings

\definecolor{clr-background}{RGB}{255,255,255}
\definecolor{clr-text}{RGB}{0,0,0}
\definecolor{clr-string}{RGB}{163,21,21}
\definecolor{clr-namespace}{RGB}{0,0,0}
\definecolor{clr-preprocessor}{RGB}{128,128,128}
\definecolor{clr-keyword}{RGB}{0,0,255}
\definecolor{clr-type}{RGB}{43,145,175}
\definecolor{clr-variable}{RGB}{0,0,0}
\definecolor{clr-constant}{RGB}{111,0,138} % macro color
\definecolor{clr-comment}{RGB}{0,128,0}

% \long\def\Gobble{
% \lstset{%
%     language=C++,
%     morekeywords=[1]{byte, text},
%     morekeywords=[2]{},
%     morekeywords=[3]{perspetive},
%     morekeywords=[4]{Context, Let, Representation},
% }
% \lstset{
% 	basicstyle=\color{black}, % any text
% 	commentstyle=\upshape,
% 	stringstyle=\color{eclipse-red},
% 	identifierstyle=\color{clr-variable}, % just about anything that isn't a directive, comment, string or known type
% 	directivestyle=\color{blue!30!black}\itshape\ttfamily, % preprocessor commands
% 	keywordstyle=[1]\color{clr-type},
% 	keywordstyle=[2]\color{clr-constant}, % you'll need to define these or use a custom language
%     commentstyle=\itshape,
%     keywordstyle=[1]\bfseries\color{eclipse-purple},
%     keywordstyle=[2]\bfseries,
%     stringstyle=\ttfamily,
%     breaklines=true,
%     columns={fullflexible},
%     texcl=true,
%     fontadjust,
%     escapechar=`,
%     mathescape,
%     extendedchars=false,
%     showstringspaces=true,
%     escapeinside={/*}{*/},
%     escapebegin={\tweak},
% }

\lstset{%
    language=C++,
    morekeywords={Constants, Types},
	basicstyle=\color{black}, % any text
	commentstyle=\color{eclipse-green},
	stringstyle=\color{eclipse-red},
	identifierstyle=\color{clr-variable}, % just about anything that isn't a directive, comment, string or known type
	directivestyle=\color{blue}\itshape, % preprocessor commands
	% listings doesn't differentiate between types and keywords (e.g. int vs return)
	% use the user types color
	keywordstyle=\color{clr-type},
	keywordstyle={[2]\color{clr-constant}}, % you'll need to define these or use a custom language
    commentstyle=\itshape,
    keywordstyle=\bfseries\color{eclipse-purple},
    stringstyle=\ttfamily,
    breaklines=false,
    columns={fullflexible},
    texcl=true,
    fontadjust,
    escapechar=`,
    mathescape,
    extendedchars=false,
    showstringspaces=true,
    escapeinside={/*}{*/},
    escapebegin={\works},
}

\begingroup
\catcode`@=\active
\catcode13=13\relax
\gdef\works{%    note the global \gdef
X
  \catcode`@=\active
  \def@##1@{%
        \noindent\textbf{\refstepcounter{segment}\arabic{segment}.~##1.\qquad}%
  }%
}%
\endgroup
%{%
%    \makeatletter
%%%    \catcode13=13\relax% Make ASCII 13 active to define it later
%    \gdef\ntweak{%
%        \lst@CalcColumn% Record the column position of slash-star
%        \xdef\slashstarposition{\the\numexpr\@tempcnta+2}
%%        \itshape% Apply some formatting here, not relevant.
%%%        \def^^M{% What should I do whenever I see ASCII 13?
%            \\% first, break the line
%%            \hbox{}\hskip\slashstarposition\lst@width% then, indent to proper column
%        }%
%    }%
%}
\begin{document}


\begin{code}
/* @The Store@ 
A distinct portion of mini-lisp implementation is devoted to frugal management  
of memory. However, other portions of the implementation need not worry about
the details of the frugality. Here we present an abstract model of a 
``store'' and the services it provides. */ 
\end{code}

\begin{code}
/* @Primitive types characterizing the machine@ 
We begin with a  definition  of the basic types that characterize our presumed 
JVM like machine i.e., byte addressable, two-complement and where words (W) are 
32 bits wide, half words (H) are 16 bits, and bytes (B) are 8 bits. */

#include <cstdint> // Standard header providing width integer types 

typedef int8_t   byte; /// Byte                         | 8 bits signed integer  | (rarely used)
typedef int16_t  H;    /// Half a word (2 bytes)        | 16 bits signed integer | (used as handle) 
typedef int32_t  W;    /// A word of 32 bits (2 halves) | 32 bits signed integer | (used as pair) 
\end{code}

\begin{code}

union Pair { // The different perspectives of a pair.
  perspective(W cons: 32)        /// I.   | A single word that can 
  perspective(H car, cdr:16)     /// II.  | A pair of car and cdr, each in a word.
  perspective(H:16; H rest:16)   /// III. | An no-data item in the pool of pairs. 
  perspective(H value, next:16)   /// IV.  | An data item in the stack used in the parser. 
};
\end{code}

\begin{code}
struct S { // An S expression represented by its handle
  H h; // the handle
  property bool atom(); 
  property bool pair();
  property bool null();  
  property variable(Pair) asPair(); /// Interpreting a handle as that of a pair, retrieves the pair behind (mutable)  
  property constant(Text) asText(); /// Interpreting the text of an atom from a handle representing an atom (immutable) 
};


Context Store {
  Provides function S make(S, S);           /// Returns (handle of) pair with given values of (handles of) its two components
  Provides function S make(Text);           /// Returns (handle of) atom with given text; 
  Provides procedure free(S);               /// Marks an S expession handle previously returned by make as no longer in use 
}
\end{code}

Context Store { 
  Provides constant(H) $M_1$;                 /// How many atoms are initially available 
  Provides constant(H) $M_2$;                 /// How many pairs are initially available 
  Provides function H available();            /// How many pairs remain available for allocation 
  Provides function H allocated();            /// How many pairs were allocated 
  Provides function H available();            /// How many chars remain available for allocation 
  Provides function H allocated();            /// How many chars were allocated 
  Provides constant(W) $m$;
}

Context Store { // Prototype for store.h
  Provides data array(char) A;
  Provides data array(pair) P;
}


Context Store { 
  Let H $M_1$ = 1 << 12;
  Let H $M_2$ = (1 << 15) - $M_1$ + 3; 
  Let W $m$ = $M_1$ + $M_2$ * sizeof (union pair);
  Let H $NIL_SIZE$ = sizeof "NIL";
}

static union  {
  char block[Store::$m$];
  struct {
   char A0[Store::$M_1$ - Store::$NIL_SIZE$];
   char A[Store::$NIL_SIZE$] = { 'N', 'I', 'L', '\\0' };
   pair P[Store::$M_2$];
  };
} memory; 


Context Store { 
  Let array(pair) P = memory.P - 1;
  Let array(pair) P0 = memory.P - 1;
  Let array(pair) P1 = memory.P + $M_2$;
  Let array(char) A  = memory.A;
  Let array(char) A0  = memory.A0;
  Let array(char) A1  = memory.A + $M_1$;
}

Context Atoms { Let H $h_0$ = Store::A0 - Store::A, $h_1$ = 0;  } 
Context Pairs { Let H $h_0$ = memory.P - Store::P, $h_1$ = Store::P1 - memory.P;  } 

Context Store {   
  Let H $h_0$ = min(Atoms::$h_0$, Pairs::$h_0$), $h_1$ = max(Atoms::$h_1$, Pairs::$h_1$);
}

Context Pairs { Let H $n$ = $h_1$ - $h_0$ + 1; };
Context Atoms { Let H $n$ = $h_1$ - $h_0$ + 1; }; 
Context Store { Let H $n$ = $h_1$ - $h_0$ + 1; };

Context Store {
  H mark(H h)   { return h + (1 << 15); } 
  H marked(H h) { return h < Atoms::$h_0$ || h > Atoms::$h_1$; } 
}

/* @Static memory@ 
All memory allocations are from a pre-allocated fixed 
contigious memory block, the \emph{pool} of $n$ bytes comprising:
\begin{enumerate} 
\item \emph{Pool of atoms}. A sub-block $N_1$ \emph{bytes}, from which 
internal representations of atoms are allocated. 
\item \emph{Pool of pairs}. A sub-block $n_2$ \emph{words},
from which internal representations of pairs are allocated. 
\end{enumerate}
*/ 
Constants {
    $N_1$ = 1<<13,
    $N_2$ = 1<<15 -1,    
    $N$ = $N_1$*sizeof B + $N_2$*sizeof W,    
}

static B M[N]; 
\end{code}


\end{document}
