\documentclass[fleqn]{beamer}
\usepackage[english]{babel}

\usepackage{amsmath,amssymb}
\usepackage{graphicx}

% vertical separator macro
\newcommand{\vsep}{
  \column{0.0\textwidth}
    \begin{tikzpicture}
      \draw[very thick,black!10] (0,0) -- (0,7.3);
    \end{tikzpicture}
}

% More space between lines in align
\setlength{\mathindent}{0pt}

% Beamer theme
\usetheme{AnnArbor}
\usefonttheme[onlysmall]{structurebold}
\mode<presentation>
\setbeamercovered{transparent=10}

% align spacing
\setlength{\jot}{0pt}

%\setbeamertemplate{navigation symbols}{}%remove navigation symbols

\title{Mini-Lisp}
\author{Yossi Gil}
\institute[Department of Computer Science, The Technion]{CS236703--Programming Languages}
\date{\today}

\begin{document}
\begin{frame}
  \titlepage
\end{frame}


\begin{frame}{Lisp?}
\begin{block}{
Philip Greenspun's Tenth\footnote{there are  no other rules, just the tenth one} Rule of Programming
}
"Any sufficiently complicated C or Fortran program contains an ad-hoc, informally-specified bug-ridden slow implementation of half of Common Lisp.“
\end{block}
* Note:  

  \begin{columns}[T]
    \column{0.5\textwidth}
    Left column
    \begin{itemize}
      \item Item 1
      \item Item 2
      \item Item 3
    \end{itemize}
    \column{0.5\textwidth}
    \begin{block}{Right Column}
      this is a block
    \end{block}
  \end{columns}
\end{frame}
\begin{frame}
  \frametitle{Slide 2}
  erster Satz\\
  zweiter Satz

  \pause

  dritter Satz
\end{frame}
\begin{frame}
  \frametitle{Slide 3}
  \uncover<1>{erster Satz}\\
  \uncover<3>{zweiter Satz}\\
  \uncover<2>{dritter Satz}
\end{frame}
\end{document}