§ מזהים לעומת מילולונים
      בכדי להעמיק בהבנת ההבדל בין מזהה לבין מילולון, נעיין בחידה הבאה מאת ר' אברהם
      אבן עזרא (הראב'ע, 1089–1164): תחילת שמי כמו אמצע שמי, אמצע שמי כמו תחילת שמי,
      סוף שמי כמו מחצית סוף שמי. מהו שמי? בקריאה ראשונה, חידה זו נראית כפרדוקסלית.
      ראשית, תחילת השם המבוקש זהה לאמצעו, ומכאן, נובע שאמצע השם זהה לתחילתו, ויש
      לתמוה על כן מדוע ראה בעל החידה להוסיף פרט מיותר זה. אך קשה מכך היא הקושיה כיצד
      סוף השם יכול להיות שווה למחצית סוף השם? הלא הדבר יתכן רק אם סוף השם הוא המספר~0?

      המיסתורין יוסר בכתיבה של החידה מחדש תןך אבחנה בין מילולון ומזהה. תחילת
      שמי כמו אמצע "שמי", אמצע שמי כמו תחילת "שמי", סוף שמי כמו מחצית סוף
      "שמי". מהו שמי?  האבחנה הזו מאפשרת לנו להבין כי המילה שמי מציינת את שמו
      של מי אשר חד את החידה, ואילו "שמי" מציין את עצמו, כלומר את הסדרית "שמי"
      עצמה. מכאן ברור כי התשובה לשאלה היא השם "משה".†{הניסוח המקורי של הראב"ע
      קל יותר לפתרון: כי תרצה לדעת שמי // קח אמצע שמי והוא ראש שמי // קח ראש
    שמי והוא אמצע שמי // קח סוף שמי וחלקהו לשניים // ומצאת את שמי. ולמתקדמים
  הנה חידה נוספת לנעוץ בה את שיניהם: אמור מה שם צבי נכבד // אשר חציו חצי חציו
// ועוד בשמו חצי נוסף // ורואיו יאמרו איו?}

      ומי אשר הראב"ע מרובע מדי לטעמו, ידרש לדבר מנחם המשעמם, לאמור: בתאוריה יש
      ארבע אותיות אהו"י, אבל במציאות יש רק שלוש כדי להבין זאת, נוסיף מרכאות,
      המציינות מילולונים. כלומר נכתוב "תאוריה" כדי לציין את הסדרית בת שש התוים,
      ונכתוב- תאוריה- כאשר הכוונה למזהה של מושג מופשט.  ב"תאוריה" יש ארבע
      אותיות 'א' 'ה' 'ו' 'י', אבל ב"מציאות" יש רק שלוש†{}

וכיצד כל זאות קשור לנושא?

בפרק זה, נעסוק במופעים של פרדוקסים אלגנטיים דומים בתחום של שפות התכנות. בדיוק
    כמו הביצה והתרנגולת, הפרדוקסליות הטמונה בפרדוקסים אלו היא קלה ביותר להבנה.
נדגים זאת במספר היתרים שונים של \מונח{פרדוקס הביצה והתרנגולת}, חוק השבות, הביטוי
    הארכאי "צבת בצבת עשויה", ועוד.

לעיתים יהיו לפרדוקסים היתרם אלגנטיים, קצרים ופשוטים (אך לא נכונים בעליל). אך
    במרבית המקרים, טיבם של פרדוקסים הוא זה שהיתרם בא רק לאחר חקירה עמקנית, שהיא
    פחות אלגנטית מהפרדוקס עצמו.

§ הביצה והתרנגולת
גם מי שאינו אורניתולוג (וגם מי שאינו יודע מה ומי הוא אורניתולוג(יודע כי כל
תרנגולת היא אפרוח שהתפתח לאחר שבקע מביצה, וכל ביצה הוטלה על ידי תרנגולת קודמת.
\מונח[פרדוקס הביצה והתרנגולת]{פרדוקס הביצה והתרנגולת} המיוחס
לאריסטו\הערת␣שוליים{384-322 לפנה"ס, יליד סטאגירה, מגדולי הפילוסופים ביוון
העתיקה, ובנו של ניקומכוס, רופא החצר של מלך מוקדון.}, שואל מה אם הביצה קדמה
לתרנגולת, או שמא, התרנגולת קדמה לביצה.
\begin{itemize}

• הטענה שהתרנגולת קדמה לביצה לא יכולה להיות נכונה, כי התרנגולת הראשונה הייתה
אפרוח שבקע מביצה. • אם תמצי לומר שהביצה קדמה לתרנגולת, מיד יטען כנגדך שהביצה
הזו הוטלה על ידי תרנגולת.

ההנחה הסמויה בפרדוקס זה היא ששרשרת הזמן, כשהיא נמתחת לאחור, היא סופית, כלומר
שגילו של היקום הוא סופי.

\end{itemize}
נדגים זאת במספר היתרים שונים של \מונח{פרדוקס הביצה והתרנגולת}.
\begin{enumerate}
• \גיבור{היתר תנ"כי:} התרנגולת, עם שאר בעלי הכנף, נבראה במאמרו של אלוהים ביום
  החמישי לבריאה ככתוב, "וַיִּבְרָא אֱלֹהִים אֶת-הַתַּנִּינִם הַגְּדֹלִים; וְאֵת כָּל-נֶפֶשׁ הַחַיָּה הָרֹמֶשֶׂת אֲשֶׁר
  שָׁרְצוּ הַמַּיִם לְמִינֵהֶם וְאֵת כָּל-עוֹף כָּנָף לְמִינֵהוּ וַיַּרְא אֱלֹהִים כִּי-טוֹב.", בראשית א' כ"א.).
    עיון מפורט בתיאור הבריאה, אינו מותיר ספק: התרנגולת, נבראה ראשונה בין שאר
    העופות שהלא נאמר "וַיְבָרֶךְ אֹתָם אֱלֹהִים, לֵאמֹר פְּרוּ וּרְבוּ וּמִלְאוּ אֶת-הַמַּיִם בַּיַּמִּים וְהָעוֹף
    יִרֶב בָּאָרֶץ. ".

• \גיבור{היתר יווני:} \מונח[אלוהים]{אלוהים} ברא הן את התרנגולת והן את הביצה. אף
    זה הינו היתר דתי לפרדוקס, וריאציה להיתר הראשון. בגרסא זו, נבראו ע"י האל
    התרנגולות והביצים, בבת אחת. פתרון זה עוקף את ההנחה שמקורן של תרנגולות בהכרח
    מביצים.

• \גיבור{היתר אבולוציוני:} לדידם של אלו המאמינים בתורת האבולוציה, אין הכרח שכל
    ביצה ממנה נוצרה תרנגולת היא של תרנגולת. למעשה, מדובר בהנחה פשטנית אשר אינה
    מתארת נכונה את היווצרות החיים. כל מין התפתח ממין קודם וקדום על ידי שורה של
    מוטציות גנטיות אקראיות שהתרחשו במרוצת השנים. באמצעות מנגנונים של מוטציות
    \הערת␣שוליים{למעשה, אין זה מדויק. לקורא המעוניין בהעשרה ביולוגית, נרחיב כי
        מנגנון בשם "רקומבינציה" (שאינו נחשב ל"מוטציה") יוצר מגוון ביולוגי
        בין-דורי אף באדם, אשר באמצעותו, באפיק נפרד להורשה "רגילה", מתקבל מגוון
        גנטי במין מסוים. לפרטים נוספים.} וברירה טבעית, התפתחו לאורך השנים,
  באופן הדרגתי, יצורים שהם התרנגולות אותם אנחנו מכירים היום. הנ"ל-סותר במידת מה
  את הגדרת הפרדוקס גם כן, וגם לו וריאנטים רבים. באופן כללי, לפי רובן התרנגולות
  התפתחו מאב קדמון כלשהו בתהליך הדרגתי, כך שלשאלת "מי קדם למי" לא נותרת משמעות
  רבה.

• \גיבור{ההיתר ה"נכון":} התרנגולת והביצה- חד המה. התהליך האבולוציוני שאחראי
ליצירת תרנגולות כולל שתי שרשראות שונות, אך מצומדות. השרשראות הללו מתחילות בתא
הראשון בו היו חיים, כאשר השרשרת הראשונה היא שהשפיעה על מהלך חייו של יצור
התרנגולת-ביצה, כלומר על מהלך חייו כצעיר, ואילו השרשרת השנייה השפיעה על מהלך
חייו כבוגר.

לשם הבנת היתר זה נעיין בניסוי המחשבתי של ריצ'רד דוקינס \הערת␣שוליים{ביולוג
אבולוציוני בריטי, הידוע בשל ספרי המדע הפופולרי שחיבר. יש יטענו שהוא סטיבן
הוקינג בגרסתו הביולוגית.} לבירור מקורו של האדם. יקח הקורא תמונה של עצמו, ויניח
מעליה תמונה של אביו. מעליה יניח הקרוא תמונה של אבי-אביו, וימשיך כך הלאה עד
לראשית הזמן.

בראש הערימה ימצא היצור החי הראשון, מין חד תא, או וירוס, או היווצרות היולית
ראשונה של החיים. במעבר על הערימה, נגלה שכל תמונה דומה מאור לזו שהונחה עליה. אך
אין הדבר מוביל למסקנה שכל התמונות דומות. יחס הדימיון אינו טרנזיטיבי, וודאי
שהתמונה הראשונה אינה דומה כלל לתמונתו של הקורא.

כעת, נכליל את הניסוי של דוקינס באופן הבא. במקום להניח תמונות בודדות בערימה,
נניח על הערימה סרטונים. נתחיל בסרטון המתאר בתמצית את ההתפתחות הגופנית האישית של
הקורא, מהעת שבה היה ביצית מופרית, ועד ליום מותו. על סרטון זה נניח סרטון דומה של
אביו, וכן הלאה.

שני מונחים שונים טבעה הביולוגיה בהקשר זה. ה\גיבור{פילוגנזה} מתארת את התפתחות
המין הביולוגי, והיא זו שמבארת את מאמרו של דרווין, לפיו מוצאו של האדם מהקוף.
לעומתה, ה\גיבור{אונתוגנזה} אינה נוגעת למין הביולוגי, כי אם לפרט מסויים ממין זה,
והיא זו המתארת את מהלך ההתפתחות הגופנית של הפרט. האונתוגנזה של הקורא, של אביו
ושל אביו מתוארת בסרטונים. הפילוגנזה של המין האנושי, מתוארת על ידי ערימת
הסרטונים. \הערת␣שוליים{למונחים הללו יש משמעות בתחום שפות התכנות. בלמדנו את מבנה
העצם השייך למחלקה, נגלה שהאונתוגנזה של העצם, עוקבת אחרי הפילוגנזה של המחלקה.
כלומר, אם מחלקה יורשת ממחלקה אחרת, הרי מהלך הבניה של עצם מהמחלקה, יעקוב אחרי
מהלך הירושה של המחלקה.}

בערימת הסרטונים הזו, כל סרטון דומה לזה שמעליו ולזה שמתחתיו. בראש הערימה, נמצא
את הסרטון המתאר את מהלך חייו של היצור החי הראשון, ובתחתיתה, את הסרטון המתאר את
הפיכתו של הקורא עצמו מילד לגבר. \הערת␣שוליים{הסליחה עם הקוראות שמלל זה מעדיף
שלא לדמותן לתרנגולות. לעומת זאת, השוואת הקורא הזכר לתרנגול נכונה וראויה היא,
שכן פשר המילה "גבר" בשפה העברית הוא גם זכר אנושי בוגר, וגם תרנגול.}

מי קדם למי? הילד לגבר, או שמא הגבר לילד? ערימת הסרטונים הזו נותנת לנו את
התשובה. הילד והגבר הם שלבים שונים בהתפתחות של פרט, מן היצור החי הראשון, דרך
הסבא רבא של הקורא, ועד לקורא עצמו. ואף כך הביצה והתרנגולת. הם שלבים שונים במהלך
ההתפתחות של יצור חי, נקודות זמן שונות על פני סרטון האונתוגנזה של הקורא ושל כל
אבות אבותיו.

על פי היתר זה, הביצה והתרנגולת הם שלבים שונים בהתפתחותו של יצור חי, והם נוצרו
במהלך האיבולוציה כשינויים בסרטוני האונתוגנזה. בדיקת ביצה הדינוזאורים מעלה השערה
שהשלב הביצתי נוצר קודם לשלב התרנגולתי שבסרטון. אך אין צורך להסתמך על כך. הביצה
והתרנגולת נוצרו במהלך הפילוגנזה של האונתוגנזה.

נשים לב כי הוכחה זו, וכל דרך החשיבה הזו הנוגעת להיתר זה, היא למעשה \מונח
{רדוקציה} של אופן יצירת החיים. הסתכלנו על שרשרת בעלי חיים המקושרת ביחסי "התפתח
מ-", שמתחילה בנו ונגמרת בתא החי הראשון. הבעיה של יצירת החיים נחשבת פרדוקסלית
הרבה פחות, וגם לה היתרים רבים (דתיים, ביולוגיים וכו'), אך הם נחשבים בעיני רבים
"מוזרים" הרבה פחות, שכן יש לכל היתר שכזה תימוכין וראיות. \end{enumerate}
ראינו אם כן פרדוקס משובב נפש. פרדוקס שקל מאוד להבין אותו, קל להתירו באמצעים מטאפיזיים.

§ המקלדת והמחשב

כמקרה הביצה והתרנגולת,כך מקרה המקלדת והמחשב, וכך תמיד בתהליכים אבולוציוניים, גם
של עופות וגם של מכונות: הקפיצה הנחשונית, הבלתי ניתנת לתפישה, היא אשליה בלבד.
המציאות תמיד מורכבת משורה של צעדים קטנים והגיונייםכמקרה הביצה והתרנגולת כך מקרה
המקלדת והמחשב, וכך תמיד בתהליכים אבולוציוניים, גם של עופות וגם של מכונות:
הקפיצה הנחשונית, הבלתי ניתנת לתפישה, היא אשליה בלבד. המציאות תמיד מורכבת משורה
של צעדים קטנים והגיוניים…
§ צבת בצבת עשויה
חז"ל-הבחינו בבעיה דומה של רקורסיה אינסופית מעין זו בסוגיה התלמודית הידועה בשם
    "צבת בצבת עשויה". בגמרא, במסכת פסחים דף נ"ד עמוד א' נאמר:
\begin{quote}
צבתא בצבתא מתעבדא וצבתא קמייתא מאן עבד הא לאי בריה בידי שמים
\end{quote}

§ חוק השבות
סעיף 4ב' לחוק השבות, תש"י- 1950 קובע:
\צטט\ע|לענין חוק זה, "יהודי"- מי שנולד לאם יהודיה או שנתגייר, והוא אינו בן
דת אחרת|.
===
בבואו להגדיר את המילה "יהודי", חוק השבות משתמש בגוף ההגדרה במילה זו עצמה.
הגדרות המשתמשות במונח המוגדר כחלק מההגדרה של המונח עצמו, נקראות הגדרות
רקורסיביות.

הגדרות רקורסיביות מופיעות גם בדתות אחרות: ע"פ השריעה (ההלכה המוסלמית), מוסלמי
הוא מי שנולד לאב מוסלמי או שהפך למוסלמי באמצעות אמירת העדות, הלא היא השהאדה:
\begin{quote}
\begin{Arabic}
  \ע|اشهد ان لَا إِلٰهَ إِلَّا الله وان مُحَمَّدا رَسُولُ الله|
\end{Arabic}
\end{quote}
(אני מעיד כי אין אלוהים לבד מאללה, וכי מוחמד הוא שליח אללה). בפני שלושה
מוסלמים. אנו רואים כי גם ההלכה המוסלמית מגדירה רקורסיבית את התשובה לשאלה "מיהו
מוסלמי?"

נוכל לזהות בשתי ההגדרות את  שלושת המרכיבים של הגדרה רקורסיבית: האיברים
האטומיים, הבנאים, וכלל השלמות: 
\begin{itemize}
  ✦ \ע|איברים אטומיים|. בסיס הרקורסיה הוא \ע|איברים אטומיים|, כלומר איברים
  של~$S$ אשר אינם נבנים מאיברים אחרים בקבוצה:
  \begin{enumerate}
  ✦ אם נביט על סעיף 4ב' של חוק השבות כעל הגדרה רקורסיבית של קבוצת היהודים, סביר
  שנאמר שאברהם אבינו ושרה אימנו הם האיברים האטומיים של הקבוצה, כלומר הם יהודים
  בזכות עצמם.
  ✦ בהסתכלות דומה על השריעה, סביר להסיק מוחמד ואולי עוד כמה מתלמידיו, הם
  מוסלמים מכוח עצמם בלבד. כל שאר המוסלמים נקבעים בדרך אחרת.
  \end{enumerate}
  ✦ \ע|בנאי איברים|. בהגדרה השרעית הרקורסיבית של קבוצות
  המוסלמים יש שני בנאים:
  \begin{enumerate}
  ✦ הבנאי שמאפשר לקבוע כי אדם מסויים הוא מוסלמי, אם אביו מוסלמי. בנאי זה הוא בנאי
  אונארי, משום שבנאי זה מתחיל מאיבר יחיד בקבוצה (גבר שהוא מוסלמי), ומאפשר
  "לבנות" איבר חדש מהאיבר הקיים.
  ✦ הכלל המגדיר כמוסלמי כמי שאמר את השהאדה בפני שלושה מוסלמים אחרים, הוא בנאי
  טרנארי משום שבנאי זה מתסמך על שלושה איברים בקבוצה המוגדרת רקורסיבית (הלא היא
  קבוצת המוסלמים), כדי לבנות איבר חדש בקבוצה.
  \end{enumerate}
  גם חוק השבות מגדיר בנאים
  \begin{enumerate}
  ✦ בנאי אונארי. 
  אונארי (אמהות). 
  ✦ בנאי טרנרי (ככל הנראה).  החוק אמנם אינו מגדיר
  במדוייק מהו גיור, אך ברור כי הגדרה מדוייקת של הגיור, תכלול רקורסיה באמצעות בנאי
  איברים ובפרט, ידרש כי חברי בית הדין המחליט על הגיור יהיו יהודים בעצמם.
  \end{enumerate}

 \item ע|שלמות ההגדרה|. כזכור, הגדרה רקורסיבית של הקבוצה~$S$ כוללת תמיד בתוכה מרכיב
  הדורש שאין ב-$S$ איברים אחרים מלבד האיברים האטומיים ואלו שנוצרו באמצעות
  בנאים. בדרך הדרישה שבמרכיב זה של אינה נאמרת במפורש, אלא משתמעת מהניסוח. כך
  למשל מניסוח חוק השבות, ברור כי ההגדרה מתכוונת לאמר שמי שאינו מקיים את התנאים
  המנויים בסעיף, אינו יהודי. אך הקביעה כי כל מי שאמו אינו יהודיה ושלא התגייר
  איננו יהודי, אינה מופיעה בחוק כלשונה אלא משתמעת ממנו.
\end{itemize}
\endinput
§ הצבת הראשונה
ובתרגום לעברית: "צבתא בצבתא מתעבדא" פירושו, "הצבת אינה נעשית אלא בצבת אחרת.",
או (כמאמר המכתם) "צבת בצבת עשויה". כלומר, כדי לייצר צבת, שהיא כלי עשוי ברזל
המשמש לאחיזת מטילי ברזל בעת ליבונם באש ועיצובם, יש ללבן באש מטיל ברזל, ואחר כך
לעצבו לצורת צבת. אבל, לשם כך, יש להשתמש בצבת שהייתה קודמת לה.

כיצד אם כן נוצרה הצבת הראשונה, ומי עשה אותה? ("וצבתא קמייתא מאן עבד?") יתורגם
למי "עשה את הצבת הראשונה?". הפתרון המוצע על ידי הגמרא הוא בקביעה שהצבת הראשונה
נבראה על ידי אלוהים. מסכת אבות שבמשנה מפרטת את נסיבות הבניה של הצבת הראשונה: יש
האומרים כי צבת ראשונה זו נעשתה על ידי אלוהים עם גמר בריאת העולם. ביום שישי
הראשון, בזמן "בין השמשות", שעה שכל הדברים האחרים כבר נבראו, בעת של בין קודש לבין
חול, בה הבורא התכונן לנפוש ממלאכתו בשבת, הוא פינה הבורא לטפל בענינים
של הרגע האחרון, ולברוא עשרה מוצרים או יותר שיש בהם משום קסם או נס: למשל, באר
נודדת שתלווה את בני ישראל במדבר, כאלפיים וחמש מאות שנה לאחר מכן.

בכלל מוצרים אלו נכללה, יש האומרים, הצבת הראשונה.

פתרון ניסי שכזה אינו בא בחשבון עבור שפות תכנות. מסכת פסחים מציגה גם דרך אחרת
שבה יוצרה הצבת הראשונה על ידי יציקה לתוך דפוס נחושת. במונחים של שפות תכנות, דרך
יצור כזו מתאימה לכתיבה בשפת מכונה של מהדר בעבור שפת תכנות חדשה.

§ המספר השלם החיובי הקטן ביותר שאי אפשר לתאר בתריסר מילים או פחות

המשפט "הַטִּבְעִי הַקָּטָן בְּיוֹתֵר מֵאֵלּו שֶׁאֵינָם גְּדִירִים בְּפָחוֹת מִתְּרֵיסָר מִלִּים" הוא מופע של
{\([^{}]*\)}/פרדוקס של ברי. לִכְאוֹרָה, ברור שקיימים המונים של טבעיים מהסוג הנדון במשפט, שכן
מספר הטבעיים גדול בהרבה ממספר ההגדרות. יש הלא אינסוף טבעיים ולעומתם, מספר
ההגדרות האפשריות הוא סופי, על אף היותו עצום ורב\הערת␣שוליים{ואף מספר זה הוא
אפסי לאין שיעור לעומת המספר של גרהאם, מספר טבעי שהוא כה גדול, שהיקום כולו אינו
מספיק לכתיבתו, אפילו לא בצורת מגדל חזקות, ואפילו אם נניח שכל ספרה תופסת את נפח
פלאנק בלבד. בכל זאת, את המספר של גרהאם של ניתן להגדיר בפחות מתריסר מילים:
"המספר הגדול ביותר ששימש אי פעם בהוכחה מתימטית עד שנת 1980". }. שהרי, יגדל
מספר המילים בשפה העברית ככל שיגדל, סופי הוא יוותר עדיין. ממילא, מספר הצירופים
בני פחות מתריסר מילים בשפה העברית, לא יוכל להיות אינסופי.
נסמן על כן בְּ-\שי{X} את הקבוצה האינסופית של המספרים שאינם גדירים. הרי ודאי שימצא
ב-\שי{X} אחד, \שי{x}, הקטן שבחבריו, כלומר כזה \שי{x} הנקבע על ידי: \[
 ∃
x∈X\mbox{}∀ x'∈X\mbox{}x'≠x\mbox{}⇒ x'>x
\] אלא, וכאן נגלה הפרדוקס שבדבר, שכן המשפט שהובא בפתיח נצרך לתשע של מילים בלבד
      כדי להגדיר את \שי{x} זה עצמו, ועל כן על כרחך נמצאת אומר~$x∉ X$.

ההיתר של הפרדוקס נגלה מהבנת השוני בין הגדרה מתימטית "רגילה" ובין שפה טבעית.
הפרדוקס שבמשפט, אשר כתוב בשפה טבעית, נובעת מכך שהמשפט מדבר על משפטים אחרים בשפה
הטבעית. בשפה טבעית קיימת אפשרות לנסח בשפה עצמה משפטים הנוגעים למשפטים אחרים
בשפה עצמה. עצימת העין נוכח הקשיים שבמגבלות של התייחסות עצמית, היא זו שמביאה
לסתירה. בניסוח המשפט האמור או כל גִּרְסָה אחרת שלו\הערת␣שוליים{ ובפרט, המשפט
  שמופיע בכותרת סעיף זה (ואגב, הערת שוליים זו, אף היא דוגמא לטקסט בשפה טבעית
  המתייחס לעצמו.)
}, טמונה הנחה סמויה לפיה קיימת משמעות מדוייקת למונחים "גדיר", "בר-הגדרה" או "ניתן לתיאור".

נכון, נדמה כאילו אנחנו יודעים לזהות הגדרה מדוייקת כשזו מופיעה לפנינו, ולדחות
מלפנינו כל הגדרה ערטילאית או לא מדוייקת. אך מתברר כי אי אפשר לעשות זאת בהיסח
הדעת במבנים שכליים כגון תיאוריה מתימטית או שפה טבעית, שהם מורכבים מספיק כדי
להכיל "התייחסות עצמית". "התייחסות עצמית" זו היא שעומדת מאחורי הפרדוקס של ברי,
כמו גם הפרדוקס של ראסל \גיבור{("תהי~$U$ הקבוצה של כל הקבוצות שאינן מכילות את
עצמן, האם~$U∈U$?")} ופרדוקסים אחרים\הערת␣שוליים{ובהם פרדוקס הגלב, הפרדוקס של
בהארטירהארי, והפרדוקס של גרלינג-נלסון, הדן בלוגיקות שאינן עוסקות בעצמן. }.

בפרט, המשפט שבפתיח הוא הגדרה העוסקת בהגדרות. כיוון שמשפט זה הוא הגדרה בעצמו,
הוא עוסק בעצמו. אמרנו שהפרדוקס שבמשפט נובע מההנחה הסמוייה שהמונח "גדיר", הוא
אכן מוגדר היטב. ציינו, שההטעייה המבריקה שבמשפט היא שאכן, בדרך כלל, קל להבחין
בין משהו שניתן להגדרה, ומשהו שאינו ניתן להגדרה. הפרדקוס יופרך משנשים לב לכך
שההבחנה הקלה בין הגדרה ובין "לא הגדרה" נכשלת כשהיא נדרשת להגדרות, כמו המשפט
שבפתיח, שמתייחסות לעצמן. הפרדוקס של ברי מוכיח מגבלה מהותית של שפה
אנושית\הערת␣שוליים{המונח שפה אנושית כולל גם שפות טבעיות מהאטרוסקית אל שְׂפַת
הפקצות, ושפות מלאכותיות, מהאספרנטו ועד הננדורינית שדיברו אותה בני לילית שבחבל
ננדור שבארץ התיכונה. }. בפרט, ניתן (במאמץ הגדרתי לא מבוטל) לנסח משפט מתימטי
שמשמעותו היא הבאה

\begin{mybox}
תהי H שפה אנושית שהיא מורכבת מספיק כדי לאפשר:
\begin{itemize}
• דיון על השפה בתוך השפה עצמה,
• דיון על קבוצות לא חסומות בגודלן.
\end{itemize}
הרי בתוך השפה H ישנם משפטים שלא ניתן לתת להם משמעות עקבית עם שאר מרכיבי השפה.
\end{mybox}

נדגיש שתי אלו:
\begin{itemize}
• אי אפשר ליצור את הפרדוקס בלא שהשפה H תהיה מסוגלת להכיל היגדים בדבר "קבוצת כל
מי שאפשר לתאר ב-H באמצעות תריסר מילים". הפרדוקס נשען על היגד זה.

• לעומת זאת, אין צורך לדרוש שהשפה H תהיה מורכבת דיה כדי לדון במספרים או
במתימטיקה. כל שנדרש הוא האפשרות לדון בקבוצות לא חסומות. ניתן לנסח את הפרדוקס של
ברי כך שיסוב על סדריות, ואפשר גם לנסות לכתוב: \גיבור{"הַפְּרִיט הָרִאשׁוֹן שֶׁיִּמָּכֵר
בְּ-\שי{EBAY} מִבֵּין אֵלּוּ שֶׁתֵּאוּרָם נִדְרָשׁ לְיוֹתֵר מִתְּרֵיסָר מִלִּים"}. ניסוח זה מדגים היטב את
הדרישה לדיון בקבוצה בלתי חסומה. אין במשפט משום כל, אם קבוצת הפריטים שעשויה
להימכר ב-\שי{EBAY} היא קטנה ממספר התיאורים בני תריסר מילים או פחות. לעומת זאת,
הפרדוקס נוצר אם מספר הפריטים שיכול להמכר ב-\שי{EBAY} הוא בלתי חסום, ולכן גדול
ממספר התיאורים הללו. קורט גדל הכיר בעובדה שמשפט מתימטי מוביל לפרדוקס אם המשפט
טוען שהוא עצמו אינו נכון. אבל הוא גם הבחין בכך שאין פרדוקס במשפט הטוען שהוא
עצמו אינו יכיח. משפט אי השלמות הראשון של גדל הוא ניסוח מתימטי מדוייק של הטענה
לעיל בדבר שפה טבעית.

\end{itemize}

\section{משפט גדל}
קורט גדל הכיר בעובדה שמשפט מתימטי מוביל לפרדוקס אם המשפט טוען שהוא עצמו אינו
נכון. אבל הוא גם הבחין בכך שאין פרדוקס במשפט הטוען שהוא עצמו אינו יכיח. משפט אי
השלמות הראשון של גדל הוא ניסוח מתימטי מדוייק של הטענה לעיל בדבר שפה טבעית.

\begin{mybox}
בכל תיאוריה מתימטית~$T$ אשר מקיימת את התנאים הבאים:
\begin{itemize}
• העדר סתירות
• מורכבת מספיק כדי להכיל את האקסיומות של המספרים הטבעיים.
\end{itemize}
קיים משפט~$T$, כך ש-$ T∈T$, הוא נכון, אך לא יכיח בתיאוריה~$T$.
\end{mybox}
האינטואיציה של הוכחת משפט אי השלמות הראשון של גדל מתחילה בדיון בפרדוקס של ברי.
אלא שבתיאוריות המתימטיות של גדל, קל למשפטים שמדברים על עצמם. הסיבה היא שבמערכת
מתימטית שהיא מספיק מורכבת כדי להכיל את המספרים הטבעיים מצטיינת בתכונות הבאה:

\begin{itemize}
• כל משפט, הוכחה או טענה, ניתנים לקידוד כמספר טבעי בודד. (קידוד זה נקרא קידוד
  גדל, ואפשר לבנות אותו למשל באמצעות התרגום של כל משפט לכתיב ה-ASCII).
• כל משפט שעוסק במספרים, עוסק לפיכך גם במשפטים, וגם בהוכחות של משפטים.
\end{itemize}
 המשפט~$T$ נבנה באופן הבא: נסתכל על כל ההוכחות האפשריות בתיאוריה~$T$, נקודד
 כל הוכחה כזו כמספר טבעי, ואחר נבנה מספר טבעי אחר, מספר גדל של התיאוריה~$T$,
 שהוא גם נכון כמשפט, וגם לא יכיח בתיאוריה~$T$. מספר גדל של תיאוריה מסוימת,
 הוא קידוד של משפט נכון אך לא יכיח בתיאוריה. הבניה של מספר גדל היא מפורשת. היא
 בונה מספר כזה המקודד משפט שהוכחתו שונה מזו מכל ההוכחות הקיימות בשפה. ואינה
 נובעת משיקולי ספירה.

 במילים אחרות, משפט אי השלמות של גדל אומר שתיאוריה מתימטית אם היא לא טריביאלית,
 היא לא שלמה.

§ הסכנה שבתכניות המשעשעות המדפיסות את עצמן

מהי תכנית המדפיסה את עצמה? ובכן, הגדרה נאיבית תהיה תכנית אשר הפלט שלה היא היא
עצמה, אך נחדד, שהרי גם התכנית הריקה תענה על הגדרה זו, ולא נרצה פתרון פשטני כל
כך שייחשב פתרון תקף לבעיה אלגנטית זו. לא נרצה גם שתכנית שמקבלת את עצמה כקלט
טקסטואלי ומדפיסה טקסט זה תיחשב, שכן מדובר בפתרון טכני בלבד.

לפיכך, ההגדרה היא כדלקמן: תכנית המדפיסה את עצמה, \מונח{דפסן}, היא תכנית לא ריקה
אשר לא מקבלת קלט והפלט היחידי שלה הוא התכנית עצמה (כך נימנע גם מלכלול תכניות
אשר מדפיסות לאורך הזמן את כל הפלטים האפשריים, בזה אחר זה, עד אשר תדפיס בעת
מסוים את עצמה). תכניות מעין אלו מהוות אבן שואבת במדעי המחשב, וניתן להתייחס
אליהן בצורה מתמטית באופן הבא: אם נתייחס לסביבת הביצוע כאל פונקציה (מקבוצת
התכניות אל קבוצת הפלטים), נקבל כי תכנית המדפיסה את עצמה היא נקודת
שבת\הערת␣שוליים{\שי{Fixed Point}. נציין כי למונח הרחבות רבות בתחום הטופולוגיה
המתמטית, המכלילות את המונח למרחבים מטריים שונים. לפרטים נוספים.}.

נראה מספר דוגמאות\הערת␣שוליים{ראוי להדגיש נקודה חשובה זו.} לתכניות המדפיסות את עצמן:

\lstinputlisting[language=Java]{sources/quine.java}
\lstinputlisting[language=Perl]{sources/quine.perl}
\lstinputlisting[language=Python]{sources/quine.py}

ולהלן דוגמה ב-שיא חד, המלווה בהסבר ובהרחבה.

מהו הטריק\הערת␣שוליים{סיבה נוספת להתעניינות בנושא זה היא הופעתו כשאלה בשיעורי הבית, בסמסטר בו נכתב סיכום זה.} שבזכותו עובדות תכניות אלו? ניתן לחלק באופן גס את הקוד ל-2 חלקים:
\begin{enumerate}
• מערך מחרוזות ו/או אוסף קבועים, אשר מכילים את קוד הביצוע של התכנית.
• קוד הביצוע של התכנית, אשר מכיל הוראות להדפסה פעמיים של מערך המחרוזות המדובר, וכן קבועים נוספים הדרושים לשכפול מדויק של הקוד לתוך מה שיודפס.
\end{enumerate}
בריצת התכנית יודפס, כאמור, פעמיים המערך והתווים הרלוונטיים- בפעם הראשונה עבור הדפסת החלק הראשון של התכנית (מערך המחרוזות), ובפעם השנייה עבור הדפסת החלק השני- קוד הביצוע ממש.

עם זאת, טמונה בקודים מעין אלו סכנה של ממש. תכניות כאלו עשויות להוות כלי להחדרה של \מונח{וירוסים}\הערת␣שוליים{למעשה, מינוח מדויק יותר הוא סוס טרויאני, מונח שעל משמעותו ניתן וראוי לדון רבות.} בידי זֵדִים\הערת␣שוליים{הסבר מלא למילה זֵד שֵם ז: בלשון המקרא אדם רע, רשע; "טָפְלוּ עָלַי שֶׁקֶר זֵדִים" (תהלים קיט סט). [מילון רב-מילים]}. כתב על כך \מונח{קן תומפסון}\הערת␣שוליים{אבי !UNIX חלוץ אמריקאי בתחום מדעי המחשב, ידוע בשל תרומתו לפיתוח שפות התכנות B,Go והגדרת UTF-8} במאמרו\הערת␣שוליים {להבנה מלאה יותר של אופן הפעולה של סוס טרויאני, ניתן ללחוץ כאן.
 }.

נתחיל בתיאור מנגנון רלוונטי. נניח שנתון לנו קובץ המקור של המהדר של שפת \סי,
ונרצה להכניס בו שינוי מסוים- נרצה שבהינתן תו הבקרה \קד{/v}, תודפס מפלצת
\קד{ASCII} הזו. כיצד נעשה זאת? ובכן, בהינתן קוד המקור האידיאלי הבא של המהדר,
נגלה כי אין זו משימה קשה במיוחד:

מדובר בקוד המקבל תווים בשפה, ומחזיר את התו הרלוונטי עבור כל מקרה. היות שהמהדר
של שפת \סי כתוב בעצמו בשפת \סי, המהדר של השפה "מכיר" את התווים המיוחדות האלו,
ויודע "מה לעשות".

על מנת להשלים את הקוד על מנת שיכיל את התוספת שלנו (ידוע שהדפסת מפלצות
\קד{ASCII} שכאלו מביאה למורת רוח מרובה בקרב המשתמשים), נבצע את השינוי הבא:
הוספנו את השורה המתאימה, וננסה עתה להדר את הקובץ החדש של המהדר באמצעות המהדר
הישן שברשותנו. אך, אבוי, שוד ושבר, נקבל שגיאה! הרי המהדר הישן לא יודע מפלצות
\קד{ASCII} מהן, ולא יודע מהו התו \קד{/v}.

נרצה לפיכך "לאלף" את המהדר הישן להכיר תוסף זה, על מנת שיוכל להדר אותו, ולהפוך
אותו ל\מונח{מהדר} תקני המכיל את השינוי. ניגש אפוא לקובץ המקור של המהדר הישן,
ונוסיף בו את השינוי הבא (אחרון, מבטיחים!): כעת, המהדר הישן מכיר את התו \קד{/v}.
נניח שקיימת מחרוזת תווים כלשהי, המייצגת את מפלצת ה-\קד{ASCII} המועדפת עלינו,
וכי היא מיוצגת ע"י ה-11 בדוגמה שלהלן. נהדר באמצעותו את המהדר החדש שכתבנו, ונקבל
\מונח{קובץ} \קד{binary} שמכיל \מונח{מהדר} חדש, המטמיע את התוספת החדשה. ניתן
באמצעות תוצר זה להדר תכניות שיבצעו תוספת זו.

באופן דומה, אך זהה מבחינה רעיונית, ניתן, לאחר מאמץ מחשבתי ניכר, לתאר נזקים
כבירים אף יותר מהופעה של מפלצת (מאיימת ככל שתהיה) על צג המחשב. דוגמה לכך היא
פריצה לחשבונות פרטיים במערכות \שי{UNIX}, ע"י הוספת פרצה, לפיה ניתן יהיה להתחבר
לכל חשבון באמצעות סיסמא כלשהי (נניח \קד{iAmHackerHoHoHo}). בשלב הבא, נבצע את
התהליך כפי שביצענו בדוגמה הקודמת, ובאמצעותו נקבל את מערכת \שי{UNIX} החדשה, בה
תהיה קיימת פרצה זו. הפרצה תהיה מקודדת ומוטמעת במערכת החדשה, ללא יכולת זיהוי.
זהו \מונח{סוס טרויאני} עמיד בפני התקפות.
