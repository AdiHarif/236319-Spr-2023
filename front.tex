\maketitle


\hfill

\begin{minipage}[l]{0.34688\textwidth}
\זעיר
וַיֹּאמֶר יְהוָה הֵן עַם אֶחָד וְשָׂפָה אַחַת לְכֻלָּם וְזֶה הַחִלָּם
לַעֲשׂוֹת וְעַתָּה לֹא יִבָּצֵר מֵהֶם כֹּל אֲשֶׁר יָזְמוּ לַעֲשׂוֹת
הָבָה נֵרְדָה וְנָבְלָה שָׁם שְׂפָתָם אֲשֶׁר לֹא יִשְׁמְעוּ אִישׁ שְׂפַת
רֵעֵהוּ וְהָאָרֶץ הָיְתָה תֹהוּ וָבֹהוּ וְחֹשֶׁךְ עַל-פְּנֵי תְהוֹם.⏎
\end{minipage}

"וכי אין כל אלו דקדוקי עניות?", אתה עשוי לשאול, ולהוסיף ולהקשות: "וכי חשוב הדבר
כל כך אם כדי להציג טכסט מסויים, עלי לכתוב printf או שמא writeln דווקא?"

אכן, לפרקים, אין הדבר חשוב כלל וכלל, שהלא אם איש אמיד, נשוא פנים ורב נכסים אתה,
מה לך לטרוח בכל אלו. השלך נא במטותא שתי פרוטות או אסימון שחוק אחד אל קופתו של
מתכנת קשה יום, והוא אשר ייגע עצמו בכל הקטנות הללו, ובסופו של עמל יומו, הוא יבצע
את כל אלו בעבורך.

אלא, שאם נסתחפה שדך, ומתכנת אתה, אין בידך ברירה: על כרחך אתה למד את דקדוקי
העניות הללו. שאם תקליד חו"ח Printf רק חרס תעלה בידך ומפח נפש יהיה מנת חלקך. יש
לכתוב \שי{printf}, כך ולא אחרת, כך חשקה נפשו של הגולם איתו נגזר עליך לעבוד, ואם
חפץ אתה שהוא יציית להגיגיך, עליך לבטאם בשפה המובנת לו.

אומר אתה, "ובכן, אין בכך כלום! אחבוש ספסל ואלמד את שפתו! חשקה נפשו ב-printf
דווקא? יהי כן! מה בכך? סוף סוף, כמה מילים יש לו?"

דא עקא, שאמנם מילים הרבה אין לו, אבל, לא שפה אחת יש לו, ולא שבעים לשונות הוא
דובר. אלפי אלפים של לשונות יש לגולם הזה, וחכמי הדור מוסיפים עליהם מדי יום. גם
אם ירבו לך משמיים בריאות, כח ועזוז כַּשַּׁלְחוּפָה ארוכת השנים, אין מספיקין בידך, ולא
תוכל ללמוד את כל הלשונות המשונות הללו.

אם כן, בואה, התחכמה לו! אין לך ללמוד לא לשון אחת ולא שבעים,
אלא את תורת הסוד המסתתרת
מאחורי כל השפות. למדת אותה, ומצא לך! לא תהא שפה העומדת בפני כוחותיך. ולא זו
בלבד, אם חריף שכל ואמיץ תבונה אתה, אף אתה לחכם הדור תחשב, ותוכל אף אתה לשעשע
נפשך ולהוסיף אף אתה לשון כזו משלך.



\renewcommand\contentsname{תוכן העניינים}
\renewcommand\listfigurename{רשימת האיורים}
\renewcommand\listtablename{רשימת הטבלאות}

\cleardoublepage
\addcontentsline{toc}{subsection}{תוכן העניינים}
\dominitoc
\tableofcontents

\cleardoublepage
\addcontentsline{toc}{section}{רשימת האיורים}
\listoffigures

\cleardoublepage
\addcontentsline{toc}{section}{רשימת הטבלאות}
\listoftables

\cleardoublepage
\addcontentsline{toc}{section}{רשימת התכניות}
\listof{תכנית}{רשימת התכניות}
