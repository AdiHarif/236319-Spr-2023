§ למידה פרוצדרולית בניגוד ללמידה דקלרטיבית 
מומחים מבדילים בין למידה פרוצדרולית, ובין למידה דקלרטיבית. בלמידה דקלרטיבית,
אנו לומדים מושגים ועובדות, שהבנתם ועיבודם דורשת מאמץ קוגניטיבי לא פשוט: משפטים,
הוכחות, תהליכים מורכבים כמו אינטגרציה בחלקים, הגדרות, פרוטוקולים, אלגוריתמים
וכו', הם דוגמאות ללמידה דקלרטיבית. לימוד נהיגה או שחיה, הוא לימוד פרוצדורלי:
הידע הנדרש לשם הפעלת כלי רכב מנועי אינו רב. אבל, יש צורך להתאמן רבות כדי להגיע
בו להישגים.

הלימוד בקורס הזה הוא לימוד פרוצדורלי: אמנם ניתקל בהרבה הגדרות, אבל הבנתן אינה
דורשת בדרך כלל מאמץ חשיבתי. מה שנרצה להרכיש לסטודנטים בקורס היא היכולת להפעיל
את ההגדרות לשם יעול הלימוד של שפות תכנות חדשות, רכישת אופני מחשבה חדשים, יכולת
להבין ולהשתתף בשיח המקצועי בנושא של שפות תכנות, וגם, היכולת לפתח בעצמך שפות
חדשות. 

לשם כך, מומלץ מאוד מאוד להשתתף בהרצאות: לשאול שאלות, לסכם תוך כדי הרצאה,
ולהשתתף בלמידה באורח אקטיבי. ישנם סיכומים שונים ברשת, וישנן אף הקלטות וידאו של
מהדורה (ישנה מאוד) של הקורס[11], אבל, אין תחליף לנוכחות, ולביצוע אמיתי של
תרגילי הבית. 

§"גמישות מחשבתית"
קורס זה מטרתו לסייע לסטודנטים ללמוד ולהכיר במהירות שפות תכנות חדשות, להגיע
לשליטה עמוקה יותר בשפות התכנות שהם מכירים ובהם הם עובדים, ובעיקר, "להגמיש את
המחשבה". כדי להסביר את המושג "להגמיש את המחשבה" נספר שבאחד המחקרים, נבדק הַמִּתְאָם
בין התפוקה של מתכנתים, ובין פרמטרים שונים, ובהם השכר, שנות הניסיון, ההשכלה,
התפקיד, ועוד. התברר כי החזאי הטוב ביותר של תפוקה היה מספר שפות התכנות השונות
אותם הם מכירים. 

כמובן, מִתְאָם אינו בהכרח מעיד על סיבתיות. אדרבא, יבוא ד"ר איפכא מסתברא ויאמר
שֶׁהַמִּתְאָם הגבוה בין תפוקה ובין הפוליגלוטיות הוא תוצאה של סיבתיות הפוכה: מתכנתים
שתפוקתם גבוהה יותר, הם כאלו שעיתותיהם בידיהם, ולכן הם מתפנים יותר ללמוד שפות
חדשות. ובכל זאת, אין זה מופרך להניח קשר חשוב בין שתי התופעות. שפות תכנות שונות
מייצגות דרכי מחשבה שונות, אופנים שונים לביטוי הפשטות ולהתמודדות עם בעיות
תכנותיות. ידע רחב בשפות תכנות מסייע על כן למתכנת לבחון בעיות מזוויות שונות.
הבחינה הזו יכולה לאפשר לפתור בעיות שונות באמצעות שפות שונות. וגם אם שפת התכנות
מוכתבת ואינה ניתנת לשינוי, דרכי החשיבה השונות יכולות להציע דרכים שונות לפתור את
הבעיה גם במסגרת שפת התכנות הנתונה.

§ התיזה של ספיר-וורף
הרעיון שהשפה מעצבת את המחשבה אינו חדש. ברומן מדע בדיוני מאת סמואל ר. דילייני
אשר נכתב בשנת 1966, אחד הצדדים למלחמה בין גלקטית מפתח נשק סודי בדמות שפה טבעית
בשם בבל-17 (כשם הרומן). תכונתה של שפת בבל-17 שהיא כופה דרך מחשבה כזו על בני
המחנה היריב הלומדים אותה, שהם חשים צורך לשנות את התנהגותם ולבגוד במחנה שלהם.
בבל-17 הוא רומן בידיוני, אבל הוא נסמך על רעיון מדעי חשוב: "התיזה של ספיר-וורף",
אשר פותחה בראשית המאה ה-20 על ידי הבלשן אדווארד ספיר ותלמידו בנימין לי וורף.
לפי תיזה זו, אופן מחשבתו של אדם מושפע מאוד מהשפה בה הוא דובר, ובמיוחד משפת אמו.
גם מבלי להתעמק בתורת הבלשנות, לא קשה להשתכנע בתיזה זו. כולנו יודעים שבעברית יש
יותר ממאה שמות לעיר ירושלים, ושבערבית יש 99 שמות שונים לאלוהים, ויש להניח כי
העושר הלשוני הזה משפיע על דרך המחשבה. 
הנה שתי דוגמאות נוספות:
\begin{itemize}
         \item אנו יודעים כי בעברית יש שלושה זמנים: עבר, הווה, ועתיד, כאשר צורת
           ההווה, הקרוייה גם "בינוני" בשפה היא מנוונת. (בערבית, יש שני זמנים
           בלבד: עבר ועתיד.) לעומת זאת, בשפה הלטינית יש שישה זמנים (tenses),
           הכוללים למשל זמן לציון מאורע שאירע בעבר ומהווה עבר בעבור מאורע שיקרה
           בעתיד.  
         \item בלטינית יש שתי מילים נבדלות לדם, זה אשר זורם בגוף נקרא sanguis
       ואילו זה אשר זב ממנו נקרא cruor.  
   \end{itemize}
גם כאן, אין זה מופרך לחשוב  כי האבחנה בין זמנים שונים ובין שני סוגי הדם השונים
מכתיבה דרכי חשיבה שונות.  אנו נגלה בקורס כי שפות התכנות השונות מבטאות ומכתיבות
דרכי חשיבה שונות, לעיתים מתוך רצון מודע של מתכנן השפה להשפיע על דרכי החשיבה של
המתכנת. מסתבר למשל שיוקיהירו מצומוטו אשר פיתח את שפת התכנות רובי באמצע שנות
התשעים של המאה הקודמת, ראה לנגד עיניו את התיזה של ספיר-וורף.

§ לימוד שפות תכנות בקורס
ישנו קושי מובנה לבנות וללמד קורס בשפות תכנות: קל לדבר על "גמישות מחשבתית" כמשאת
נפש, אבל לא ברור כיצד ניתן להרכיש אותה לסטודנטים אשר מתקשים בהגמשת זמנם כך
שיכיל את מערכת הלימודים, העבודה והפנאי גם יחד.  לדעת אחדים, קורס בשפות תכנות
הוא קורס שבו הסטודנטים ילמדו שתיים-שלוש (ולעיתים אחת) שפות תכנות מתקדמות,
מעניינות, ובכך ירגילו אותם בדרכי מחשבה שונות.  אחרים ילמדו את הקורס כקורס
תיאורטי, בדומה לקורס ב"אוטומטים ושפות פורמליות", וישנה גם גישה המציעה למנות את
המושגים השונים ולפרט את דרך מימושים בשפות התכנות השונות.
הגישה אשר בה ינקוט קורס זה היא גישה משולבת. ראשית, נלמד כמה וכמה שפות תכנות,
ובראש ובראשונה את השפות הבאות: 
  \begin{itemize}
         \item ML - כדי להדגים שאפשר לכתוב תכניות גם מבלי להשתמש במשתנים, ואת
           צורת החשיבה הפונקציונלית.

         \item Prolog - כדי להדגים שאפשר לכתוב תכניות גם מבלי להשתמש בפקודות.

         \item Pascal - כדי להעשיר את הרקע ההיסטורי, כדי להבין את השפעותיה על
           שפות התכנות של ימינו, וכדי להנגיד אותה עם שפת התכנות המוכרת יותר, C.

\end{itemize}

שנית, נציב בקורס גישה שיטתית יותר לניתוח שפות תכנות, ובפרט קריטריונים שונים
לניתוחן. נדגים את הגישה הזו על שפות תכנות שונות, ונתרגל אותה לשם לימוד שפות
תכנות חדשות. במופע זה של הקורס, הסטודנטים ילמדו "כהרף עין" את השפות Go, Dart,
AWK ועוד. כמובן, בתחילה יקח זמן לסטודנטים לרכוש את המיומנות ללמוד שפות חדשות
"כהרף עין", אבל יש לקוות כי היכולת הזו תיתפתח במהלך הקורס. 
