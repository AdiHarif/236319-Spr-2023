\documentclass{article}
\RequirePackage[utf8]{inputenc}
\RequirePackage{fontspec}
\RequirePackage[table,dvipsnames]{xcolor}
\usepackage{listings}
\usepackage{forest}
\usetikzlibrary{positioning}
\begin{document}

    \forestset{%
      x tree/.style={%
          for tree={%
              font=\ttfamily\scriptsize,
              s sep'+=3pt,
              l sep'+=3pt,
              fit=band,
            },
        },
        node/.style={ellipse,minimum height=0.0ex,minimum width=0.0ex},
        library/.style={node,fill=green!20},
        identifier/.style={node,fill=red!20},
        operator/.style={node,fill=yellow!20},
        literal/.style={node,fill=blue!20},
        function/.style={node,fill=olive!20},
    }

    \begin{forest}x tree[?:,operator, 
          tikz={\node[right=of .east]{built-in operator};}
        ]
      \end{forest}
   \end{document}
     
    \begin{tabular}{b{5ex}p{0.3\textwidth}}
        %\toprule
        %\bf Notation & \bf Meaning \\
        \midrule
        \begin{forest}x tree[\E|?:|,operator, 
          tikz={\node[right=of ()]{built-in operator};}
        ]
          & built-in operator \\
      \begin{forest}x tree[$h()$,function] \end{forest} 
        &identifier of a user-defined function\\
      \begin{forest}x tree[$2$,literal] \end{forest} & 
       identifier of a library function \\
      \begin{forest}x tree[$a$,identifier] \end{forest} & 
      identifier of a user-defined variable\\
        \midrule
        %\bottomrule
  \end{tabular}
\end{tabular}
  \end{LTR}
\end{figure}\newcommand\codesize\scriptsize
\newfontfamily \listingsfont[
            Extension=.otf,
						UprightFont=*-Regular,
						BoldFont=*-Bold,
						ItalicFont=*-Italic,
						BoldItalicFont=*-BoldItalic,
]{UMTypewriter}

\newfontfamily\commentsfont[Mapping=tex-text]{Liberation Serif}
{%
    \makeatletter
    \catcode13=13\relax% Make ASCII 13 active to define it later
    \gdef\newlinewillbreakandindent{%
        \def^^M{\space}%
    }%
}

\lstdefinelanguage{Chic}{
  language=Pascal,
  keywords=[3]{Integer},
  keywords=[2]{Var, Let},
  keywords=[1]{goto,try,catch,return}
}

\lstdefinestyle{code}{
  basicstyle={\listingsfont\codesize},
  stringstyle={\listingsfont\itshape\codesize},
  commentstyle={\commentsfont\slshape},
  keywordstyle={\listingsfont\bfseries},
  keywordstyle={[2]\listingsfont\bfseries\codesize},
  keywordstyle={[3]\listingsfont\bfseries\codesize},
  showstringspaces=false,
  backgroundcolor=\color{yellow!20},
  frame=lines,
  escapechar=¢,
}

\lstdefinestyle{display}{
  style=code,
  columns={[c]fullflexible},
  xleftmargin=1ex,
  aboveskip=2pt,
  belowskip=2pt,
}

\lstset{%
  rangeprefix=\{,rangesuffix=\},
  language=Chic,
  style=display,
  backgroundcolor=\color{RoyalBlue!10},
  includerangemarker=false,
}

\lstdefinelanguage{Chic}{
  language=C++,
  keywords=[3]{Integer},
  keywords=[2]{Var, Let},
  keywords=[1]{goto,try,catch,return},
  escapeinside={/**}{*/},
  escapebegin={\bgroup\newlinewillbreakandindent},
  escapeend={\egroup}
}

\lstinputlisting[linerange=REPL-END]{Mini-Lisp.Chic/repl.cc}
\lstinputlisting{Mini-Lisp.Chic/repl.cc}
\end{document}
