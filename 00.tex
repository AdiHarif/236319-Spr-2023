\documentclass[a4paper,12pt]{book}

\usepackage{00}


⌘dominitoc
⌘מאומה 
⌘faketableofcontents

⌘כותרת{
  \fontsize{120}{140}\selectfont
     שפות תכנות
}

⌘מחבר{
הפקולטה למדעי המחשב⏎
הטכניון - מכון טכנולוגי לישראל⏎
}
⌘תחילת{מסמך}
⌘חומר␣מקדים
\bash[verbose,scriptFile=condition.sh,stdoutFile=condition.tex]
location="Jerusalem , Israel"
server="http://www.Google.com/ig/api"
request="$server?weather=$location"
wget -q -O - $request |\
tr " < >" "\012\012" |\
grep " condition data " |\
head -n 1 |\
sed -e 's/^.*="//' -e 's/"\/*//' |\
tr 'A -Z' 'a -z'
\END
⌘פרק{פרולוג}
%⌘כלול{admin}
⌘כלול{front}



⌘חומר␣עיקרי
ℂ צעדים ראשונים
⌘כלול{steps}
⌘פרק{הביצה והתרנגולת}
⌘כלול{chicken-and-egg}


⌘פרק{קידוד תווים}
⌘כלול{character-encoding}
⌘סוף{מסמך}

⌘פרק{אבני הבנין של שפות תכנות}
⌘כלול{stones}

⌘פרק{למה בכלל?}
⌘כלול{motivation}

⌘פרק{דקדוקי עניות}
⌘כלול{steps}

%⌘כלול{steps-ex}
%\flushbottom
%\clearpage

⌘תחילת{multicols}{2} 
⌘מאומה ⌘גודל␣הערת␣שוליים 
⌘printglossary[type=hebeng]
⌘סוף{multicols}
\clearpage
⌘תחילת{multicols}{2} 
\setLTR
⌘מאומה ⌘גודל␣הערת␣שוליים 
⌘printglossary[type=engheb]
⌘סוף{multicols}
⌘פרק{הִדּוּר וַהֲרָצָה שֶׁל תָּכְנִיּוֹת Java}
⌘כלול{JVM}
⌘סוף{מסמך}


⌘פרק{השם ופשרו}
⌘כלול{binding}

⌘פרק{ערכים}
⌘כלול{values}
⌘פרק{מקרה לדוגמה: שפת פסקל}
⌘כלול{pascal}
\chapter{יצוג אריתמטי של מצביעים}
⌘פרק{יצוג אריתמטי של טיפוסים}
⌘כלול{arithmetics}

⌘חומר␣אחורי

⌘סוף{מסמך}


⌘תחילת{מובאה}
⌘setLTR\cpp{cat << EOF}
⌘סוף{מובאה}


איך ללמוד שפה מהר?
קודם כל להתגבר על בעיות הדקדוק. מסתבר שהדקדוק של שפות שונות עשוי להיות שונה
מאוד, אבל יש קוים משותפים עמוקים.

§ מאפיינים דקדוקיים ויזואליים






§§אבחנה בין תיבה עליונה ותחתונה
למה קוראים להם תיבות עליונות ותחתונות? בעבר, בדפוס היתה תיבה גבוהה עם האותיות
הגדולות ותיבה נמוכה, תחתונה שבה האותיות הקטנות.


איך לומדים מהר שפה?
§§אוסף מילות מפתח

במרבית המקרים קל לנחש את המשמעות של רובם.

§§אוסף האופרטורים

גם כאן קל לנחש את המשמעות של רובם.
מיון אופרטורים: אונריים, בינריים, טרנריים.
מילולונים הם בעצם אופרטורים אפס מקומיים.

תופעות יחודיות

⌘תחילת{ציינון}
• סימן מיוחד למינוס אונרי …
• מגוון סימונים לאופרטור השונה מ
• הצבה? לא תמיד אופרטור.
⌘סוף{ציינון}

§§אוסף המילולונים
בדרך כלל קל הרבה יותר. המילולונים הם די סטנדרטיים.
מספר שלם, או מספר ממישי, תוים וסדריות.
עיין לעיל מילולוני סדריות
§§גודל השפה
האם הקלט פלט הוא חלק מהשפה או שמא הוא חלק מהסיפריה הסטנדרטית.

הנטיה היום: שפה קטנה וספריה גדולה.

§§דברנות

האם מתכנן השפה ניסה להגדיר דקדוק כזה המזכיר את השפה האנגלית.

§מאפיינים תכנוניים שניוניים
§§תכנית ריקה
האם השפה מקבלת תכנית ריקה?
§§גודל התכנית הקצרה ביותר
תכנית שאינה ריקה, אבל קצרה ביותר.
§§גודל ה⌘מהדר העצמי
נקודת שבת מענינת.
§§אורתוגונליות של התכנון
קשה בדרך כלל להבחין.

§§שיטת הביצוע
שפה ביצועית, לעומת שפה הצהרתית, שבה יש מנוע הקובע את שיטת החישוב.
⌘תחילת{ציינון}
• הדור לשפת מכונה
• הדור לשפה אחרת.
• פרשנות
• תרגום למכונה אבסטרקטית.
יש גם מכונה אבסטרקטית הצהרתית לשפת ⌘פרולוג.


