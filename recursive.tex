\newcommand\textbfit[1]{{\bfseries\itshape\selectfont{#1}}}
\newcommand\textalt[1]{{\altdavid\selectfont{#1}}}
\newfontfamily\altdavid{David CLM}
\begin{hebrew}
{מנעולן הפך כף חצץ שגזר קט איבד סתם.}
(גופן רגיל)\\
\textit{מנעולן הפך כף חצץ שגזר קט איבד סתם.}
(גופן נטוי)\\
\textbf{מנעולן הפך כף חצץ שגזר קט איבד סתם.}
(גופן מודגש)\\
\textbfit{מנעולן הפך כף חצץ שגזר קט איבד סתם.}
(גופן נטוי ומודגש)\\
\textalt{מנעולן הפך כף חצץ שגזר קט איבד סתם.}
(דוד קולמוס)\\
\textalt{\textit{מנעולן הפך כף חצץ שגזר קט איבד סתם.}}
(דוד נטוי)\\
\textalt{\textbf{מנעולן הפך כף חצץ שגזר קט איבד סתם.}}
(דוד מודגש)\\
\textalt{\textbf{\textit{מנעולן הפך כף חצץ שגזר קט איבד סתם.}}}
(דוד נטוי ומודגש)
\end{hebrew}​


§ הגדרות ובניות רקורסיבית
סעיף 4ב' לחוק השבות, תש"י - 1950 קובע:
\begin{quote}
⌘כתב␣עבה{לענין חוק זה, "יהודי" - מי שנולד לאם יהודיה או שנתגייר, והוא אינו בן דת אחרת}.
\end{quote}
חוק השבות בבואו להגדיר את המילה "יהודי" משתמש בגוף ההגדרה במילה זו עצמה. הגדרות המשתמשות במונח המוגדר כחלק מההגדרה של המונח עצמו, נקראות הגדרות רקורסיביות. על פי ההלכה המוסלמית, מוסלמי הוא מי שנולד לאב מוסלמי או שהפך למוסלמי באמצעות אמירת העדות, הלא היא השהאדה:
\begin{quote}
\begin{Arabic}
  لَا إِلٰهَ إِلَّا الله مُحَمَّدٌ رَسُولُ الله
  \end{Arabic}
\end{quote}



בפני שלושה מוסלמים.כלומר גם ההלכה המוסלמית, מגדירה רקורסיבית את התשובה לשאלה "מיהו מוסלמי?".

באופן מדוייק יותר, נאמר על קבוצה~$S$ שהיא מוגדרת באופן רקורסיבי אם ההגדרה של~$S$ מכילה את שלושת המרכיבים הבאים:
\begin{enumerate}
⌘⌘ רשימה של ⌘מונח{איברים אטומיים} של~$S$.איברים אטומייים של~$S$ אינם נבנים מאיברים אחרים בקבוצה~$S$. אם נביט על סעיף 4ב' של חוק השבות כעח הגדרה רקורסיבית של קבוצת היהודים,סביר שנאמר שאברהם אבינו ושרה אמנו הם האיברים האטומיים של הקבוצה, כלומר הם יהודים בזכות עצמם. בהסתכלות דומה על ההלכה המוסלמית, סביר שנאמר כי נביא האיסלם, הלא הוא מוחמד, ואולי עוד כמה מתלמידיו, הם מוסלמים מכוח עצמם בלבד. כל שאר המוסלמים נקבעים בדרך אחרת.
⌘⌘ רשימה של ⌘מונח{בנאי איברים} של~$S$ המאפשרים לייצר איברים נוספים ל-$S$ לבד מהאיברים האטומיים. בנאי איברים בונה איברים חדשים של הקבוצה~$S$ מתוך איברים הקיימים בה. בחוק השבות יש לפיכך שני בנאים: האחד הוא הבנאי שמאפשר לקבוע כי אדם מסויים הוא יהודי, אם אמו יהודיה. בנאי האימהות הוא בנאי ⌘מונח{אונארי}, משום שבנאי זה מתחיל מאיבר יחיד בקבוצה, אישה שהיא יהודיה, ומאפשר לייצר איבר חדש מהאיבר הקיים.

בדומה לכך, הכלל המגדיר כמוסלמי כמי שאמר את השהאדה בפני שלושה מוסלמים אחרים, הוא בנאי ⌘מונח{טרנארי} משום שבנאי זה מתסמך על שלושה איברים בקבוצה המוגדרת רקורסיבית (קבוצת המוסלמים הפעם), כדי לבנות איבר חדש בקבוצה. אמנם חוק השבות אינו מגדיר במדוייק מהו גיור, אך ברור כי הגדרה מדוייקת של הגיור, תכלול רקורסיה באמצעות בנאי איברים - נדרש כי חברי בית הדין המחליט על הגיור יהיו יהודים בעצמם.

⌘⌘ הגדרה רקורסיבית של הקבוצה~$S$ כוללת תמיד בתוכה מרכיב הדורש שאיברי~$S$ הם אטומיים או כאלו שנוצרו מכללי בנייה, ואין ב~$S$ איברים אחרים. בדרך כלל הדרישה המגולמת במרכיב זה אינה נאמרת במפורש, אלא משתמעת מהניסוח. מניסוח חוק השבות, ברור כי ההגדרה מתכוונת לאמר שמי שאינו מקיים את התנאים המנויים בסעיף, אינו יהודי. אך הקביעה כי כל מי שאמו אינו יהודיה ושלא התגייר איננו יהודי, אינה מופיעה בחוק, אלא משתמעת ממנו.
\end{enumerate}

נגדיר לדוגמה באופן רקורסיבי את הקבוצה~$R[x]$
\begin{definition}[קבוצת הפונקציות הרציונליות]
יהי~$x$ משתנה פורמלי. אזי, נגדיר רקורסיבית קבוצה ~$R[x]$ של פונקציות מהממשיים אל הממשיים, $R[x]⊆ℝ→ℝ$, 
על ידי:
\begin{enumerate}
⌘⌘\textbf{איברים אטומיים}
\begin{itemize}
⌘⌘ הפונקציה המעתיקה כל מספר ממשי אל המספר הטבעי~$1$ נמצאת בקבוצה~$R[x]$, כלומר
\begin{align}
\label{eq:1}
1∈R[x]
\end{align}
⌘⌘ הפונקציה המעתיקה כל מספר ממשי אל עצמו נמצאת בקבוצה~$R[x]$, כלומר
\begin{align}
\label{eq:x}
x∈R[x]
\end{align}
\end{itemize}

⌘⌘{בנאים}
\begin{itemize}
⌘⌘ אם הפונקציה~$r(x)$ שייכת ל~$R[x]$ אזי גם הפונקציה~$-r(x)r$ שייכת לקבוצה זו

\begin{align}
\label{eq:minus}
-r(x)∈R[x]
\end{align}
⌘⌘ אם הפונקציות~$r₁(x)$ ו~$r₂(x)$ שייכות ל~$R[x]$ אזי גם הסכום שלהן, המכפלה שלהן, והמנה שלהן שייכות ל~$R[x]$
\begin{align}
  r₁(x)+r₂(x)&∈R[x] \label{eq:plus}⏎
  r₁(x)·r₂(x)&∈R[x] \label{eq:times}⏎
  r₁(x)/r₂(x)&∈R[x] \label{eq:div}
\end{align}
\end{itemize}

⌘⌘ \textbf{מינימליות}
הקבוצה~$R[x]$ היא הקבוצה הקטנה ביותר המקיימת את התנאים
\cref{eq:1},
\cref{eq:x},
\cref{eq:minus},
\cref{eq:plus},
\cref{eq:times}
ו
\cref{eq:div}

\end{enumerate}
\end{definition}

ניסוח תמציתי ומדוייק לבנאים הוא כ\מונח{כללי היסק} כפי שהם נהוגים בתחשיב הפסוקים. כלל היסק האומר שבהתקיים הטענות $P₁, P₂,…,Pₙ$
ניתן להסיק את הטענה~$Q$ יכתב כך:
\begin{align}
\dfrac{\begin{array}{l}P₁ ⏎P₂ ⏎⋮ ⏎Pₙ\end{array}}{Q}
\end{align}
ניתן גם לכתוב את הדרישות בשורה אחת, ובלבד שהן מופרדות זו מזו,
\begin{align}
\infer Q{P₁ & P₂ &⋯& Pₙ}
\end{align}
את ארבעת הבנאים של הקבוצה$R[x]$
ניתן לכתוב כארבעה כללי היסק:

\begin{align}
 &\infer{-r(x)∈R[x]}{r(x)∈R[x]} ⏎
 &\infer{r₁(x)+r₂(x)∈R[x]}{r₁(x)∈R[x] & r₂(x)∈R[x]} ⏎
 &\infer{r₁(x)·r₂(x)∈R[x]}{r₁(x)∈R[x] & r₂(x)∈R[x]} ⏎
 &\infer{r₁(x)/ r₂(x)∈R[x]}{r₁(x)∈R[x] & r₂(x)∈R[x]}
\end{align}

תיתכנה דרכים שונות להגדרה רקורסיבית של אותה הקבוצה. למשל, ניתן להשמיט את הדרישה לאיבר אטומי~$1∈RE[x]$, משום שאפשר לקבל איבר זה באמצעות חלוקה של כל איבר בעצמו.

\end{document}

§ אלפאבית ומילים

אלפאבית הוא קבוצה, בדרך כלל סופית, של אותיות. לדוגמה הקבוצה$Σ$
המוגדרת על ידי$Σ=❴a, b, c❵$
הינה אלפאבית המכיל שתי אותיות,$a$$b$.
לאותיות באלפאבית אין משמעות מלבד העובדה שכולן שונות זו מזו. מילה מעל האלפאבית היא סדרה של אותיות מתוך האלפאבית. למשל,$caba$
היא מילה בת ארבע אותיות מעל$Σ=❴a, b, c❵$

 בהינתן אלפאבית$Σ$
   נסמן ב$Σ^*$
את הקבוצה האינסופית המכילה את כל המילים באורך סופי מעל~$Σ$, לרבות המילה הריקה, אותה בדרך כלל מסמנים ב~$ε$. בדוגמה שלנו$$
Σ^*=❴ε, a, b, c, aa, ab, ac, ba, bb, bc, ca, cb, cc, aaa,…❵$$

"לענין חוק זה, "יהודי" - מי שנולד לאם יהודיה או שנתגייר, והוא אינו בן דת אחרת".
אנו רואים שהגדרת מיהו יהודי לצורך חוק השבות היא רקורסיבית, שכן הגדרת המונח "יהודי" בחוק מבוססת על המונח "יהודי" עצמו.
אנו נאמר על קבוצה שהיא "קבוצה מוגדרת רקורסיבית", אם היא מוגדרת באופן רקורסיבי שכזה. כלומר אפשר לחזור ולוהסיף איברים לקבוצה אם הם נוצרים באופן מסוים מאיברים אחרים שהם כבר בקבוצה. יתירה מכך, טבעה של הרקורסיה הוא שאפשר לחזור ולהפעיל אותה מספר בלתי חסום של פעמים.

כשמדברים על קבוצות מוגדרות רקורסיבית נוח להשתמש בכמה מונחים מוסכמים:
איברים אטומיים
איברים מורכבים
בנאים
קל מאוד להבין אינטואיטיבית מהי קבוצה מוגדרת רקורסיבית. גם פשר המונחים האלו הוא קל. די להציץ בדוגמאות אחדות כדי לגלות את התבנית החוזרת. כיוון שקשה הרבה יותר להגדיר בצורה מדויקת את המונחים, נתחיל בכמה דוגמאות.
​3.3.2​ בנאים, איברים מורכבים, איברים אטומיים בחוק השבות
אם נעמיק ברקורסיה שבחוק השבות נגלה:
איברים מורכבים. חוק השבות קובע את הכלל הבא: אם אדם נולד לאם יהודיה, אזי הוא יהודו. כלל זה הוא בנאי., אשר מקבל פרמטר שהוא אם יהודי, יוצר איבר חדש בקבוצה. תהליך הלידה מיצר מהפרמטר איבר חדש של הקבוצה אותה אנו מגדירים. איבר בקבוצה שנוצר מאיבר אחר, נקרא איבר מורכב.
איברים אטומיים: עוד בנאי נמצא בקביעה "אם פלוני התגייר הרי הוא נחשב כיהודי", גם היא בנאי. כלל זהו הוא בדיוק בנאי. אפשר לנסח זאת כך: הבנאי שלפנינו מקבל "פרמטר".ותהליך הגיור הופך זה את הפרמטר ליהודי. המתגיירים הם היהודים האטומיים.
​3.3.3​ קבוצת הפקודות בְּPascal היא קבוצה מוגדרת רקורסיבית
האיברים האטומיים, הלא הם הפקודות האטומיות, הם שלושה:
הפקודה הריקה
פקודת ההצבה
קריאה לפרוצדורה
נשתמש בשיטה זו כדי להגדיר את אוסף הפקודות בְּPascal. האיברים האטומיים, הלא הם הפקודות האטומיות, הם שלושה:
הפקודה הריקה
פקודת ההצבה
קריאה לפרוצדורה
כללי היצירה העיקריים של הפקודות הם:
שרשור של פקודות המופרדות על ידי סימן הנקודה ופסיק (;) הוא פקודה.
פקודת תנאי, כפי שתוארה לעיל, היא פקודה, אשר מכילה בתוכה פקודה אחת או שתיים.
פקודת תנאי רבת ראשים המוגדרת באמצעות המילה השמורה case
לולאות המתארות ביצוע איטרטיבי של פקודה (אטומית או מורכבת), אף הן פקודות. יש בְּPascal שלושה סוגים של פקודות לולאה מורכבות:
for
while
repeat until
ניתן גם באופן דומה להגדיר את הביטויים בִּשְׂפַת תכנות פשוטה. הביטויים האטומיים יהיו מספרים ושמות משתנים, ואילו כללי היצירה יהיו מבוססים על פעולות החשבון וסימני הסוגריים.
​3.3.4​ דקדוקי עניות במונח "קבוצה מוגדרת רקורסיבית"
יוזהר מראש כי ההגדרה הפורמלית הזו של המונחים שבהם השתמשנו היא מסורבלת. למרבה המזל, אף כי יש חשיבות מסוימת לדקדוקי העניות אלו, ההבנה האינטואטיבית חשובה יותר.
בכל הגדר רקורסיבית של קבוצה יש שלושה מרכיבים.
רשימה של איברים אטומיים אשר משתייכים לקבוצה בכוח עצמם בלבד.
נסתכל לרגע על קבוצת הטיפוסים בִּשְׂפַת פסקל. קבוצה זו היא קבוצה מוגדרת רקורסיבית, בקבוצה זו ישנם ארבעה איברים אטומיים מסוג זה: Character, Integer, Real, Boolean. כל שאר הטיפוסים נוצרים באמצעות בנאים.
רשימה של בנאים.
בנאי הוא פונקציה שמקבלת פרמרטים שיכולים להיות:
אפס או יותר איברים של הקבוצה
אפס או יותר ערכים שאינם איברים של הקבוצה.
ומחזירה ערך שאף הוא בקבוצה.
איבר מורכב הוא איבר שנוצר מבנאי שקיבל כפרמטר איבר אחד לפחות של הקבוצה.
מערכים בִּשְׂפַת Pascal הם טיפוסים מורכבים: בנאי המערך קיבל כפרמטר את טיפוס של תא במערך, והחזיר בתמורה את טיפוס המערך של תאים מאותו טיפוס.
איבר אטומי הוא איבר שנוצר מבנאי שלא קיבל כפרמטרים ערכים של הקבוצה.
ניתן להסתכל על הטיפוסים אטומיים מכוח עצמם: Character, Integer, Real, Boolean כעל בנאים שאינם מקבלים פרמטרים כלל.
טיפוסים מנויים (enumerated types) בִּשְׂפַת פסקל, הם גם טיפוסים אטומיים (איברים אטומיים של קבוצת הטיפוסים) . הם נוצרו מבנאי שמקבלל רשימה של תגיות. תגית היא מזהה חוקי בִּשְׂפַת פסקל.
הדרישה, אשר למען הקיצור נוהגים להשמיטה, כי אין בקבוצה איברים אחרים פרט לאטומיים ולמורכבים שנוצרו באמצעות כללי היצירה. .
בדוגמה של חוק השבות, האיברים האטומיים הם אברהם אבינו, וחשוב מכך, שרה אמנו, וכל מי שנתגייר, ואילו היצירה הוא הלידה.
​3.3.5​ קבוצה שאינה מוגדרת רקורסיבית
נעיין לדוגמה בִּשְׂפַת MetaPost. שפה זו שייכת לפרדיגמה הדקלראטיבית. ליתר דיוק, השפה נועדה לשרטוט באמצעות אילוצים. המתכנת כותב רשימה של משוואות לינאריות, שהפתרון שלהן, שמחושב באופן אוטומטי על ידי מנוע הבנוי בשפה, קובע את השרטוט המבוקש.
 במדריך של השפה קבוצת הטיפוסים מוגדרת באופן הבא:
MetaPost actually has ten basic data types: numeric, pair, path, transform, (rgb)color, cmykcolor, string, boolean, picture, and pen. Let us consider these one at a time beginning with the numeric type.
\begin{itemize}
⌘⌘
Numeric quantities in MetaPost are represented in fixed point arithmetic as integer multiples of 165536, the smallest positive value, which is also available as the predefined constant epsilon…
⌘⌘
The pair type is represented as a pair of numeric quantities. We have seen that pairs are used to give coordinates in draw statements. Pairs can be added, subtracted, used in mediation expressions, or multiplied or divided by numerics…

⌘⌘ A path represents a straight or curved line that is defined parametrically…
⌘⌘ A transform can be any combination of rotating, scaling, slanting, and shifting. If p=(𝑝𝑥, 𝑝𝑦) is a pair and T is a transform,
                                       p transformed T
is a pair of the form...
⌘⌘ The color type is like the pair type, except that it has three components instead of two and each component is normally between 0 and 1…The type ‘rgbcolor’ is an alias of type ‘color’.
The cmykcolor type is similar to the color type except that it has four components instead of three…
⌘⌘ A string represents a sequence of characters…
⌘⌘ The boolean type has the constants true and false and the operators and, or, not…
⌘⌘ The picture data type is just what the name implies. Anything that can be drawn in MetaPost can be stored in a picture variable…Pictures can be added to other pictures and operated on by transforms...
⌘⌘ The main function of pens in MetaPost is to determine line thickness, but they can also be used to achieve calligraphic effects. The statement
\cc{pickup~$⟨\text{\itshape pen expression}⟩$} causes the given pen to be used in subsequent draw or drawdot statements...
\end{itemize}
אנו רואים שבשפה ישנם טיפוסים מיוחדים שאינם שכיחים בשפות אחרות: עט, צבע, תמונה, זוג, ועוד. אלא שהיחודיות הזו חסומה: אין אפשרות למתכנת בשפה להוסיף טיפוסים חדשים. כיוון שקבוצת הטיפוסים היא קטנה וסופית, ברור שהיא אינה מוגדרת רקורסיבית, שכן כל קבוצה מוגדרת רקורסיבית היא בלתי חסומה בגדלה. בפרט, אין בִּשְׂפַת MetaPost בנאים או טיפוסים מורכבים. יש בה עשרה טיפוסים, ותו לא.
כדאי להתעכב מעט על הטיפוס pair: טיפוס זה אמנם מיוצג כזוג של numeric. אבל מדובר כאן בייצוג בלבד. לא בבנית טיפוסים. אבל אין אפשרות ליצור זוג של שום טיפוס אחר. אין זוגות של תמונות, צבעים, עטים, וגם אין זוגות של זוגות. יתירה מכך, יש פעולות יחודיות לזוג, כמו טרנספורמציה שאינן נובעות מכך שהוא "מורכב" משני numeric. לפיכך, אין מדובר כאן בבנאי של זוגות.

קבוצות מוגדרות רקורסיבית
      רקורסיה אינה ענין של שפות תכנות בלבד
      סעיף 4ב' לחוק השבות, תש"י - 1950 קובע:
      "לענין חוק זה, "יהודי" - מי שנולד לאם יהודיה או שנתגייר, והוא אינו בן דת אחרת".†{הגדרה רקורסיבית דומה קיימת ביחס לשאלה "מיהו מוסלמי?". על פי ההלכה המוסלמית, מוסלמי הוא מי שנולד לאב מוסלמי, או שהפך למוסלמי באמצעות אמירת העדות, הלא היא השהאדה: لَا إِلٰهَ إِلَّا الله مُحَمَّدٌ رَسُولُ الله בפומבי. אבל, כיוון שהאמירה חייבת להעשות בפני מי שהם מוסלמים, ההתאסלמות מהווה כלל יצירה של איבר מורכב. אגב, ההמרה לנצרות, לפחות על פי כתבי הקודש דורשת גם היא אמירה, אך אמירה זו היא פרטית ולא פומבית, ככתוב באיגרת פולוס השליח אל הרומיים, פרק י"א "כִּי אִם־בְּפִיךָ תוֹדֶה שֶׁיֵּשׁוּעַ הוּא הָאָדוֹן וְתַאֲמִין בִּלְבָבְךָ שֶׁהָאֱלֹהִים הֱעִירוֹ מִן־הַמֵּתִים תִּוָּשֵׁעַ׃ כִּי בִלְבָבוֹ יַאֲמִין הָאָדָם וְהָיְתָה לּוֹ לִצְדָקָה וּבְפִיהוּ יוֹדֶה וְהָיְתָה־לּוֹ לִישׁוּעָה׃ כִּי הַכָּתוּב אֹמֵר כָּל־הַמַּאֲמִין בּוֹ לֹא יֵבוֹשׁ׃ וְאֵין הַפְרֵשׁ בֵּין הַיְּהוּדִי לַיְּוָנִי כִּי אָדוֹן אֶחָד לְכֻלָּם".}
      אנו רואים שהגדרת מיהו יהודי לצורך חוק השבות היא רקורסיבית, שכן הגדרת המונח "יהודי" בחוק מבוססת על המונח "יהודי" עצמו.
      אנו נאמר על קבוצה שהיא "קבוצה מוגדרת רקורסיבית†{20}", אם היא מוגדרת באופן רקורסיבי שכזה. כלומר אפשר לחזור ולוהסיף איברים לקבוצה אם הם נוצרים באופן מסוים מאיברים אחרים שהם כבר בקבוצה. יתירה מכך, טבעה של הרקורסיה הוא שאפשר לחזור ולהפעיל אותה מספר בלתי חסום של פעמים.
      כשמדברים על קבוצות מוגדרות רקורסיבית נוח להשתמש בכמה מונחים מוסכמים:
      1. איברים אטומיים
      2. איברים מורכבים
      3. בנאים
      קל מאוד להבין אינטואיטיבית מהי קבוצה מוגדרת רקורסיבית. גם פשר המונחים האלו הוא קל. די להציץ בדוגמאות אחדות כדי לגלות את התבנית החוזרת. כיוון שקשה הרבה יותר להגדיר בצורה מדויקת את המונחים, נתחיל בכמה דוגמאות.
      בנאים, איברים מורכבים, איברים אטומיים בחוק השבות
      אם נעמיק ברקורסיה שבחוק השבות נגלה:
      ⌘תחילת{itemize}
      ⌘⌘ איברים מורכבים. חוק השבות קובע את הכלל הבא: אם אדם נולד לאם יהודיה, אזי הוא יהודו. כלל זה הוא בנאי., אשר מקבל פרמטר שהוא אם יהודי, יוצר איבר חדש בקבוצה. תהליך הלידה מיצר מהפרמטר איבר חדש של הקבוצה אותה אנו מגדירים. איבר בקבוצה שנוצר מאיבר אחר, נקרא איבר מורכב.
      ⌘⌘ איברים אטומיים: עוד בנאי נמצא בקביעה "אם פלוני התגייר הרי הוא נחשב כיהודי", גם היא בנאי. כלל זהו הוא בדיוק בנאי. אפשר לנסח זאת כך: הבנאי שלפנינו מקבל "פרמטר".ותהליך הגיור הופך זה את הפרמטר ליהודי. המתגיירים הם היהודים האטומיים.
      קבוצת הפקודות בְּPascal היא קבוצה מוגדרת רקורסיבית
      האיברים האטומיים, הלא הם הפקודות האטומיות, הם שלושה:
      1. הפקודה הריקה
      2. פקודת ההצבה
      3. קריאה לפרוצדורה
      נשתמש בשיטה זו כדי להגדיר את אוסף הפקודות בְּPascal. האיברים האטומיים, הלא הם הפקודות האטומיות, הם שלושה:
      1. הפקודה הריקה
      2. פקודת ההצבה
      3. קריאה לפרוצדורה
      כללי היצירה העיקריים של הפקודות הם:
      1. שרשור של פקודות המופרדות על ידי סימן הנקודה ופסיק (;) הוא פקודה.
      2. פקודת תנאי, כפי שתוארה לעיל, היא פקודה, אשר מכילה בתוכה פקודה אחת או שתיים.
      3. פקודת תנאי רבת ראשים המוגדרת באמצעות המילה השמורה case
      4. לולאות המתארות ביצוע איטרטיבי של פקודה (אטומית או מורכבת), אף הן פקודות. יש בְּPascal שלושה סוגים של פקודות לולאה מורכבות:
      ⌘⌘ for
      ⌘⌘ while
      ⌘⌘ repeat until
      ניתן גם באופן דומה להגדיר את הביטויים בִּשְׂפַת תכנות פשוטה. הביטויים האטומיים יהיו מספרים ושמות משתנים, ואילו כללי היצירה יהיו מבוססים על פעולות החשבון וסימני הסוגריים.
      § דקדוקי עניות במונח "קבוצה מוגדרת רקורסיבית"
      יוזהר מראש כי ההגדרה הפורמלית הזו של המונחים שבהם השתמשנו היא מסורבלת. למרבה המזל, אף כי יש חשיבות מסוימת לדקדוקי העניות אלו, ההבנה האינטואטיבית חשובה יותר.
      בכל הגדר רקורסיבית של קבוצה יש שלושה מרכיבים.
      1. רשימה של איברים אטומיים אשר משתייכים לקבוצה בכוח עצמם בלבד.
      ⌘⌘ נסתכל לרגע על קבוצת הטיפוסים בִּשְׂפַת פסקל. קבוצה זו היא קבוצה מוגדרת רקורסיבית, בקבוצה זו ישנם ארבעה איברים אטומיים מסוג זה: Character, Integer, Real, Boolean. כל שאר הטיפוסים נוצרים באמצעות בנאים.
      1. רשימה של בנאים.
      ⌘⌘ בנאי הוא פונקציה שמקבלת פרמרטים שיכולים להיות:
      ⌘⌘ אפס או יותר איברים של הקבוצה
      ⌘⌘ אפס או יותר ערכים שאינם איברים של הקבוצה.
      ומחזירה ערך שאף הוא בקבוצה.
      ⌘⌘ איבר מורכב הוא איבר שנוצר מבנאי שקיבל כפרמטר איבר אחד לפחות של הקבוצה.
      ⌘⌘ מערכים בִּשְׂפַת Pascal הם טיפוסים מורכבים: בנאי המערך קיבל כפרמטר את טיפוס של תא במערך, והחזיר בתמורה את טיפוס המערך של תאים מאותו טיפוס.†{למען השלמות יש לציין שבנאי המערך מקבל עוד שני פרמטרים, שהם קצוות המערך. ואפשר להפליג ולצין שבנאי של מערך רב מימדי מקבל מספר זוגות של פרמטרים כנ"ל.}
      ⌘⌘ איבר אטומי הוא איבר שנוצר מבנאי שלא קיבל כפרמטרים ערכים של הקבוצה.
      ⌘⌘ ניתן להסתכל על הטיפוסים אטומיים מכוח עצמם: Character, Integer, Real, Boolean כעל בנאים שאינם מקבלים פרמטרים כלל.
      ⌘⌘ טיפוסים מנויים (enumerated types) בִּשְׂפַת פסקל, הם גם טיפוסים אטומיים (איברים אטומיים של קבוצת הטיפוסים) . הם נוצרו מבנאי שמקבלל רשימה של תגיות. תגית היא מזהה חוקי בִּשְׂפַת פסקל.
      1. הדרישה, אשר למען הקיצור נוהגים להשמיטה, כי אין בקבוצה איברים אחרים פרט לאטומיים ולמורכבים שנוצרו באמצעות כללי היצירה. .
      בדוגמא של חוק השבות, האיברים האטומיים הם אברהם אבינו, וחשוב מכך, שרה אמנו, וכל מי שנתגייר, ואילו היצירה הוא הלידה.
      § קבוצה שאינה מוגדרת רקורסיבית
      נעיין לדוגמא בִּשְׂפַת MetaPost. שפה זו שייכת לפרדיגמה הדקלראטיבית. ליתר דיוק, השפה נועדה לשרטוט באמצעות אילוצים. המתכנת כותב רשימה של משוואות לינאריות, שהפתרון שלהן, שמחושב באופן אוטומטי על ידי מנוע הבנוי בשפה, קובע את השרטוט המבוקש.
      במדריך של השפה קבוצת הטיפוסים מוגדרת באופן הבא:
      MetaPost actually has ten basic data types: numeric, pair, path, transform, (rgb)color, cmykcolor, string, boolean, picture, and pen. Let us consider these one at a time beginning with the numeric type.
      \begin{enumerate}
      ⌘⌘ Numeric quantities in MetaPost are represented in fixed point arithmetic as integer multiples of, the smallest positive value, which is also available as the predefined constant epsilon…
      ⌘⌘ The pair type is represented as a pair of numeric quantities. We have seen that pairs are used to give coordinates in draw statements. Pairs can be added, subtracted, used in mediation expressions, or multiplied or divided by numerics…
      ⌘⌘ A path represents a straight or curved line that is defined parametrically…
      ⌘⌘ A transform can be any combination of rotating, scaling, slanting, and shifting. If p=(𝑝𝑥, 𝑝𝑦) is a pair and T is a transform,
      p transformed T
      is a pair of the form...
      ⌘⌘ The color type is like the pair type, except that it has three components instead of two and each component is normally between 0 and 1…The type ‘rgbcolor’ is an alias of type ‘color’.
      ⌘⌘ The cmykcolor type is similar to the color type except that it has four components instead of three…
      ⌘⌘ A string represents a sequence of characters…
      ⌘⌘ The boolean type has the constants true and false and the operators and, or, not…
      ⌘⌘ The picture data type is just what the name implies. Anything that can be drawn in MetaPost can be stored in a picture variable…Pictures can be added to other pictures and operated on by transforms...
      ⌘⌘. The main function of pens in MetaPost is to determine line thickness, but they can also be used to achieve calligraphic effects. The statement pickup~$⟨\text{pen expression}⟩$ causes the given pen to be used in subsequent draw or drawdot statements…

  \end{enumerate}
      אנו רואים שבשפה ישנם טיפוסים מיוחדים שאינם שכיחים בשפות אחרות: עט, צבע,
      תמונה, זוג, ועוד. אלא שהיחודיות הזו חסומה: אין אפשרות למתכנת בשפה להוסיף
      טיפוסים חדשים. כיוון שקבוצת הטיפוסים היא קטנה וסופית, ברור שהיא אינה
      מוגדרת רקורסיבית, שכן כל קבוצה מוגדרת רקורסיבית היא בלתי חסומה בגדלה.
      בפרט, אין בִּשְׂפַת MetaPost בנאים או טיפוסים מורכבים. יש בה עשרה טיפוסים, ותו
      לא. כדאי להתעכב מעט על הטיפוס pair: טיפוס זה אמנם מיוצג כזוג של numeric.
      אבל מדובר כאן בייצוג בלבד. לא בבנית טיפוסים. אבל אין אפשרות ליצור זוג של
      שום טיפוס אחר. אין זוגות של תמונות, צבעים, עטים, וגם אין זוגות של זוגות.
      יתירה מכך, יש פעולות יחודיות לזוג, כמו טרנספורמציה שאינן נובעות מכך שהוא
      "מורכב" משני numeric. לפיכך, אין מדובר כאן בבנאי של זוגות.
      §§שפות פורמליות לתיאור שפות תכנות
      §§§ביטויים רגולריים
      בהינת אלפאבית ∑ של סימנים יסודיים, נגדיר כ *∑ את אוסף כל הסדריות (Strings) הסופיות שניתן לכתוב בעזרתו, ובכלל אלו את הסדרית הריקה אשר מסומנת בדרך כלל כ 𝜺. במנוחים אלו שפה פורמלית היא פשוט תת-קבוצה של *∑. ביטויים רגולריים (regular expressions), הם מכשיר להגדרת שפות פורמליות.
      קבוצת הביטויים הרגולריים מעל ∑ מוגדרת רקורסיבית באופן הבא:
      1. ביטויים אטומים:
      ⌘⌘ כל תו ב ∑ הוא ביטוי רגולרי אטומי המתאר שפה המכילה סדרית אחת בלבד, התו עצמו.
      ⌘⌘ הסימן 𝜺 הוא ביטוי רגולרי אטומי, המתאר שפה המכילה את הסדרית הריקה בלבד.
      1. כללי יצירה
      ⌘⌘ אם R1 וְ ּR2 הם ביטוייים רגולריים, אזי
      ⌘⌘ R1R2 הוא ביטוי רגולרי שהשפה שלו מורכבת מכל הסדריות שאפשר לחלק אותם לשני חלקים רצופים וזרים, כאשר החלק הראשון, הרישא של הסדרית, שייך לשפה של R1 ואילו החלק השני, הסיפא, שייך לשפה של R2.
      ⌘⌘ R1 | R2 הוא ביטוי רגולרי שהשפה שלו היא האיחוד של שתי השפות של R1 וְ R2.
      ⌘⌘ אם ּR הוא ביטוי רגולרי, אזי גם+R הוא ביטוי רגולרי, שהשפה שלו היא כל הסדריות שניתן לחלק אותן לסדריות זרות ורצופות אשר כל אחת מהן שייכת לשפה של הביטוי הרגולרי R.
      במרבית השימושים בפועל מוסיפים תַּחְבִּירִי סֻכָּר כגון:
      ⌘⌘ ?ּR הוא קיצור עבור R|𝜺
      ⌘⌘ *R הוא קיצור עבור (+R)|𝜺
      ⌘⌘ u-v הוא קיצור עבור כל התווים שמצויים בין התווים u ל v, תוך הנחה שיש סדר מוסכם של התווים באלפאבית ∑
      ⌘⌘ הסימן "." (נקודה) מתאר את הביטוי הרגולרי שמכיל אות אחת בדיוק מהאלפאבית.
      ⌘⌘ ניתן לקצר את הכתיבה באמצעות מתן שמות לביטויים רגולריים חלקיים.
      הנה דוגמא לשימוש ב תַּחְבִּירִי סֻכָּר כדי להגדיר ביטוי רגולרי בעבור מזהה חוקי בִּשְׂפַת C:
  ⌘סוף{itemize}
\begin{verbatim}
Digit=0-9
Lower=a-z
Upper=A-Z
Letter=Lower|Upper
IdLetter=Letter|\$|\_
IdCharacter=IdLetter|Digit
Identifier=CLetter IdCharacter*
\end{verbatim}

      כדאי לדעת כי ניתן לכתוב ביטוי רגולרי עבור חיתוך של השפות של שני ביטויים
      רגולרייים, וגם עבור השפה של כל הסדריות של ביטוי רגולרי אחד אשר אינן מצויות בשפה
      של ביטוי רגולרי אחר, וזאת בעבור כל שני ביטויים רגולריים שרירותיים. השימוש
      בביטויים רגולריים בתיאור הדקדוק של שפה, אינו חשוב רק למען הדיוק. ישנם כלים
      אוטומטיים אשר הקלט שלהם הוא ביטוי רגולרי ואשר מיייצרים תכניות המסוגלות לזהות
      מופעים של ביטויים רגולריים בטקסט. כלים אלו מועילים מאוד בביטויים בכתיבת מהדרים
      עבור שפות תכנות.
      דקדוקי BNF
      כוח הביטוי של ביטויים רגולריים הוא מוגבל ביותר. כך למשל, לא ניתן לבטא באמצעות
      ביטוי רגולרי את הדרישה שהסוגריים בתכנית מאוזנים. לפיכך, השימוש בביטויים
      רגולריים מוגבל להגדרות פשוטות של אבני הבנין של השפה: משתנים, הערות, מספרים
      וכו'. להגדרות מורכבות יותר, יש להשתמש במנגנון הידוע בשם דקדוק חסר הקשר
      (Context Free Grammar), אשר נכתב בדרך כלל בשיטת הסימון הידועה בשם Backus Naur
      Form או BNF. כתיב אחר לדקדוקים אלו הוא הדיאגרמות שתוארו לעיל.
      הגדרת דקדוק בשיטת סימון זו מורכבת מארבעה חלקים:
      1. קבוצה של סימנים סופיים, Terminals. (ה Terminals קרויים
      לעיתים גם Tokens או אסימונים). הדקדוק מתאר שפה מעל האלפאבית
      שיוצרים הסימנים הסופיים.
      2. קבוצה של סימנים לא סופיים, Non Terminal Symbols, המשמשים
      ככלי עזר להגדרת הדקדוק. סימני עזר אלו דומים מעט לשימוש בשמות
      לביטויים רגולריים חלקיים, אלא, שהגדרתם של סימני העזר הללו
      יכולה להיות רקורסיבית, וניתן להגדירן יותר מאשר פעם אחת.
      3. קביעה של אחד מהסימנים הלא סופיים כסימן התחלה: Start Symbol
      4. אוסף של כללי גזירה, כאשר לכלל גזירה יש שני חלקים: ראש הכלל
      הכתוב בצד שמאל של הכלל, הוא תמיד סימן לא סופי, ואילו גוף הכלל
      הכתוב בצידו הימני, הוא סדרית (היכולה להיות ריקה) של סימנים
      סופיים ולא סופיים. כללי הגזירה נכתבים כך שישנו חץ המוביל מראש
      הכלל אל גופו. בפועל, נוהגים להשמיט את המרכיבים 1 עד 3 של
      הגדרת הדקדוק ולהסתפק בכללי הגזירה לבדם. קל להבחין בין סימנים
      סופיים ולא סופיים בכללי הגזירה, משום שסימן סופי לא יופיע
      לעולם בראש כלל. גם סימן ההתחלה ברור בדרך כלל מההקשר.
      הנה דוגמא:
      \begin{align}
        S &→E ⏎
        E &→a E b ⏎
        E &→𝜺 ⏎
      \end{align}

      בדוגמא זו בדקדוק ישנם שלושה כללי גזירה, אשר מקריאתם מתגלה כי:
      1. הסימנים הלא סופיים הם S וְ E
      2. סימן ההתחלה הוא S
      3. הסימנים הלא סופיים הם a וְ b
      השפה המוגדרת על ידי הדקדוק חסר ההקשר הזה היא פשוטה ביותר, והיא מכילה את כל
      הסדריות שבראשן יש מספר n (שיכול להיות 0) של a ואחריהם n מופעים של הסימן b.
      ניתן להוכיח (ולא נעשה זאת כאן), כי לא ניתן להגדיר שפה זו באמצעות ביטויים
      רגולריים.

      הנה דוגמא המהווה קטע של הדקדוק של שְׂפַת פסקל:

\begin{derivation}
      \begin{align}
pascal-program→program identifier program-heading ; block . ⏎
program-heading→𝜺 ⏎
program-heading→(identifier-list) ⏎
identifier-list→identifier ⏎
identifier-list→identifier-list, identifier ⏎
block→block1 ⏎
block→label-declaration ; block1 ⏎
block1→block2 ⏎
block1→constant-declaration ; block2 ⏎
block2→block3 ⏎
block2→type-declaration ; block3 ⏎
block3→block4 ⏎
block3→variable-declaration ; block4 ⏎
block4→block5 ⏎
block4→proc-and-func-declaration ; block5 ⏎
block5→begin statement-list end ⏎
…⏎
type-declaration→type type-declarator ⏎
type-declaration→type-declaration ; type-declarator ⏎
type-declarator→identifier=type ⏎
…⏎
type→identifier ⏎
type→record field-list \=end=⏎
field-list→𝜺 ⏎
      \end{align}
\end{derivation}

      קל לזהות בהגדרת דקדוק זו את סימן ההתחלה. לשם הנוחות סימנו את הסימנים הסופיים כגופן וצבע מיוחדים. מהגדרת הדקוק הזו אנו למדים למשל:
      1. תכנית Pascal מתחילה תמיד במילה program ומסתיימת בסימן ". "
      2. לתכנית יש שם שאחריו יכולה להפועים רשימת מזהים העטופה בסוגריים עגולים.
      3. בראש התכנית יש ארבעה פרקי הגדרות החייבים להופיע בסדר קבוע: הגדרת תוויות, הגדרת קבועים, הגדרת טיפוסים והגדרת משתנים.
      4. כל אחד מארבעת מפרקי ההגדרות הוא אופציונלי.
      5. בהגדרת פרק הטיפוסים אם מופיעה המילה type אזי אחריה חייבת להופיע הגדרת טיפוס אחת לפחות.
      6. הגדרות הטיפוסים חייבות להיות מופרדות בסימן ";"
      7. כל הגדרת טיפוס בודדת מכילה מזהה, סימן שיווין, ואחריו גוף הטיפוס, שיכול להיות מזהה או רשומה.

      והנה דוגמא לתכנית פשוטה (וסרת טעם) המצייתת לדקדוק לעיל:

\begin{verbatim}
program p;
type
shalem=integer;
student=record
end;
begin
end.
\end{verbatim}

      הגדרת הדקדוק של שְׂפַת תכנות באמצעות דקדוק חסר הקשר לא נועדה למען הדיוק בלבד. ישנם
      כלים אוטמטיים המאפשרים תרגום של דקדוק חסר הקשר כזה לתכנית ניתוח, אשר לוקחת טכסט
      נתון, ובונה בעבורו את אופן גזירתו מהדקדוק. אופן הגזירה הזה נקרא "עץ גזירה"
      (ַParse Tree) אשר מהווה הוכחה כי הטכסט אמנם נגזר מהדקדוק. הפורמליזם של דקדוק
      BNF חזק יותר מהפורמליזם של ביטויים רגולריים שכן הוא מתיר הגדרות רקורסיביות. כך
      למשל, בהגדרת הדקדוק של Pascal נמצא הגדרות רקורסיביות שבהן הסימן הלא סופי
      statement-list מוגדר באמצעות הסימן הלא סופי statement ולהיפך:

      ⌘begin{align}
statement-list→statement
statement-list→statement-list ; statement
statement→𝜺
statement→variable :=expression
statement→begin statement-list end
statement→if expression then statement
statement→if expression then statement else statement
statement→case expression of case-list end
statement→while expression do statement
statement→repeat statement-list until expression
statement→for varid :=for-list do statement
statement→procid
statement→procid(expression-list)
statement→goto label
statement→with record-variable-list do statement
statement→label : statement
      \end{align}

      הגדרות רקורסיביות מעין אלו הינן חיוניות בהגדרת שפות תכנות מודרניות, אך הן אינן
      ניתנות להיעשות בביטויים רגולריים. דקדוקי EBNF EBNF הוא קיצור של Extended BNF.
      פורמליזם זה דומה בעיקרו לפורמליזם של דקדוק BNF, אלא שגופו של כלל הגזירה יכול
      להיות ביטוי רגולרי מעל אוסף הסימנים, הסופיים והלא סופיים כאחד. שימוש בביטויים
      רגולריים כאלו הוא בבחינת תַּחְבִּירִי סֻכָּר לדקדוקי BNF. ההרחבה עצמה אינה מאפשרת
      הגדרת שפות פורמליות נוספות פרט לאלו הניתנות להגדרה בדקדוק BNF, אך ניתן באמצות
      הרחבה זו להגדיר שפות פורמליות ביתר תמציתיות.

      הנה שכתוב של קטע הדקדוק הראשון של Pascal שהבאנו כאן, תוך שימוש בשיטות הסימון
      של EBNF.

      ⌘begin{align}
Pascal -program→program identifier [(identifier {,identifier})] ; block .
block→[label-declaration;]
               [constant-declaration;]
               [type-declaration;]
               [variable-declaration ;]
      begin statement-list end
…
type-declaration→type ַtype-declarator {; type-declaration}
type-declarator→identifier=type
type→identifier | record field-list end
field-list→𝜺
      \end{align}

      כדאי לשים לב לכך שהכתיב של ביטויים רגולריים בגוף כלל הגזירה של EBNF הוא מעט
      שונה. למעלה, בדוגמא הזו השתמשנו בכתיב על פיו * חזרה אפס או יותר פעמים מסומנת על
      ידי עטיפה הביטוי החוזר בסוגריים מסולסלים, המעוצבים טיפוגרפית בדוגמא כך: {} כך
      למשל תת הביטוי המופיע בגופו של כלל הגזירה הראשון לעיל
      identifier {,identifier}
      מציין רשימה של אחד או יותר מזהים המופרדים בפסיקים.
      ⌘תחילת{itemize}
      ⌘⌘ ביטוי אופציונלי עטוף בסוגריים מרובעים, המעוצבים טיפוגרפית בדוגמא כך: [] כך למשל תת הביטוי
      [label-declaration;]
      מציין שה label-declaration שאחריו יש סימן ; הוא אופציונלי.
      ⌘⌘ עוד נשים לכך שהדוגמא מניחה כללי קדימות של האופרטורים היוצרים את הביטוי הרגולרי, בפרט
      identifier | record field-list end
      מתפרש כך:
      identifier | (record field-list end)
      ולא כך:
      (identifier | record) field-list end
      דקדוק ה EBNF של שְׂפַת תכנות מסוימת עשוי להשתמש בשיטת סימון מעט אחרת ואולי אף כללי קדימות אחרים. בדרך כלל, יכול הקורא הנבון להסיק את שיטת הסימון מתוך הקריאה, ואילו הקורא הסכל יאלץ לעיין בנספח, בהקדמה או בתוספת אחרת למסמך הראשי, ואשר בהם אולי תימצא הגדרה מדוייקת של שיטת הסימון.
  ⌘סוף{itemize}
