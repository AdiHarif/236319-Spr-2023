§§ קבוצות מוגדרות רקורסיבית 
סעיף 4ב' לחוק השבות, תש"י - 1950 קובע:
\צטט\עבה{לענין חוק זה, "יהודי" - מי שנולד לאם יהודיה או שנתגייר, והוא אינו בן
דת אחרת}.===
נשים לב לכך כי בבואו להגדיר את המילה "יהודי", חוק השבות משתמש בגוף ההגדרה במילה
זו עצמה. הגדרות המשתמשות במונח המוגדר כחלק מההגדרה של המונח עצמו, נקראות הגדרות
רקורסיביות.

הגדרות רקורסיביות מופיעות גם בדתות אחרות: ע"פ השריעה (ההלכה המוסלמית), מוסלמי
הוא מי שנולד לאב מוסלמי או שהפך למוסלמי באמצעות אמירת העדות, הלא היא השהאדה:
\צטט
\begin{Arabic}
  \עבה{اشهد ان لَا إِلٰهَ إِلَّا الله وان مُحَمَّدا رَسُولُ الله}
\end{Arabic}
===
(אני מעיד כי אין אלוהים לבד מאללה, וכי מוחמד הוא שליח אללה). בפני שלושה
מוסלמים. אנו רואים כי גם ההלכה המוסלמית מגדירה רקורסיבית את התשובה לשאלה "מיהו
מוסלמי?".

נאמר על קבוצה~$S$ שהיא מוגדרת באופן רקורסיבי (או בנוייה באופן רקורסיבי, או
לעיתים גם בנויה באופן אינדוקטיבי ) אם ההגדרה של~$S$ מבדילה בין שני סוגים של
איברים: איברים אטומיים ואיברים מורכבים.
איברים מורכבים נוצרים באמצעות בנאי איברים מאיברים אטומיים ואיברים מורכבים
אחרים:
\החל{ציינון}
✦ \עבה{איברים אטומיים}. בסיס הרקורסיה הוא \מונח[איבר אטומי]{איברים אטומיים},
כלומר איברים של~$S$ אשר אינם נבנים מאיברים אחרים בקבוצה. אם נביט על סעיף 4ב' של
חוק השבות כעח הגדרה רקורסיבית של קבוצת היהודים, סביר שנאמר שאברהם אבינו ושרה
אמנו הם האיברים האטומיים של הקבוצה, כלומר הם יהודים בזכות עצמם. בהסתכלות דומה
על השריעה, סביר להסיק מוחמד ואולי עוד כמה מתלמידיו, הם מוסלמים מכוח עצמם בלבד.
כל שאר המוסלמים נקבעים בדרך אחרת.

✦ \עבה{בנאי איברים}. הרקורסיה עצמנה נבנית באמצעות \מונח{בנאי איברים}, שהם כללים
המאפשרים לייצר איברים נוספים ל-$S$ מתוך איברים קיימים. איבר הנוצר על ידי בנאי
איברים, נקרא מורכב. בנאי איברים בונה איברים מורכבים של הקבוצה~$S$ מתוך איברים
אטומיים ואיברים מורכבים הקיימים בה. בהגדרה השרעית הרקורסיבית של קבוצות
המוסלמים יש שני בנאים:
\ספרר
✦ הבנאי שמאפשר לקבוע כי אדם מסויים הוא מוסלמי, אם אביו מוסלמי. בנאי זה הוא בנאי
\מונח{אונארי}, משום שבנאי זה מתחיל מאיבר יחיד בקבוצה, גבר שהוא מוסלמי, ומאפשר
"לבנות" איבר חדש מהאיבר הקיים.
✦ הכלל המגדיר כמוסלמי כמי שאמר את השהאדה בפני שלושה מוסלמים אחרים, הוא בנאי
\מונח{טרנארי} משום שבנאי זה מתסמך על שלושה איברים בקבוצה המוגדרת רקורסיבית (
הלא היא קבוצת המוסלמים), כדי לבנות איבר חדש בקבוצה.
===

גם חוק השבות מגדיר בנאי אונארי (אמהות). החוק אמנם אינו מגדיר
במדוייק מהו גיור, אך ברור כי הגדרה מדוייקת של הגיור, תכלול רקורסיה באמצעות בנאי
איברים ובפרט, ידרש כי חברי בית הדין המחליט על הגיור יהיו יהודים בעצמם.
\סוף{ציינון}

\פסקה{הערות}
\החל{אבגוד}
✦ לעיתים נתייחס לאיברים האטומיים של קבוצה מוגדרת רקורסיבית כבנאים שהם nullary,
כלומר בנאים שאינם מקבלים ארגומנטים.
✦ גדלן של קבוצות המוגדרות רקורסיביות הוא בלתי חסום בדרך כלל, שכן תמיד ניתן
להשתמש בבנאים כדי ליצור איברים נוספים.
✦ קבוצה מוגדרת רקורסיבית יכולה להיות בעלת גודל סופי:
\ציינן
✦ אם הגדרת הקבוצה מכילה יחס שקילות, שגורים לכך שהפעלה אינסופית של בנאים, יוצרת
רק אוסף סופי של איברים שקולים.
✦ אם הבנאים אינם כאלו שתמיד ניתן להפעילם.
===
\סוף
{אבגוד}
הגדרה רקורסיבית מתאפייינת גם בתכונה נוספת:
\החל{ציינון}
✦ \עבה{שלמות ההגדרה}. הגדרה רקורסיבית של הקבוצה~$S$ כוללת תמיד בתוכה מרכיב
הדורש שאין ב-$S$ איברים אחרים מלבד האיברים האטומיים ואלו שנוצרו באמצעות בנאים.
בדרך הדרישה שבמרכיב זה של אינה נאמרת במפורש, אלא משתמעת מהניסוח. כך למשל מניסוח
חוק השבות, ברור כי ההגדרה מתכוונת לאמר שמי שאינו מקיים את התנאים המנויים בסעיף,
אינו יהודי. אך הקביעה כי כל מי שאמו אינו יהודיה ושלא התגייר איננו יהודי, אינה
מופיעה בחוק כלשונה אלא משתמעת ממנו.
\סוף{ציינון}

§§ הגדרה רקורסיבית של קבוצת הפונקציות הרציונליות

נגדיר לדוגמה באופן רקורסיבי את~$ℚ₁$, קבוצת הפונקציות הרציונליות במשתנה אחד. כל
איבר~$f$ בקבוצה~$ℚ₁$ הוא פונקציה חלקית מ-ℝ אל~$ℝ$ ($ℝ$ היא קבוצת
המספרים הממשיים). כלומר~$f:ℝ⇸ℝ$. הכוונה במונח פונקציה חלקית היא שייתכן כי קיים
ערך מסויים~$ℝ∈x$, שעבורו ערך הפונקציה~$f)x($ אינו
מוגדר. אנו נשתמש בסימון~$⊥$ כדי לציין את הערך הלא מוגדר. ניתן לכן
לכתוב~$f:ℝ→ℝ∪❴⊥❵$. בניסוח אחר,~$ℚ₁⊆ℝ⇸ℝ$, כלומר~$ℚ₁$ היא קבוצה חלקית של קבוצת
הפונצקיות החלקיות מ-$ℝ$ אל~$ℝ$.

\החל{definition}\label{definition:rationals}
הקבוצה ב-$ℚ₁$, קבוצת הפונקציות הרציונליות במשתנה אחד, מוגדרת על ידי שלושת
התנאים הבאים:
\החל{ספרור}
✦ \עבה{איברים אטומיים של קבוצת הפונקציות הרציונליות}
\החל{ציינון}
✦ הפונקציה~$U$, המעתיקה כל מספר ממשי אל המספר הטבעי~$1$,
\begin{equation*}
  ∀ x∈ℝ∙ U(x)=1,
\end{equation*}
נמצאת בקבוצה~$ℚ₁$,
כלומר
\begin{equation}\label{eq:1}
  U∈ℚ₁
\end{equation}
✦ פונקצית הזהות,~$I$, המעתיקה כל מספר ממשי אל עצמו,
\begin{equation*}
  ∀ x∈ℝ∙ I(x)=x
\end{equation*}
נמצאת בקבוצה~$ℚ₁$, כלומר
\begin{equation}\label{eq:x}
  I∈ℚ₁
\end{equation}
\סוף{ציינון}
✦ \עבה{בנאים של קבוצת הפונקציות הרציונליות}
\החל{ציינון}
✦ אם הפונקציה~$f$ שייכת ל-$ℚ₁$ אזי גם הפונקציה~$-f$ שייכת לקבוצה זו, כלומר
\begin{equation}\label{eq:minus}
  -f∈ℚ₁.
\end{equation}
✦ אם שתי הפונקציות~$f₁$ ו-$f₂$ שייכות ל-$ℚ₁$ אזי גם הסכום שלהן, המכפלה
שלהן, והמנה שלהן שייכות ל-$ℚ₁$, כלומר
\begin{align}
  f₁+f₂ & ∈ℚ₁, \label{eq:plus} ⏎
  f₁·f₂ & ∈ℚ₁ \label{eq:times} ⏎
  f₁/f₂ & ∈ℚ₁. \label{eq:div}
\end{align}
\סוף{ציינון}
✦ \עבה{שלמות ההגדרה: אין פונקציות רציונליות חוץ מהאטומיות ואלו שנוצרו באמצעות
הבנאים}
⏎ הקבוצה~$ℚ₁$ היא הקבוצה הקטנה ביותר של פונקציות המקיימת את התנאים
\פנה{eq:1},
\פנה{eq:x},
\פנה{eq:minus},
\פנה{eq:plus},
\פנה{eq:times}
ו-\פנה{eq:div}.
\סוף{ספרור}
\סוף{definition}

§§ כתיב של כללי היסק

ניסוח תמציתי ומדוייק לבנאים הוא כ\מונח{כללי היסק} כפי שהם נהוגים בתחשיב
הפסוקים. כלל היסק האומר שבכל פעם שמתקיימות ההנחות~$P₁,P₂,…,Pₙ$ ניתן להסיק את
המסקנה~$Q$ יכתב כך: \[
  \dfrac{\begin{array}{c}P₁ ⏎P₂ ⏎⋮ ⏎Pₙ\end{array}}{Q}
\] ניתן גם לכתוב את הדרישות בשורה אחת, ובלבד שהן מופרדות זו מזו, \[
  \infer Q{P₁ & P₂ &⋯& Pₙ}
\] לדוגמה, את הבנאי \פנה{eq:plus} של הקבוצה~$ℚ₁$, ניתן לכתוב ככלל היסק:
\begin{equation*}
  \infer{f₁+f₂∈ℚ₁}{f₁∈ℚ₁ & f₂∈ℚ₁}
\end{equation*}
בכלל היסק זה יש שתי הנחות~$P₁=f₁∈ℚ₁$ ו-$P₂=f₂∈ℚ₁$. כל אחת מההנחות צריכה להיקרא
ככמת אוניברסלי כפי שהוא מופיע בתחשיב הפסוקים, כלומר, עבור כל בחירה של
פונקציה~$f₁$ המקיימת~$f₁∈ℚ₁$ ולכל בחירה של פונקציה~$f₂$ המקיים~$f₂∈ℚ₁$ נובעת
המסקנה~$Q=f₁+f₂∈ℚ₁$. בניסוח אחר כלל ההיסק אומר כי \[
  ∀f₁∀f₂❨f₁∈ℚ₁∧f₂∈ℚ₁→f₁+f₂∈ℚ₁❩.
\] ניתן לנסח את ארבעת הבנאים של הקבוצה~$ℚ₁$, כלומר \פנה{eq:minus},
\פנה{eq:plus},
\פנה{eq:times}
ו-\פנה{eq:div},
ככלל היסק אחד:
\begin{equation*}
  \infer{-f₁,f₁+f₂, f₁·f₂,f₁/f₂∈ℚ₁}{f₁∈ℚ₁&f₂∈ℚ₁}
\end{equation*}

ניתן גם לנסח את הגדרת האיברים האטומיים של הקבוצה~$ℚ₁$, כלומר \פנה{eq:1}
ו-\פנה{eq:x},
כשני כללי היסק אשר קבוצת ההנחות שלהן ריקה,
\begin{equation*}
  \begin{array}{ccc}
    \infer{I∈ℚ₁}{} &  & \infer{1∈ℚ₁}{}
  \end{array}\hfill
\end{equation*}
או בקיצור, ככלל אחד,
\begin{equation*}
  \infer{1, I∈ℚ₁}{}
\end{equation*}
לדוגמה, \[
  \frac {I+1}{I·I -3·I+1}
\] הוא איבר ב-$ℚ₁$,
שמיייצג את הפונקציה~$f(x)=(x+1)/(x²-3x+1)$.

ניתן להשתמש במבנה ההגדרה הרקורסיבי של קבוצה בהגדרות רקורסיביות נוספות המתייחסות
לקבוצה ולאיבריה.

\begin{definition}[ערך של פונקציה רציונלית]
עבור פונקציה רציונלית~$f∈ℚ₁$, ועבור כל מספר ממשי~$x∈ℝ$ נגדיר את~$f(x)$
רקורסיבית
\begin{equation}\label{eq:value}
  \begin{array}{cc}
    U(x)=1                            & I(x)=x ⏎ ⏎
    \infer{-f(x)=-x₁}{f(x)=x₁}        & \infer{(f₁+f₂)(x)=x₁+x₂}{f₁(x)=x₁ & f₂(x)=x₂} ⏎ ⏎
    \infer{(f₁·f₂)(x)=x₁·x₂}{f₁(x)=x₁ & f₂(x)=x₂}                         &
    \infer{(f₁/f₂)(x)=x₁/x₂}{f₁(x)=x₁ & f₂(x)=x₂}
  \end{array}
\end{equation}
\end{definition}

§§ אינדוקצית מבנה
הגדרות רקורסיביות מאפשרות לנו להוכיח טענות באינדוקציה הידועה בשם אינדוקצית
מבנה. באינדוקציה כזו, אנו מוכיחים ראשית כי הטענה נכונה עבור כל האיברים האטומיים
של קבוצה. בצעד האינדוקציה נעבור על כל בנאי האיברים: לגבי כל בנאי נניח שהטענה
נכונה לגבי כל האיברים עליהם פועל, ונוכיח כי הטענה נכונה גם עבור האיבר אשר אותו
יצר הבנאי. ניתן גם להסתכל על הוכחות באינדוקצית מבנה כאינדוקציה על מספר
ההפעלות~$n$ של בנאים לשם יצירת~$f$.

נוכיח לדוגמה את הטענה הפשוטה הבאה עבור ההגדרה הרקורסיבית של הקבוצה~$ℚ₁$
(\פנה{definition:rationals}) וההגדרה של ערך הפונקציה מעל \פנה{eq:value}.

\begin{claim}
  עבור כל מספר רציונלי~$q∈ℚ$, ועבור כל פונקציה רציונלית~$f∈ℚ₁$, מתקיים כי
  \begin{equation}\label{eq:Q}
    f(q)∈ℚ∪❴⊥❵
  \end{equation}
  כלומר~$f(q)$ אינו מוגדר, או שהוא מספר רציונלי.
\end{claim}

\begin{proof}
  \mbox{}
  \תאר
  ✦[בסיס האינדוקציה] אם~$n=0$ אז~$f$ הוא איבר אטומי של~$ℚ₁$,
  ואז~$f=1$ או~$f=I$ וברור שאם~$q$ רציונלי, אז גם~$f(q)∈❴1,q❵$. ולכן
  \פנה{eq:Q} מתקיימת עבור~$n=0$.
  ✦ [צעד האינדוקציה] נניח שהטענה \פנה{eq:Q} מתקיימת עבור כל~$n'$, כאשר~$n'<n$
  ונוכיח אותה עבור~$n$.
  נסתכל על איבר~$f∈ℚ₁$ אשר נוצר מהפעלה של~$n$ בנאים, ונניח ש-$n>0$ כלומר~$f$
  נוצר על ידי הפעלה של בנאי. בנאי זה הוא אחד מארבעת הבנאים \פנה{eq:minus},
  \פנה{eq:plus}, \פנה{eq:times}, או \פנה{eq:div}.
  מכאן, \[
    f(q)∈❴-q₁,q₁+q₂,q₁·q₂,q₁/q₂❵.
  \] כיוון שמספר הפעולות הבנאים לשם יצירת הפונקציות~$f₁$ ו-$f₂$ קטן ממש מ-$n$
  הנחת האינדוקציה מתקיימת לגביהן, ולכן, גם~$q₁$ וגם~$q₂$ חייבים להיות רציונליים
  אם הם מוגדרים, ולכן גם~$f(q)$, אם הוא מוגדר, חייב להיות מספר רציונלי.
  ===
\end{proof}

שימוש בהגדרות רקורסיביות באיפיון של שפת תכנות

הגדרות רקורסיביות משמשות לעיתים קרובות באיפיון של שפות תכנות. אוסף הביטויים
המותר לשימוש בשפה, אוסף הפקודות, ואוסף הטיפוסים, כמעט תמיד מוגדרים רקורסיבית.

§§ הגדרה קרוסיבית של ביטויים בשפות תכנות

בשפת \פסקל הביטוי
\begin{PASCAL}
(-12+sin(13.4)) * x
\end{PASCAL}
הוא ביטוי מורכב המכיל בתוכו שלושה ביטויים אטומיים \קד{12}, \קד{13.4}
ו-\קד{x}. בביטוי מופיע הבנאי הבינארי של החיבור, הבנאי הבינארי של הכפל,
ושלושה בנאים אונאריים: סימן המינוס החד מקומי (\קד{-$·$}), הפונקציה החד מקומית
\קד{sin($·$)}, וגם בנאי אונארי נוסף המאפשר לעטוף ביטוי בסוגרים. על פי בנאי
זה, אם~$E$ הוא ביטוי אזי גם \קד{($E$)} הוא ביטוי, ועל כן, כיוון
ש-\קד{-12+sin(13.4)} הוא ביטוי אזי גם \קד{(-12+sin(13.4))} הוא ביטוי.

כל הגדרה רקורסיבית של קבוצת הביטויים המותרים לשימוש בשפת תכנות מסויימת, כוללת
קביעה מי הם הביטויים האטומיים, כלומר ביטויים שאינם מכילים בתוכם ביטויים אחרים.
בדרך כלל ביטויים האטומיים הם משני סוגים: ליטרלים כגון המספר השלם \קד{21} והמספר
הממשי \קד{13.4} ושמות, כגון \קד{x}. השיערוך של ביטוי מוגדר רקורסיבית. השיערוך
של ליטרל, הוא ערכו של הליטרל, ואילו שיערוך של שם נעשה באמצעות חיפוש השמות של
הערך המתאים לשם.

מלבד הביטויים האטומיים, ההגדרה הרקורסיבית של קבוצת הביטויים המותרים לשימוש,
כוללת בתוכה גם רשימה של בנאי ביטויים. בנאי ביטויים אונארי המופיע כמעט בכל שפת
תכנות נפוץ הוא זוג סוגריים: אם~$E$ הוא ביטוי, אזי גם~$⌘)E⌘($ הוא ביטוי.
גם אופרטורים הם בנאי ביטויים נפוצים: אופרטורים אונאריים, כגון סימן המינוס החד
מקומי (\קד-) הם בנאי ביטויים אונאריים, שכן, אם~$E$ הוא ביטוי,
אזי גם~$⌘-E$ הוא ביטוי. אופרטורים בינאריים, כגון החיבור (קד+) והכפל (\קד*) הם
בנאי ביטויים בינאריים, שכן אם~$E₁$ ו-$E₂$ הם ביטויים אזי גם~$E₁⌘+E₂$
ו~$E₁⌘*E₂$ הוא ביטוי. בשפת התכנות~\CPL, כמו גם שפות תכנות אחרות, יש גם
אופטור טרנרי, על פיו אם~$E₁$,~$E₂$ ו-$E₃$ הם ביטויים אזי גם
$E₁⌘?E₂⌘:E₃$ הוא ביטוי.

בנאי הביטויים כוללים בתוכם גם פונקציות. אם~$⌘f$ היא פונקציה~$n$-מקומית
עבור ו-$E₁,E₂,…,Eₙ$ הם ביטויים אזי גם~$⌘f⌘)E₁⌘,E₂⌘,…⌘,Eₙ⌘($ הוא ביטוי.
מקרה מעניין הוא זה שבו~$n=0$. פונקציה 0-מקומית יוצרת היא בנאי nullary, שיוצרת
ביטוי אטומי ללא שימוש בביטויים אחרים.
בשפת~\CPL, הכתיב~$⌘{f()}$
המשמש לקריאה לפונקציה ללא ארגומנטים, מטעים שפונקציה זו מהווה בנאי
nullary.
בשפת פסקל, הכתיב~$⌘f$ המשמש לקריאה לפונקציה כזו מטעים שפונקציה זו היא ביטוי
אטומי.

האבחנה שגם אופרטורים וגם פונקציות הם בנאי ביטויים, מאפשרת לנו לזהות שהההבדל
ביניהם הוא דקדוקי יותר מאשר עקרוני:

\החל{אבגוד}
✦ אופרטורים נכתבים בדרך כלל בכתיב של infix, כלומר, בין הארגומנטים שלהם.
אופרטורים אונאריים יכולים להיכתב בכתיב של prefix, כלומר לפני הארגומנט שלהם,
או בכתיב של postfix, כלומר אחרי הארגומנט שלהם. לעומת זאת, פונציות נכתבות
תמיד בכתיב של prefix.
הביטוי הבא בשפת C מדגים אופרטרים הכתובים ב-prefix, postfix, ו-infix,
ופונקציות בנות 0, 1, ו-3 ארגומנטים הכתובות כולן ב-postfix.
\begin{PASCAL}
     g(a) ? f(++a,b++,h()) : a+b
\end{PASCAL}
✦ פונקציות המקבלות ארגומנט אחד או יותר נכתבות תמיד עם סימני סוגריים, ולכן, יש
רק דרך אחת לפרש ביטוי שבו הבנאים היחידים הם פונקציות. 

לעומת זאת, אורפטורים נכתבים בדרך כלל ללא סוגריים. סדר הקדימויות של האופרטורים
הוא זה הקובע את מבנה הביטוי.  הכתיב של אופרטורים קצר יותר, אך גם מחייב
שימוש בסימני סוגריים כדי להתגבר על סדר הקדימויות.

✦ השמות של פונקציות נראים בדרך כלל כמו "שמות רגילים", הכוללים בתוכם אותיות
אנגליות, ולעיתים גם ספרות, וסימנים מפרידים. לעומת זאת, שמות האופרטרים הם בדרך
כלל סדרה של סימני פיסוק. בכל זאת, ישנן שפות רבות בהן לפחות למקצת האופרטורים יש
שמות שהם מילים. למשל, שמו של האופרטור~$∧$, בשפת פסקל, הוא המילה
השמורה~$⌘{and}$.

מצד שני, קיימות שפות תכנות בהן ניתן לתת לפונקציות שמות המורכבים מסימני פיסוק
בלבד.

✦ אופרטורים בדרך כלל מוגדרים על ידי שפת התכנות, בעוד שפונקציות מוגדרות על ידי
  המתכנת. יחד עם זאת, ישנן שפות תכנות רבות המאפשרות למתכנת להגדיר אופרטורים
  חדשים שאינם מוכרים על ידי השפה. ומנגד, ישנן שפות תכנות שבהן ישנן פונקציות
  המוגדרות על ידי השפה, ולא על ידי המשתמש.
\סוף{אבגוד}

§§ פקודות
לבד מביטויים, תכניות גם מכילות פקודות אשר ביצוען מביא לשינוי של מצב התכנית. גם
גם קבוצת הפקודות מוגדרת בדרך כלל רקורסיבית, והגדרה זו כוללת בתוכה
הגדרה של בנאי פקודות ופקודות אטומיות.

פקודה אטומית יכולה להכיל בתוכו ביטוי, והביטוי יכול שיהיה ביטוי מורכב, גם אם
הפקודה עצמה היא אטומית.
לדוגמה, בשפת פסקל, פקודות אטומיות מכילות גם את פקודת ההצבה. כך, 
\begin{PASCAL}
x:=(-12+sin(13.4))*x
\end{PASCAL}
היא פקודה אטומית שבה מחושב ערכו של הביטוי \קד{(-12+sin(13.4))*x},
ולאחר החישוב, הערך מוצב אל תוך המשתנה \קד{x}.
בפקודה זו נוכל לזהות תתי מרכיבים: למשל תת-הביטוי \קד{sin(13.4}.
 בכל זאת, הפקודה היא פקודה אטומית, שכן לא ניתן לזהות בה תת-מרכיב שהוא \פקודה
 בעצמו.  באופן כללי יותר, איבר מקבוצה המוגדרת רקורסיבית הוא איבר אטומי אם
לא ניתן לזהות בתוכו תת-איבר אחד או יותר השייך לאותה קבוצה. אבל איבר אטומי אינו
בהכרח לא פריק: האטומיות נובעת מכך שלא כל שנדרש הוא שלא ניתן לזהות בתוכה תת-איבר
מאותה קבוצה.

לבד מהצבה, בשפת התכנות פסקל ישנן פקודות אטומיות משלושה סוגים נוספים:
\החל{ספרור}
✦ \עבה{קריאה לפרוצדורה}. זוהי פקודה כגון
\begin{PASCAL}
WriteLn('Hello, World')
\end{PASCAL}
אשר קוראת לפרוצדורה המוגדרת בשפה, או פרוצדורה המוגדרת על ידי המתכנת, כמו
\begin{PASCAL}
ComputeSolution(1,5,6,x)
\end{PASCAL}
✦ \עבה{פקודת קפיצה}. זוהי פקודה כגון
\begin{PASCAL}
goto 999
\end{PASCAL}
אשר בה בקרת הזרימה מועברת למקום אחר בתכנית. בדוגמה שלנו, הפקודה המסומנת בתגית
999.
✦ \עבה{הפקודה הריקה}. זוהי פקודה שאינה עושה דבר. אין צורך לכתוב דבר כדי
להשתמש בפקודה זו, והיא משמשת בעיקר לצורך בניית פקודות מורכבות
\סוף{ספרור}
בשפת פסקל יש גם בנאי פקודות. החשובים ביותר הם אלו:
\החל{ספרור}
✦ \עבה{בנאי הבלוק}. סדרה של פקודות המופרדות על ידי סימן הנקודה ופסיק (\קד{;})
והעטופה במילים \מש{begin} ו\מש{end} גם היא פקודה. בנאי הבלוק הוא בנאי רב
מקומי, שיכול לקבל מספר כלשהו של פקודות, מורכבות או אטומיות, ולבנות מהם פקודה
אחת. זו לדוגמה פקודה מורכבת הנוצרת על ידי בנאי הבלוק משתי פקודות אטומיות.
\begin{PASCAL}
begin
  a:=b;
  goto 999
end
\end{PASCAL}
נשים לב לכך שסימן הנקודה ופסיק (\קד{;}) מפריד בין פקודות ואינו חלק מהפקודה.
לכן,
\begin{PASCAL}
begin
  a:=b;
  goto 999;
end
\end{PASCAL}
היא פקודה מורכבת הנוצרת משלוש פקודות אטומיות, שהאחרונה בהן ריקה. בנאי הבלוק הוא
גם nullary, וגם \begin{PASCAL}
begin
end
\end{PASCAL}
היא פקודה.
✦ \עבה{בנאי לולאת ה-\קד{while}}. בנאי לולאת ה-\קד{while} הוא בנאי אונארי:
אם~$C$ היא פקודה ו-$E$ הוא ביטוי בוליאני, אזי גם לולאת ה\קד{while}
\begin{PASCAL}
while ⌘$E$⌘ do ⌘$C$⌘
\end{PASCAL}
היא פקודה. על פי בנאי זה,
\begin{PASCAL}
while x > sin(x) do x :=sin(x)
\end{PASCAL}
✦ \עבה{בנאי התנאי חלקי}. גם בנאי התנאי חלקי הוא בנאי אונארי. לפי בנאי זה,
אם~$C$ היא פקודה ו-$E$ הוא ביטוי בוליאני, אזי גם פקודת התנאי החלקי היא פקודה:
\begin{PASCAL}
if ⌘$E$⌘ then ⌘$C$⌘
\end{PASCAL}
✦ \עבה{בנאי התנאי המלא}. אם~$C₁$ ו-$C₂$ הן פקודות ו-$E$ הוא ביטוי בוליאני, אזי
גם פקודת התנאי המלא היא פקודה:
\begin{PASCAL}
if ⌘$E$⌘ then ⌘$C₁$⌘ else ⌘$C₂$⌘
\end{PASCAL}
היא פקודה.
/סוף{ספרור}


לךעמותבמרבית שפת תכנות היא קבוצה המוגדרת רקורסיבית, ובתוך קבוצה זו, אנו יכולים לזהות
את הפקודות האטומיות. אולם, כפי שנראה, 
לעומת זאת, ה\פקודה הבאה בפסקל,
\החל{PASCAL}
begin
c := a;
a := b;
b := c;
end
\end{PASCAL}
אינה פקודה אטומית, שכן ניתן לאתר בה מרכיב שהוא \פקודה בעצמו.


§§§ טיפוסים
במרבית שפות התכנות, קבוצת הטיפוסים בהם ניתן להשתמש בשפה, גם היא מוגדרת
רקורסיבית. לדוגמה, מערכת הטיפוסים בשפת התכנות Nִִִim, ניתן לתיאור רקורסיבי פשוט:
\החל{תיאור}
✦ [טיפוסים אטומיים] אלו הם טיפוסים שאינם מורכבים משום טיפוס אחר.
אחדים מטיפוסים אלו הם טיפוסים מוגדרים מראש בשפת Nim
\החל{תיאור}
✦ {טיפוסי מספר שלם:}
אלו כוללים את הטיפוסים
int, int8, int16, int32, ו-int64,
כמו גם את הגירסאות חסרות הסימן של טיפוסים אלו:
uint, uint8, uint16, uint32, ו-uint64.
✦ {טיפוסי מספר ממשי:} הכוללים בתוכם את הטיפוסים float, float32, ו-float64.
✦ {הטיפוס char :} הכוללים בתוכם את הטיפוסים float, float32, ו-float64.
:
והם מכילים בתוכם את הטיפוסים הבאים:
\סוף{תיאור}
✦ [בנאי טיפוסים] אלו הם טיפוסים שאינם מורכבים משום טיפוס אחר.
\סוף{תיאור}


קבוצת ה\טיפוסים שֶׁל שפת \סי (כמו גם קבוצת הטיפוסים שֶׁל שפת פסקל) אף היא קבוצה
המוגדרת רקורסיבית: ישנם טיפוסים מורכבים, אשר ניתן לזהות כי הם מורכבים מיחידות
קטנות יותר, אשר אף הן טיפוסים. בטיפוס רשומה למשל ניתן לזהות כיחידות קטנות יותר
את טיפוסי השדות הבונים את הרשומה. אנו אומרים שהטיפוס שֶׁל רשומה הוא \מונח{טיפוס
מורכב}, משום שיש בו תתי-יחידות אשר אף הן טיפוסים.

לעומת ה\מונח[טיפוס מורכב]{טיפוסים המורכבים}, ישנם טיפוסים אטומיים, כלומר
טיפוסים אשר לא ניתן לזהות בתוכם טיפוסים אחרים. הטיפוס שֶׁל מספרים שלמים או הטיפוס
שֶׁל מספרים ממשיים, הם דוגמאות לטיפוסים כאלו.

המילה השמורה \מש{int} בשפת \סי \מזהה את הטיפוס האטומי שֶׁל מספר שלם.  מילים
שמורות המשמשות כ\מזהים נקראות \מונח{מזהה שמור}.

ניתן להשתמש בשיטת ההגדרה הרקורסיבית, כדי להגדיר את קבוצת הפקודות בפסקל,

\החל{ספרור}
✦ \עבה{פקודות אטומיות} פקודות אטומיות הן פקודות שאינן בנויות מפקודות אחרות. בשפת \סי יש שני סוגים של פקודות אטומיות,
\החל{ספרור}
✦ \עבה{פקודה ריקה} פקודה שאינה מבצעת דבר, נקראת הפקודה הריקה. הפקודה הריקה נכתבת בשפת אמצעות סימן הנקודה ופסיק~\cc{;}
✦ \עבה{פקודת ביטוי} בשפת \סי, כל ביטוי שאחריו מופיע סימן הנקודה ופסיק~\cc{;}
\begin{CPP}
  a; f(); b=2; 2;++i; 1-1;
\end{CPP}
\סוף{ספרור}
קריאה לפרוצדורה
כללי היצירה העיקריים של הפקודות הם:
שרשור של פקודות המופרדות על ידי סימן הנקודה ופסיק (;) הוא פקודה.
פקודת תנאי, כפי שתוארה לעיל, היא פקודה, אשר מכילה בתוכה פקודה אחת או שתיים.
פקודת תנאי רבת ראשים המוגדרת באמצעות המילה השמורה case
לולאות המתארות ביצוע איטרטיבי של פקודה (אטומית או מורכבת), אף הן פקודות. יש בְּPascal שלושה סוגים של פקודות לולאה מורכבות:
for
while
repeat until
\סוף{ספרור}
\סוף{ספרור}

string, char, ו-bool,
וכו', וישנם בנאי טיפוסים המאפשרים להגדיר טיפוסים מורכבים מתוך טיפוסים
סוף{ציינון}

מערכים בִּשְׂפַת Pascal הם טיפוסים מורכבים: בנאי המערך קיבל כפרמטר את טיפוס של תא
במערך, והחזיר בתמורה את טיפוס המערך של תאים מאותו טיפוס.†{למען השלמות יש לציין
שבנאי המערך מקבל עוד שני פרמטרים, שהם קצוות המערך. ואפשר להפליג ולצין שבנאי של
מערך רב מימדי מקבל מספר זוגות של פרמטרים כנ"ל.}
איבר אטומי הוא איבר שנוצר מבנאי שלא קיבל כפרמטרים ערכים של הקבוצה.

ניתן להסתכל על הטיפוסים אטומיים מכוח עצמם: Character, Integer, Real, Boolean
כעל בנאים שאינם מקבלים פרמטרים כלל. טיפוסים מנויים (enumerated types) בִּשְׂפַת
פסקל, הם גם טיפוסים אטומיים (איברים אטומיים של קבוצת הטיפוסים) . הם נוצרו מבנאי
שמקבלל רשימה של תגיות. תגית היא מזהה חוקי בִּשְׂפַת פסקל.
