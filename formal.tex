אלפאבית הוא קבוצה, בדרך כלל סופית, של איברים הקרויים אותיות או סמלים. לדוגמה
הקבוצה
\begin{equation*}
❴⌘{a},⌘{b},⌘{c}❵,
\end{equation*}
הינה אלפאבית המכיל שלוש אותיות,~$⌘a$,~$⌘b$ ו-$⌘c$.

לאותיות האלפאבית אין משמעות מלבד העובדה שכולן שונות זו מזו. בפרק זה נשתמש בגופן
שונה כדי להבדיל בין האות עצמה, ובין מה שהאות מציינת: לכן, הכתיב~$⌘a$ מכוון אל
האות הראשונה באלפאבית הלטיני, ולא למה שאות זו מציינת, ואילו הכתיב~$a$ יתייחס
למה שהאות הראשונה מציינת:
למשל, במשוואה הריבועית \[
  a x²+bx+c=0
\] הכתיב~$a$ מתייחס למקדם של~$x²$.

\דוגמה |האלפאבית של שפת~\CPL|
האלפאבית~$Σ_C$ המשמש לכתיבת תכניות בשפת~\CPL מכיל 95 תווים
אותם אפשר לחלק, לשם נוחות, לקבוצות הבאות:
\begin{equation}\label{alpahet:C}
  Σ_C=
  Σ_{\text{upper}}∪ Σ_{\text{lower}}∪
  Σ_{\text{digit}}∪
  Σ_{\text{special}}∪
  Σ_{\text{space}}.
\end{equation}
\החל{ספרור}
✦ \ע|26 אותיות אנגליות גדולות| \[
  Σ_{\text{upper}}=❴⌘A,⌘B,⌘C,⌘D,⌘E,⌘F,⌘G,⌘H,⌘I,⌘J,⌘K,⌘L,⌘M,⌘N,⌘O,⌘P,⌘Q,⌘R,⌘S,⌘T,⌘U,⌘V,⌘W,⌘X,⌘Y,⌘Z❵.
\] ✦ \ע|26 אותיות אנגליות קטנות| \[
  Σ_{\text{lower}}=
  ❴⌘a,⌘b,⌘c,⌘d,⌘e,⌘f,⌘g,⌘h,⌘i,⌘j,⌘k,⌘l,⌘m,⌘n,⌘o,⌘p,⌘q,⌘r,⌘s,⌘t,⌘u,⌘v,⌘w,⌘x,⌘y,⌘z❵.
\] ✦ \ע|10 ספרות| \[
  Σ_{\text{digit}}=❴⌘0,⌘1,⌘2,⌘3,⌘4,⌘5,⌘6,⌘7,⌘8,⌘9❵.
\] ✦ \ע|29 אותיות מיוחדות| \[
  Σ_{\text{special}}=
  Σ_{\text{punctuation}}∪
  Σ_{\text{wrapping}}∪
  Σ_{\text{arithmetic}}∪
  Σ_{\text{other}}∪
  Σ_{\text{space}}.
\] המתחלקות באופן הבא:
\החל{ספרור}
✦ \ע|8 אותיות פיסוק| \[
  Σ_{\text{punctuation}}=❰⌘., ⌘,, ⌘?, ⌘!, ⌘:, ⌘;, ⌘', ⌘"❱.
\] ✦ \ע|6 אותיות אריתמטיות| \[
  Σ_{\text{arithmetic}}=❰⌘+, ⌘*, ⌘/, ⌘-, ⌘<, ⌘>❱.
\] ✦ \ע|6 אותיות סוגריים| \[
  Σ_{\text{wrapping}}=❰⌘(, ⌘), ⌘[ ⌘], ⌘❴, ⌘❵,❱.
\] ✦ \ע|9 אותיות אחרות| \[
  Σ_{\text{other}}=❰⌘&,⌘\textbackslash, ⌘\textasciicircum, ⌘\_, ⌘|, ⌘∿, ⌘\$,
  ⌘\%, ⌘#,❱.
\] \סוף{ספרור}
✦ \ע|6 אותיות רווח| \[
  Σ_{\text{space}}=❰\text{space},\text{tab},
  \text{horizontal tab}, \text{new line},
  \text{vertical tab}, \text{form feed}❱.
\] היצוג הגרפי של אותיות אלו הינו בלתי נראה, ולכן
כתבנו כאן את שמות האותיות, ולא את היצוג הגרפי שלהן.
\סוף{ספרור}

מילה מעל האלפאבית היא סדרה סופית של אותיות מתוך האלפאבית. למשל,~$⌘{caba}$ היא
מילה בת ארבע אותיות מעל האלפאבית~$❴⌘a,⌘b,⌘c❵$. בהינתן אלפאבית~$Σ$, נסמן
ב-$Σ^*$ את הקבוצה האינסופית המכילה את כל המילים באורך סופי מעל~$Σ$, לרבות
המילה הריקה, אותה בדרך כלל מסמנים ב-$ε$. בדוגמה שלנו
\begin{equation}
  ❴⌘a,⌘b,⌘c❵^*=❴ε,⌘a,⌘b,⌘c,⌘{aa},⌘{ab},⌘{ac},⌘{ba},⌘{bb},⌘{bc},⌘{ca},⌘{cb},⌘{cc},⌘{aaa},⌘{aab},…❵
\end{equation}

ניתן גם להגדיר את~$Σ^*$ רקורסיבית. בהגדרה זו, יהיה איבר אטומי אחד, המילה
הריקה~$ε$, ובנאי אונארי שמאפשר להאריך כל מילה ב-$Σ^*$ באות מתוך~$Σ$:

\החל{definition}[המילים מעל אלפאבית]
בהנתן אלפאבית~$Σ$ אזי~$Σ^*$, קבוצת ה\ע|המילים הפורמליות| מעל~$Σ$, מוגדרת
באמצעות הבנאי הנולארי (המגדיר איבר אטומי אחד ויחיד)
\begin{equation}
  \infer{ε∈Σ^*}{}
\end{equation}
והבנאי האונארי:
\begin{equation}
  \infer{wσ∈Σ^*}{w∈Σ^* &σ∈Σ}
\end{equation}
\סוף{definition}

בהסתמך על הגדרה רקורסיבית זו, נגדיר רקורסיבית את~$|w|$, מספר התווים במילה~$w$:
\החל{definition}[אורך מילה]\label{definition:length}
עבור~$w∈Σ^*$
\begin{equation}
  |w|=\begin{cases}
    |w'|+1 & w=w'σ ⏎
    0 & w=ε. ⏎
  \end{cases}
\end{equation}
\סוף{definition}

כך נקבל ש-$|ε=0|$,~$|⌘a|=1$,~$|⌘{caa}|=3$.

שפה פורמלית~$L$ מעל~$Σ$ היא אוסף של מילים הלקוחות מ-$Σ^*$, כלומר~$L⊆Σ^*$.

\דוגמה|שפת התכנות~\CPL כשפה פורמלית|
שפת התכנות~\CPL מגדירה שפה פורמלית~$L₀$ מעל האלפאבית
$Σ_C$ \פנה|eq:alphabet:C|:
נאמר על מילה~$w$
(כאשר~$w∈Σ_C^*$),
כי היא שייכת לשפה~$L₀$
אם ורק אם~$w$ היא תכנית חוקית בשפת~\CPL.

למעשה, כל שפת תכנות מגדירה גם שפה פורמלית. זוהי השפה אשר מילותיה הן תכניות
חוקיות בשפת התכנות. אולם, הגדרת שפת תכנות אינה מצטמצמת להגדרת השפה הפורמלית
הזו. הגדרת שפת התכנות כוללת גם מתן משמעות לכל תכנית חוקית.

§§ קבוצת הפונקציות הרציונליות כשפה פורמלית

נגדיר לדוגמה באופן רקורסיבי את השפה הפורמלית~$L₁$ שכל מילה בה יכולה להתפרש
כאיבר ב-$ℚ₁$.
\אבגד
✦ המילים האטומיות בשפה זו יהיו האותיות הבודדות~⌘U ו-⌘I. כמילים בשפה~$L₁$, לא
תהיה למילים אלו משמעות, כשנגדיר פונקציה המעניקה משמעות לכל מילה בשפה
הפורמלית~$L₁$, המשמעות של~⌘U תהיה פונקצית היחידה, והמשמעות
של~⌘I תהיה פונקצית הזהות.
✦ בנוסף נשתמש בסימנים~⌘/,~⌘*,~⌘+, ו-⌘- בתוך בנאי המילים. בשפה הפורמלית לא תהייה
לסימנים אלו משמעות, אך כשנגדיר פונקציה המעניקה משמעות לכל מילה בשפה
הפורמלית~$L₁$, פונקציה זו תפרש
ארבעה סימנים אלו כאופרטורים האריתמטיים.
✦ על ששת הסימנים האלו נוסיף גם את הסימנים ⌘(ו-⌘) בכדי להבטיח שתהיה רק דרך
אחת לתת משמעות למילה בשפה הפורמלית~$L₁$.===

\החל{definition}[השפה הפורמלית של הפונקציות הרציונליות]
\label{definition:L1}
השפה~$L₁$, היא שפה פורמלית מעל האלפאבית
\begin{equation}\label{eq:Q:alphabet}
  ❴⌘U, ⌘I, ⌘(, ⌘), ⌘/, ⌘*, ⌘+, ⌘-❵
\end{equation}
המוגדרת על ידי שני איברים אטומיים
\begin{align}
   & \infer{⌘U∈L₁}{} ⏎
   & \infer{⌘I∈L₁}{}
\end{align}
ועל ידי ארבעה בנאים:
\begin{align}
  & \infer{⌘)⌘-w⌘(∈L₁}{w∈L} \label{eq:Q:minus}⏎
   & \infer{⌘)w₁⌘+w₂⌘(∈L₁}{w₁∈L₁ & w₂∈L₁}\label{eq:Q:plus}⏎
   & \infer{⌘)w₁⌘*w₂⌘(∈L₁}{w₁∈L₁ & w₂∈L₁}\label{eq:Q:times}⏎
   & \infer{⌘)w₁⌘/w₂⌘(∈L₁}{w₁∈L₁ & w₂∈L₁}\label{eq:Q:div}
\end{align}
\סוף{definition}
כדאי לשים לב להבדלים בין הגדרה זו ובין ההגדרה הקודמת של הקבוצה~$L₁$
(\פנה|definition:rational|). הגדרה הנוכחית אינה מניחה ידע במתימטיקה או בפעולות
החשבון. איבר של הקבוצה היא סדרה ללא משמעות אותיות הלקוחה מהאלפאבית
\פנה|eq:Q:alphabet|.


  הנה עוד כמה שפות פורמליות מעל האלפאבית~$❴⌘a,⌘b,⌘c❵$:
שלא תשתננה גם אם תיקראנה מסופן.
\ספרר
✦ השפה~$L₄$ שפת המילים שהאותיות שלהן מופיעות בסדר אלפאביתי לא יורד.
✦ השפה~$L₅$ שפת המילים שמכילות חמש אותיות או יותר שאף אחת מהן אינה~$⌘b$.
✦ השפה~$L₆$ שפת המילים שכל אותיותיהן שונות.
===

§§ ההיררכיה של חומסקי

שפה פורמלית יכולה להיות סופית או אינסופית. שפה סופית ניתנת תמיד לתאור מדוייק
באמצעות מניית כל המילים בשפה. ברשימה לעיל כל השפות הן שפות אינסופיות מלבד השפה
האחרונה~$L₆$ המכילה בדיוק 13 מילים
\begin{equation}\label{eq:L6}
  L₆=❴ε,⌘a,⌘b,⌘c,⌘{ab},⌘{ac},⌘{ba},⌘{bc},⌘{ca},⌘{cb},⌘{abc},⌘{acb},⌘{bac},⌘{bca},⌘{cab},⌘{cba}❵.
\end{equation}
את השפה האינסופית~$L₅$ ניתן להגדיר תוך שימוש ב\פנה|definition:length|
\begin{equation}\label{eq:L5}
  L₅=❴w \,|\, w∈Σ^*, |w|≥5❵.
\end{equation}

כל ההגדרות המדוייקות של השפות הפורמליות
$L₆$ \cref{eq:L6},
$L₅$ \cref{eq:L5},
ו-~$L₁$
(\cref{definition:L1})
היו שונות זו מזו,
וכולן היו, במובן מסויים אד-הוק.
כלומר, בחרנו בעבור כל שפה פורמלית, בשיטת הגדרה המתאימה לה. הגדרות של שפות תכנות משתמשות לעיתים קרובות בהגדרות אד-הוק, אבל לעיתים קרובות יותר, הן משתמשות במנגנונים כלליים להגדרת שפות פורמליות.
ארבעת המנגנונים העיקריים הם:
\החל{ספרור}
✦ \ע|ביטויים רגולריים| (\E|regular expressions|)
אותם נכיר ב-\cref{section:regular}.
✦ \ע|שפות חסרות הקשר| (\E|regular expressions|)
אותם נכיר ב-\cref{section:regular}.

קבוצת כל השפות מעל~$Σ$ היא לכן קבוצת כל הקבוצות החלקיות של~$Σ^*$, כלומר קבוצת
החזקה של~$Σ^*$, אותה נסמן ב-$𝒫Σ^*$. נסמן ב-\textbf{Finite} את קבוצת כל השפות
הסופיות. ראינו כבר כי \[
  L₆∈\text{\textbf{Finite}}.
\] נסמן ב-\textbf{Regular} את קבוצת כל השפות הפורמליות הרגולריות, כלומר
השפות שאותן ניתן לתאר באמצעות ביטוי רגולרי.
כאמור \[
  L₃,L₄,L₅∈\text{\textbf{Regular}}.
\] נסמן ב-\textbf{CFG} את קבוצת כל השפות הפורמליות חסרות ההקשר, כלומר אותן
השפות שאותן ניתן לתאר באמצעות דקדוק חסר הקשר.
כאמור \[
  L₁,L₂∈\text{\textbf{CFG}}.
\] מתברר כי:
\begin{equation*}
  \text{\bfseries Finite}⊊\text{\bfseries Regular}⊊\text{\bfseries CFG}⊊
  \text{\bfseries CSG}⊊𝒫Σ^*.
\end{equation*}
כלומר, ניתן לתאר את כל השפות הסופיות באמצעות ביטוויים רגולריים, אך לא כל השפות
שאפשר לתאר אותן בביטוי רגולרי הן סופיות. בנוסף, כל שפה שאפשר לתאר באמצעות
ביטויים רגולריים, ניתן גם לתאר באמצעות דקדוק חסר הקשר, אך יש שפות שניתן לתאר
באמצעות דקדוקים חסרי הקשר, ושאי אפשר לתאר באמצעות ביטויים רגולריים. יתירה מכך,
ישנן שפות אותן לא ניתן לתאר באמצעות דקדוקים חסרי הקשר.

עוד מתברר כי
\begin{equation}
  L₀∉\text{\bfseries CFG}
\end{equation}
כלומר, לא ניתן לתאר את שפת~\CPL באמצעות דקדוקים חסרי הקשר.

ביטויים רגולריים, אותם נכיר ב\cref{section:regular} מאפשרים להגדיר שפות
פשוטות כגון~$L₄$ ו-$L₅$.
✦ \ע|דקדוקים חסרי הקשר רגולריים|
✦ \ע|דקדוקים תלויי הקשר|
✦ \ע|הגדרה באמצעות תכנית|
\סוף{ספרור}

הגדרה פורמלית של השפות~$L₁$ ו-$L₂$ דורשת שימוש במנגנון שנכיר בהמשך, דקדוקים
חסרי הקשר (\E|context free grammars|). הגדרה מדיוקת של השפות~$L₃$ ו-$L₄$ דורשת
שימוש במנגנון אחר, ביטויים רגולריים
מרבית השימושים בפועל מוסיפים תַּחְבִּירִי סֻכָּר כגון:


\begin{figure}[H]
  \centering
\begin{tikzpicture}
\node[above,ellipse,minimum height=12em,minimum width=24em,draw,fill=yellow,opacity=1] (f) {};
\node[above,ellipse,minimum height=10em,minimum width=20em,draw,fill=magenta,opacity=1] (e) {};
\node[above,ellipse,minimum height=8em,minimum width=16em,draw,fill=orange,opacity=1] (d) {};
\node[above,ellipse,minimum height=6em,minimum width=12em,draw,fill=olive,opacity=1] (c) {};
\node[above,ellipse,minimum height=4em,minimum width=9em,draw,fill=green,opacity=1] (b) {};
\node[above,ellipse,minimum height=2em,minimum width=6em,draw,fill=red,opacity=1] (a) {Finite};


\path (a.north) node[above] {Regular}
    (b.north) node[above] {Context Free}
    (c.north) node[above] {Context Sensitive}
    (d.north) node[above] {Recursively Enumerable}
    (e.north) node[above] {$\wp \Sigma^*$};

\draw[label distance=-4pt] (c.north) ++ (4em,-2em) node[minimum size=3pt,shape=circle,inner sep=0pt,fill=blue,draw=black,label=60:\scriptsize$L_2$]{};
\draw[label distance=-4pt] (a.north) ++ (1.5em,-1.4em) node[minimum size=3pt,shape=circle,inner sep=0pt,fill=blue,draw=black,label=60:\scriptsize$L_6$]{};
\draw[label distance=-4pt] (b.north) ++ (1.5em,-1.4em) node[minimum size=3pt,shape=circle,inner sep=0pt,fill=blue,draw=black,label=60:\scriptsize$L_4$]{};
\end{tikzpicture}

  \caption{ההיררכיה של חומסקי}
\end{figure}

