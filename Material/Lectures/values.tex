\documentclass[aspectratio=169]{beamer}
% Remove navigation bar
\setbeamertemplate{navigation symbols}{}
% Tikz package
\usepackage{tikz}
\usetikzlibrary{positioning}


\begin{document}
\begin{frame}{Values}
\begin{description}
    \item [Values] runtime creatures, passed as parameter, returned from functions, stored in variable
    \item [Types] Classification of values, mostly compile time creatures 
    \item [Variables] Where values are stored, mostly runtime 
\end{description}
\end{frame}

\begin{frame}{Forms of Values}
\begin{description}
    \item [Atomic] do not contain other values. 
    \item [Compound] composed from other (compound or atomic) values.
\end{description}
\end{frame}

\begin{frame}[fragile]{Atomic Values \#1/2}
Characterize the programming language
\begin{description}
\item [Numeric] extremely typifying; may be non-scalar, e.g., primitive 2D points; lots of operations.  
    \begin{enumerate}
        \item Only approximate the sets $\mathbb Z$ and $\mathbb S$; 
        \item Balance efficiency (hardware match) and portability
    \end{enumerate}
\item [Boolean] typifying, less essential than numeric, several operations
\item [Symbolic:] only operation is comparison, 
\end{description}
\end{frame}

\begin{frame}[fragile]{Atomic Values \#2/2}
\begin{description}
\item [Numeric] \ldots
\item [Boolean] \ldots
\item [Symbolic] our main interest
\item [Meta] encapsulate meta data about the computation
\begin{description}
    \item Function values as in C, 
    \begin{verbatim}
int fibonnaci(int n) { 
    int (*fibo)(int whatever_optional_name) = fibonnaci;
    return n <= 2 ? 1 : fibo(n-1) + fibonacci(n-2);
    \end{verbatim}
    \item Closure values as in ML
    \item Generator values as in Python
    \item Class values as in \begin{verbatim}
        Class<?> c = new Foo().bar().class 
    \end{verbatim}
    in Java
    \item Reference values as in C++, not values, but refer to a
    a cell in memory that contains a value.
    \begin{verbatim}
        max(int &a, int &b) { return a > b ? a : b; }
    \end{verbatim}
\end{description}
\end{description}
\end{frame}

\begin{frame}{Compound Values}
{Characterize the programming language}
    \begin{itemize}
        \item Non-Recursive: constructed by tuple, branding, disjoint union, maps (tabular, non-functions) from other values.  
        \item Recursive (contain values of the same ``kind"): lists, lists of lists, trees and DAGs; require recursive type constructor, e.g., datatype in ML 
        \item Circular: contain directed loops, e.g., linked lists: require recursive type constructor
    \end{itemize}
\end{frame}

\begin{frame}{Symbolic Values}
Atomic value, which can only compared to another atomic value;
    \begin{itemize}
        \item Define a set $\Sigma$ of a distinct symbols/letters
        \item $\Sigma$ may be finite, e.g., the character type in many languages 
        \item $\Sigma$ may be infinite (unbounded), e.g., the string type in many languages, e.g., atoms in Mini-Lisp. 
        \item $\Sigma$ may even be a singleton; the only value of the unit type; one can design a programming language 
        without any ``atomic value"
    \end{itemize}
Only operation: Given $\sigma_1, \sigma_2\in\Sigma$ determine whether $\sigma_1=\sigma_2$; in the case $\Sigma$ is a singleton, this
operation always returns true.
\end{frame}

\begin{frame}{Symbolic Data Structures}
Compound values; built from symbolic values; ladder of generality
\scriptsize
\begin{columns}
        \column{0.5\textwidth}
    \begin{enumerate}
        \item $\Sigma^*$ The set of all strings (lists) of letters from the alphabet; used
        in programming languages such as Snobol, Bash, Makefile, 
        \begin{itemize}
            \item $\epsilon \in \Sigma^*$
            \item $\Sigma \subset \Sigma^*$; if $\sigma\in\Sigma$ and $\alpha\in \Sigma^*$, then $\sigma\alpha\in\Sigma^*$ 
        \end{itemize}
    \item $\Sigma^R$ set of all ``ropes" over $\Sigma$
        \item $\Sigma^T$ The set of all trees over $\Sigma$, e.g., Prolog
        \begin{itemize}
        \item If $t_1, t_2, \ldots,t_n \in \Sigma^T$ for  $n\ge0$, 
            and $\sigma \in \Sigma$ 
            then $\sigma(t_1,\ldots,t_n) \in \Sigma^T$  
        \end{itemize}
        \item  $\Sigma^S$, trees with ``typing''/signature, define set of expressions in a typed programming language, let $ \Sigma = \Sigma_0 \cup \Sigma_1\cup\Sigma_2\cdots$ 
         then for all $n\ge0$, $\sigma(t_1,\ldots,t_n) \in \Sigma^S$, 
            but  only if $\sigma\in\Sigma_n$ 
        \end{enumerate}
        \column{0.5\textwidth}
        \begin{enumerate}\setcounter{enumi}{4}
        
        \item $\Sigma^L$ set of all items and lists of items/lists, e.g., Mini-Lisp without 
        dotted pairs
        \begin{itemize}
        \item $\Sigma\subset\Sigma^L$
        \item If $s_1,\ldots,s_n \in\Sigma^L$, then the list
            $(s_1 \ldots s_n \in \Sigma^L$.
        \end{itemize}
    \item $\Sigma^X$ set of all S-expressions over $T$: 
        \begin{itemize}
        \item $\Sigma\subset\Sigma^X$, $\epsilon\in \Sigma^X$
        \item If $\sigma_1,\ldots,\sigma_n \in\Sigma$, then the list
            $(\sigma_1 \ldots \sigma_n \in \Sigma^L$.
        \end{itemize}
    \item $\Sigma^G$ set of all directed graphs with labels from $\Sigma$ 
        \end{enumerate}
    \end{columns}
\end{frame}

\begin{frame}{Embedding of Symbolic Data Structure}
Embedding, is also emulation:
\begin{itemize}
    \item Strings $\Sigma^*$ can emulate the alphabet $\Sigma$ (but not vice versa)
    \item Ropes $\Sigma^R$ can emulate strings $\Sigma^*$ (and vice versa)
    \item Trees $\Sigma^T$ can emulate ropes $\Sigma^R$ 
    \item Signature tree $\Sigma^T$ can emulate signature trees $\Sigma^S$ (and vice versa)
    \item List of lists/items $\Sigma^L$ contains $\Sigma^T$ 
    \item S-expression $\Sigma^X$ can emulate $\Sigma^L$ 
    
\end{itemize}

All these can be defined and may be useful over a singleton alphabet. 
\end{frame}

\begin{frame}{Definition of S-Expressions}
\begin{itemize}
\item $\epsilon \in S^X$, $\epsilon \not \in\Sigma$
\item $\Sigma \in S^X$
\item If $x_1,x_2\in\Sigma^S$ then $[x_1.x_2]\in S^x$ (dotted pair)
\end{itemize}
\end{frame}
\begin{frame}{Embedding of Symbolic Data Structure}
Embedding, is also emulation:
\begin{itemize}
    \item Strings $\Sigma^*$ can emulate the alphabet $\Sigma$ (but not vice versa)
    \item Ropes $\Sigma^R$ can emulate strings $\Sigma^*$ (and vice versa)
    \item Trees $\Sigma^T$ can emulate ropes $\Sigma^R$ 
    \item Signature tree $\Sigma^T$ can emulate signature trees $\Sigma^S$ (and vice versa)
    \item List of lists/items $\Sigma^L$ contains $\Sigma^T$ 
    \item S-expression $\Sigma^X$ can emulate $\Sigma^L$ 
    
\end{itemize}

All these can be defined and may be useful over a singleton alphabet. 


\end{frame}

\begin{frame}{Manipulation of S-Expressions}
Operations:
\begin{itemize}
\item Cons: full binary operator 
\item Car, Cdr: partial unary operators; ``failure'' is a meta-value
\end{itemize}
Predicates
\begin{itemize}
\item Atom: determine whether $x$ is an atom  
\item Eq: treat two atoms as symbol and check for equality
\end{itemize}
\end{frame}

\begin{frame}{Semantics of S-Expressions}
Recursive definition, if expression $x\in S^X$ is
\scriptsize
\begin{itemize}
\item $x = \epsilon\not\in\Sigma$ (\texttt{NIL} in Lisp): semantics is $\epsilon$ 
    (\texttt{NIL} in Lisp)
    
\item $x = \sigma\in\Sigma$,  atom: look it up in a symbol table (initially very minimal, or even empty) 
\item Cons of $x_1$ and $x_2$, i.e., $x= [x_1,x_2]$, apply semantics of $x_1$ as function to evaluated argument $x_2$
\item Recursion stops if the function or atom are pre-defined; very small set of pre-defined functions and atoms.
\end{itemize}
In $\Sigma^L$ , given $\ell\in\Sigma^*$
\begin{enumerate}
\item $\ell = \epsilon\not\in\Sigma$, the empty list;  semantics is $\epsilon$ 
    is as in $S^X$
\item $\ell = \sigma\in\Sigma$,  semantics  is as in $S^X$, lookup
\item Non-empty list $(s_0,s_1,\ldots s_n$ (where $n\ge0$),
    then apply function $s_0$ to list of arguments $(s_1,\ldots s_n$ (where $n\ge0$),
\end{enumerate}
In trees , e.g., evaluation of expression in C, Pascal, etc.
\begin{enumerate}
\item if $t=\sigma\in\Sigma^T$, lookup as in $S^X$
        \item If $t= \sigma(t_1,\ldots,t_n) \in \Sigma^T$ semantics
        of function $\sigma$ (as found in lookup) applied to $t_1,\ldots,t_n$
\end{enumerate}
Main concept: function \emph{application}; parameter passing to function application
make it possible define and then to lookup functions/atoms/symbols which are not pre-defined.
\end{frame}

\begin{frame}{The Evaluation Function}
Computes the semantics of an S-expression
\begin{itemize}
    \item A mathematical function over the set $\Sigma^X$
    \item Returns another value in $\Sigma^X$
    \item Not all $x\in S^X$ have semantics
    \item Can be implemented by a function/functions in a programming 
    \item Mathematically partial function: value may be undefined 
    on some $x\in\S^X$; only two outcomes:
    \begin{enumerate}
        \item Another member of $S^X$
        \item Failure
    \end{enumerate}
        There is no notion of order of evaluation
    \item In programming language may fail, invoking the function: many outcomes:
    \begin{enumerate}
        \item Another member of $S^X$
        \item Failure with an error message (there could be many different errors)
    \end{enumerate}
        Order of evaluation may matter may matter
\end{itemize}
\end{frame}

\begin{frame}[fragile]{The Evaluation Algorithm in C}
Some atoms are pre-defined.
\begin{verbatim}
  S eval(S s) { 
    return s.atom() ? lookup(s) : apply(s.car(), s.cdr();
  }
}
\end{verbatim}
Structure of functions: a list of 3 items:
    \begin{enumerate}
        \item Special tag, identifying function type
        \begin{itemize} 
           \item Is this function atomic (pre-defined), or should it searched.
           \item Is this function (pseudo) normal, or it is strict
        \end{itemize} 
        \item Arguments:
        \begin{itemize} 
           \item List of atoms: names of arguments
           \item Single atom: name of list of arguments 
        \end{itemize} 
    \end{enumerate}
\end{frame}

\begin{frame}[fragile]{The Application Algorithm in C}
\begin{verbatim}
S apply(S f, S actuals) {
    S before = alist;
    S lambda = f.eval(); 
    S result = apply(lambda.$1$(), lambda.$2$(), lambda.$3$(), actuals);
    alist = before;
    return result;
  }
}
\end{verbatim}
\end{frame}

\begin{frame}[fragile]{Auxiliary Apply}
\begin{verbatim}
  S apply(S tag, S formals, S body, S actuals) {
    S arguments = map(actuals, normal(tag) ? identity : eval);
    if (formals.atom())   
      bind_item(formals, arguments);
    else { M("list of arguments",actuals);
      align(formals, actuals, MISSING_ARGUMENT, REDUNDANT_ARGUMENT);
      bind_list(formals, arguments);
    }
    return (native(tag) ? exec : eval)(body);
  }
\end{verbatim}
\end{frame}

\begin{frame}[fragile]{Aligning Lists}
\begin{verbatim}
    S align(S s1, S s2, S e1, S e2) { return  
      s1.null()  &&  s2.null() ?  NIL0 : 
     !s1.null()  &&  s2.null() ? s1.error(e1) : 
      s1.null()  && !s2.null() ? s2.error(e2) : 
        align(s1.cdr(), s2.cdr(), e1, e2);  
    }
\end{verbatim}
\end{frame}

\begin{frame}[fragile]{Binding L}
\begin{verbatim}
    void bind_list(S formals, S arguments) {
      if (formals.null() || arguments.null()) return; 
      bind_item(formals.car(), arguments.car());
      bind_list(formals.cdr(), arguments.cdr());
    }
\end{verbatim}
\end{frame}

\begin{frame}[fragile]{The A List and Item Binding}
\begin{verbatim}
  extern S alist = NIL;
  extern S bind_item(S k, S v) { return alist = k.cons(v).cons(alist), v; }

\end{verbatim}
\end{frame}

\begin{frame}[fragile]{The A List and Item Binding}
\begin{verbatim}
  extern S lookup(S id) { 
    for (S s = alist; s.pair(); s = s.cdr())
        if (s.car().car().eq(id))
          return s.car().cdr(); 
    return lookup(id, globals); 
  } 
  
  extern S lookup(S id, S list) {
    return 
      list.null() ?  id.error(UNDEFINED_ATOM): 
      list.car().car().eq(id) ? list.car().cdr() : 
      lookup(id, list.cdr()); 
  }
\end{verbatim}
\end{frame}
\end{document}

 
