שעת לילה מאוחרת, סטודנטית צעירה משתמשת בתכנת הדואר האלקטרוני שלה 
כדי לכתוב מכתב לאהובה. 
היא מתקנת את מיליה, 


יש תכניות מחשב מופלאות באמת. 
אמנם התרגלנו כבר לנפלאות ה⌘דואל, אך אנו עדיין עומדים משתאים 
   לנוכח תכניות המסוגלות להביס מתחרים אנושיים בשעשועוני ידע, 
   או הבוראות עולם וירטואלי, תלת-מימדי חי נושם ופועם. 
אך עם כל אלו, עלינו לזכור, כי אחרי ככלות הכל, תכנית מחשב אינה אלא
  יותר מהגשמה של פונקציה מתימטית: 
  תכנית מחשב, כל תכנית מחשב, מתרגמת ⌘קלט ל⌘פלט.
  
התזה של ⌘צרץ ו⌘טיורינג, מנסה לאפיין מה


מדוע עלינו ללמוד שפות תכנות?
⌘תחילת{ציינון}
• כדי שנוכל ללמוד מהר שפות תכנות חדשות ולהיות פרודוקטיביים בהם
• לבעיות מסוימות מתאימות שפות מיוחדות
• פרדיגמות מעשירות את שיטת המחשבה.
למשל, תכנות פונקציונלי, מבטל את הצורך שלנו במשתנים.
• המצאת שפות חדשות
שפות DSL.
⌘סוף{ציינון}

הרציונל של השפה: כמה משפטים שמתארים את שיטת התכנון של השפה.

דוגמאות לרציונלים למשל של ⌘פסקל או של ⌘סי.


ספר זה מנוסח בלשון זכר, כי זוהי הצורה הבלתי מסומנת בעברית. כלומר, אם מדובר
בנקבה (או בנקבות) בלבד - משתמשים במין נקבה. בכל מקרה אחר - דהיינו: זכר, מינים
מעורבים או מין לא ידוע - משתמשים בצורה הבלתי מסומנת, הנקראת גם "לשון זכר". איני
רואה כל צורך להתנצל על כך. 

