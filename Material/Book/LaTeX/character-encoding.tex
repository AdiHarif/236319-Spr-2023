§ מהו קידוד?
במרבית שפות התכנות, הטיפוס \מונח{תו}, כמו הטיפוס \מונח{מספר שלם} הוא
\מונח{טיפוס אטומי}: כלומר, זהו טיפוס שהוא אקסיומה של מערכת הטיפוסים, ואשר משמש
הן בפני עצמו, והן לבניית טיפוסים באמצעות בנאי הטיפוסים. כזכור, לטיפוס יש מספר
תפקידים, ובהם קביעת משמעות של פעולות, קביעת חוקיות של פעולות, סיווג נתונים
ועוד.

אחד מהתפקידים החשובים של טיפוס הוא קביעת הקידוד, כלומר קביעת הדרך שבה הערך
המופשט בשפת התכנות, מתורגם ליצוג בביטים, כלומר \מונח{קידוד}, ובכיוון ההפוך,
כיצד יש לתרגם סדרת ביטים לערך מופשט, כלומר \מונח{פענוח}.

יוזכר כי \שי{Type Punning} הוא הפרת הקידוד, כלומר עקיפת כללי התרגום שבין הערך
לבין הייצוג שלו כסדרת ביטים. מעקף כזה יכול להתבצע באמצעות \begin{itemize}
✦ פירוש סדרת ביטים שלא בדרך שבה היא קודדה, לדוגמה,
\begin{itemize}
✦ פירוש ערך של מצביע בן 64 ביטים, כמספר בנקודה צפה בשיטת הקידוד של המשלים לאחד
✦ קבלת ג'יבריש של אותיות היא תוצאה של \שי{Type Punning} של פירוש: התכנה אשר מציגה את התווים, מפרשת את תוכנה של סדרת בתים כסדרת תווים, כאשר ההתאמה של בית (או מספר בתים) לתו גרפי נעשית באמצעות כללי תרגום שונים מאלו ששימשו לתרגום תו גרפי אל בית (או מספר בתים) אצל התכנה שבה השתמשה סדרת התווים,
\end{itemize}
או באמצעות הפעולה ההפוכה, כלומר,
✦ כתיבה של סדרת ביטים אל מקום בזכרון שלא באמצעות כללי ה\מונח{קידוד}, לדוגמא
\begin{itemize}
✦ כתיבת המספר 93 לתוך תא בזיכרון שנועד לשם איחסון תו גרפי
✦ שמירת ערכו של מצביע בתוך תא שנועד להיקרא כבלוק של 64 ביטים שמייצגים מספר ממשי.
\end{itemize}
\end{itemize}

§ מדוע קידוד תווים?
מתברר כי קידוד של הטיפוס האטומי "תו" הוא מסובך הרבה יותר מהקידוד של מספרים שלמים. הסיבה לך היא שמשמעות המונח "תו" נתונה לפרשנויות רבות ושונות. למשל, בעברית, ישנם 22 אותיות, ו-5 אותיות סופיות. האם נדרוש תו נפרד לכל אחת מהאותיות הסופיות? (התשובה הנפוצה בעברית היא "כן", אך היא אינה אוניברסלית. לעומת זאת, בערבית התשובה היא גם כן וגם לא, ובשפות אחרות, התשובה הנפוצה היא "לא".)

ועוד אפשר לשאול האם נדרש תו נפרד לכל אחד מסימני הניקוד? ואם כן, האם יש להקצות תו נפרד לסימן הניר מאוד של חטף חיריק, ומה בדבר המפיק? האם הוא שונה מהדגש הקל? ומהדגש החזק? ומסימן השורוק? והאם יש צורך בתו נפרד עבור הנון ההפוכה? והאם התווים של אידיש שונים או נבדלים מאלו של העברית?

ברוך שהחלטות אלו תלויות בצרכים. אם אתה עוסק בכתבים רליגיוזית מימי הביניים, יש לך צורך ממשי בליגטורה
\begin{center}
\Huge{ﭏ}
\normalsize{}
\end{center}
שבה יש חיבור של האות א' עם האות ל', ואם אתה עוסק בבנקאות, יש לך צורך בליגטורה
\begin{center}
\Huge{₪}
\normalsize{}
\end{center}
אשר בה יש חיבור של האות ש' עם האות ח'. ואם אינך דובר עברית כלל, אין לך צורך באף אחד מהתווים העבריים, אבל אולי תרצה להשתמש בתו מיוחד כדי לאיית את המילה façade כהלכה, ואם כזה הוא רצונך, הרי שפת התכנות שבה כתוב מעבד התמלילים שלך, צריכה להיות מודעת לתשובה לשאלה האם ç הוא תו, ואם אין כך הדבר, יש לה צורך לדעת כיצד להתגבר על מכשלה זו.

ההחלטה של מהו "תו" היא החלטה קשה. במהלך השנים, ניתנו תשובות רבות, שונות, וסותרות לשאלה זו. קידוד יוניקוד הוא ניסיון לתת מענה כולל לשאלה זו ולתמוך בכל מגוון השפות, ויהיו אלו שפות שבשימוש נרחב כמו ספרדית, שפות שנכחדו, כמו ארמית קיסרית, ושפות מומצאות כגון השפה הקלינגונית.

גם עתה, לאחר ניסיונות אדירים אלו, אין הסכמה. אפילו קידוד השפה העברית אינו אחיד, כאשר קיימת שיטה אחת מרכזית:
\begin{itemize}
✦ \שי{Unicode}: יוניקוד
ועוד מספר שיטות, פחות נפוצות, כמו:
✦8859-8 \שי{ISO}: מערכות מחשוב ישנות
✦ \שי{1255-Windows}: הרחבה של 8859-8 \שי{ISO}, בשימוש מערכת הפעלה חלונות
✦ 862 \שי{Code page}: בשימוש במחשבי DOS
✦ \שי{Mac OS Hebrew}: בשימוש במחשבים של חברת אפל
\end{itemize}

§ מהם תווים (characters)
מספר התווים שניתן לייצג תלוי במערכת הקידוד בה משתמשים על מנת לייצג אותם. מערכות קידוד שונות יכולות לייצג מספר שונה של תווים. במערכות קידוד בינריות זה תלוי באורך רצף הביטים שנבחר על מנת לייצג תו, ואילו רצפי ביטים נבחרו על מנת לייצג תווים.

קבוצות של תווים גדלו והתפתחו במהלך השנים. \שי{Morse Code} יצא בשנת 1840 ושימש לקידוד של כל אות לטינית, ספרה ועוד כמה תווים עכשיו סדרה של לחיצות קצרות וארוכות במכשיר הטלגרף. \שי{Baudot Code} הומצא בשנת 1870 ונהייה תקן לתקשורת בטלגרף בשנת 1930, הוא מבסוס על 5 ביטים, ולכן ניתן לקודד 32 תווים בלבד. \שי{CDC Display Code} יצא בשנת 1960 ושימש במחשבי CDC עליהם פתוחה שפת התכנות פסקל, הוא מבוסס על 6 ביטים ולכן יכול לייצג 64 תווים שונים ללא הבדלה בי ןאות גדולה לקטנה. גם BCD יצא בשנת 1960 ומבוסס על 6 ביטים.

בצורות הקידוד שהוזכרו לעיל לא ניתן להשתמש בימינו כי יש צורך בהצגה של מספר הרבה יותר גדול של תווים.

קוד ASCII גם הוצג בערך בשנת 1960 והוא מבוסס 7 ביטים לתו, על כן 128 אפשרויות לתווים, והוא נמצא בשימוש גם היום. EBCDIC יצא בסביבות 1965 עם 8 ביטים לתו (עם זאת לא נעשה שימוש בכל 256 האפשרויות).

תקן נפוץ היום לאופני קידוד הוא Unicode (פירוט עליו בהמשך), עם זאת אין תאימות מלאה של כל אופני הקידוד לתקן זה.

§ קידוד תווים
