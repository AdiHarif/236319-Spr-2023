§§ חיפוש שמות, קישור לשמות, וטיפול בשגיאות
כזכור, ה-a-list היא רשימה של \E|dotted pairs| שכל אחד מהם הוא קישור בין שם לערך.
נגדיר פונקציה רקורסיבית, \E|lookup| אשר מקבלת שני פרמטרים: id שהוא אטום,
ו-a-list, שהיא רשימה במבנה של \E|a-list| כפי שהצגנו אותו, כלומר סדרה של
dotted pairs אשר ה-car של כל אחד מהם הוא אטום, וה-cdr שלו הוא משמעותו של האטום:
\begin{KERNEL}
(defun lookup (id a-list) ; lookup id in an a-list
  (cond ; Three cases to consider:
    ((null a-list) ; (i) a-list was exhausted.
      (error UNDEFINED id))
    ((eq id (car (car a-list))) ; (ii) found in first dotted pair
      (cdr (car a-list))) ; return value part of dotted pair
    (t (lookup id (cdr a-list))))) ; (iii) otherwise, recursive call on remainder of a-list
\end{KERNEL}
גם בפונקציה lookup אנו רואים קריאה רקורסיבית עם cond המתפצל לשלושה מקרים: במקר
ה
של רשימה ריקה הפונקציה נכשלת, במקרה שראש הרשימה מתאים לתנאי, הפונקציה מצליחה.
הפונקציה קוראת לעצמה רקורסיבית בכל מקרה אחר.

בהגדרת lookup נעשה שימוש בפונקציה אטומית נוספת, \E|error|. פונקציה זו
מדפיסה את כל הפרמטרים שלה וגורמת לאינטרפרטר לעבור לסבב הבא של לולאת ה-\E|REPL|.

נשתמש ב-error כדי להגדיר את הפונקציה \E|bind|. זוהי פונקציה רקורסיבית אשר תשמש כחלק
ממימוש אלגוריתם השיערוך בליספ. תפקידה במנגנון הוא בשיערוך של ביטוי המכיל הפעלה
של פונקציה, כמו למשל בביטוי
\begin{MINI}
  (exists name '(atom car cdr cond cons eq error eval set))
\end{MINI}
שנתקלנו בו קודם במימוש של is-atomic באמצעות \E|exists|. בקריאה ל-exists,
יש לקשור בין \E|x| ו-\E|xs|, שמות הפרמטרים הפורמליים לפונקציה, ובין ערכי
הארגומנטים האקטואליים. בהפעלה זו ערכי הארגומנטים הם:
\begin{itemize}
  ✦ \E|name|, או, ליתר דיוק, ביטוי ה-\E|S| שאותו מציין האטום \E|name|.
  ✦ הרשימה
  \begin{MINI}
    (atom car cdr cond cons eq error eval set)
  \end{MINI}
  או ליתר דיוק הביטוי המתקבל מתוצאת השיערוך של
  \begin{MINI}
  '(atom car cdr cond cons eq error eval set)
  \end{MINI}
\end{itemize}

הפונקציה bind אחראית לביצוע קישור זה בזמן הפעלה של פונקציה בתהליך השיערוך.
בפרט, bind מקבלת שתי רשימות באורך שווה: רשימת פרמטרים (פורמליים) (\E|formals|)
ורשימה ערכים---הארגומנטים (\E|actuals|). פרמטר שלישי ל-\E|bind| הוא רשימה קיימת
של קישורים, כלומר רשימה במבנה של \E|a-list|.

bind מחזירה רשימת קישורים חדשה, המכילה רשימה של קישורים בין פרמטרים וארגומנטים,
ובהמשכה, רשימת הקישורים ההתחלתית, שהועברה כפרמטר שלישי ל-bind. כל קישור כזה הוא
זוג, אשר ה-car שלו הוא הפרמטר, אטום של מיני-ליספ, וה-cdr שלו הוא הארגומנט, ביטוי-\E|S|
שערכו נקבע בזמן ההפעלה, אשר אותו מציין הפרמטר.

\begin{KERNEL}
(defun bind (formals actuals a-list) ; Bind parameters to arguments, and prepend to a-list
  (cond ((null formals) ; no more formals left
        (cond ((null actuals) a-list) ; no more actuals left, binding done-- return a-list
              (t (error missing-formals)))) ; more actuals than formals
        ((null actuals) ; formals is not nil but actuals is, i.e., more formals than actuals
          (error missing-actuals))
        (t ; both formals and actuals are not empty
          (cons ; create new binding and prepend it to result of recursive call
            (cons (car formals) (car actuals)) ; new dotted pair defines single binding
            (bind (cdr formals) (cdr actuals) a-list))))) ; recursive call
\end{KERNEL}

§§ שיערוך של ביטויי~\E|S|
\begin{minipage}{0.9\textwidth}
  \centering
\newcommand\exception[1]{{\textcolor{red}{#1}}}
\begin{mdframed}[backgroundcolor=Lavender!20]
  \footnotesize
  בהינתן ביטוי מורכב \E|$s$|, אלגוריתם השיערוך של ליספ מנסה
  להציג אותו כרשימה בת~$n+1$ איברים \[
    s=(s₀\;\;s₁\;\;⋯\;\;sₙ).
\] \begin{enumerate}
    ✦ \exception{
      אם הביטוי אינו רשימה, כלומר, אם הביטוי הנתון הוא \E|dotted pair| (כמו
      \E|(a.b)| למשל) השיערוך נכשל.}
    ✦ אם הרשימה ריקה, תוצאת השיערוך היא האטום \T|nil|.
    ✦ אחרת, המשערך מסתכל על הרשימה כעל עץ שיערוך, כלומר כעל קריאה
    לפונקציה~$s₀$ המקבלת~$n$ ארגומנטים \E|$s₁$,…,$sₙ$|, והשיערוך של~$s$ מתבצע
    על פי הסתכלות זו:
    \begin{quote}
      \begin{enumerate}
        ✦ \ע|שיערוך רקורסיבי של~$s₀$, הפונקציה אותה יש להפעיל.|
        \begin{itemize}
          ✦ \exception{אם השיערוך של~$s₀$ נכשל, אזי גם השיערוך של~$s$ נכשל.}
          ✦ אחרת, שיערוך זה מחזיר ביטוי~\E|S| שנסמן~$s₀'$.
          ✦ \exception{השיערוך של~$s$ נכשל אם~$s₀'$ אינו פונקציה.}
          ✦ \exception{השיערוך של~$s$ נכשל גם אם~$s₀'$ הוא פונקציה, אך כזו
            שאינה מצפה ל-$n$ ארגומנטים בדיוק.}
        \end{itemize}
        ✦ \ע|שיערוך הארגומנטים.|
        \begin{itemize}
          ✦ המשערך ממשיך כעת לשערך רקורסיבית את הארגומנטים לפונקציה~$s₀'$,
          הלא הם הביטוים~\E|$s₁$,…,$sₙ$|. נסמן את תוצאות השיערוך הללו
          ב-\E|$s₁'$,…,$sₙ'$|.

          ✦ \exception{אם השיערוך הרקורסיבי של אחד מבין
            הביטוים~\E|$s₁$,…,~$sₙ$| נכשל, אז גם השיערוך של הביטוי~$s$ כולו
            נכשל.}
        \end{itemize}

        ✦ \ע|הפעלת הפונקציה על הארגומנטים.|

        \begin{itemize}
          ✦ תוצאת השיערוך של הביטוי~$s$ היא תוצאת הקריאה לפונקציה~$s₀'$
          על~$n$ הביטוים \E|$s₁'$,…,~$sₙ'$|.

          ✦ \exception{אם הפעלה זו נכשלת, השיערוך של הביטוי~$s$ כולו נכשל.}
        \end{itemize}
      \end{enumerate}
    \end{quote}
  \end{enumerate}
\end{mdframed}
\end{minipage}

\begin{minipage}{0.9\textwidth}
  \centering
\newcommand\exception[1]{{\textcolor{red}{#1}}}
\begin{mdframed}[backgroundcolor=Lavender!20]
  \footnotesize
  בהינתן ביטוי מורכב \E|$s$|, אלגוריתם השיערוך של ליספ מנסה
  להציג אותו כרשימה בת~$n+1$ איברים \[
    s=(s₀\;\;s₁\;\;⋯\;\;sₙ).
\] \begin{enumerate}
    ✦ \exception{
      אם הביטוי אינו רשימה, כלומר, אם הביטוי הנתון הוא \E|dotted pair| (כמו
      \E|(a.b)| למשל) השיערוך נכשל.}
    ✦ אם הרשימה ריקה, תוצאת השיערוך היא האטום \T|nil|.
    ✦ אחרת, המשערך מסתכל על הרשימה כעל עץ שיערוך, כלומר כעל קריאה
    לפונקציה~$s₀$ המקבלת~$n$ ארגומנטים \E|$s₁$,…,$sₙ$|, והשיערוך של~$s$ מתבצע
    על פי הסתכלות זו:
    \begin{quote}
      \begin{enumerate}
        ✦ \ע|שיערוך רקורסיבי של~$s₀$, הפונקציה אותה יש להפעיל.|
        \begin{itemize}
          ✦ \exception{אם השיערוך של~$s₀$ נכשל, אזי גם השיערוך של~$s$ נכשל.}
          ✦ אחרת, שיערוך זה מחזיר ביטוי~\E|S| שנסמן~$s₀'$.
          ✦ \exception{השיערוך של~$s$ נכשל אם~$s₀'$ אינו פונקציה.}
          ✦ \exception{השיערוך של~$s$ נכשל גם אם~$s₀'$ הוא פונקציה, אך כזו
            שאינה מצפה ל-$n$ ארגומנטים בדיוק.}
        \end{itemize}
        ✦ \ע|שיערוך הארגומנטים.|
        \begin{itemize}
          ✦ המשערך ממשיך כעת לשערך רקורסיבית את הארגומנטים לפונקציה~$s₀'$,
          הלא הם הביטוים~\E|$s₁$,…,$sₙ$|. נסמן את תוצאות השיערוך הללו
          ב-\E|$s₁'$,…,$sₙ'$|.

          ✦ \exception{אם השיערוך הרקורסיבי של אחד מבין
            הביטוים~\E|$s₁$,…,~$sₙ$| נכשל, אז גם השיערוך של הביטוי~$s$ כולו
            נכשל.}
        \end{itemize}

        ✦ \ע|הפעלת הפונקציה על הארגומנטים.|

        \begin{itemize}
          ✦ תוצאת השיערוך של הביטוי~$s$ היא תוצאת הקריאה לפונקציה~$s₀'$
          על~$n$ הביטוים \E|$s₁'$,…,~$sₙ'$|.

          ✦ \exception{אם הפעלה זו נכשלת, השיערוך של הביטוי~$s$ כולו נכשל.}
        \end{itemize}
      \end{enumerate}
    \end{quote}
  \end{enumerate}
\end{mdframed}
\end{minipage}

§§ שיערוך ביטוי מורכב וקריאה לפונקציה

תיארנו למעלה כיצד נעשה שיערוך של אטום. נותר לתאר כיצד מתבצע השיערוך של ביטוי
מורכב. בתמצית, שיערוך של ביטוי מורכב נעשה התייחסות ל-\E|CAR| של הביטוי כפונקציה
שאותה יש להפעיל על ה-\E|CDR| של הביטוי. הפעלה זו מתייחסת ל-\E|CDR| של הביטוי
כרשימת הארגומנטים שיש להעביר לפונקציה שב-\E|CAR|.

עלינו עוד להדגים כיצד נעשית קריאה לפונקציה. לשם כך, נשתמש בפונקציה המוגדרת מראש
defun, המשמשת להגדרת פונקציות חדשות, כדי להגדיר את הפונקציה הרקורסיבית append
המוסיפה פריט בסופה של רשימה 
\begin{MINI}
(defun append(x xs) ; append item x to end of list of xs
  (cond ((null xs) ; no more xs, recursion base¢…¢
          (cons x nil)) ; ¢…¢ return a list containing x
        (t ; recursive call
          (cons
            (car xs) ; prepend first of xs to¢…¢
            (append x (cdr xs)))))) ; ¢…¢ result of recursive call on remaining xs
\end{MINI}

לאחר ביצוע הגדרה זו, כלומר לאחר שיערוך הביטוי מעלה, יווצר קישור בין השם append
ובין גוף הפונקציה \E|append|, כלומר ביטוי ה-$λ$
\begin{MINI}
(lambda (x xs)
  (cond ((null xs) (cons x nil))
        (t (cons
              (car xs)
              (append x (cdr xs))))))
\end{MINI}

וקישור זה יתווסף לתחילת ה-\E|a-list|. ויזואלית,
אם ה-a-list נראתה טרם ההגדרה כך,
\begin{LTR}
  \begin{tikzpicture}[list/.style={rectangle split, rectangle split parts=2,
          draw,minimum height=3ex, fill=blue!20,rectangle split horizontal}, >=stealth, start chain, node distance=3ex]
    \foreach \x/\y/\z in {%
        h/t/t,
        i/nil/nil
      } {%
        \node[on chain, list,font=\tt\scriptsize] (\x) {\y};
        \node[below=4 ex of \x.one,anchor=north west,align=left,font=\tt\scriptsize,color=red] (temp) {\z};
        \draw[->,bend left] (\x.one south) .. controls+(270:0.3) and+(120:0.6) .. (temp.north west);
      }

    \node[on chain] (j) {\huge$⋯$};
    \draw[*->] let \p1=(i.two), \p2=(j.center) in (\x1,\y2)--(j);

    \foreach \a/\b in {h/i} {%
        \draw[*->] let \p1=(\a.two), \p2=(\b.center) in (\x1,\y2)--(\b);
      }

    \node[above=of h] (A) {a-list};
    \draw[->] (A.south)--(h);
  \end{tikzpicture}
\end{LTR}
אזי, אחרי ההגדרה של append ה-a-list תראה כך
\begin{LTR}
  \begin{tikzpicture}[list/.style={rectangle split, rectangle split parts=2,
          draw,minimum height=3ex, fill=blue!20,rectangle split horizontal}, >=stealth, start chain, node distance=3ex]
    \foreach \x/\y/\z in {%
    g/append/{%
    (lambda⏎
    \quad (x xs)⏎
    \quad (cond ⏎
    \quad\quad((null xs) (x))⏎
    \quad\quad (t (cons (car x) ⏎
    \quad\quad\quad\quad\quad\quad(append x (cdr xs))))))},
    h/t/t,
    i/nil/nil
    } {%
    \node[on chain, list,font=\tt\scriptsize] (\x) {\y};
    \node[below=4 ex of \x.one,anchor=north west,align=left,font=\tt\scriptsize,color=red] (temp) {\z};
    \draw[->,bend left] (\x.one south) .. controls+(270:0.3) and+(120:0.6) .. (temp.north west);
    }

    \node[on chain] (j) {\huge$⋯$};
    \draw[*->] let \p1=(i.two), \p2=(j.center) in (\x1,\y2)--(j);

    \foreach \a/\b in {g/h, h/i} {%
        \draw[*->] let \p1=(\a.two), \p2=(\b.center) in (\x1,\y2)--(\b);
      }

    \node[above=of g] (A) {a-list};
    \draw[->] (A.south)--(g);
  \end{tikzpicture}
\end{LTR}
בקריאה ל-append כמו
\begin{MINI}
(append 'a '(b c))
\end{MINI}

פונקציית השיערוך תאתר בתוך ה-a-list את ביטוי ה-$λ$ הקשור לפונקציה, כלומר
את הביטוי
\begin{MINI}
(lambda
  (x xs)
  (cond ((null xs) (x))
        (t (cons
              (car x)
              (append x (cdr xs))))))
\end{MINI}
ותפרק אותו לשלושת
מרכיביו: האטום lambda המזהה את הביטוי, רשימת הפרמטרים הפורמליים \mini{(x xs)}
והביטוי לחישוב,
\begin{MINI}
(cond ((null xs) (x))
      (t (cons
            (car x)
            (append x (cdr xs)))))
\end{MINI}
אותו נסמן לרגע ב-$ϕ$. הקריאה לפונקציה append מתבצעת על ידי שיערוך הפרמטרים
האקטואליים, כלומר שיערוך של הפרמטר \mini{'a} לכדי הערך \mini{A} ושיערוך של
הפרמטר \mini{'(b c)} לכדי הערך \mini{(B C)}, ולאחר מכן חישוב הביטוי~$ϕ$ בתנאים
שבהם הפרמטר x משתערך לאטום \mini{a} והפרמטר xs משתערך לרשימה \mini{(b c)}.

קריאה לפונקציה בפרוצדורת השיערוך בליספ נעשית באמצעות יצירת קישורים בין שמות
הפרמטרים הפורמליים וערכי הפרמטרים האקטואליים (פרמטרים אקטואליים נקראים גם
\ע|ארגומנטים|). במקרה של הפונקציה append מתווספים שני \E|dotted pairs|
\mini{(x.a)} ו-\mini{(xs.(b c))} ל-\E|a-list|.
ויזואלית, ה-a-list תראה כך
\begin{LTR}
  \begin{tikzpicture}[list/.style={rectangle split, rectangle split parts=2,
          draw,minimum height=3ex, fill=blue!20,rectangle split horizontal}, >=stealth, start chain, node distance=2.5ex]
    \foreach \x/\y/\z in {%
    e/x/a,
    f/xs/(b c),
    g/append/{%
    (lambda⏎
    \quad (x xs)⏎
    \quad (cond⏎
    \quad\quad((null xs) (x))⏎
    \quad\quad (t (cons (car x) ⏎
    \quad\quad\quad\quad\quad\quad(append x (cdr xs))))))},
    h/t/t,
    i/nil/nil
    } {%
    \node[on chain, list,font=\tt\scriptsize] (\x) {\y};
    \node[below=4 ex of \x.one,anchor=north west,align=left,font=\tt\scriptsize,color=red] (temp) {\z};
    \draw[->,bend left] (\x.one south) .. controls+(270:0.3) and+(120:0.6) .. (temp.north west);
    }

    \node[on chain] (j) {\huge$⋯$};
    \draw[*->] let \p1=(i.two), \p2=(j.center) in (\x1,\y2)--(j);

    \foreach \a/\b in {e/f, f/g, g/h, h/i} {%
        \draw[*->] let \p1=(\a.two), \p2=(\b.center) in (\x1,\y2)--(\b);
      }

    \node[above=of e] (A) {a-list};
    \draw[->] (A.south)--(e);
  \end{tikzpicture}
\end{LTR}
לאחר הקריאה לפונקציה, יש לשחזר את ערכה של ה-a-list לערכה המקורי טרם הקריאה לה.

כאשר הפונקציה append קוראת לעצמה רקורסיבית בפעם הראשונה, יתבצע קישור נוסף בין
הפרמטרים האקטואליים והפורמליים, וגם זאת על ידי תוספת של שני \E|dotted pairs| נוספים:
\mini{(x.a)} ו-\mini{(xs.(c))}.

לאחר ביצוע קישורים אלו, ה-a-list תראה כך
\begin{LTR}
  \begin{tikzpicture}[list/.style={rectangle split, rectangle split parts=2,
          draw,minimum height=3ex, fill=blue!20,rectangle split horizontal}, >=stealth, start chain, node distance=3ex]
    \foreach \x/\y/\z in {%
    c/x/a,
    d/xs/(c),
    e/x/a,
    f/xs/(b c),
    g/append/{%
    (lambda⏎
    \quad (x xs)⏎
    \quad (cond⏎
    \quad\quad((null xs) (x))⏎
    \quad\quad (t (cons (car x) ⏎
    \quad\quad\quad\quad\quad\quad(append x (cdr xs))))))},
    h/t/t,
    i/nil/nil
    } {%
    \node[on chain, list,font=\tt\scriptsize] (\x) {\y};
    \node[below=4 ex of \x.one,anchor=north west,align=left,font=\tt\scriptsize,color=red] (temp) {\z};
    \draw[->,bend left] (\x.one south) .. controls+(270:0.3) and+(120:0.6) .. (temp.north west);
    }

    \node[on chain] (j) {\huge$⋯$};
    \draw[*->] let \p1=(i.two), \p2=(j.center) in (\x1,\y2)--(j);

    \foreach \a/\b in {c/d, d/e, e/f, f/g, g/h, h/i} {%
        \draw[*->] let \p1=(\a.two), \p2=(\b.center) in (\x1,\y2)--(\b);
      }

    \node[above=of c] (A) {a-list};
    \draw[->] (A.south)--(c);
  \end{tikzpicture}
\end{LTR}
כלומר הקישורים החדשים יסתירו את הקישורים הקודמים.

בקריאה הרקורסיבית השניה של append לעצמה, יווצרו שני קישורים נוספים,
\mini{(x.a)} ו-\mini{(xs.())}, וה-a-list תראה כך
\begin{LTR}
  \usetikzlibrary{chains,arrows}
  \begin{tikzpicture}[list/.style={rectangle split, rectangle split parts=2,
          draw,minimum height=3ex, fill=blue!20,rectangle split horizontal}, >=stealth, start chain, node distance=3ex]

    \foreach \x/\y/\z in {%
    a/x/a,
    b/xs/(),
    c/x/a,
    d/xs/(c),
    e/x/a,
    f/xs/(b c),
    g/append/{%
    (lambda⏎
    \quad (x xs)⏎
    \quad (cond⏎
    \quad\quad((null xs) (x))⏎
    \quad\quad (t (cons (car x) ⏎
    \quad\quad\quad\quad\quad\quad(append x (cdr xs))))))},
    h/t/t,
    i/nil/nil
    } {%
    \node[on chain, list,font=\tt\scriptsize] (\x) {\y};
    \node[below=4 ex of \x.one,anchor=north west,align=left,font=\tt\scriptsize,color=red] (temp) {\z};
    \draw[->,bend left] (\x.one south) .. controls+(270:0.3) and+(120:0.6) .. (temp.north west);
    }

    \node[on chain] (j) {\huge$⋯$};
    \draw[*->] let \p1=(i.two), \p2=(j.center) in (\x1,\y2)--(j);

    \foreach \a/\b in {a/b, b/c, c/d, d/e, e/f, f/g, g/h, h/i} {%
        \draw[*->] let \p1=(\a.two), \p2=(\b.center) in (\x1,\y2)--(\b);
      }

    \node[above=of a] (A) {a-list};
    \draw[->] (A.south)--(a);
  \end{tikzpicture}
\end{LTR}
כאמור, בכל פעם שבה הפונקציה append חוזרת מקריאה רקורסיבית, ה-a-list ישוחזר,
ושני הקישורים בין הפרמטרים פורמליים לאקטואליים שנעשו למען הקריאה הרקורסיבית
יוסרו ממנו.

§§ שיערוך דחוי
אלגוריתם השיערוך כפי שתואר עד כה אינו מספיק כדי לשערך את העץ ב\פנה|איור:עץ|
כהלכה. הסיבה היא שהאופרטור הטרנארי \LR{\texttt{$·$?$·$:$·$}} אשר מצוי בשורש העץ, מחשב את
הארגומנט
הראשון שלו, ובהתאם לתוצאת החישוב, מחשב את הארגומנט השני או השלישי, אבל לא את
שניהם גם יחד. למימוש של האופרטור הטרנארי

בפרוצדורת השיערוך שתוארה למעלה, יש לחשב את כל הארגומנטים לפונקציה
(או אופרטור), טרם שמפעילים את הפונקציה עצמה.

אלגוריתם השיערוך הפשוט גם אינו מתאים לפונקציות כגון quote ו-lambda אשר אינן
משערכות את הפרמטרים שלהן כלל. הפרוצדורה גם אינה מתאימה לפונקציות כגון cond בהן
השיערוך של חלקים מגוף הפונקציה תלוי בערכים המחושבים בחלקים אחרים של הגוף.

נתקן את אלגוריתם השיערוך כך שיתמוך בפונקציות שהן \E|normal|, על ידי הוספת התנאי
הבא:
\begin{mdframed}[backgroundcolor=Lavender!20]
  \footnotesize
  בשיערוך הביטוי \[
    s=(s₀\;\;s₁\;\;⋯\;\;sₙ).
\] אם שיערוך הביטוי~$s₀$ מחזיר פונקציה שהיא
  normal כלומר ביטוי~$λ$ שהאטום הראשון שבו הוא nlambda (ולא \E|lambda|) אזי יש
  להעביר לפונקציה זו כארגומנטים את הביטוים~\E|$s₁$,…,$sₙ$| ולא את הערכים
  המשוערכים שלהם~\E|$s₁'$,…,$sₙ'$|.
\end{mdframed}

§§ הפונקציה eval והסביבה

בנוסף לפונקציות האטומיות ולפונקציות המוגדרות מראש, שפת מיני-ליספ תומכת, כמו
מרבית הניבים של ליספ, כפונקציה \E|eval| אשר מקבלת ביטוי~\E|S| ומשערכת אותו.
הפונקציה eval גם היא נחשבת, מסיבות טכניות, לפונקציה אטומית. אנחנו נדגים כיצד
ניתן לממש אותה באמצעות מימוש פונקציה דומה \E|evaluate| כפונקצית ספרייה.

ישנו הבדל קטן בין \E|eval|, הפונקציה האטומית של ליספ, ובין \E|evaluate|, אותה
נממש כפונקציית ספרייה. שתי הפונקציות מקבלות ביטוי-\E|S| ומשערכות אותו, אלא
שהפונקציה \E|evaluate| מקבלת פרמטר נוסף, המתאר את הקישורים שנעשו בתכנית עד כה.
לאוסף זה של ההגדרות, קוראים \ע|סביבה|. הסביבה ממומשת בשפת ליספ באמצעות
מבנה נתונים הקרוי \E|a-list|. הפונקציה eval משתמשת אף היא בסביבה, אלא שבתוך
המימוש של מיני-ליספ, אין צורך להעביר את הסביבה כפרמטר: רשימת ה-\E|a-list| נגישה
לפונקציה \E|eval| כמין משתנה גלובלי, אותו אין צורך להעביר כפרמטר מפורש, וניתן
לשנותו מכל נקודה בתכנית. העדר הצורך להעביר את ה-\E|a-list| כפרמטר ל-eval הוא זה
אשר מבחין בינה ובין \E|evaluate| אשר אין לה גישה לערכים גלובליים, ועל כן היא
נדרשת לקבל את הסביבה, המיוצגת באמצעות-\E|a-list| כפרמטר.

המימוש של האלגוריתם שמאחורי \E|eval| (כלומר המימוש של הפונקציה \E|evaluate|)
נעשה כולו במיני-ליספ, תוך שימוש בפונקציות אטומיות או פונקציות מוגדרות מראש, אשר
כאמור גם הן ממומשות באמצעות הפונקציות האטומיות. מימוש זה הוא פשוט וקצר יחסית
ועולה לכדי פחות ממאה שורות קוד.

§§ הפונקציות evaluate ו-apply

מתברר שאפשר לממש את אלגוריתם השיערוך גם בליספ עצמה. לשם כך נגדיר פונקציה
evaluate המקבלת ביטוי~\E|S|, וסביבה במבנה של \E|a-list|, ואשר משערכת את הביטוי
בהתאם לקישורים שבסביבה הנתונה. הפונקציה evaluate קוראת לעצמה ברקורסיה כדי לשערך
את מרכיבי הביטוי.

כאשר evaluate תזהה ביטוי~$λ$ היא תקרא לפונקצית עזר apply אשר גם אותה נגדיר
בהמשך, ואשר אחראית להפעיל ביטוי~$λ$ על הפרמטרים המועברים לו. אם ביטוי ה-$λ$ הוא
\E|eager|, אז apply תקרא ל-evaluate כדי לשערך את הפרמטרים.

מבנה הקריאות והרקורסיות ההדדיות שבמימוש של evaluate ופונקציות העזר שלה מתואר
ב\פנה|איור:שיערוך|.
\begin{figure}[H]
 \scriptsize
  \centering
  \input Forests/eval-call-graph.tikz 
  \caption[רקורסיות הדדיות במימוש של eval]{רקורסיות הדדיות בין הפונקציה evaluate ופונקציות העזר שלה}
  \תגית|איור:שיערוך|
\end{figure}

אנו רואים באיור גם ש-apply קוראת לפונקציה הרקורסיבית bind (שאותה כבר הגדרנו),
וזאת כדי לקשור בין הפרמטרים הפורמליים והאקטואליים. האיור מראה גם ש-evaluate
משתמשת בכמה פונקציות נוספות שהוצגו כבר: הפונקציה הרקורסיבית lookup כדי לשערך את
ערכו של אטום, ו-is-atomic הקוראת ל-exists הרקורסיבית כדי לזהות אטומים המזהים
פונקצית אטומיות.

בנוסף לכל הפונקציות הללו, evaluate נעזרת גם בפונקציה \E|evaluate-atomic|, אשר
מצידה משתמשת בכמה פונקציות עזר משלה, בהמשך כדי לממש את הפונקציות האטומיות שבהם
משתמשת מיני-ליספ. שיערוך של אטומים, יחד עם מימוש הפונקציות האטומיות, מהווה את
בסיס הרקורסיה של \E|evaluate|. השיערוך של כל ביטוי~\E|S|, מורכב ככל שיהיה
מתורגם על ידי evaluate לסדרה מתאימה של שיערוך של אטומים והפעלות של פונקציות
אטומיות.

הגוף של evaluate עצמה הוא קצר, והוא מבוסס על אבחנה בין שלוש אפשרויות שונות ביחס
לביטוי אותו היא נדרש לשערך.

\minipage\textwidth
\begin{KERNEL}
(defun evaluate(S-expression a-list) ; evaluate S-expression in environment defined by an a-list
  (cond ((atom S-expression) ; recursion base: lookup of atom in a-list
          (lookup S-expression a-list))
        ((is-atomic (car S-expression)) ; case of handling atomic functions
          (evaluate-atomic S-expression a-list))
        (t ; recursive step---lambda applied to parameters
          (apply (evaluate (car S-expression) a-list) ; find lambda expression
                  (cdr S-expression) ; find actual parameters
                  a-list))))
\end{KERNEL}
\endminipage

אם הביטוי ש-evaluate מקבלת הוא אטום, evaluate מפעילה את \E|lookup| כדי למצוא את
הערך הקשור אליו ברשימת ה-\E|a-list|. אם לעומת הביטוי הוא ביטוי מורכב, אזי
\E|dotted pair| אשר ה-car שלו הוא פונקציה, וה-cdr שלו הוא רשימה של
ארגומנטים שיש להעביר לפונקציה. אם ה-car הוא שם של פונקציה אטומית, נדרש טיפול
מיוחד, ואז evaluate קוראת לפונקציה evaluate-atomic אשר מבצעת זאת.
בכל מקרה אחר, evaluate מוצאת את הפונקציה אותה יש להפעיל על ידי השיערוך
\begin{MINI}
(evaluate (car S-expression) a-list)
\end{MINI}
קריאה רקורסיבית זו תחזיר את ביטוי ה-$λ$ אותו יש להפעיל על הפרמטרים. evaluate
גם מחשבת את הפרמטרים באמצעות הקריאה-\T|(cdr S-expression)|, וקוראת לפונקציה
apply אשר מפעילה את ביטוי ה-$λ$ על הפרמטרים.

הפונקציה apply מצידה מפרקת את ביטוי ה-$λ$ לשלושת מרכיביו (תגית, רשימת פרמטרים
וביטוי לחישוב), ומעבירה מרכיבים אלו לפונקצית העזר apply-decomposed-lambda
\begin{KERNEL}
(defun apply(lambda-expression actuals a-list)
  (apply-decomposed-lambda
    (car lambda-expression) ; tag=lambda or nlambda
    (car (cdr lambda-expression)); list of formal parameters
    (car (cdr (cdr lambda-expression))); body
    actuals
    a-list))
\end{KERNEL}
הפונקציה apply-decomposed-lambda משערכת את גוף הפונקציה, בסביבה הכוללת קישור בין
הפרמטרים הפורמליים לאקטואליים. קישור זה נעשה באמצעות הפונקציה bind אותה כבר
הגדרנו.
\begin{KERNEL}
(defun apply-decomposed-lambda(tag formals body actuals a-list)
  (evaluate body
    (cond ((eq tag 'nlambda) (bind formals actuals a-list))
          ((eq tag 'lambda) (bind formals (evaluate-list actuals a-list) a-list))
          (t (error unknown-lambda tag))
)))
\end{KERNEL}

האבחנה בין הסוגים השונים של ביטוי ה-$λ$, נעשית באופן הבא בפונקציה
apply-decomposed-lambda
\begin{itemize}
  ✦
  אם התגית של ביטוי ה-$λ$ היא האטום nlambda הקישור נעשה
  ללא שיערוך הפרמטרים
  האקטואליים.
  ✦ אם לעומת זאת התגית היא האטום lambda הקישור בין הפרמטרים הפורמליים
  לאקטואליים נעשה לאחר שיערוך של הפרמטרים האקטואליים באמצעות הפונקציה
  \E|evalauate-list|. ✦
  אם התגית אינה אף אחד משני האטומים הללו השיערוך נכשל.
\end{itemize}

ההגדרה של הפונקציה evaluate-list המיועדת לשיערוך רשימת הפרמטרים האקטואליים היא
באמצעות הפעלה רקורסיבית של evaluate על כל אחד מאיברי הרשימה, ושרשור התוצאות
לכדי רשימה אחת.
\begin{KERNEL}
(defun evaluate-list(S-expressions a-list)
  (cond ((null S-expressions) nil) ; no more S-expressions to evaluate
    (t (cons
          (evaluate (car S-expressions) a-list) ; evaluate first S-Expression
          (evaluate-list (cdr S-expressions) a-list))))) ; recursive call on remainder
\end{KERNEL}

§§ מימוש הפונקציות האטומיות של מיני-ליספ בתוך הפונקציה evaluate
תיארנו את הפונקציה evaluate המממשת את מרבית הפעולות שבאלגוריתם השיערוך של ליספ.
נותר עוד לתאר את המימוש של הפונקציה evaluate-atomic אשר נועדה לטיפול ב-8
הפונקציות האטומיות של השפה.

נעיין ראשית ב\פנה|טבלה:אטומיות| המפרטת את כל שנדרש כדי \ע|להשתמש| בפונקציות
הללו.

מימוש הפונקציות האטומיות בתוך evaluate נדרש לקיים את המפרט שבטבלה. כמובן, לא
ניתן לממש את הפונקציות האטומיות עצמן. תפקידה של הפונקציה evaluate הוא לזהות
שהיא נדרשת לשערך פונקציה אטומית, ואז להעביר את המשימה לפונקציה האטומית המתאימה.
מרביתה של הפונקציה evaluate-atomic היא פעולות הכנה לקראת המטרה העיקרית שלנו:
\begin{quote}
  מימוש הפונקציות האטומיות בעבור פונקצית הספרייה evaluate וזאת בתוך הפונקציה
  \E|evaluate-atomic|.
\end{quote}

הצעד הראשון במימוש של evaluate-atomic הוא פירוק הביטוי אותו יש לשערך לשני
חלקים: שם הפונקציה האטומית, והפרמטרים לפונקציה זו.
\begin{KERNEL}
(defun evaluate-atomic (S-expression a-list)
  (apply-atomic (car S-expression) (cdr S-expression) a-list))
\end{KERNEL}

הפונקציה evaluate-atomic צריכה לטפל בכל הפונקציות האטומיות, כמו גם בפונקציה
eval אשר אינה אטומית, אבל עדיין לא טופלה. אנו רואים \פנה|טבלה:אטומיות| ש-cond
היא הפונקציה היחידה מבין כל אלו שהסמנטיקה שלה היא \E|normal|.

\minipage\textwidth
במימוש של הפונקציה \E|apply-atomic| נפריד ראשית בין cond ובין כל שאר הפונקציות
האטומיות.
\begin{KERNEL}
(defun apply-atomic (atomic actuals a-list)
  (cond ((eq 'cond atomic) ; special case: cond has normal semantics
            (evaluate-cond actuals a-list)) ; don't evaluate actuals
        (t (apply-eager-atomic ; all other atomics are eager
              atomic (evaluate-list actuals a-list) a-list))))
\end{KERNEL}
\endminipage

במקרה ששם הפונקציה האטומית cond נשתמש בפונקצית עזר, \E|evaluate-cond|. במקרה
שהשם הוא אחר, נמשיך עם הפונקציה \E|apply-eager-atomic|.

המימוש של evaluate-cond הוא ברקורסיה פשוטה על רשימת ה-test-forms
\begin{KERNEL}
(defun evaluate-cond(test-forms a-list)
  (cond ((null test-forms) nil) ; if no more test-forms, return nil
        ((evaluate (car (car test-forms)) a-list) ; evaluate condition part of the first test-form
        (evaluate (car (cdr (car test-forms))) a-list)) ; if true, evaluate value part (second part) of the first test-form
        (t (evaluate-cond (cdr test-forms) a-list)))) ; otherwise, recurse on rest of test-forms
\end{KERNEL}

כדי לטפל בשאר הפונקציות שב\פנה|טבלה:אטומיות| ובפונקציה eval נשים לב לכך שכל אלו
מלבד eval אינן נדרשות יותר ל-a-list וכולן, לבד מ-error מצפות לפרמטר אחד או שני
פרמטרים.

הפונקציה apply-eager-atomic מטפלת במקרים המיוחדים של \E|eval| ושל \E|error|
ומעבירה את המשך הטיפול ל-apply-trivial-atomic
\begin{KERNEL}
(defun apply-eager-atomic (atomic actuals a-list)
  (cond
    ((eq atomic 'error) (error actuals))
    ((eq atomic 'eval) (evaluate (car actuals) a-list))
    ((null (cdr actuals)) (apply-trivial-atomic atomic (car actuals) NIL)) ;~1 arguments atomic
    (t (apply-trivial-atomic atomic (car actuals) (car (cdr actuals)))) ;~2 arguments atomic
))
\end{KERNEL}

כעת הפונקציה apply-trivial-atomic משתמשת ב-cond כדי לטפל בששת הפונקציות האטומיות שנותרו:
\begin{KERNEL}
(defun apply-trivial-atomic (atomic first second)
  (cond ((eq atomic 'atom) (atom first))
        ((eq atomic 'car) (car first))
        ((eq atomic 'cdr) (cdr first))
        ((eq atomic 'cons) (cons first second))
        ((eq atomic 'eq) (eq first second))
        ((eq atomic 'set) (set first second))
        (t (error something-went-wrong atomic))))
\end{KERNEL}

§§ הקוד המלא של evaluate ופונקציות העזר שלה
\תגית|פרק:eval|

\immediate \closeout \kernelFile
\begin{LTR}
  \lstinputlisting[language=Mini,style=display,
    numbers=left,
    stepnumber=1,
    numbersep=2pt,
    xleftmargin=3ex,
    numberblanklines=false,
    numberstyle=\tiny\bf,
    backgroundcolor=\color{olive!10}
  ]{\jobname.kernel.lisp}
\end{LTR}


