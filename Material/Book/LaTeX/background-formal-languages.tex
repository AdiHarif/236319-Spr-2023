שפה פורמלית~$L$
מעל אלפבית~$Σ$ היא אוסף של מילים פורמליות הלקוחות מ-$Σ^*$, כלומר~$L⊆Σ^*$.

\דוגמה|שפת התכנות~\CPL כשפה פורמלית|
שפת התכנות~\CPL מגדירה שפה פורמלית~$L₀$ מעל האלפבית~$Σ_C$ \פנה|eq:alphabet:C|:
נאמר על מילה~$w$
(כאשר~$w∈Σ_C^*$),
כי היא שייכת לשפה~$L₀$
אם ורק אם~$w$ היא תכנית חוקית בשפת~\E|\CPL|.

למעשה, כל שפת תכנות מגדירה גם שפה פורמלית. זוהי השפה אשר מילותיה הן תכניות
חוקיות בשפת התכנות. אולם, הגדרת שפת תכנות אינה מצטמצמת להגדרת השפה הפורמלית
הזו. הגדרת שפת התכנות כוללת גם מתן \ע|משמעות| לכל תכנית חוקית.

§§ קבוצת הפונקציות הרציונליות כשפה פורמלית

נגדיר לדוגמה באופן רקורסיבי את השפה הפורמלית~$L₁$ שכל מילה בה יכולה להתפרש
כאיבר ב-$ℚ₁$.
\אבגד
✦ המילים האטומיות בשפה זו יהיו האותיות הבודדות~⌘U ו-⌘I. כמילים בשפה~$L₁$, לא
תהיה למילים אלו משמעות, כשנגדיר פונקציה המעניקה משמעות לכל מילה בשפה
הפורמלית~$L₁$, המשמעות של~⌘U תהיה פונקצית היחידה, והמשמעות
של~⌘I תהיה פונקצית הזהות.
✦ בנוסף נשתמש בסימנים~⌘/,~⌘*,~⌘+, ו-⌘- בתוך בנאי המילים. בשפה הפורמלית לא תהייה
לסימנים אלו משמעות, אך כשנגדיר פונקציה המעניקה משמעות לכל מילה בשפה
הפורמלית~$L₁$, פונקציה זו תפרש
ארבעה סימנים אלו כאופרטורים האריתמטיים.
✦ על ששת הסימנים האלו נוסיף גם את הסימנים ⌘(ו-⌘) כדי להבטיח שתהיה רק דרך
אחת לתת משמעות למילה בשפה הפורמלית~$L₁$.===

\החל{definition}[השפה הפורמלית של הפונקציות הרציונליות]
\label{definition:L1}
השפה~$L₁$, היא שפה פורמלית מעל האלפבית
\begin{equation}\label{eq:Q:alphabet}
  ❴⌘U, ⌘I, ⌘), ⌘), ⌘/, ⌘*, ⌘+, ⌘-❵
\end{equation}
המוגדרת על ידי שני איברים אטומיים
\begin{align}
   & \infer{⌘U∈L₁}{} ⏎
   & \infer{⌘I∈L₁}{}
\end{align}
ועל ידי ארבעה בנאים:
\begin{align}
   & \infer{⌘)⌘-w⌘)∈L₁}{w∈L} \label{eq:Q:minus}⏎
   & \infer{⌘)w₁⌘+w₂⌘)∈L₁}{w₁∈L₁                 & w₂∈L₁}\label{eq:Q:plus}⏎
   & \infer{⌘)w₁⌘*w₂⌘)∈L₁}{w₁∈L₁                 & w₂∈L₁}\label{eq:Q:times}⏎
   & \infer{⌘)w₁⌘/w₂⌘)∈L₁}{w₁∈L₁                 & w₂∈L₁}\label{eq:Q:div}
\end{align}
\סוף{definition}
כדאי לשים לב להבדלים בין הגדרה זו ובין ההגדרה הקודמת של הקבוצה~$L₁$
(\פנה|definition:rational|). הגדרה הנוכחית אינה מניחה ידע במתימטיקה או בפעולות
החשבון. איבר של הקבוצה היא סדרה ללא משמעות אותיות הלקוחה מהאלפבית
\פנה|eq:Q:alphabet|.
§§ דוגמאות לשפות פורמליות
הנה כמה דוגמאות של שפות פורמליות מעל האלפבית~$❴⌘a,⌘b,⌘c❵$:
\ספרר
 ✦ השפה הפורמלית~$L₃$ של כל המילים שאפשר לחלק אותן לשלוש מילים רצופות, זרות, וזהות. כמה מילים בשפה זו (מעל
✦ השפה~$L₃$ שפת הפלינדרומים (שפת ההפכפכות), כלומר השפה המכילה את כל המילים שלא תשתננה גם אם תיקראנה מסופן.
✦ השפה~$L₄$ שפת המילים שהאותיות שלהן מופיעות בסדר אלפאביתי לא יורד.
✦ השפה~$L₅$ שפת המילים שמכילות חמש אותיות או יותר שאף אחת מהן אינה~$⌘b$.
✦ השפה~$L₆$ שפת המילים שכל אותיותיהן שונות.
===
