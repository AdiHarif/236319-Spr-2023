§§ אלפבית
אלפבית הוא קבוצה, בדרך כלל סופית, של איברים הקרויים אותיות או סמלים. לדוגמה
הקבוצה
\begin{equation} \label{eq:alphabet:a:b:c}
  ❴⌘a,⌘b,⌘c❵,
\end{equation}
הינה אלפבית המכיל שלוש אותיות,~$⌘a$,~$⌘b$ ו-$⌘c$.

לאותיות האלפאבית אין משמעות מלבד העובדה שכולן שונות זו מזו. בפרק זה נשתמש בגופן
שונה כדי להבדיל בין האות עצמה, ובין מה שהאות מציינת: לכן, הכתיב~$⌘a$ מכוון אל
האות הראשונה באלפאבית הלטיני, ולא למה שאות זו מציינת, ואילו הכתיב~$a$ יתייחס
למה שאות זו \ע|מציינת|: למשל, במשוואה הריבועית \[
  a x²+bx+c=0
\] הכתיב~$a$ מתייחס למקדם של~$x²$.

\דוגמה |האלפאבית של שפת~\CPL|
האלפבית~$Σ_C$ המשמש לכתיבת תכניות בשפת~\CPL מכיל 95 תווים
אותם אפשר לחלק, לשם נוחות, לקבוצות הבאות:
\begin{equation}\label{alpahet:C}
  Σ_C=
  Σ_{\text{upper}}∪Σ_{\text{lower}}∪
  Σ_{\text{digit}}∪
  Σ_{\text{special}}∪
  Σ_{\text{space}}.
\end{equation}
\begin{enumerate}
  ✦ \ע|26 אותיות אנגליות גדולות| \[
    Σ_{\text{upper}}=❴⌘A,⌘B,⌘C,⌘D,⌘E,⌘F,⌘G,⌘H,⌘I,⌘J,⌘K,⌘L,⌘M,⌘N,⌘O,⌘P,⌘Q,⌘R,⌘S,⌘T,⌘U,⌘V,⌘W,⌘X,⌘Y,⌘Z❵.
\] ✦ \ע|26 אותיות אנגליות קטנות| \[
    Σ_{\text{lower}}=
    ❴⌘a,⌘b,⌘c,⌘d,⌘e,⌘f,⌘g,⌘h,⌘i,⌘j,⌘k,⌘l,⌘m,⌘n,⌘o,⌘p,⌘q,⌘r,⌘s,⌘t,⌘u,⌘v,⌘w,⌘x,⌘y,⌘z❵.
\] ✦ \ע|10 ספרות| \[
    Σ_{\text{digit}}=❴⌘0,⌘1,⌘2,⌘3,⌘4,⌘5,⌘6,⌘7,⌘8,⌘9❵.
\] ✦ \ע|29 אותיות מיוחדות| \[
    Σ_{\text{special}}=
    Σ_{\text{punctuation}}∪
    Σ_{\text{wrapping}}∪
    Σ_{\text{arithmetic}}∪
    Σ_{\text{other}}∪
    Σ_{\text{space}}.
\] המתחלקות באופן הבא:
  \begin{enumerate}
    ✦ \ע|8 אותיות פיסוק| \[
      Σ_{\text{punctuation}}=❰⌘., ⌘,, ⌘?, ⌘!, ⌘:, ⌘;, ⌘', ⌘"❱.
\] ✦ \ע|6 אותיות אריתמטיות| \[
      Σ_{\text{arithmetic}}=❰⌘+, ⌘*, ⌘/, ⌘-, ⌘<, ⌘>❱.
\] ✦ \ע|6 אותיות סוגריים| \[
      Σ_{\text{wrapping}}=❰⌘), ⌘), ⌘[ ⌘], ⌘❴, ⌘❵,❱.
\] ✦ \ע|9 אותיות אחרות| \[
      Σ_{\text{other}}=❰⌘&,⌘\textbackslash, ⌘\textasciicircum, ⌘\_, ⌘|, ⌘∿, ⌘\$,
      ⌘\%, ⌘#,❱.
\] \end{enumerate}
  ✦ \ע|6 אותיות רווח| \[
    Σ_{\text{space}}=❰\text{space},\text{tab},
    \text{horizontal tab}, \text{new line},
    \text{vertical tab}, \text{form feed}❱.
\] היצוג הגרפי של אותיות אלו הינו בלתי נראה, ולכן
  כתבנו כאן את שמות האותיות, ולא את היצוג הגרפי שלהן.
\end{enumerate}

\ע|מילה| מעל אלפבית היא סדרה סופית של אותיות מתוך האלפבית. למשל,~$⌘{caba}$ היא
מילה בת ארבע אותיות מעל האלפבית~$❴⌘a,⌘b,⌘c❵$. בהינתן אלפבית~$Σ$, נסמן
ב-$Σ^*$ את הקבוצה האינסופית המכילה את כל המילים באורך סופי מעל~$Σ$, לרבות
המילה הריקה, אותה בדרך כלל מסמנים ב-$ε$. בדוגמה שלנו
\begin{equation}
  ❴⌘a,⌘b,⌘c❵^*=❴ε,⌘a,⌘b,⌘c,⌘{aa},⌘{ab},⌘{ac},⌘{ba},⌘{bb},⌘{bc},⌘{ca},⌘{cb},⌘{cc},⌘{aaa},⌘{aab},…❵
\end{equation}

ניתן גם להגדיר את~$Σ^*$ רקורסיבית. בהגדרה זו, יהיה איבר אטומי אחד, המילה
הריקה~$ε$, ובנאי אונארי שמאפשר להאריך כל מילה ב-$Σ^*$ באות מתוך~$Σ$:

\החל{definition}[המילים מעל אלפבית]
בהנתן אלפבית~$Σ$ אזי~$Σ^*$, קבוצת ה\ע|המילים הפורמליות| מעל~$Σ$, מוגדרת
באמצעות הבנאי הנולארי (המגדיר איבר אטומי אחד ויחיד)
\begin{equation}
  \infer{ε∈Σ^*}{}
\end{equation}
והבנאי האונארי:
\begin{equation}
  \infer{wσ∈Σ^*}{w∈Σ^* &σ∈Σ}
\end{equation}
\סוף{definition}

בהסתמך על הגדרה רקורסיבית זו, נגדיר רקורסיבית את~$|w|$, מספר התווים במילה~$w$:
\החל{definition}[אורך מילה]\label{definition:length}
עבור~$w∈Σ^*$
\begin{equation}
  |w|=\begin{cases}
    |w'|+1 & w=w'σ ⏎
~0      & w=ε. ⏎
  \end{cases}
\end{equation}
\סוף{definition}

כך נקבל ש-$|ε=0|$,~$|⌘a|=1$,~$|⌘{caa}|=3$.

בהינת אלפבית~$Σ$ נגדיר כ-$Σ*$ את אוסף כל המחרוזות (Strings) הסופיות שניתן לכתוב
בעזרתו, ובכלל אלו את \ע|המילה הריקה| אשר מסומנת בדרך כלל כ-𝜺.  
