§§ הסביבה וה-\E|association list|.

השיערוך של ביטויי-\E|S| מוגדר רקורסיבית. בסיס הרקורסיה הוא בשיערוך של אטומים.
שיערוך של אטום אינו נדרש לבצע חישוב, אלא רק למצוא את משמעותו של האטום הנתון.
אטום בליספ הוא מזהה \E|(identifier)|, וכמו בשפות תכנות אחרות, משמעותו של המזהה
נמצאת בטבלת הסמלים \E|(symbol table)|. טבלת הסמלים היא מבנה נתונים הקושר בין
שמות ובין משמעותם, ומציאת המשמעות נעשית על ידי חיפוש בטבלה זו.

טבלת הסמלים בליספ בנויה במבנה נתונים הידוע בשם \E|association list|
או בקיצור \E|a-list|. ה-\E|a-list| היא רשימה של פריטים אשר כל אחד מהם מבטא
קישור של שם לערך. כל פריט הוא dotted pair אשר ה-car שלו הוא שם (המיוצג כאטום)
ואילו ה-cdr של הפריט הוא משמעותו של השם (שהיא~\E|S-Expression|, אטומי או מורכב.
אם משמעותו של האטום foo היא העץ~$ϕ$ אז הפריט הנשמר ב-\E|a-list|
הוא~\E|$(\text{foo}.ϕ)$|. הקישור בין השם והביטוי מתבטא בעובדה ששני אלו מחוברים
ברשומת \E|cons|.

בשפת מיני-ליספ, הסביבה מיוצגת ברשימה הקרויה-\E|a-list| המנוהלת כמחסנית, כלומר
פריטים מתווספים ומוסרים ממנה רק בתחילתה. החיפוש אחר משמעות ברשימה נעשה סדרתית,
בסריקה המתחילה בתחילת הרשימה. חיפוש המשמעות של האטום foo נעצר בזוג הראשון שבו
מרכיב השם הוא \E|foo|. החיפוש נכשל אם לא נמצא זוג כזה. כל הפעולות על הסביבה
נעשות בזמן שיערוך התכנית.

מעיון חוזר בדוגמה למעלה ניתן להסיק כי הסביבה בנויה גם בשפת~\CPL כמעין מחסנית
של מילונים. פתיחה של בלוק בסימן \texttt{❴} מוסיפה מילון חדש בראש מחסנית. הגדרה
של שם מתווספת לסביבה, באמצעות עדכון המילון שבראש המחסנית. הגדרות אלו מוסרות
מהסביבה עם סיום הבלוק, כלומר בסימן \texttt{❵}, אשר גורם לסילוק המילון שבראש המחסנית.

ישנם הבדלים טכניים לא מעטים בין הסביבה של~\CPL וזו של מיני-ליספ, אולם ישנו הבדל
מהותי אחד בין השתיים: פעולות על הסביבה במיני-ליספ נעשות בזמן שיערוך, הכולל בתוכו
מרכיב של חישוב. בניגוד לכך, הפעולות על הסביבה ב-\CPL, ובכלל הוספת שמות לסביבה,
וחיפוש שמות בתוכה, נעשים בזמן הידור, ועוד טרם להרצה.

כאשר האינטרפרטר של ליספ מתחיל את פעולתו, ה-\E|a-list| מכילה קישורים בעבור כמה
אטומים המוגדרים מראש בשפה. במיני-ליספ מספיק להניח כי ה-\E|a-list| מכילה קישורים
בעבור שני האטומים שבקבוצה~$\mathcal B$ בלבד, כלומר הרשימה תיתן משמעות ל-\E|t| ו-\E|nil|.
התוכן המינימלי של ה-\E|a-list| הוא
\begin{LISP}
((t.t)
     (nil.nil))
\end{LISP}

הפריט הראשון ברשימה הוא dotted pair הקובע שהמשמעות של האטום \T|t| היא אטום זה
עצמו. הפריט השני ברשימה קובע שמשמעותו של האטום \T|nil| היא אטום זה עצמו גם כן.

לפיכך, כאשר האינטרפרטר של ליספ יקרא את האטום \T|t| הוא ידפיס בתגובה \T|T|,
וכאשר הוא יקרא את האטום \T|nil| הוא ידפיס בתגובה \T|NIL|.
\begin{SAMPLE}
> t
T
> nil
NIL
\end{SAMPLE}

§§ הפונקציה set
הפונקציה \T|set| שבליספ היא פונקציה אטומית המקבלת שני פרמטרים, שהראשון בהם
הוא אטום, והשני הוא ביטוי~\E|S| כלשהו. הפונקציה מוסיפה ל-\E|a-list| פריט שהוא
dotted pair המבטא קישור בין האטום לביטוי. במרבית הניבים של ליספ יש ואריאנטים
רבים של \T|set| אולם במיני-ליספ נסתפק ב-\T|set| בלבד.

אם נזין לאינטרפרטר של ליספ את הביטוי
\begin{SAMPLE}
> (set foo (bar baz))
\end{SAMPLE}
כבקשה לקשור את האטום \T|foo| לרשימה \T|(bar baz)|, הרי ניתקל בשגיאות, שכן
האינטרפרטר ינסה לשערך את שני הארגומנטים של הפונקציה \E|set| טרם שהוא יפעיל
פונקציה זו. אלא ששיערוך זה יכשל, שכן לאף אחד משלושת האטומים המופיעים בקריאה
ל-set אין משמעות מלכתחילה.

בכדי להתגבר על מכשלה זו יש להשתמש בפונקציה \T|quote|, פונקציה של ארגומנט אחד
אשר \ע|אינה| משערכת את הארגומנט שלה, אלא מחזירה אותו כמו שהוא.
כך למשל
\begin{SAMPLE}
> (quote foo)
FOO
> (quote (bar baz))
(BAR BAZ)
\end{SAMPLE}
קשירת האטום \T|foo| לרשימה \T|(bar baz)| יכול להעשות באמצעות
\begin{SAMPLE}
> (set (quote foo) (quote (bar baz)))
(BAR BAZ)
> foo
(BAR BAZ)
\end{SAMPLE}
נשים לב לכך שהשיערוך של הביטוי
\begin{LISP}
(set (quote foo) (quote (bar baz)))
\end{LISP}
אינו במטרה לחשב ערך כלשהו, אלא במטרה לקשור שם לערך. נזכר שבאופן כללי פעולת
השיערוך בליספ כוללת שלושה סוגים של פעולות: מציאת משמעות של שם, הגדרת משמעות של
שם, וחישוב שהוא הפעלת פונקציה על ביטוי~\E|S| או ביטויי~\E|S|. בכל זאת, כל
פונקציה בליספ חייבת להחזיר ערך.

במקרה של הפונקציה set הערך המוחזר הוא ערכו של הפרמטר השני שלה. בתגובה לקלט
\begin{LTR}
\begin{quote}
\T|(set (quote foo) (quote (bar baz)))|
\end{quote}
\end{LTR}
האינטרפרטר ידפיס את הערך \LR{\T|(BAR BAZ)|}:
\begin{SAMPLE}
> (set (quote foo) (quote (bar baz)))
(BAR BAZ)
\end{SAMPLE}

הפונקציה set מוסיפה זוג לתחילת ה-\E|a-list|: לכן, אם התוכן של רשימת
ה-\E|a-list| לפני הקריאה ל-set היה
\begin{LISP}
((t.t)
     (nil.nil))
\end{LISP}

\pagebreak[3]
אזי אחרי הקריאה תוכן רשימה זו יהיה
\begin{LISP}
((foo.(bar baz))
     (t.t)
     (nil.nil))
\end{LISP}

ויזואלית, נשרטט את רשימת ה-\E|a-list| לפני הקריאה כך,
\begin{LTR}
  \begin{tikzpicture}[list/.style={rectangle split, rectangle split parts=2,
          draw,minimum height=3ex, fill=blue!20,rectangle split horizontal}, >=stealth, start chain, node distance=3ex]
    \foreach \x/\y/\z in {%
        h/t/t,
        i/nil/nil
      } {%
        \node[on chain, list,font=\tt\scriptsize] (\x) {\y};
        \node[below=4 ex of \x.one,anchor=north west,align=left,font=\tt\scriptsize,color=red] (temp) {\z};
        \draw[->,bend left] (\x.one south) .. controls+(270:0.3) and+(120:0.6) .. (temp.north west);
      }

    \node[on chain,font=\tt\scriptsize] (j) {nil};
    \draw[*->] let \p1=(i.two), \p2=(j.center) in (\x1,\y2)--(j);

    \foreach \a/\b in {h/i} {%
        \draw[*->] let \p1=(\a.two), \p2=(\b.center) in (\x1,\y2)--(\b);
      }
    \node[above=of h] (A) {a-list};
    \draw[->] (A.south)--(h);
  \end{tikzpicture}
\end{LTR}
ואחרי השיערוך של הביטוי
{\setLTR
\begin{quote}
  \T|(set (quote foo) (quote (bar baz)))|
\end{quote}
}
תראה ה-\E|a-list| כך,
\begin{LTR}
  \begin{tikzpicture}[list/.style={rectangle split, rectangle split parts=2,
          draw,minimum height=3ex, fill=blue!20,rectangle split horizontal}, >=stealth, start chain, node distance=3ex]
    \foreach \x/\y/\z in {%
        g/foo/(bar baz),
        h/t/t,
        i/nil/nil
      } {%
        \node[on chain, list,font=\tt\scriptsize] (\x) {\y};
        \node[below=4 ex of \x.one,anchor=north west,align=left,font=\tt\scriptsize,color=red] (temp) {\z};
        \draw[->,bend left] (\x.one south) .. controls+(270:0.3) and+(120:0.6) .. (temp.north west);
      }

    \node[on chain,font=\tt\scriptsize] (j) {nil};
    \draw[*->] let \p1=(i.two), \p2=(j.center) in (\x1,\y2)--(j);

    \foreach \a/\b in {g/h,h/i} {%
        \draw[*->] let \p1=(\a.two), \p2=(\b.center) in (\x1,\y2)--(\b);
      }
    \node[above=of g] (A) {a-list};
    \draw[->] (A.south)--(g);
  \end{tikzpicture}
\end{LTR}
שפת מיני-ליספ יכולה לאתחל את ה-a-list באמצעות
\begin{LIBRARY}
(set (quote t) (quote t))
(set (quote nil) (quote nil))
\end{LIBRARY}
לא ניתן \ע|להסיר| פריטים מתוך ה-\E|a-list|. אבל, ניתן \ע|להסתיר| קישור באמצעות
הוספת פריט לרשימה, אשר יסתיר את הקישור הקודם. בפרט, אם נכתוב כעת
\begin{SAMPLE}
> (set (quote foo) (quote ((baz))))
(BAZ)
\end{SAMPLE}
הרי התוכן של ה-a-list יהיה
\begin{LISP}
((foo.((baz)))
     (foo.(bar baz))
     (t.t)
     (nil.nil))
\end{LISP}
וויזואלית כך,
\begin{LTR}
  \begin{tikzpicture}[list/.style={rectangle split, rectangle split parts=2,
          draw,minimum height=3ex, fill=blue!20,rectangle split horizontal}, >=stealth, start chain, node distance=3ex]
    \foreach \x/\y/\z in {%
        f/foo/((baz)),
        g/foo/(bar baz),
        h/t/t,
        i/nil/nil
      } {%
        \node[on chain, list,font=\tt\scriptsize] (\x) {\y};
        \node[below=4 ex of \x.one,anchor=north west,align=left,font=\tt\scriptsize,color=red] (temp) {\z};
        \draw[->,bend left] (\x.one south) .. controls+(270:0.3) and+(120:0.6) .. (temp.north west);
      }
    \node[on chain,font=\tt\scriptsize] (j) {nil};
    \draw[*->] let \p1=(i.two), \p2=(j.center) in (\x1,\y2)--(j);

    \foreach \a/\b in {f/g, g/h,h/i} {%
        \draw[*->] let \p1=(\a.two), \p2=(\b.center) in (\x1,\y2)--(\b);
      }
    \node[above=of f] (A) {a-list};
    \draw[->] (A.south)--(f);
  \end{tikzpicture}
\end{LTR}
ואם ננסה לשערך את \T|foo| לאחר ההסתרה הזו, נקבל את הערך הנוצר מהקישור החדש
\begin{SAMPLE}
> foo
((BAZ))
\end{SAMPLE}
וזאת משום שהחיפוש ב-a-list מתחיל מתחילתה, ועוצר במקום הראשון שבו הוא מצליח.

§§ סימן ה-\E|quote| לסימון הפונקציה quote
\תגית|פרק:quote|

כתיב מקוצר לביטוי
\begin{LTR}
\begin{quote}
  \setLTR \[
    \text{(\E|quote~$s$|)}
\] \end{quote}
\end{LTR}
המפעיל את הפונקציה quote על הביטוי~$s$, הוא \[
  's
\] כלומר סימן מרכאה בודד
לפני הביטוי~$s$. הטיפול בכתיב זה נעשה על ידי
ה-Reader וה-Printer של ליספ, והם אינם חלק מהמשערך של השפה.
הכתיב המקוצר פועל כאשר הביטוי~$s$ הוא אטום בודד, למשל,
\begin{SAMPLE}
> 'foo
FOO
> (quote 'foo)
'FOO
\end{SAMPLE}
וגם כאשר~$s$ הוא ביטוי מורכב כגון
\begin{SAMPLE}
> '(a (b c))
(A (B C))
\end{SAMPLE}
הקריאה ל-\E|set| בשימוש בכתיב המקוצר היא אכן קצרה יותר,
\begin{SAMPLE}
> (set 'foo '(bar baz))
(BAR BAZ)
> foo
(BAR BAZ)
\end{SAMPLE}
הכתיב המקוצר של הפונקציה quote יכול להיות מקונן, כלומר הביטוי המורכב עליו
מופעלת \E|quote| יכול להכיל הפעלה של \E|quote| בעצמו:
\begin{SAMPLE}
> '(quote foo)
'FOO
\end{SAMPLE}
גם סימן המרכאה לציון quote גם הוא יכול להיות מקונן, ואת הביטוי
\begin{LISP}
(quote ((quote a) (quote b)))
\end{LISP}
ניתן לכתוב בקיצור כך
\begin{LISP}
'('a 'b)
\end{LISP}
גם אם לא נשתמש בסימן ה-\E|quote| האינטרפרטר יעשה זאת בעבורינו:
המקרים שבהם יש צורך להשתמש בפונקציה quote מקוננת אינם נפוצים, אולם הם מופיעים
מדי פעם.
\begin{SAMPLE}
> (quote (quote ((quote a) (quote b))))
'('a 'b)
\end{SAMPLE}



§
§ ביטויי~$λ$
מתכנת בליספ יכול להוסיף פריטים לטבלת הסמלים באמצעות הפונקציות \E|set|
ו-\E|defun|.

הפונקציה defun מאפשרת להגדיר פונקציות חדשות. נגדיר לדוגמה פונקציה בשם mirror
המקבלת פרמטר x שהוא ביטוי~\E|S| מורכב, ומחזירה ביטוי מורכב אחר שבו ה-car וה-cdr
שב-x הוחלפו.
\begin{LISP}
(defun mirror ; define a new function named mirror
  (x) ; which expects a single parameter, named x
  (cons (cdr x) (car x)); and whose body is this S-expression
)
\end{LISP}
אנו רואים שהפונקציה defun מקבלת שלושה פרמטרים: שם הפונקציה אותה יש להגדיר,
רשימת הפרמטרים ה\ע|פורמליים| לפונקציה, וגוף הפונקציה. הדוגמה גם מראה שהערות
בליספ מתחילות בסימן~";" (נקודה ופסיק) ונמשכות עד סוף השורה.

אחרי הגדרה זו של \E|mirror| באמצעות \E|defun|, נוכל להשתמש בפונקציה החדשה
שהגדרנו כך:
\begin{SAMPLE}
> (mirror '(a.b))
(B.A)
\end{SAMPLE}

מקובל לקרוא ל-\E|a-list| גם סביבה \E|(environment)|. הסביבה היא אוסף של קישורים
בין שמות למשמעותם, והיא זו שנותנת משמעות לשמות. ראינו כבר שהפונקציה \E|set|
מוסיפה קישור לסביבה. גם הפונקציה \E|defun| מוסיפה קישור לסביבה: שיערוך הביטוי
\begin{LISP}
(defun mirror (x) (cons (cdr x) (car x)))
\end{LISP}
מביא לקשירה בין השם mirror ובין מימוש הפונקציה. שיערוך הביטוי \T|(mirror
'(a.b))| משתמש בשם זה, למציאת מימוש הפונקציה.

המימוש של פונקציה כולל את ביטוי ה-\E|S| המהווה את גוף הפונקציה. בדוגמה שלנו ביטוי
זה הוא
\begin{LISP}
(cons (cdr x) (car x))
\end{LISP}
אבל, מלבד גוף הפונקציה המימוש כולל גם את רשימת הפרמטרים. כל \ע|פרמטר|
הוא אטום, אשר בזמן הפעלת הפונקציה יוחלף בביטוי-\E|S|. האטום מהווה לכן את שמו של
הביטוי. הביטוי המחליף את האטום קרוי \ע|ארגומנט|. כיוון שערכם של הפרמטרים
אינו ידוע בזמן הגדרה של פונקציה. לעיתים משתמשים במונח \ע|פרמטרים פורמליים|.
הארגומנטים נקראים לעיתים גם \ע|פרמטרים אקטואליים|.

בפונקציה mirror יש פרמטר אחד אשר מיוצג ברשימה \T|(x)|. שמות הפרמטרים הפורמליים
קובעים כיצד יש לחשב את גוף הפונקציה: בכל מקום שבגוף הפונקציה מופיע שם של פרמטר
פורמלי. בזמן הקריאה לפונקציה וחישוב הגוף יש להחליף שם זה בארגומנט האקטואלי,
כלומר, הערך המחושב.

הצירוף של גוף הפונקציה ושמות הפרמטרים אליה נקרא ביטוי-$λ$. ביטוי כזה
מייצג פונקציה אנונימית. ביטוי ה-$λ$ בעבור mirror הוא
\begin{LISP}
  (lambda (x) (cons (cdr x) (car x)))
\end{LISP}

באופן כללי ביטוי~$λ$ הוא רשימה בת שלושה פריטים בדיוק:
\ספרר
✦ האטום \T|lambda| המזהה את הרשימה כביטוי~$λ$.
✦ רשימה של אטומים, המייצגת את שמות הפרמטרים הפורמליים לפונקציה. בביטוי שלמעלה,
זו רשימה המכילה איבר אחד בלבד, האטום \E|x|, שהוא שמו של הפרמטר הפורמלי
לפונקציה.
✦ גוף הפונקציה, כלומר הביטוי שהשיערוך שלו אחרי הקשירה בין הפרמטרים
הפורמליים לאקטואליים, יתן את ערכה של הפונקציה. בביטוי שלמעלה גוף הפונקציה הוא
\lisp{(cons (cdr x) (car x))}.
===

נשים לב ששם הפונקציה אינו חלק מביטוי ה-$λ$, שכן ביטוי זה מציין פונקציה
\ע|אנונימית|. הפונקציה defun יוצרת ביטוי-$λ$ וקושרת אותו לשם הפונקציה המוגדרת.
מסיבות של יעילות, מרבית המימושים של ליספ מנהלים את הקישורים בין שמות הפונקציות
ובין הגוף שלהן באמצעות מנגנונים יעודיים. במיני-ליספ הקישורים הללו מנוהלים
באמצעות ה-\E|a-list|:

אם תוכן ה-\E|a-list| הוא
\begin{LISP}
(
  (foo.(bar baz))
  (t.t)
  (nil.nil)
)
\end{LISP}
אז לאחר שיערוך הביטוי
\begin{LISP}
(defun mirror (x)
  (cons (cdr x) (car x)))
\end{LISP}
תוכן ה-\E|a-list| יהיה
\begin{LISP}
((mirror.
        (lambda (x)
        (cons (cdr x) (car x))))
     (foo.(bar baz))
     (t.t)
     (nil.nil))
\end{LISP}

ניתן במיני-ליספ לייצר ביטוי-$λ$ מבלי להשתמש ב-\E|defun|, בשימוש בפונקציה
\E|quote|,
\begin{LISP}
  '(lambda (x) (cons (cdr x) (car x)))
\end{LISP}
ולכן ניתן להגדיר את הפונקציה mirror תוך שימוש ב-set במקום ב-\E|defun|,
\begin{LISP}
(set mirror
  '(lambda (x) (cons (cdr x) (car x))))
\end{LISP}
לחילופין, ניתן להשתמש בפונקציה lambda המקבלת שני פרמטרים: הראשון שבהם הוא רשימה
של אטומים, המייצגים את שמות הפרמטרים הפורמליים לפונקציה, והשני הוא גוף הפונקציה
אשר אותו יש לשערך בקריאה לפונקציה, בסביבה הכוללת קישורים בין הפרמטרים הפורמליים
ובין האקטואליים. תוצאת השיערוך של קריאה לפונקציה lambda היא ביטוי~\E|S| שנראה
בדיוק כמו הקריאה לפונקציה.
\begin{SAMPLE}
> (lambda (x) (cons (cdr x) (car x)))
(LAMBDA (X) (CONS (CDR X) (CAR X)))
\end{SAMPLE}
ניתן לחקות את הפעולה של defun באמצעות set ושימוש בביטויי~$λ$.
\begin{LISP}
(set mirror
  (lambda (x) (cons (cdr x) (car x))))
\end{LISP}
לאחר הקישור בין השם mirror ובין ביטוי ה-$λ$ המתאר את מימוש הפונקציה \E|mirror|,
האטום mirror ישוערך לביטוי~$λ$ זה. בקריאה \T|(mirror '(a.b))| תוחלף המילה
mirror בביטוי זה, אשר מגדיר כיצד לבצע את הקישור בין הפרמטרים האקטואליים
והפורמליים.

ניתן גם לקרוא לפונקציה מבלי לנקוב בשמה, אלא תוך שימוש בביטוי המתאר את המימוש
שלה:
\pagebreak[3]
\begin{SAMPLE}
> (
    '(lambda (x)
      (cons (cdr x) (car x)))
    '(a.b)
)
(B.A)
\end{SAMPLE}
ניתן להשמיט את סימן ה-\E|quote|, ולנצל את העובדה שהאטום \T|lambda| אינו רק
הפתיח של ביטוי~$λ$ אלא גם פונקציה המחזירה ביטוי כזה,
\begin{SAMPLE}
> (
    (lambda (x) (cons (cdr x) (car x)))
    '(a.b)
)
(B.A)
\end{SAMPLE}

§§ שיערוך מותנה ורקורסיה

שיערוך מותנה פירושו שהשיערוך מתבצע כאשר תנאי מסוים מתקיים, והוא אינו מתבצע
כאשר התנאי אינו מתקיים. בליספ שיערוך מותנה מתבצע באמצעות הפונקציה \E|cond|
אשר מכלילה את פקודות ה-if וה-switch שיש בשפת-\E|\CPL|.

נשתמש ב-cond כדי להגדיר פונקציה zcar המחזירה את ה-car של הפרמטר אם
הוא ביטוי מורכב, ואת הפרמטר עצמו אם הוא אטום
\begin{LISP}
(defun zcar(x)
  (cond ; check a number of options
    ((atom x) x) ; if parameter x is an atom, return it
    (t (car x)) ; else, if 'T' (all other cases), return its car
))
\end{LISP}
גוף הפונקציה zcar הוא הביטוי
\begin{LISP}
  (cond ((atom x) x) (t (car x)))
\end{LISP}
שהוא קריאה לפונקציה cond עם שני פרמטרים שכל אחד מהם הוא רשימה בת שני איברים:
\begin{LTR}
  \begin{itemize}
    ✦ \lisp{((atom x) x)}
    ✦ \lisp{(t car(x))}
  \end{itemize}
\end{LTR}
cond היא פונקציה רב מקומית המקבלת מספר כלשהו של פרמטרים שכל אחד מהם נקרא
\E|test-form|. כל \E|test-form| הוא רשימה בת שני פריטים שכל אחד מהם הוא ביטוי~\E|S|
אשר עשוי להיות משוערך במהלך פעולתה של \E|cond|. הפריט הראשון ב-\E|test-form|
נקרא test-condition והפריט השני נקרא \E|test-value|.

הפונקציה cond עוברת על ה-test-forms שקיבלה כפרמטרים לפי סדרם, ובכל אחד כזה היא
משערכת את ה-\E|test-condition|.
\אבגד
✦ אם תוצאת השיערוך היא \E|nil|, כלומר ה-test-condition אינו מתקיים, אזי cond
\ע|אינה| משערכת את ה-\E|test-value|, וממשיכה ל-test-form הבא.
✦ אם לעומת זאת תוצאת השיערוך אינה \E|nil|, כלומר ה-test-condition מתקיים, אזי
cond משערכת את ה-\E|test-value|, ומחזירה את תוצאת השיערוך הזו, \ע|מבלי|
להמשיך לשאר ה-\E|test-forms|.
✦ אם cond ממצה את רשימת ה-\E|test-forms| מבלי להיתקל באף \E|test-condition|
שמתקיים, הרי cond מחזירה את הערך \E|nil|.
===
על פי תיאור זה של הפונקציה \E|cond|, הביטוי \begin{LISP}
  (cond ((atom x) x) (t (car x)))
\end{LISP}
שבגוף הפונקציה zcar ישוערך לכן ל-x אם x הוא אטום, ול-\E|car| של \E|x|, אם~x
איננו אטום.

מקובל להשתמש תמיד באטום \E|t| דווקא כדי לציין את התנאי שמתקיים תמיד, כלומר חלק
ה-else של פקודת ה-\E|if|, אבל טכנית ניתן היה להגדיר את zcar תוך שימוש בביטוי
אחר שערכו אינו \E|nil|, למשל
\begin{LISP}
(defun zcar(x)
  (cond ((atom x) x) ('(any S expression) (car x))))
\end{LISP}
הנה דוגמה נוספת בה cond מופעלת עם מספר רב יותר של test-forms
\begin{LISP}
(defun is-atomic(name) ; determine whether name denotes a atomic function
  (cond ((eq name 'atom) t)
        ((eq name 'car) t)
        ((eq name 'cdr) t)
        ((eq name 'cond) t)
        ((eq name 'cons) t)
        ((eq name 'eq) t)
        ((eq name 'error) t)
        ((eq name 'eval) t)
        ((eq name 'set) t)
        (t nil)))
\end{LISP}
הפונקציה is-atomic משתמשת בסידרה של test-forms כדי לבדוק אם הפרמטר שלה הוא
אטום המציין פונקציה אטומית. אם תנאי זה מתקיים, הפונקציה מחזירה את האטום
\E|t|. (מסיבות טכניות אנו מניחים כאן שגם \T|eval| היא פונקציה אטומית.)
האטום \E|nil| יוחזר בכל מקרה שהפרמטר אינו אטום, או שהוא אטום שאינו מציין
פונקציה אטומית. לשם כך, משתמשת
is-atomic בפונקציה \T|eq|, שהיא פונקציה אטומית, המשמשת לבדיקה אם הפרמטרים
שלה הם שני אטומים השווים זה לזה.

כדאי לשים לב לכך שה-test-form האחרון שברשימה (כלומר הרשימה \T|(t nil)|) מיותר,
שהרי אם אף אחד מה-test-conditions שב-test-forms שקדמו לו אינו מתקיים, ממילא
cond תחזיר \E|nil|. בכל זאת, מקובל להוסיף את ה-test-form \T|(t nil)| אף אם אינו
נחוץ, כדי להדגיש שערך ברירת המחדל הוא \E|nil|.

דרך קצרה יותר להגדיר את הפונקציה is-atomic היא באמצעות בדיקה אם הפרמטר
שלה מצוי ברשימה המכילה את שמות כל הפונקציות האטומיות
\begin{KERNEL}
(defun is-atomic(name); determine whether name denotes an atomic function
  (exists name '(atom car cdr cond cons eq error eval set)))
\end{KERNEL}
במימוש זה, אטום הוא שמה של פונקציה אטומית אם הוא מצוי בתוך רשימה של האטומים
המציינות פונקציה אטומית. הפונקציה exists עצמה אינה מצויה במיני-ליספ, אבל
קל לממש אותה ברקורסיה.
\begin{LISP}
(defun exists (x xs) ; determine whether atom x is in list xs
  (cond ; three case to consider
    ((eq xs nil) nil) ; (i) list of xs is exhausted
    ((eq x (car xs)) t) ; (ii) item x is first in xs
    (t (exists x (cdr xs))))) ; (iii) otherwise, search recursively for x on rest of xs
\end{LISP}
הפונקציה \T|exists| מקבלת שני פרמטרים: הפרמטר הראשון הוא אטום, המסומן ב-\T|x|
והפרמטר השני הוא רשימה של אטומים המסומנת ב־\T|xs|. על פי הגדרה זו, הפונקציה
\T|exists| מקבלת אטום המסומן ב-\T|x| ורשימה של אטומים המסומנת ב-\T|xs|,
ומחזירה~t אם האטום x מצוי בין פריטי הרשימה-\T|xs| ו-\T|nil| אחרת.

חישוב רקורסיבי דורש תמיד בדיקה של תנאי בכדי להבחין בין שני מצבים בהם עשוי
להיתקל החישוב: המצב של בסיס הרקורסיה, אשר בו \ע|לא ניתן| לבצע קריאות רקורסיביות
נוספות, והמצב הכללי אשר בו \ע|קיימת אפשרות| להפעלה רקורסיבית נוספת. פעמים רבות
התנאי הוא משולש ומבחין בין שלושה מצבים: המצב של בסיס הרקורסיה, המצב של עצירה
מוקדמת של החישוב הרקורסיבי, אשר בו האפשרות להפעלה רקורסיבית נוספת \ע|אינה
מנוצלת|, והמצב בו \ע|נדרשת| הפעלה רקורסיבית נוספת בכדי לסיים את החישוב.
.
הפונקציה \T|exists| משתמשת בתנאי משולש כזה, ומעבירה לכן ל-\T|cond| שלושה
\E|test-forms|:
\ספרר
✦ \T|((eq xs nil) t)| כלומר, אם רשימת ה-\T|xs| ריקה, ברור ש-\T|s| אינו מצוי
בתוכה ואז \T|exists| מחזירה \T|nil|. תנאי זה מתאר את המצב הראשון, של בסיס
הרקורסיה. אם תנאי זה אינו מתקיים, \T|exists| עוברת לבדוק את התנאי הבא:

✦ \T|((eq x (car xs)) t)| כלומר, הפונקציה בוחנת את הפריט הראשון ברשימה \T|xs|,
על ידי חישוב הביטוי \T|(car xs)| (מובטח שחישוב \T|(car xs)| יצליח שכן
ה-\E|test-form| הקודם שבו נבדק אם הרשימה ריקה נכשל) ומשווה אותו ל-\T|x|. אם
מתקיים שיוויון, החיפוש הצליח. במצב זה של עצירה מוקדמת, הפונצקיה-\T|exists|
מחזירה \T|t| מבלי להידרש לקריאות רקורסיביות נוספות.

במקרה שו ההשוואה נכשלת, השיערוך ממשיך ל-\E|test-form| הבא.

✦ \T|(t (exists x (car xs)))|, כלומר, במקרה שהרשימה \T|xs| אינה ריקה, ו-\T|x|
אינו שווה לפריט הראשון שב-\T|xs|, הפונקציה \T|exists| קוראת לעצמה רקורסיבית
לחיפוש של \T|x| בשארית הרשימה אשר מתקבלת מחישוב \T|(cdr xs)|.
===
הבדיקה אם הרשימה ריקה נעשית באמצעות חישוב התנאי \T|(eq xs nil)|. השוואה של אטום
ל-nil היא נפוצה בחישובים רקורסיביים. מיני-ליספ מגדירה פונקציה הנקראת null
המיועדת לשם ביצוע השוואה זו:
\begin{LIBRARY}
(defun null(x) (eq x nil))
\end{LIBRARY}

ניזכר שהפונקציה eq היא פונקציה אטומית, במובן זה שלא ניתן לממש אותה באמצעות
פונקציות אחרות, שהרי מלבד פונקציה זו, אין אף פונקציה אטומית שמאפשרת לבדוק את
תוכנו של אטום. לעומת זאת, הפונקציה null שבמיני-ליספ ניתנת להגדרה
באמצעות eq, והיא אכן נמנית ברשימת הפונקציות המוגדרות מראש.

מימוש exists באמצעות null יראה כך
\begin{KERNEL}
(defun exists (x xs) ; determine whether atom x is in list xs
  (cond ; Three cases to consider:
    ((null xs) xs) ; (i) list of xs is exhausted
    ((eq x (car xs)) t) ; (ii) item x is first in xs
    (t (exists x (cdr xs))))) ; (iii) otherwise, recurse on rest of xs
\end{KERNEL}

אנו מבדילים בין שני סוגים של סמנטיקות של שיערוך של פונקציה:
\begin{description}
  ✦ [eager] בקריאה לפונקציה כגון set שהסמנטיקה שלה היא \E|eager|, הארגומנטים
  לפונקציה משוערכים \ע|טרם| הקריאה לפונקציה, ורק ערכי השיערוך מועברים לפונקציה.
  הפונקציה אינה יכולה לדעת מה היו ערכי הארגומנטים לפני ששוערכו.

  פונקציות בעלות סמנטיקה שהיא eager מיוצגות על ידי ביטוי~$λ$ שהאטום הראשון שבו
  הוא \T|lambda|.

  ✦ [normal] לעומת זאת, בקריאה לפונקציה כגון quote שהסמנטיקה שלה היא
  \E|normal|, הארגומנטים לפונקציה \ע|אינם| משוערכים טרם הקריאה לפונקציה,
  והפונקציה יכולה לבחור אם לשערך את הארגומנטים.

  פונקציות בעלות סמנטיקה שהיא normal מיוצגות על ידי ביטוי~$λ$ שהאטום הראשון שבו
  הוא האטום \T|nlambda|.†{%
  שימוש זה ב-nlambda היה קיים במימושים הראשונים של ליספ, אולם ברוב המימושים המודרניים
  הוא בוטל מסיבות של יעילות. במקומו, נוסף מה שקוראים מקרו \E|(macro)|
  שהוא דומה אך לא זהה ל-\E|nlambda|.}

  כלומר, אנו מבחינים בין שני סוגים של ביטויי~\E|$λ$| בליספ: כאלו שמשערכים את
  הפרמטרים שלהם לפני הפעלתם והמסומנים על ידי האטום \T|lambda| וכאלו שאינם עושים
  זאת, והמסומנים על ידי האטום \T|nlambda|.
\end{description}

הסמנטיקה של \ע|כל הפונקציות| וכמעט כל האופרטורים בשפות כמו \E|\CPL|, היא
\E|eager|. כמפורט ב\פנה|טבלה:אטומיות| שבהמשך, הסמנטיקה של כל הפונקציות
האטומיות של מיני-ליספ היא \E|eager|, זאת מלבד הפונקציה \E|cond|, אשר הסמנטיקה
שלה היא \E|normal|. הסיבה לכך היא ש-cond מדמה במיני-ליספ את פקודת התנאי \E|(if
command)| של שפות אחרות. למעשה, cond מכלילה את האופרטור הטרנארי
\LR{\texttt{$·$?$·$:$·$}}, כך שיוכל לבדוק מספר תנאים, ולא תנאי אחד בלבד.

הפונקציה quote שבמיני-ליספ אינה אטומית שכן ניתן להגדיר אותה באמצעות ביטוי
ה-nlambda הפשוט הבא
\begin{LISP}
(nlambda (x) x)
\end{LISP}
לא ניתן להגדיר פונקציות שהן normal באמצעות \E|defun|, אך ניתן להוסיף את הפונקציה
quote לרשימת ה-a-list באמצעות set
\begin{LIBRARY}
(set 'quote '(nlambda (x) x))
\end{LIBRARY}
כפי שראינו מיני-ליספ מגדירה גם פונקציה lambda שהסמנטיקה שלה היא normal המאפשרת
ליצור ביטוי~$λ$ מבלי להשתמש ב-quote
\begin{SAMPLE}
> (lambda (x) (cons (cdr x) (car x)))
(LAMBDA (X) (CONS (CDR X) (CAR X)))
\end{SAMPLE}
נוכל לכן להגדיר את הפונקציה lambda באמצעות
\begin{LIBRARY}
(set 'lambda '(nlambda(parameters body) ('lambda parameters body)))
\end{LIBRARY}
כלומר, הפונקציה lambda היא פונקציה שמקבלת שני פרמטרים: parameters
ו-\E|body|. הפונקציה אינה משערכת פרמטרים אלו ומחזירה ביטוי lambda המוגדר על ידם
\begin{LISP}
  ('lambda parameters body)
\end{LISP}
גם הפונקציה nlambda מסייעת להגדרת ביטויי nlambda באופן דומה, וניתן להגדירה
באמצעות
\begin{LIBRARY}
(set 'nlambda '(nlambda (parameters body) ('nlambda parameters body)))
\end{LIBRARY}
כלומר, גם הפונקציה nlambda היא פונקציה שמקבלת שני פרמטרים שאינם משוערכים
(\E|parameters| ו-\E|body|) ומחזירה ביטוי nlambda המוגדר על ידי
שני הפרמטרים הללו
\begin{LISP}
  ('nlambda parameters body)
\end{LISP}
לאחר הגדרה זו של הפונקציה nlambda הכתיב
\begin{LISP}
  (nlambda x y)
\end{LISP}
זהה לכתיב
\begin{LISP}
  ('nlambda x y)
\end{LISP}
ללא תלות בערכיהם של x ו-\E|y|.

ניתן גם להגדיר את הפונקציה defun באמצעות ביטוי nlambda
\begin{LIBRARY}
(set 'defun '(nlambda (name parameters body)
  (set name (lambda parameters body))))
\end{LIBRARY}
כלומר, הפונקציה defun היא ביטוי nlambda אשר מקבל שלושה פרמטרים: \E|name|,
\E|parameters| ו-\E|body|, ואשר קושר (באמצעות \E|set|) את הפרמטר אל ביטוי
ה-lambda המוגדר על ידי שני הפרמטרים האחרים, והכל מבלי לשערך את הפרמטרים שלו.

ניתן גם להגדיר את הפונקציה ndefun המגדירה פונקציה שהסמנטיקה שלה היא normal
\begin{LIBRARY}
(set 'ndefun '(nlambda (name parameters body)
    (set name (nlambda parameters body))))
\end{LIBRARY}
גוף ההגדרה זהה לזה של defun אלא ש-ndefun יוצרת ביטוי nlambda במקום ביטוי
\E|lambda|.

לאחר שהגדרנו את הפונקציה ndefun נוכל להגדיר את quote באמצעותה
\begin{LISP}
(ndefun quote(x) x)
\end{LISP}
הפונקציות \E|quote|, \E|defun|, \E|ndefun|, \E|lambda| ו-\E|nlambda| אינן
משערכות את הפרמטרים שלהן אף פעם. כאשר יש צורך בשיערוך הפרמטרים ניתן לקרוא
לפונקציה \E|eval|. נגדיר לדוגמה את הפונקציה setq אשר מאפשרת לקשור ערך לאטום ללא
צורך לבצע quoting על שם האטום
\begin{LISP}
(ndefun setq(a value)
  (set a (eval value)))
\end{LISP}
השימוש בסמנטיקה שהיא normal אינו מתעורר רק בהגדרת פונקציות כמו quote ו-defun
המשמשות לביצוע הגדרות. נגדיר למשל את הפונקציה \T|?:| אשר מתנהגת באופן דומה
לאופרטור בשם זה בשפת~\CPL
\begin{LISP}
(ndefun ?: (condition true-value false-value)
  (cond ((eval condition) (eval true-value))
        (t (eval false-value))))
\end{LISP}
כדוגמה נוספת, נגדיר פונקציות \lisp{||} ו-\lisp{&&} הדומות לאופרטורים בשמות אלו
בשפת~\CPL, כלומר חישוב של הפונקציות הבוליאניות של and ו-or בשיטה הידועה בשם
\E|short-circuit|: ראשית שיערוך של הפרמטר הראשון, ואם זה קובע את תוצאת הפונקציה
הבוליאנית, הימנעות משיערוך הפרמטר השני. אחרת, ערכה של הפונקציה הבוליאנית הוא תוצאת
שיערוך הפרמטר השני
\begin{LISP}
(ndefun && (x y)
  (cond ((eval x) (eval y))
        (t nil)))
(ndefun || (x y)
  (cond ((eval x) t)
        (t (eval y))))
\end{LISP}


