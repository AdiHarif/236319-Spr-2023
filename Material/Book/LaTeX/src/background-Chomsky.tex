  שפה פורמלית יכולה להיות סופית או אינסופית. שפה סופית ניתנת תמיד לתאור מדוייק
  באמצעות מניית כל המילים בשפה. ברשימה לעיל כל השפות הן שפות אינסופיות מלבד השפה
  האחרונה~$L₆$ המכילה בדיוק 13 מילים
  \begin{equation}\label{eq:L6}
    L₆=❴ε,⌘a,⌘b,⌘c,⌘{ab},⌘{ac},⌘{ba},⌘{bc},⌘{ca},⌘{cb},⌘{abc},⌘{acb},⌘{bac},⌘{bca},⌘{cab},⌘{cba}❵.
  \end{equation}
  את השפה האינסופית~$L₅$ ניתן להגדיר תוך שימוש ב\פנה|definition:length|
  \begin{equation}\label{eq:L5}
    L₅=❴w \,|\, w∈Σ^*, |w|≥5❵.
  \end{equation}

  כל ההגדרות המדוייקות של השפות הפורמליות
  $L₆$ \cref{eq:L6},
  $L₅$ \cref{eq:L5},
  ו-~$L₁$
  (\cref{definition:L1})
  היו שונות זו מזו, וכולן היו, במובן מסויים אד-הוק. כלומר, בחרנו בעבור כל שפה
  פורמלית, בשיטת הגדרה המתאימה לה. הגדרות של שפות תכנות משתמשות לעיתים קרובות
  בהגדרות אד-הוק, אבל לעיתים קרובות יותר, הן משתמשות במנגנונים כלליים להגדרת
  שפות פורמליות.
  ארבעת המנגנונים העיקריים הם:
  \begin{enumerate}
    ✦ \ע|ביטויים רגולריים| (\E|regular expressions|)
    אותם נכיר ב-\cref{section:regular}.
    ✦ \ע|שפות חסרות הקשר| (\E|regular expressions|)
    אותם נכיר ב-\cref{section:regular}.

    קבוצת כל השפות מעל~$Σ$ היא לכן קבוצת כל הקבוצות החלקיות של~$Σ^*$, כלומר
    קבוצת החזקה של~$Σ^*$, אותה נסמן ב-$𝒫Σ^*$. נסמן ב-\textbf{Finite} את קבוצת
    כל השפות הסופיות. ראינו כבר כי \[
      L₆∈\text{\textbf{Finite}}.
\] נסמן ב-\textbf{Regular} את קבוצת כל השפות הפורמליות הרגולריות, כלומר
    השפות שאותן ניתן לתאר באמצעות ביטוי רגולרי.
    כאמור \[
      L₃,L₄,L₅∈\text{\textbf{Regular}}.
\] נסמן ב-\textbf{CFG} את קבוצת כל השפות הפורמליות חסרות ההקשר, כלומר אותן
    השפות שאותן ניתן לתאר באמצעות דקדוק חסר הקשר.
    כאמור \[
      L₁,L₂∈\text{\textbf{CFG}}.
\] מתברר כי:
    \begin{equation*}
      \text{\bfseries Finite}⊊\text{\bfseries Regular}⊊\text{\bfseries CFG}⊊
      \text{\bfseries CSG}⊊𝒫Σ^*.
    \end{equation*}
    כלומר, ניתן לתאר את כל השפות הסופיות באמצעות ביטוויים רגולריים, אך לא כל
    השפות שאפשר לתאר אותן בביטוי רגולרי הן סופיות. בנוסף, כל שפה שאפשר לתאר
    באמצעות ביטויים רגולריים, ניתן גם לתאר באמצעות דקדוק חסר הקשר, אך יש שפות
    שניתן לתאר באמצעות דקדוקים חסרי הקשר, ושאי אפשר לתאר באמצעות ביטויים
    רגולריים. יתירה מכך, ישנן שפות אותן לא ניתן לתאר באמצעות דקדוקים חסרי הקשר.

    עוד מתברר כי
    \begin{equation}
      L₀∉\text{\bfseries CFG}
    \end{equation}
    כלומר, לא ניתן לתאר את שפת~\CPL באמצעות דקדוקים חסרי הקשר.

    ביטויים רגולריים, אותם נכיר ב\cref{section:regular} מאפשרים להגדיר שפות
    פשוטות כגון~$L₄$ ו-$L₅$.
    ✦ \ע|דקדוקים חסרי הקשר רגולריים|
    ✦ \ע|דקדוקים תלויי הקשר|
    ✦ \ע|הגדרה באמצעות תכנית|
  \end{enumerate}

  הגדרה פורמלית של השפות~$L₁$ ו-$L₂$ דורשת שימוש במנגנון שנכיר בהמשך, דקדוקים
  חסרי הקשר (\E|context free grammars|). הגדרה מדיוקת של השפות~$L₃$ ו-$L₄$
  דורשת שימוש במנגנון אחר, ביטויים רגולריים מרבית השימושים בפועל מוסיפים תַּחְבִּירִי
  סֻכָּר כגון:

  \begin{figure}[H]
    \centering
\begin{tikzpicture}
\node[above,ellipse,minimum height=12em,minimum width=24em,draw,fill=yellow,opacity=1] (f) {};
\node[above,ellipse,minimum height=10em,minimum width=20em,draw,fill=magenta,opacity=1] (e) {};
\node[above,ellipse,minimum height=8em,minimum width=16em,draw,fill=orange,opacity=1] (d) {};
\node[above,ellipse,minimum height=6em,minimum width=12em,draw,fill=olive,opacity=1] (c) {};
\node[above,ellipse,minimum height=4em,minimum width=9em,draw,fill=green,opacity=1] (b) {};
\node[above,ellipse,minimum height=2em,minimum width=6em,draw,fill=red,opacity=1] (a) {Finite};


\path (a.north) node[above] {Regular}
    (b.north) node[above] {Context Free}
    (c.north) node[above] {Context Sensitive}
    (d.north) node[above] {Recursively Enumerable}
    (e.north) node[above] {$\wp \Sigma^*$};

\draw[label distance=-4pt] (c.north) ++ (4em,-2em) node[minimum size=3pt,shape=circle,inner sep=0pt,fill=blue,draw=black,label=60:\scriptsize$L_2$]{};
\draw[label distance=-4pt] (c.east) ++ (4em,-2em) node[minimum size=3pt,shape=circle,inner sep=0pt,fill=blue,draw=black,label=60:\scriptsize$L_1$]{};
\draw[label distance=-4pt] (a.north) ++ (1.5em,-1.4em) node[minimum size=3pt,shape=circle,inner sep=0pt,fill=blue,draw=black,label=60:\scriptsize$L_6$]{};
\draw[label distance=-4pt] (b.north) ++ (1.5em,-1.4em) node[minimum size=3pt,shape=circle,inner sep=0pt,fill=blue,draw=black,label=60:\scriptsize$L_4$]{};
\end{tikzpicture}

    \caption{ההיררכיה של חומסקי}
  \end{figure}
