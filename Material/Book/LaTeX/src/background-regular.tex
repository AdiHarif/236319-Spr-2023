ביטויים רגולריים הם מכשיר חשוב להגדרת שפות פורמליות מעל אלפאבית נתון. ביטויים
רגולריים דומים במבנה שלהם לביטויים חשבוניים: הם עצים, אשר בצמתים פנימיים נמצא
אופרטורים בעלי arity ידוע, כאשר בעלים נמצאים סימנים מתוך האלפאבית.

§§ ביטויים רגולריים כביטויים מעל חתימה
בהינתן אלפאבית~$Σ$, נבנה מעליו חתימה~$Γ$, כאשר~$Γ=Γ₀∪Γ₁∪Γ₂$ וכן
\begin{equation*}
  \begin{split}
    Γ₀ &=Σ^* ⏎
    Γ₁ &=❴⌘*❵ ⏎
    Γ₂ &=❴⌘;,⌘|❵
  \end{split}
\end{equation*}
נגדיר כעת את~$\RE(Σ)$ קבוצת הביטויים הרגולריים מעל האלפאבית~$Σ$ כקבוצת
הביטויים מעל החתימה~$Γ$,
\begin{equation*}
  \RE(Σ)=E(Γ)
\end{equation*}
עלים של ביטויים אלו הם מילים מהאלפאבית~$Σ$ ובכלל אלו המילה הריקה~$ε$.
לצמתים פנימיים המסומנים ב-⌘* יש בן אחד בדיוק. לצמתים פנימיים המסומנים ב-⌘;
או ב-⌘| יש שני בנים בדיוק.

\פנה|figure:regexp| מראה כמה ביטויים רגולריים הנבנים כביטויים מעל
האלפאבית~$❴⌘a,⌘b,⌘c❵$
\begin{figure}
  \centering
  \begin{tikzpicture}[every node/.style={shape=circle,
          draw, align=center,
          top color=red!10, bottom color=blue!20}]
    \usetikzlibrary{trees,chains}
    \begin{scope}[start chain=growing right,minimum size=2em]
      \node[on chain,circle,draw]{$⌘;$}
      child {%
          node{$⌘*$}
          child{%
              node{$⌘|$}
              child{node{$⌘a⌘b$}}
              child{node{$⌘c$}}
            }
      }
      child{%
        node{$⌘*$}
        child{node{$⌘b$}}
      };
      \node[on chain,circle,draw,xshift=8ex]{$⌘*$}
      child {%
          node{$⌘|$}
          child {node{$⌘a$}}
          child {node{$⌘|$} child{node{$ε$}} child{node{$⌘c$}}}
        }
      ;
    \end{scope}
  \end{tikzpicture}
  \caption[ביטויים רגולריים כביטויים מעל חתיחה]{ביטויים רגולריים כביטויים מעל החתימה~$Γ=Γ₀∪Γ₁∪Γ₂$, הכוללת את
  העלים~$Γ₀=❴⌘a, ⌘b, ⌘c❵^*$, האופראטור האונארי~$Γ₁=❴⌘*❵$ ושני האופראטורים
  הבינאריים,~$Γ₂=❴⌘;,⌘|❵$}.
  \label{figure:regexp}
\end{figure}

§§ קבוצת הביטויים הרגולריים כשפה פורמלית

הצגנו את קבוצת הביטויים הרגולרים כסוג של עצים מעל חתימה שנבנתה מעל אלפאבית
נתון. אולם, ניתן להציג את קבוצת הביטויים הרגולריים באופן ישיר כשפה
פורמלית.
יהי~$Σ=❴σ₁,…,σₙ❵$ ונניח כי ארבעת סימני הפיסוק ⌘|, ⌘*, ⌘(ו-⌘) אינם שייכים ל-$Σ$.
נגדיר אלפאבית מורחב~$Σ'=Σ∪❴⌘|,⌘*,⌘),⌘($ ונגדיר את~$\RE(Σ)$ כשפה פורמלית
מעל~$Σ'$, כלומר,
$\RE(Σ)⊂❨Σ'❩^*$.
\החל{definition}[ביטויים רגולריים כשפה פורמלית]
\label{definition:re}
הקבוצה~$\RE(Σ)$, \ע|השפה הפורמלית של הביטויים הרגולריים מעל~$Σ$|, היא השפה
פורמלית מעל האלפבית המורחב~$Σ'$ המוגדרת רקורסיבית באמצעות חמשת כללי ההיסק
הבאים:
\begin{align}
  \infer{σ∈\RE(Σ)}{σ∈Σ} ⏎
  \infer{⌘{()}∈\RE(Σ)}{} ⏎
  \infer{⌘)r₁r₂⌘)∈\RE(Σ)}{r₁∈S(Σ)   & r₂∈\RE(Σ)} ⏎
  \infer{⌘)r₁⌘|r₂⌘)∈\RE(Σ)}{r₁∈S(Σ) & r₂∈\RE(Σ)} ⏎
  \infer{⌘)r⌘*⌘)∈\RE(Σ)}{r∈\RE(Σ)}
\end{align}
\סוף{definition}
שני כללי ההיסק הראשונים בהגדרה זו מגדירים את הביטויים הרגולריים האטומיים
הכוללים כל אות~$σ$,~$σ∈Σ$, וגם את המילה בת שתי האותיות~$⌘{()}$ הלקוחה
מתוך~$❨Σ'❩^*$. שלושת הכללים הבאים אחריהם מגדירים את בנאי הביטויים.

\begin{english}
  \begin{tabularx}\textwidth{lX}
    $r₁=a$ ⏎
    $r₂=()$ ⏎
    $r₃=(r₁b)=(ab)~$ ⏎
    $r₄=(r₂|r₃)=(()|(ab))~$ ⏎
    $r₅=(r₄*)$
  \end{tabularx}
\end{english}

§§ המשמעות של ביטויים רגולריים
כדי לתת משמעות לשפה הפורמלית של הביטויים הרגולריים, נגדיר פונקצית המעניקה
משמעות לכל ביטוי רגולרי מעל~$Σ$. משמעות זו תהיה בעצמה שפה פורמלית, או קבוצת
מילים. כדי להבחין בין ביטוי רגולרי שהוא מילה מעל~$Σ'$, ובין המשמעות של הביטוי,
נשתמש ב-$r$ לציון הביטוי הרגולרי כמילה, וב-$⟦r⟧$ לציון משמעותה של מילה זו.

נסכם את הסימונים שבהם השתמשנו עד כה ואת היחסים ביניהם.

\begin{tabularx}\textwidth{lX}
  $Σ$                  & האלפבית הנתון, למשל~$Σ=❴⌘a,⌘b,⌘c❵$ ⏎
  $r∈\RE(Σ)$           & הסימון~$r$ מתייחס למילה בשפה הפורמלית~$\RE(Σ)$, ולא למשמעות
  המילה. ⏎
  $\RE(Σ)⊆❨Σ'❩^*$      & הקבוצה~$\RE(Σ)$ היא שפה פורמלית מעל האלפבית~$Σ'$. ⏎
  $Σ'=Σ∪❴⌘),⌘),⌘|,⌘*❵$ & האלפבית~$Σ'$ מתקבל מהאלפבית המקורי בתוספת ארבעה
  סימני פיסוק שלא היו בו. ⏎
  $⟦r⟧⊆Σ^*$            & הסימון~$⟦r⟧$ מתייחס למשמעות של המילה~$r$, שהיא קבוצה של מילים
  מעל האלפבית הנתון~$Σ$. ⏎
  $⟦r⟧∈℘Σ^*$           & $⟦r⟧$ שייכת לקבוצת השפות הפורמליות מעל~$Σ$. ⏎
  $⟦·⟧:\RE(Σ)→℘Σ^*$    & הפונקציה~$⟦·⟧$, כפונקציה של ארגומנט אחד, היא פונקציה
  מהשפה הפורמלית~$\RE(Σ)$ (שפה אשר מוגדרת מעל האלפבית המורחב~$Σ'$) אל קבוצת
  השפות הפורמליות מעל~$Σ$ (האלפבית המקורי). ⏎
  $⟦·⟧:❨Σ'❩^*⇸℘Σ^*$    &
  הפונקציה~$⟦·⟧$ היא פונקציה חלקית מקבוצות המילים מעל האלפבית המורחב, אל קבוצת
  השפות הפורמליות מעל~$Σ$ (האלפבית המקורי). הפונקציה היא חלקית, כי לא כל מילה
  הכתובה באלפבית~$Σ'$, היא ביטוי רגולרי חוקי, כלומר ניתנת להתקבל באמצעות
  ההגדרה \פנה|definition:regular|.
\end{tabularx}

גם הגדרת~$⟦r⟧$, פונקצית המשמעות של ביטוי רגולרי היא רקורסיבית, ורקורסיה זו
תואמת את הרקורסיה שבהגדרה \פנה|definition:re|.

\החל{definition}[פונקצית המשמעות של ביטויים רגולריים]
\label{definition:regular}
בהינתן ביטוי רגולרי~$r$, חמשת כללי ההיסק הבאים קובעים את משמעותו
\begin{enumerate}
  ✦ \[
    \infer{⟦r⟧=❴r❵}{r=σ & σ∈Σ}
\] כלומר, המשמעות של ביטוי רגולרי שהוא אות~$σ$
  באלפבית~$Σ$ היא השפה הפורמלית המכילה את המילה~$σ$ בלבד.
  ✦ \[
    \infer{⟦⌘{()}⟧=❴ε❵}{}
\] כלומר, המשמעות של הביטוי הרגולרי ⌘(⌘) היא השפה
  הפורמלית המכילה בתוכה את המילה הריקה בלבד.
  ✦ \[
    {\infer{⟦(r₁|r₂)⟧=⟦r₁⟧∪⟦r₂⟧}{r₁∈\RE(Σ)&r₂∈\RE(Σ)}}
\] כלומר, אם~$r₁$ ו-$r₂$
  הם ביטויים רגולריים, אזי~$⟦⌘)r₁⌘|r₂⌘)⟧$, המשמעות של הביטוי הרגולרי~$⌘)r₁⌘|r₂⌘)$, היא
  קבוצה המילים המתקבלת מאיחוד קבוצות המילים שהן המשמעויות של~$r₁$ ושל~$r₂$.
  ✦ \[
    \infer{⟦⌘(r₁r₂⌘)⟧=❴w₁w₂\,|\,w₁∈⟦r₁⟧∧w₂∈⟦r₂⟧❵}{r₁∈\RE(Σ) & r₂∈\RE(Σ)}
\] כלומר, אם~$r₁$ ו-$r₂$ הם ביטויים רגולריים, אזי~$⟦⌘)r₁r₂⌘)⟧$, המשמעות של
  הביטוי הרגולרי~$⌘)r₁⌘r₂⌘)$ היא הקבוצה של כל המילים שאפשר לחלק אותן לשתי מילים
  עוקבות, כך שהאחת נמצאת בתוך~$⟦r₁⟧$, השפה שהיא המשמעות של הביטוי~$r₁$, ואילו
  המילה האחרת נמצאת בתוך~$⟦r₂⟧$, השפה הפורמלית שהיא המשמעות של הביטוי~$r₂$.
  ✦ \[
    \infer
    {⟦⌘)r⌘*⌘)⟧=⟦ε⟧∪⟦r⟧∪⟦⌘)rr⌘)⟧∪⟦⌘)⌘)rr⌘)r⌘(⟧∪⋯}
    {r∈\RE(Σ)},
\] כלומר, אם~$r$ הוא ביטוי רגולרי, אזי~$⟦⌘)r⌘*⌘)⟧$, המשמעות של הביטוי
  הרגולרי~$⌘(r⌘*⌘)$, היא קבוצת כל המילים המילים שאפשר לחלק אותן ל-$0≤n$ מילים
  עוקבות שכל אחת מהן לקוחה מתוך~$⟦r⟧$, השפה הפורמלית שהיא המשמעות של הביטוי
  הרגולרי~$r$. \end{enumerate}
\סוף{definition}

נסתכל לדוגמה על הביטוי הרגולרי~$r$, \[
  r=⌘{(((a)((a|(b|c)))*)(c))}.
\] כדי למצוא את המשמעות של~$⟦r⟧$, עלינו לבדוק כיצד~$r$ נבנה רקורסיבית.

מציין את השפה~$L₃$, השפה של כל המילים המתחילות באות~$a$ ומסתיימות באות~$c$
\פסקה{דוגמאות}
המש הרגולרי \[
  ((a)((a|(b|c)))*)(c)
\] \[
  ⟦((a)((a|(b|c)))*)(c)⟧=L₃.
\] הביטוי הרגולרי \[
  ((a*)(b*))(c*)
\] מציין את השפה~$L₄$, שפת המילים שהאותיות שלהן מופיעות בסדר אלפאביתי לא יורד \[
  ⟦ (((a)*);((b)*));((c*))⟧=L₄.
\] 


§§ כתיב מקוצר לביטויים רגולריים
הכתיב שהוצע ב\פנה|definition:re| הוא ארכני, שכן כל הפעלה של בנאי מוסיפה זוג של
סוגריים לביטוי. נהוג להשתמש בכתיב מקוצר המסתמך על שלוש מוסכמות:
\ספרר
✦ יש שימוש באסוציאטיביות של ⌘| ושל השרשור כדי להשמיט סוגריים. כך למשל, במקום
$(a|(b|c))$
כותבים בקיצור
$(a|b|c)$
✦ כללי קדימות, לפיהם הסימן \* הוא בעל הקדימות הגבוהה ביותר, והסימן ⌘| בעל
הקדימות הנמוכה ביותר מאפשרים להשמיט סימני סוגריים נוספים. כך למשל, במקום
$(a*);(b*)$
נכתוב~$a*b*$.
✦ סימן השרשור (⌘;) מושמט. במקום
$a;b;c$
נכתוב בקיצור~$abc$.
===

בכתיב זה הביטוי הרגולרי המתאר את השפה~$L₄$ נכתב כ-$a*b*c*$. עוד נהוג שלא
להקפיד על ההבדלה בין הביטוי כמילה, ובין המשמעות שלו, ולכן, בדרך כלל כותבים
בקיצור~$L₄=a*b*c*$ במקום~$L₄=⟦a*b*c*⟧$. בכתיב המקוצר, הביטוי
הרגולרי~$(a|c)(a|c)(a|c)(a|c)(a|c)*$ מתאר את השפה השפה~$L₅$ המכילה את כל המילים
בנות חמש אותיות או יותר שאף אחת מהן אינה האות ⌘b.

ישנם כלים רבים המאפשרים לבצע חיפוש והחלפה בטקסט תוך שימוש בביטויים רגולריים.
כלים אלו מוסיפים קיצורים משלהם, אך כמה קיצורים מופיעים באורח זהה כמעט בכל כלי
המאפשר למשתמש בו להגדיר ביטויים רגולריים.

\ספרר
✦ עבור~$σ₁,σ₂,…σₖ∈Σ$, הסימון
$[σ₁σ₂⋯σₖ]$
הוא קיצור לביטוי
$(σ₁|σ₂|⋯|σₖ)$
כלומר, ביטוי המתאים לכל מילה שהיא תו אחד בדיוק מבין
$σ₁,σ₂,…,σₖ$.
כך למשל ⌘{+|-}
יכתב כ-⌘{[+-]}.

✦ אם~$r$ הוא ביטוי רגולרי, אזי~$r?$ הוא קיצור לביטוי הרגולרי~$(r|ε)$. כך למשל
השפה המתאימה לביטוי הרגולרי ⌘{?[+-]} מכילה שלוש מילים~$❴ε, ⌘a, ⌘b❵$.

✦ אם~$r$ הוא ביטוי רגולרי, אזי~$r+$ הוא קיצור לביטוי הרגולרי~$rr*$, כלומר מילה
המכילה מופע אחד או יותר של~$r$.

✦ בהנחה שיש סדר מוסכם של התווים באלפבית~$Σ$, הסימון~$σ₁-σ₂$ כשהוא מופיע בתוך
סוגריים מרובעים, הוא קיצור של רשימת כל התווים בין~$σ₁$ ובין~$σ₂$.
כך למשל הביטוי הרגולרי
\begin{quote}
  ⌘{[0-9]+}
\end{quote}
מתאר סדרה לא ריקה של ספרות עשרוניות, ואילו
\begin{quote}
  ⌘{[-+]?[0-9]+}
\end{quote}
הסימן "." (נקודה) מתאר את הביטוי הרגולרי שמכיל אות אחת בדיוק מהאלפבית.
====

כדאי לדעת כי ניתן לכתוב ביטוי רגולרי עבור חיתוך של השפות של שני ביטויים
רגולרייים, וגם עבור השפה של כל המילים בשפה של ביטוי רגולרי אחד אשר אינן מצויות בשפה
של ביטוי רגולרי אחר, וזאת בעבור כל שני ביטויים רגולריים שרירותיים. לשימוש
בביטויים רגולריים בתיאורה של שפה פורמלית יש גם השלכה מעשית: ישנם כלים אוטומטיים
אשר הקלט שלהם הוא ביטוי רגולרי ואשר מיייצרים תכניות המסוגלות לזהות
מופעים של ביטויים רגולריים בטקסט. כלים אלו מועילים מאוד בכתיבת מהדרים עבור שפות
תכנות.
