\def\Title{פתח דבר}
\def\wrapMe{\relax}
\makeatletter
\ifx\documentclass\@twoclasseserror % after \documentclass
  \ifx\@onlypreamble\@notprerr % after \begin{document}
  \relax 
  \else % before \begin{document}
    \relax
  \fi
\else % before \documentclass
  \def \wrapMe{
    \def\cname{\jobname}
    \documentclass[12pt,reqno,fleqn]{article}

\def\run#1#2{%
  \immediate\write18{%
    % texfot --interactive 
    xelatex -shell-escape --jobname="#1"
    "\gdef\string\cname{#1}\gdef\string\Title{#2}\string\input\space\jobname"
  }%
}
\ifx\cname\undefined
%\run{motivation}{על התחום}
%\run{techniques}{טכניקות לניתוח שפות תכנות}
%\run{lexical}{ניתוח וסיווג האסימונים} 
%\run{lisp}{מיני ליספ ושיערוך}
%\run{background}{רקע מתימטי}
\run{universal}{מאוניברסליות לאלגנטיות}
%\run{recursion-games}{שעשועים רקורסיביים}
%\run{JVM}{הידור והרצה של תכניות Java}
%\run{character-encoding}{קידוד תווים}
\expandafter\stop
\fi

\usepackage{00}
\begin{document}
\מחבר{
יוסי גיל ⏎
הפקולטה למדעי המחשב⏎
הטכניון - מכון טכנולוגי לישראל⏎
}
\title{\Title}
\maketitle
\begin{multicols}{2}%
  \scriptsize\bfseries
  \tableofcontents
  \listoftables
  \listoffigures%
\end{multicols}%
\input \cname
\end{document}

  }
\fi
\makeatother
\wrapMe


§ למידה פרוצדרולית בניגוד ללמידה דקלרטיבית 

מומחים מבדילים בין למידה פרוצדרולית, ובין למידה דקלרטיבית. בלמידה דקלרטיבית,
אנו לומדים מושגים ועובדות, שהבנתם ועיבודם דורשת מאמץ קוגניטיבי לא פשוט: משפטים,
הוכחות, תהליכים מורכבים כמו אינטגרציה בחלקים, הגדרות, פרוטוקולים, אלגוריתמים
וכו', הם דוגמאות ללמידה דקלרטיבית. לימוד נהיגה או שחיה, הוא לימוד פרוצדורלי:
הידע הנדרש לשם הפעלת כלי רכב מנועי אינו רב. אבל, יש צורך להתאמן רבות כדי להגיע
בו להישגים.

הלימוד בקורס הזה הוא לימוד פרוצדורלי: אמנם ניתקל בהרבה הגדרות, אבל הבנתן אינה
דורשת בדרך כלל מאמץ חשיבתי. מה שנרצה להרכיש לסטודנטים בקורס היא היכולת להפעיל
את ההגדרות לשם יעול הלימוד של שפות תכנות חדשות, רכישת אופני מחשבה חדשים, יכולת
להבין ולהשתתף בשיח המקצועי בנושא של שפות תכנות, וגם, היכולת לפתח בעצמך שפות
חדשות. 

לשם כך, מומלץ מאוד מאוד להשתתף בהרצאות: לשאול שאלות, לסכם תוך כדי הרצאה,
ולהשתתף בלמידה באורח אקטיבי. ישנם סיכומים שונים ברשת, וישנן אף הקלטות וידאו של
מהדורה (ישנה מאוד) של הקורס[11], אבל, אין תחליף לנוכחות, ולביצוע אמיתי של
תרגילי הבית. 

§"גמישות מחשבתית"
קורס זה מטרתו לסייע לסטודנטים ללמוד ולהכיר במהירות שפות תכנות חדשות, להגיע
לשליטה עמוקה יותר בשפות התכנות שהם מכירים ובהם הם עובדים, ובעיקר, "להגמיש את
המחשבה". כדי להסביר את המושג "להגמיש את המחשבה" נספר שבאחד המחקרים, נבדק הַמִּתְאָם
בין התפוקה של מתכנתים, ובין פרמטרים שונים, ובהם השכר, שנות הניסיון, ההשכלה,
התפקיד, ועוד. התברר כי החזאי הטוב ביותר של תפוקה היה מספר שפות התכנות השונות
אותם הם מכירים. 

כמובן, מִתְאָם אינו בהכרח מעיד על סיבתיות. אדרבא, יבוא ד"ר איפכא מסתברא ויאמר
שֶׁהַמִּתְאָם הגבוה בין תפוקה ובין הפוליגלוטיות הוא תוצאה של סיבתיות הפוכה: מתכנתים
שתפוקתם גבוהה יותר, הם כאלו שעיתותיהם בידיהם, ולכן הם מתפנים יותר ללמוד שפות
חדשות. ובכל זאת, אין זה מופרך להניח קשר חשוב בין שתי התופעות. שפות תכנות שונות
מייצגות דרכי מחשבה שונות, אופנים שונים לביטוי הפשטות ולהתמודדות עם בעיות
תכנותיות. ידע רחב בשפות תכנות מסייע על כן למתכנת לבחון בעיות מזוויות שונות.
הבחינה הזו יכולה לאפשר לפתור בעיות שונות באמצעות שפות שונות. וגם אם שפת התכנות
מוכתבת ואינה ניתנת לשינוי, דרכי החשיבה השונות יכולות להציע דרכים שונות לפתור את
הבעיה גם במסגרת שפת התכנות הנתונה.

§ התיזה של ספיר-וורף
הרעיון שהשפה מעצבת את המחשבה אינו חדש. ברומן מדע בדיוני מאת סמואל ר. דילייני
אשר נכתב בשנת 1966, אחד הצדדים למלחמה בין גלקטית מפתח נשק סודי בדמות שפה טבעית
בשם בבל-17 (כשם הרומן). תכונתה של שפת בבל-17 שהיא כופה דרך מחשבה כזו על בני
המחנה היריב הלומדים אותה, שהם חשים צורך לשנות את התנהגותם ולבגוד במחנה שלהם.
בבל-17 הוא רומן בידיוני, אבל הוא נסמך על רעיון מדעי חשוב: "התיזה של ספיר-וורף",
אשר פותחה בראשית המאה ה-20 על ידי הבלשן אדווארד ספיר ותלמידו בנימין לי וורף.
לפי תיזה זו, אופן מחשבתו של אדם מושפע מאוד מהשפה בה הוא דובר, ובמיוחד משפת אמו.
גם מבלי להתעמק בתורת הבלשנות, לא קשה להשתכנע בתיזה זו. כולנו יודעים שבעברית יש
יותר ממאה שמות לעיר ירושלים, ושבערבית יש 99 שמות שונים לאלוהים, ויש להניח כי
העושר הלשוני הזה משפיע על דרך המחשבה. 
הנה שתי דוגמאות נוספות:
\begin{itemize}
         \item אנו יודעים כי בעברית יש שלושה זמנים: עבר, הווה, ועתיד, כאשר צורת
           ההווה, הקרוייה גם "בינוני" בשפה היא מנוונת. (בערבית, יש שני זמנים
           בלבד: עבר ועתיד.) לעומת זאת, בשפה הלטינית יש שישה זמנים (tenses),
           הכוללים למשל זמן לציון מאורע שאירע בעבר ומהווה עבר בעבור מאורע שיקרה
           בעתיד.  
         \item בלטינית יש שתי מילים נבדלות לדם, זה אשר זורם בגוף נקרא sanguis
       ואילו זה אשר זב ממנו נקרא cruor.  
   \end{itemize}
גם כאן, אין זה מופרך לחשוב  כי האבחנה בין זמנים שונים ובין שני סוגי הדם השונים
מכתיבה דרכי חשיבה שונות.  אנו נגלה בקורס כי שפות התכנות השונות מבטאות ומכתיבות
דרכי חשיבה שונות, לעיתים מתוך רצון מודע של מתכנן השפה להשפיע על דרכי החשיבה של
המתכנת. מסתבר למשל שיוקיהירו מצומוטו אשר פיתח את שפת התכנות רובי באמצע שנות
התשעים של המאה הקודמת, ראה לנגד עיניו את התיזה של ספיר-וורף.

§ לימוד שפות תכנות בקורס
ישנו קושי מובנה לבנות וללמד קורס בשפות תכנות: קל לדבר על "גמישות מחשבתית" כמשאת
נפש, אבל לא ברור כיצד ניתן להרכיש אותה לסטודנטים אשר מתקשים בהגמשת זמנם כך
שיכיל את מערכת הלימודים, העבודה והפנאי גם יחד.  לדעת אחדים, קורס בשפות תכנות
הוא קורס שבו הסטודנטים ילמדו שתיים-שלוש (ולעיתים אחת) שפות תכנות מתקדמות,
מעניינות, ובכך ירגילו אותם בדרכי מחשבה שונות.  אחרים ילמדו את הקורס כקורס
תיאורטי, בדומה לקורס ב"אוטומטים ושפות פורמליות", וישנה גם גישה המציעה למנות את
המושגים השונים ולפרט את דרך מימושים בשפות התכנות השונות.
הגישה אשר בה ינקוט קורס זה היא גישה משולבת. ראשית, נלמד כמה וכמה שפות תכנות,
ובראש ובראשונה את השפות הבאות: 
  \begin{itemize}
         \item ML - כדי להדגים שאפשר לכתוב תכניות גם מבלי להשתמש במשתנים, ואת
           צורת החשיבה הפונקציונלית.

         \item Prolog - כדי להדגים שאפשר לכתוב תכניות גם מבלי להשתמש בפקודות.

         \item Pascal - כדי להעשיר את הרקע ההיסטורי, כדי להבין את השפעותיה על
           שפות התכנות של ימינו, וכדי להנגיד אותה עם שפת התכנות המוכרת יותר, C.

\end{itemize}

שנית, נציב בקורס גישה שיטתית יותר לניתוח שפות תכנות, ובפרט קריטריונים שונים
לניתוחן. נדגים את הגישה הזו על שפות תכנות שונות, ונתרגל אותה לשם לימוד שפות
תכנות חדשות. במופע זה של הקורס, הסטודנטים ילמדו "כהרף עין" את השפות Go, Dart,
AWK ועוד. כמובן, בתחילה יקח זמן לסטודנטים לרכוש את המיומנות ללמוד שפות חדשות
"כהרף עין", אבל יש לקוות כי היכולת הזו תיתפתח במהלך הקורס. 


⌘יחידה*{נהלי הקורס} 
      נהלי הקורס המלאים והמחייבים זמינים לכל דיכפין, ואילו מבארים היטב עניינים כגון דרישות קדם, דרישות צמודות, תרגילי הבית, מבנה המבחן, וכיוצ"ב.
      אִם לְךָ אוֹבֶה
      שעות הקבלה של המרצה הן בתיאום של 24 שעות מראש, באמצעות מערכת הזמנות אלקטרונית. שעות הקבלה מיועדות לטיפול בבעיות אישיות, ולסיוע אישי

      בנוסף, ניתן לפנות גם בדוא"ל למרצה בשאלות אישיות, ויהיו אלו הקנטרניות ביותר\footnote{דרך משל,  "שותפי העבדקן צבע זקנו בשני, צבע המזכיר לי אנשובי, אשר איני יכול לשאת את ריחו. מה עלי לעשות?" היא לגיטימית.}.
      \begin{itemize}

     \item 
      פניות בדוא"ל אל המרצה בנושא החומר של הקורס, להבהרת קושיה מתחום החומר הנלמד, הסבר נוסף, חידוד הסבר קיים, וכו', תזכה להתעלמות או למתן התשובה: אני משיב לשאלות כלליות אך ורק בשעות הקבלה (או, בר"ת, אם לך אוב"ה \footnote{ ויפטיר החרזן הלץ: "אִם לְךָ אוֹבֶה, יעזרני יהוה"}). שאלות מסוג אלו, יש להם ענין לקהל הלומדים כולו ומקומן ברשות הרבים,  בקבוצת הפייסבוק של הקורס, או באתר השו"ת.  פניה פומבית נותנת מידע ועוזרת לסטודנטים אחרים, ומאפשרת גם תגובה של סטודנטים אחרים בקורס. 
      
      המבקש הסבר אישי המיוחד לו, יטרח ויגיע באופן אישי לשעות הקבלה שם יקבל הסבר כזה בסבר פנים יפות.
    \item 
     פניות בנושאים מנהלתיים, כגון שעות ההרצאה, מבחנים, וכיוצ"ב, צריכות לבוא מהאחראי האקדמי מטעם אגודת הסטודנטים, המרכז אותן, שוקל את שיקולי כלל הסטודנטים, מעריך את חשיבותן, ומצרפן לפניות אחרות בהתאם לצורך. רוב רובן של פניותיו של הרכז האקדמי נענות בחיוב, וכולן זוכות להתייחסות רצינית מאוד.
\end{itemize}
⌘יחידה*{ספרי לימוד וחומר עזר}
      קורס זה נבנה על יסודה של המהדורה השניה של ספרו של David Watt נושא השם "Programming Languages Concepts and Paradigms". אבל, במקומות רבים מאוד נרחיב מעבר לכתוב בספר. ספר הלימוד הזה מתבסס על שלוש שפות תכנות:
      \begin{itemize}
      \item Pascal
      \item ֵML
      \item Ada
     \end{itemize} 
      הספר מעט מיושן, ולעיתים משתמש במונחים באורח אידיוסינקרטי מעט. אמנם הספר יצא לאור במהדורה שלישית, אך לא נשתמש בה כיוון שהיא זונחת את שְׂפַת התכנות ML לטובת שְׂפַת התכנות Haskell.
      ניתן למצוא את השקפים של המהדורה השלישית באינטרנט יחד עם חומר נוסף באתר המחבר:
      http://www.dcs.gla.ac.uk/~daw/books/PLDC/
      ניתן גם לרכוש את הספר כאן:

      http://preview.tinyurl.com/watt2
⌘יחידה*{מבחן}
      המבחן יערך בחומר סגור, יכלול 10 שאלות, אשר משקל כל אחת מהן 10 נק'. קושיין של השאלות והזמן הנדרש לפתירתן אינו זהה. דף השער של המבחן בקורס מצוי כאן. האתר http://safot.cs.technion.ac.il מרכז שאלות ממבחנים קודמים ונועד לאימון בפתרון שאלות ממבחנים קודמים (בעיקר), לקראת הכנה למבחן.
      במבחן תופיע שאלה אחת לפחות מבין כל אחת מארבעת הקטיגוריות הבאות:
      \begin{enumerate}
      \item תרגילי בית של הסמסטר
      \item מבחנים משנים אחרונות (ללא הגדרה מדוייקת של טווח)
      \item תרגילים מחוברת השקפים
      \item שאלות מאתר ההכנה למבחן: 
    ⌘שי{ 
      http://safot.cs.technion.ac.il
      }
      שאלות מסוג זה תיקראנה שאלות "ממוחזרות". כיוון שהקטיגוריות אינן זרות, התנאי יכול (אך לא חייב) להתקיים באורח חופף, ועל כן לא מובטח כי במבחן תהיינה 4 שאלות ממוחזרות.   בכל זאת,  המטרה הכללית היא כ-40\% מהניקוד במבחן יהיה מבוסס על שאלות ממוחזרות.
      \end{enumerate}
