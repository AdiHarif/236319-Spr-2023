קבוצת \ע|העצים הפורמאליים| מעל אלפאבית~$Γ$, המסומנת~$𝓣(Γ)$ מכילה את העצים בהם
כל צומת נושא תגית שהיא איבר של~$Γ$. דרגת הצמתים בעץ פורמלי אינה חסומה. כל צומת
בעץ יכול להכיל מספר כלשהו של בנים.

עץ פורמלי יכול להיות
\ציינן
:קאטומי, זאת קורה כאשר בעץ מורכב מצומת בודד שהוא עלה המתוייג
באיבר כלשהו של~$Γ$. עץ ב-$𝓣(Γ)$ יכול להיות גם עץ מורכב, ובמקרה זה שורש העץ הוא
צומת פנימי המתוייג באחד מבין האיברים של~$Γ$ אשר לו מספר כלשהו של בנים שכולם
עצים ב-\E|$𝓣(Γ)$|.

\begin{definition}[עצים מעל אלפאבית]
  בהנתן אלפאבית~$Γ$ אזי,~\E|$𝓣(Γ)$|, קבוצת העצים מעל~$Γ$ מוגדרת באמצעות הבנאי
  ה-$n$ מקומי
\begin{equation}
    \label{infer:tree}
    \infer{γ(t₁,…,tₙ)∈𝓣(Σ)}{γ∈Γ & n≥0 & t₁∈𝓣(Γ) & t₂∈𝓣(Γ)&⋯& tₙ∈𝓣(Σ)}
  \end{equation}
\end{definition}

כמה איברים של קבוצת העצים מעל האלפאבית בן שלוש האותיות~$❴a,b,c❵$ הם \[
  a(),b(c()),a(b(),c()), a(b(a())), a(a(),b(),c())
\]

כלל ההיסק~\ref{infer:tree} שבהגדרה הוא הבנאי היחיד של קבוצת העצים הפורמאליים,
והוא מתאר הן עצים אטומיים והן עצים מורכבים. לפי כלל זה עץ אטומי הוא עץ נולארי,
כלומר עץ "ערירי" אשר אין לו בנים, אשר צריך להיכתב לכן כ-~$γ()$. בכל זאת, מקובל
להשמיט את סימני הסוגריים עבור עצים ותתי-עצים שאין להם בנים, ולכתוב עץ כזה כ-$γ$
במקום~$γ()$.

בכתיב מקוצר זה, העץ הפורמאלי
\begin{equation}
  a(a(a(),ab(),abc()),b(b(),ab(c())),c(c(a(ab())))
\end{equation}
מעל האלפאבית האינסופי~$❴a,b,c❵^*$ יכתב כך
\begin{equation}
  a(a(a,ab,abc),b(b,ab(c)),c(c(a(ab)))
\end{equation}
\cref{figure:tree} מדגים את הטופולוגיה של עץ זה.

\begin{figure}[H]
  \centering
  \scriptsize
  \forestset{%
    x tree/.style={%
        for tree={%
            math content,
            s sep'+=-3pt,
            circle,
            fit=band,
          },
      },
  }
  \begin{forest}
    s tree [a
          [a,[a][ab][abc]]
          [b,[b][ab[c]]]
          [c,[c[a[ab]]]]
      ]
  \end{forest}
\caption[עץ פורמאלי מעל האלפאבית~$❴a,b,c❵^*$]
  {העץ הפורמאלי~$a(a(a,ab,abc),b(b,ab(c)),c(c(a(ab))))$ 
  מעל האלפאבית~$❴a,b,c❵^*$}
  \label{figure:tree}
\end{figure}

נוכל להגדיר את הקבוצה~$𝓣(Γ)$ גם כשפה פורמלית מעל אלפאבית מורחב~$Γ'$, הכולל גם
את זוג סימני הסוגריים ואת סימן הפסיק.
\begin{equation}
  Γ'=Γ∪❴⌘{(},⌘{.},⌘{)}❵.
\end{equation}

