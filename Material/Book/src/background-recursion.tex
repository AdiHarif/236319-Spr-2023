נאמר על קבוצה~$S$ שהיא מוגדרת באופן \ע|רקורסיבי| (או בנוייה באופן רקורסיבי, או
לעיתים גם בנויה באופן \ע|אינדוקטיבי|) אם ההגדרה של~$S$ מבדילה בין שני סוגים של
איברים: איברים אטומיים (\E|atomic|) ואיברים מורכבים (\E|compound|). איברים
מורכבים נוצרים באמצעות בנאי (\E|constructor|) איברים מאיברים אטומיים ואיברים
מורכבים אחרים. הגדרה רקורסיבית מתאפייינת גם בתכונה נוספת: 
\begin{itemize}
  ✦ \ע|שלמות ההגדרה|. הגדרה רקורסיבית של הקבוצה~$S$ כוללת תמיד בתוכה מרכיב
  הדורש שאין ב-$S$ איברים אחרים מלבד האיברים האטומיים ואלו שנוצרו באמצעות בנאים.
  בדרך כלל הדרישה שבמרכיב זה של אינה נאמרת במפורש, אלא משתמעת מהניסוח.
  \end{itemize}

\פסקה{הערות}
\החל{אבגוד}
✦ לעיתים נתייחס לאיברים האטומיים של קבוצה מוגדרת רקורסיבית כבנאים שהם nullary,
כלומר בנאים שאינם מקבלים ארגומנטים.
✦ גדלן של קבוצות המוגדרות רקורסיביות הוא בלתי חסום בדרך כלל, שכן תמיד ניתן
להשתמש בבנאים כדי ליצור איברים נוספים.
✦ קבוצה מוגדרת רקורסיבית יכולה להיות בעלת גודל סופי:
\ציינן
✦ אם הגדרת הקבוצה מכילה יחס שקילות, שגורם לכך שהפעלה אינסופית של בנאים, יוצרת
רק אוסף סופי של איברים שקולים.
✦ אם הבנאים אינם כאלו שתמיד ניתן להפעילם.
===
\סוף
{אבגוד}

§§ הגדרה רקורסיבית של קבוצת הפונקציות הרציונליות

נגדיר לדוגמה באופן רקורסיבי את~$ℚ₁$, קבוצת הפונקציות הרציונליות במשתנה אחד. כל
איבר~$f$ בקבוצה~$ℚ₁$ הוא פונקציה חלקית מ-$ℝ$ (קבוצת המספרים הממשיים) אל~$ℝ$ .
כלומר~$f:ℝ⇸ℝ$. הכוונה במונח פונקציה חלקית (\E|partial function|) היא שייתכן כי
קיים ערך מסויים~$ℝ∈x$, שעבורו ערך הפונקציה~\E|$f(x)$| אינו מוגדר. אנו נשתמש
בסימון~$⊥$ כדי לציין את הערך הלא מוגדר. ניתן לכן לכתוב~$f:ℝ→ℝ∪❴⊥❵$. בניסוח
אחר,~$ℚ₁⊆ℝ⇸ℝ$, כלומר~$ℚ₁$ היא קבוצה חלקית של קבוצת הפונצקיות החלקיות מ-$ℝ$
אל~$ℝ$.

\החל{definition}\label{definition:rationals}
הקבוצה ב-$ℚ₁$, קבוצת הפונקציות הרציונליות במשתנה אחד, מוגדרת על ידי שלושת
התנאים הבאים:
\begin{enumerate}
  ✦ \ע|איברים אטומיים של קבוצת הפונקציות הרציונליות|
  \begin{itemize}
    ✦ הפונקציה~$U$, המעתיקה כל מספר ממשי אל המספר הטבעי~$1$,
    \begin{equation*}
      ∀ x∈ℝ∙ U(x)=1,
    \end{equation*}
    נמצאת בקבוצה~$ℚ₁$,
    כלומר
    \begin{equation}\label{eq:1}
      U∈ℚ₁
    \end{equation}
    ✦ פונקצית הזהות,~$I$, המעתיקה כל מספר ממשי אל עצמו,
    \begin{equation*}
      ∀ x∈ℝ∙ I(x)=x
    \end{equation*}
    נמצאת בקבוצה~$ℚ₁$, כלומר
    \begin{equation}\label{eq:x}
      I∈ℚ₁
    \end{equation}
  \end{itemize}
  ✦ \ע|בנאים של קבוצת הפונקציות הרציונליות|
  \begin{itemize}
    ✦ אם הפונקציה~$f$ שייכת ל-$ℚ₁$ אזי גם הפונקציה~$-f$ שייכת לקבוצה זו, כלומר
    \begin{equation}\label{eq:minus}
-f∈ℚ₁.
    \end{equation}
    ✦ אם שתי הפונקציות~$f₁$ ו-$f₂$ שייכות ל-$ℚ₁$ אזי גם הסכום שלהן, המכפלה
    שלהן, והמנה שלהן שייכות ל-$ℚ₁$, כלומר
    \begin{align}
      f₁+f₂ &∈ℚ₁, \label{eq:plus} ⏎
      f₁·f₂ &∈ℚ₁ \label{eq:times} ⏎
      f₁/f₂ &∈ℚ₁. \label{eq:div}
    \end{align}
  \end{itemize}
  ✦ \ע|שלמות ההגדרה: אין פונקציות רציונליות חוץ מהאטומיות ואלו שנוצרו באמצעות
  הבנאים| ⏎
  הקבוצה~$ℚ₁$ היא הקבוצה הקטנה ביותר של פונקציות המקיימת את התנאים
  \פנה|eq:1|,
  \פנה|eq:x|,
  \פנה|eq:minus|,
  \פנה|eq:plus|,
  \פנה|eq:times|
  ו-\פנה|eq:div|.
\end{enumerate}
\סוף{definition}

§§ כתיב של כללי היסק

ניסוח תמציתי ומדוייק לבנאים הוא ככללי היסק \E|(inference rules)| כפי שהם נהוגים
בתחשיב הפסוקים. כלל היסק האומר שבכל פעם שמתקיימות ההנחות~$P₁,P₂,…,Pₙ$ ניתן
להסיק את המסקנה~$Q$ יכתב כך: \[
  \dfrac{\begin{array}{c}P₁ ⏎P₂ ⏎⋮ ⏎Pₙ\end{array}}{Q}
\] ניתן גם לכתוב את הדרישות בשורה אחת, ובלבד שהן מופרדות זו מזו, \[
  \infer Q{P₁ & P₂ &⋯& Pₙ}
\] לדוגמה, את הבנאי \פנה|eq:plus| של הקבוצה~$ℚ₁$, ניתן לכתוב ככלל היסק:
\begin{equation*}
  \infer{f₁+f₂∈ℚ₁}{f₁∈ℚ₁ & f₂∈ℚ₁}
\end{equation*}
בכלל היסק זה יש שתי הנחות~$P₁=f₁∈ℚ₁$ ו-$P₂=f₂∈ℚ₁$. כל אחת מההנחות צריכה להיקרא
ככמת אוניברסלי כפי שהוא מופיע בתחשיב הפסוקים, כלומר, עבור כל בחירה של
פונקציה~$f₁$ המקיימת~$f₁∈ℚ₁$ ולכל בחירה של פונקציה~$f₂$ המקיים~$f₂∈ℚ₁$ נובעת
המסקנה~$Q=f₁+f₂∈ℚ₁$. בניסוח אחר כלל ההיסק אומר כי \[
  ∀f₁∀f₂❨f₁∈ℚ₁∧f₂∈ℚ₁→f₁+f₂∈ℚ₁❩.
\] ניתן לנסח את ארבעת הבנאים של הקבוצה~$ℚ₁$, כלומר \פנה|eq:minus|,
\פנה|eq:plus|,
\פנה|eq:times|
ו-\פנה|eq:div|,
ככלל היסק אחד:
\begin{equation*}
  \infer{-f₁,f₁+f₂, f₁·f₂,f₁/f₂∈ℚ₁}{f₁∈ℚ₁&f₂∈ℚ₁}
\end{equation*}

ניתן גם לנסח את הגדרת האיברים האטומיים של הקבוצה~$ℚ₁$, כלומר \פנה|eq:1|
ו-\פנה|eq:x|,
כשני כללי היסק אשר קבוצת ההנחות שלהן ריקה,
\begin{equation*}
  \begin{array}{ccc}
    \infer{I∈ℚ₁}{} &  & \infer{1∈ℚ₁}{}
  \end{array}\hfill
\end{equation*}
או בקיצור, ככלל היסק אחד שגוזר שתי מסקנות מקבוצת הנחות ריקה
\begin{equation*}
  \infer{1, I∈ℚ₁}{}
\end{equation*}
לדוגמה, \[
  \frac {I+1}{I·I-3·I+1}
\] הוא איבר ב-$ℚ₁$, שמיייצג את הפונקציה~$f(x)=(x+1)/(x²-3x+1)$.

ניתן להשתמש במבנה ההגדרה הרקורסיבי של קבוצה בהגדרות רקורסיביות נוספות המתייחסות
לקבוצה ולאיבריה.

\begin{definition}[ערך של פונקציה רציונלית]
  עבור פונקציה רציונלית~$f∈ℚ₁$, ועבור כל מספר ממשי~$x∈ℝ$ נגדיר את~$f(x)$
  רקורסיבית
  \begin{equation}\label{eq:value}
    \begin{array}{cc}
      U(x)=1                            & I(x)=x ⏎ ⏎
      \infer{-f(x)=-x₁}{f(x)=x₁}        & \infer{(f₁+f₂)(x)=x₁+x₂}{f₁(x)=x₁ & f₂(x)=x₂} ⏎ ⏎
      \infer{(f₁·f₂)(x)=x₁·x₂}{f₁(x)=x₁ & f₂(x)=x₂}                         &
      \infer{(f₁/f₂)(x)=x₁/x₂}{f₁(x)=x₁ & f₂(x)=x₂}
    \end{array}
  \end{equation}
\end{definition}

§§ אינדוקצית מבנה
הגדרות רקורסיביות מאפשרות לנו להוכיח טענות באינדוקציה הידועה בשם אינדוקצית
מבנה. באינדוקציה כזו, אנו מוכיחים ראשית כי הטענה נכונה עבור כל האיברים האטומיים
של קבוצה. בצעד האינדוקציה נעבור על כל בנאי האיברים: לגבי כל בנאי נניח שהטענה
נכונה לגבי כל האיברים עליהם פועל, ונוכיח כי הטענה נכונה גם עבור האיבר אשר אותו
יצר הבנאי. ניתן גם להסתכל על הוכחות באינדוקצית מבנה כאינדוקציה על מספר
ההפעלות~$n$ של בנאים לשם יצירת~$f$.

נוכיח לדוגמה את הטענה הפשוטה הבאה עבור ההגדרה הרקורסיבית של הקבוצה~$ℚ₁$
(\פנה|definition:rationals|) וההגדרה של ערך הפונקציה מעל \פנה|eq:value|.

\begin{claim}
  עבור כל מספר רציונלי~$q∈ℚ$, ועבור כל פונקציה רציונלית~$f∈ℚ₁$, מתקיים כי
  \begin{equation}\label{eq:Q}
    f(q)∈ℚ∪❴⊥❵
  \end{equation}
  כלומר~$f(q)$ אינו מוגדר, או שהוא מספר רציונלי.
\end{claim}

\begin{proof}
  \mbox{}
  \begin{description}
    ✦ [בסיס האינדוקציה] אם~$n=0$ אז~$f$ הוא איבר אטומי של~$ℚ₁$,
    ואז~$f=1$ או~$f=I$ וברור שאם~$q$ רציונלי, אז גם~$f(q)∈❴1,q❵$. ולכן
    \פנה|eq:Q| מתקיימת עבור~$n=0$.
    ✦ [צעד האינדוקציה] נניח שהטענה \פנה|eq:Q| מתקיימת עבור כל~$n'$, כאשר~$n'<n$
    ונוכיח אותה עבור~$n$.
    נסתכל על איבר~$f∈ℚ₁$ אשר נוצר מהפעלה של~$n$ בנאים, ונניח ש-$n>0$ כלומר~$f$
    נוצר על ידי הפעלה של בנאי. בנאי זה הוא אחד מארבעת הבנאים \פנה|eq:minus|,
    \פנה|eq:plus|, \פנה|eq:times|, או \פנה|eq:div|.
    מכאן, \[
      f(q)∈❴-q₁,q₁+q₂,q₁·q₂,q₁/q₂❵.
\] כיוון שמספר הפעולות הבנאים לשם יצירת הפונקציות~$f₁$ ו-$f₂$ קטן ממש מ-$n$
    הנחת האינדוקציה מתקיימת לגביהן, ולכן, גם~$q₁$ וגם~$q₂$ חייבים להיות רציונליים

    אם הם מוגדרים, ולכן גם~$f(q)$,
    אם הוא מוגדר, חייב
    להיות מספר רציונלי.
  \end{description}
\end{proof}

הגדרות רקורסיביות משמשות לעיתים קרובות באיפיון של שפות תכנות. אוסף הביטויים
המותר לשימוש בשפה, אוסף הפקודות, ואוסף הטיפוסים, כמעט תמיד מוגדרים רקורסיבית.


