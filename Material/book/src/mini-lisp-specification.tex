§§ ביטויי S, אטומים, ואלפאבית
שפת מיני-ליספ סובבת סביב איברי הקבוצה
\begin{equation}\label{eq:S:mini}
  𝓢_{\text{Mini-Lisp}}=𝓢(Γ_{\text{Mini-Lisp}}
\end{equation}
\ע|קבוצת ביטויי ה-S| הבנוייה מעל הקבוצ (האינסופית)~$Γ_{\text{Mini-Lisp}}$ של
ה\ע|אטומים| של מיני-ליספ.

רקורסיבית, כל אטום הינו ביטוי-\E|S|, וה-cons של כל שני ביטויי~\E|S|, גם הוא
ביטוי~\E|S|. בניסוח אחר, מתקיים   \[
  Γ_{\text{Mini-Lisp}}⊆𝓢_{\text{Mini-Lisp}}.
\] ושהקבוצה~$𝓢$ סגורה תחת הפעולה הבינארית של \E|cons|.

אטום במיני-ליספ הוא \ע|מילה| לא ריקה מעל האלפאבית~$Σ_{\text{Mini-Lisp}}$, 
\begin{equation}
  Γ_{\text{Mini-Lisp}}=Σ_{\text{Mini-Lisp}}⁺
\end{equation}

נוח למנות תווי האלפאבית בקבוצות, 
\begin{equation}\label{alphabet:C}
  Σ_{\text{Mini-Lisp}}=
  Σ_{\text{upper}}∪
  Σ_{\text{digit}}∪
  Σ_{\text{other}}.
\end{equation}
כאשר~$Σ_{\text{Mini-Lisp}}$ מכיל 62 תווים:
\begin{enumerate}
  ✦ \ע|26 אותיות אנגליות גדולות| \[
    Σ_{\text{upper}}=❴⌘A,⌘B,⌘C,⌘D,⌘E,⌘F,⌘G,
    ⌘H,⌘I,⌘J,⌘K,⌘L,⌘M,⌘N,⌘O,⌘P,⌘Q,⌘R,⌘S,⌘T,⌘U,⌘V,⌘W,⌘X,⌘Y,⌘Z❵.
\] \relax
  ✦ \ע|10 ספרות| \[
    Σ_{\text{digit}}=❴⌘0,⌘1,⌘2,⌘3,⌘4,⌘5,⌘6,⌘7,⌘8,⌘9❵.
\] \relax

  ✦ \ע|26 סימנים מיוחדים| \[
    Σ_{\text{other}}=❴
    ⌘!,
    ⌘",
    ⌘#,
    ⌘\$,
    ⌘\%,
    ⌘&,
    ⌘*,
    ⌘+,
    ⌘,,
    ⌘\,
    ⌘-,
    ⌘/,
    ⌘:,
    ⌘<,
    ⌘=,
    ⌘>,
    ⌘?,
    ⌘@,
    ⌘\textbackslash,
    ⌘\textasciicircum,
    ⌘\_,
    ⌘`,
    ⌘❴,
    ⌘|,
    ⌘❵,
    ⌘\textasciitilde
  ❵
\] \relax
\end{enumerate}

האלפאבית $\Sigma_{\text{Mini-Lisp}}$ לקבוצות~\cref{eq:alphabet:C} לתתי-קבוצות
נעשתה לשם מנייה שיטתית של התווים באלפאבית.  אין לחלוקה זו כל משמעות סמנטית. 

§§ דקדוק אבסטרקטי של שפת מיני-ליספ
הדקדוק של שפת מיני-ליספ תומך בכתיב הרשימות  ובסימן ה-\E|quote| (\cc{'}) לציון
הפונקציה \E|quote|. 

דקדוק אבסטרקטי של שפת מיני-ליספ ניתן כרשימה של חוקי
גזירה של דקדוק חסר הקשר:
\begin{equation}
  \label{eq:ambigious}
  \begin{split}
    S &→A ⏎
    S &→S₁\, ⌘.\, S₂⏎
    S₁&→S ⏎
    S₂&→S ⏎
    S &→⌘[\, S\, ⌘] ⏎
    S &→⌘(\, L ⌘\,) ⏎
    L &→S \, L\,\, |\, \,ε ⏎
    S &→⌘'\,S ⏎
  \end{split}
\end{equation}

במילים, תכנית בשפת מיני-ליספ היא ביטוי~\E|S|, המסומן בדקדוק באות~$S$, אשר יכול
ללבוש את הצורות הבאות:
\ספרר
✦ זוג של שני ביטויי~\E|S| המופרדים בסימן הנקודה (\cc{.}), המהווים יחד רשומת
  \E|cons|, או \E|dotted pair|.
✦ ביטוי~\E|S| העטוף בזוג סוגריים מרובעים (הסימן~\cc[ והסימן~\cc]).
✦ רשימה~$L$ של אפס או יותר ביטויי~\E|S| העטופים בזוג סוגריים עגולים (הסימן~\cc(והסימן~\cc)).
✦ ביטוי~\E|S| אשר לפניו מופיע סימן ה-\E|quote| (\cc{'})
✦ אטום בודד המסומן כ-$A$ בדקדוק.
===
בדקדוק \cref{eq:unambigious}, נוכל לזהות אסימונים משלושה סוגים:
\begin{enumerate}
✦ ארבעה סימני פיסוק:~$⌘[$,~$⌘]$,~$⌘($, ו-$⌘)$.
✦ שני אופרטורים: האופרטור הבינארי \E|dot|, אשר מסומן בנקודה ($⌘.$), אשר
  הוא שם אחר לפונקציה הדו-מקומית \lisp{cons}, והאופרטור האונארי \E|quote|
  המסומן בסימן המרכאה ($⌘'$), אשר הוא שם אחר לפונקצית הספריה החד-מקומית \lisp{quote}
✦ מספר בלתי חסום של אטומים, אשר ניתן לחשוב עליהם הן כמזהים והן כמילולונים.
\end{enumerate}

\lstdefinelanguage{Verbatim}{}
\lstnewenvironment{Verbatim}{\lstset{language=Verbatim}}{}
הביטוי הרגולרי הבא מתאר את האפשרויות השונות לכתיבת אטומים בשפה
\begin{LTR}
\begin{verbatim}
[a-zA-Z0-9!"#$%&*+,\-/:<=>?@\^_`{|}~]+
\end{verbatim}
\end{LTR}

ובמילים, אטום הוא סדרה באורך אחד או יותר של תווים הלקוחים מהאלפאבית של שפת ליספ:
כל תווי ה-\E|ASCII| הגרפיים (כלומר תווים שאינם תווי בקרה וניתן להדפיס אותם),
לבד משמונת התווים הבאים \begin{itemize}
    ✦ ששת התווים המופיעים בדקדוק, הכוללים, כפי שראינו, ארבעה סימני פיסוק ושני אופרטורים.
    ✦ תו הרווח, אשר נועד בעיקר להפריד בין ביטויי~\E|S| המופיעים ברשימה, אך יכול
    להופיע בכל מקום אחר בדקדוק. כיוון שרווח זה אינו יכול להופיע בתוך אטום, הרי
    אם הוא מתגלה בזמן שהאינטרפרטר של ליספ קורא סדרת תווים של אטום, הוא יסיים את
    הסדרה.
    ✦ התו נקודה-ופסיק (⌘;) המשמש לציון הערה. הערה במיני-ליספ, כמו בליספ, מתחילה
    בסימן זה, ונמשכת עד תום השורה. האינטרפטר של ליספ מתעלם מהערות.
\end{itemize}
    מיני-ליספ מתעלמת לא רק מרווחים, אלא גם מסימני שורה החדשה, מסימן הטאבולציה,
    ומכל תו בקרה אחר של \E|ASCII|. כל אלו דינם כדין רווח: הם עשויים לסיים את רצף
    התווים המהווה אטום, אך לבד מכך, השפה מתעלמת מהם. מיני-ליספ אינה תומכת
    ב-\E|Unicode| או בכל קידוד תווים אחר שאינו \E|ASCII|.

§§ קדימויות ואסוציאטיביות 
הדקדוק~\ref{eq:eval} הוא אמנם דקדוק מדויק של מיני-ליספ, אך הוא אינו חד משמעי.
ניתן לגזור מהדקדוק את התכנית \lisp{a.b.c} בשני אופנים שונים:
\begin{enumerate}
✦ \ע|גזירה שמאלית ביותר| (\E|left most derivation|) בה מחליפים בכל פעם את
  המשתנה הדקדוקי השמאלי ביותר. \פנה|איור:שמאלית| מתאר גזירה זו, ואת עץ
  הגזירה של התכנית \lisp{a.b.c} הנוצר ממנה.

\begin{figure}[H]
  \caption[גזירה שמאלית ביותר]{גזירה שמאלית ביותר של התכנית \LR{a.b.c} על פי הדקדוק הרב-משמעי של
  ליספ~\ref{eq:ambigious}, ועץ הגזירה המתקבל ממנה}
  \תגית|איור:שמאלית|
\centering
\scriptsize
\def⌘#1{\texttt{\bfseries\textcolor{olive}{#1}}}
\begin{LTR}
\begin{minipage}{0.4\textwidth}
\begin{equation}\label{eq:left}
 \catcode`\*=\active
  \def*#1{\textcolor{Red}{#1}}
  \begin{split}
      S &→*{S₁} \, ⌘.\, S₂⏎
      &→*S \, ⌘.\, S₂\, ⌘.\, S₂⏎
      &→* A \, ⌘.\, S₂\, ⌘.\, S₂⏎
      &→⌘a \, ⌘.\, *{S₂}\, ⌘.\, S₂⏎
      &→⌘a \, ⌘.\, *S \, ⌘.\, S₂⏎
      &→⌘a \, ⌘.\, *A\, ⌘.\, S₂⏎
      &→⌘a \, ⌘.\, ⌘b\, ⌘.\, *{S₂}⏎
      &→⌘a \, ⌘.\, ⌘b\, ⌘.\, *S⏎
      &→⌘a \, ⌘.\, ⌘b\, ⌘.\, *A⏎
      &→⌘a \, ⌘.\, ⌘b\, ⌘.\, ⌘c⏎
  \end{split}
\end{equation}
\end{minipage}`
\quad
\begin{minipage}{0.4\textwidth}
\begin{forest}
  for tree [$S$,nt
  [$S₁$,nt
    [$S₁$,nt [$S$,nt[$A$,nt[⌘a,terminal]]]]
    [⌘.,terminal]
    [$S₂$,nt [$S$,nt [$A$,nt [⌘b,terminal]]]]
  ]
  [⌘.,terminal]
  [$S₂$,nt [$S$,nt [$A$,nt [⌘c,terminal]]]]
]
\end{forest}
\end{minipage}
\end{LTR}
\end{figure}
ביטוי ה-\E|S| המתקבל עבור התכנית במקרה של גזירה כזו הוא
\begin{quote}\scriptsize
\Forest{s tree [{},cons [{},cons [⌘a,atom],[⌘b,atom]] [⌘c,atom] ] }
\end{quote}
מתאים לאסוציאטיביות שמאלית של אופרטור ה-\E|dot|.

✦ \ע|גזירה ימנית ביותר| (\E|right most derivation|) בה מחליפים בכל פעם את
המשתנה הדקדוקי הימני ביותר: \פנה|איור:שמאלית| מתאר גזירה זו, ואת עץ הגזירה של
התכנית \lisp{a.b.c} הנוצר ממנה.

\begin{figure}[H]
  \caption[גזירה ימנית ביותר]{
    גזירה ימנית ביותר של התכנית \LR{a.b.c} על פי הדקדוק הרב-משמעי של ליספ \cref{eq:ambigious}, ועץ הגזירה המתקבל ממנה}

  \תגית|איור:ימנית|
\centering
\begin{LTR}
  \begin{minipage}{0.4\textwidth}
\scriptsize
\def⌘#1{\texttt{\bfseries\textcolor{olive}{#1}}}
\begin{equation}\label{eq:right}
 \catcode`\*=\active
  \def*#1{\textcolor{Red}{#1}}
  \begin{split}
    S &→S₁ \, ⌘.\, *{S₂}⏎
      &→S₁ \, ⌘.\, S₁\, ⌘.\, *{S₂}⏎
      &→S₁ \, ⌘.\, S₁\, ⌘.\, *S⏎
      &→S₁ \, ⌘.\, S₁\, ⌘.\, *A⏎
      &→S₁ \, ⌘.\, *{S₁}\, ⌘.\, ⌘c⏎
      &→S₁ \, ⌘.\, *S\, ⌘.\, ⌘c⏎
      &→S₁ \, ⌘.\, *A\, ⌘.\, ⌘c⏎
      &→*{S₁} \, ⌘.\, ⌘b\, ⌘.\, ⌘c⏎
      &→*S \, ⌘.\, ⌘b\, ⌘.\, ⌘c⏎
      &→*A \, ⌘.\, ⌘b\, ⌘.\, ⌘c⏎
      &→⌘a \, ⌘.\, ⌘b\, ⌘.\, ⌘c⏎
  \end{split}
\end{equation}
\end{minipage}
\quad
  \begin{minipage}{0.4\textwidth}
\scriptsize
\begin{forest}
 [$S$,nt
   [$S₁$,nt [$S$,nt [$A$,nt [⌘a,terminal]]]]
  [⌘.,terminal]
  [$S₂$,nt
    [$S₁$,nt [$S$,nt [$A$,nt[⌘b,terminal]]]]
    [⌘.,terminal]
    [$S₂$,nt [$S$,nt [$A$,nt [⌘c,terminal]]]]
  ]
]
\end{forest}
\end{minipage}
\end{LTR}
\end{figure}
ביטוי ה-\E|S| המתקבל עבור התכנית במקרה של גזירה כזו מתאים לאסוציאטיביות ימנית
של אופרטור ה-\E|dot|.

\begin{quote}\scriptsize
\Forest{s tree [{},cons [⌘a,atom],[{},cons [⌘b,atom] [⌘c,atom]]] }
\end{quote}
\end{enumerate}

התכנית \lisp{'a.b} מדגימה דו משמעות מסוג אחר: בגזירה שמאלית ביותר נקבל כי
האופרטור quote מופעל על האטום \lisp{a}; בגזירה ימנית ביותר, האופרטור מופעל על
הזוג \lisp{a.b}.

כדי להימנע מדו-משמעות נשכתב את הדקדוק כך:
\newcommand \Or{\;|\;}
\newcommand \Then{\, }
\begin{equation}\label{eq:unambigious}
  \begin{split}
     S &→Q \, P ⏎
    P &→⌘{.} P \Or ε ⏎
    Q &→⌘{'} \Then Q \Or E ⏎
     E &→A \Or ⌘(\Then l \Then ⌘) \Or ⌘[ \Then S \Then ⌘] ⏎
     L &→S \Then L \Or ε ⏎
  \end{split}
\end{equation}
בניסוח מילולי, דקדוק זה משמעו כי ביטוי~\E|S|, המסומן בדקדוק באות~$S$, הוא סדרה
של אחד או יותר מרכיבים מסוג~$Q$, המופרדים באופרטור ה-\E|dot|.
המרכיב~$Q$ הוא ביטוי-\E|S| אשר אינו כולל בתוכו אופרטור זה, אלא כשהוא
מקונן באחד משני סוגי הסוגריים. בפרט המרכיב~$Q$ מתחיל בסידרה של אפס או יותר
מופעים של האופרטור \E|quote| המופעלים זה בזה אחר זה על המשתנה הדקדוקי
מסוג~$E$ אשר יכול
ללבוש את אחת מבין הצורות הבאות
\ספרר
✦ אטום בודד
✦ ביטוי~\E|S| העטוף בזוג סוגריים מרובעים (הסימן~\cc[ והסימן~\cc]).
✦ רשימה~$L$ של אפס או יותר ביטויי~\E|S| העטופים בזוג סוגריים עגולים (הסימן~\cc(והסימן~\cc)).
===
 הדקדוק~\ref{eq:unambigious} קובע כי סימן המרכאה הוא בעל קדימות גבוהה יותר מסימן
 הנקודה, וכי לסימן הנקודה אסוציאטיביות ימנית.

§§ אטומים, מילולונים, ומזהים

ניזכר שבביטויי~\E|S| כל אטום הוא סדרת סימנים נטולת משמעות. במיני-ליספ
האטום 42 הוא סדרה (בת שתי אותיות) שאין לה משמעות משלה. הא, ותו לא! במיני-ליספ
לאטום זה יכול לקבל משמעות בשתי דרכים: אם באמצעות קריאה לפונקציה
\E|set|,
\begin{LISP}
(set '42 'answer)
\end{LISP}
ואם בדרך של קריאה לפונקציה, בה נעשית קשירה של שמות פרמטרים לערכי הארגומנטים
שבהפעלה. אטום מציין את ערכו אחד הפרמטרים. כאשר ניתנת לאטום משמעות כזו, ניתן
לחשוב עליו כמזהה או אף כמשתנה. אולם, במיני-ליספ אין לאטום כל משמעות אלא אם כן
זו ניתנה לו במפורש באחת מבין שתי הדרכים האלו.

במרבית המימושים של ליספ המצב שונה: האטום 42 מייצג ערך שהוא מספר שלם. אנו אומרים
שהאטום 42 הוא \ע|מילולון| \E|(literal)|. מילולון הוא אטום שמשמעותו קבועה מראש
והוא אינו מציין ביטוי~\E|S| אחר. משמעותו של מילולון נקבעת על ידי פירוש "מילולי"
של סדרת התווים שבו, בניגוד למשמעותו של סמל הנקבעת באמצעות טבלת סמלים.

כאמור, אין מילולונים במיני-ליספ. ב-\E|Common Lisp| יש כמה סוגים של מילולונים,
הכוללים
\begin{enumerate}
  ✦ מספרים שלמים, כגון \T|-12|.
  ✦ מספרים רציונליים, כגון \T|3/7|.
  ✦ מספרים ממשיים כגון \T|3.1415926535897932384d0| ו-\T|6.02E+23|.
  ✦ מספרים מרוכבים כגון \T|#C(5-3)| אשר ערכו הוא~$5-3i$.
  ✦ מחרוזות כגון \T|"Hello, World"|
\end{enumerate}

התמיכה במילולונים נעשית על ידי הרחבת שיטת השיערוך של אטומים: אם האטום אותו יש
לשערך הוא סדרת תווים הנחזית להיות מספר או מילולון אחר, אז השיערוך של האטום נעשה
ללא היוועצות בטבלת הסמלים. תמיכה בפעולות אריתמטיות נעשית באמצעות הגדרה מראש של
קישור בין האטום~\T|+| לפעולת החיבור של מספרים, בין~\T|*| ובין פעולת הכפל,
וכו'.

בכמעט כל הניבים של ליספ, ניתן לכתוב ביטוים אריתמטיים בכתיב הרשימות, והשיערוך
שלהם יביא לתוצאה הצפויה,

\begin{SAMPLE}
> (+¢ ¢2 (*¢ ¢3¢ ¢5))
17
\end{SAMPLE}

הרחבות אלו אינן נחוצות לשם הבנה של ליספ, שכן ניתן לקודד בתוך ביטויי~\E|S| את
מספרים, וניתן באמצעות מיני-ליספ לממש את כל הפונקציות האריתמטיות הפועלות על
מספרים. אבל, מיני-ליספ אינה מסוגלת לבדוק את תוכנה של סדרת הסימנים היוצרת אטום:
הפעולה היחידה המותרת על אטום במיני-ליספ היא השוואתו לאטום אחר: כיוון שכך שפת
מיני-ליספ אינה יכולה להפוך אטום כמו \T|43217| למילולון אשר משמעותו היא המספר
הטבעי \E|$43,217$|, דהיינו, ארבעים ושלושה אלפים מאתיים ושבע עשרה.

§§ הפונקציות האטומיות של מיני-ליספ 

\newcounter{magicrownumbers}
\newcommand\rownumber{\stepcounter{magicrownumbers}\arabic{magicrownumbers}}
\begin{table}[!hbt]
  \כיתוב|הפונקציות האטומיות של מיני-ליספ|
  \תגית|טבלה:אטומיות|
  \footnotesize
  \rowcolors{2}{blue!10}{white}
  \begin{tabularx}\textwidth{r>{\small\setLR}c>{\small}c
    >{\setRL\scriptsize\raggedleft\arraybackslash\hsize=0.5\hsize\linewidth=\hsize}X
    >{\scriptsize\setLR\raggedright\arraybackslash\hsize=1.5\hsize\linewidth=\hsize}X
    }

    \toprule
    \normalsize \bfseries #                          &
    \normalsize \bfseries {\text{/arity}}שם           &
    \normalsize \bfseries סמנטיקה                     &
    \normalsize \bfseries תמצית                       &
    \multicolumn1c{\normalsize \bfseries \RL{דוגמאות}} ⏎
    \midrule

    \rownumber                                        &
    atom/1                                            &
    eager                                             &
    בדיקה אם הפרמטר הוא אטום                        &
    \lisp{(atom nil)}~$⇒$ \lisp{T} \newline
    \lisp{(atom t)}~$⇒$ \lisp{T} \newline
    \lisp{(atom '(a a))}~$⇒$ \lisp{NIL} \newline
    \lisp{(atom 'a)}~$⇒$ \lisp{T} ⏎

    \rownumber                                        &
    car/1                                             &
    eager                                             &
    חילוץ האיבר הראשון ברשימה                         &
    \lisp{(car '(b.a))}~$⇒$ \lisp{B.A} \newline
    \lisp{(car '(b a))}~$⇒$ \lisp{B} \newline
    \lisp{(car '(a))}~$⇒$ \lisp{A} \newline
    \lisp{(car 'a)}~$⇒$ \text{✗} \newline
    \lisp{(car t)}~$⇒$ \text{✗} \newline
    \lisp{(car ())}~$⇒$ \text{✗} ⏎

    \rownumber                                        &
    cdr/1                                             &
    eager                                             &
    חילוץ שארית הרשימה, כלומר הרשימה ללא האיבר הראשון &
    \lisp{(cdr '(a.b))}~$⇒$ \lisp{NIL} \newline
    \lisp{(cdr '(a b))}~$⇒$ \lisp{(b)} \newline
    \lisp{(cdr '(b))}~$⇒$ \lisp{NIL} \newline
    \lisp{(cdr t)}~$⇒$ ✗ \newline
    \lisp{(cdr ())}~$⇒$ ✗ \newline
    \lisp{(cdr nil)}~$⇒$ ✗ ⏎

    \rownumber                                        &
    cond/*                                    &
    normal                                            &
    הכללה של פקודת \E|if|. &
    \lisp{(cond (t 'A))}~$⇒$ \lisp{A} \newline
    \lisp{(cond (nil 'A) (t 'B))}~$⇒$ \lisp{B} \newline
    \lisp{(cond (nil 'A) (t 'B) (t 'C))}~$⇒$ \lisp{B} \newline
    \lisp{(cond (nil 'A) (nil 'B) (nil 'C))}~$⇒$ \lisp{nil}\newline
    \lisp{(cond)}~$⇒$ \lisp{nil} ⏎

    \rownumber                                        &
    cons/2                                            &
    eager                                             &
    הוספת איבר בתחילת רשימה                           &
    \lisp{(cons 'a '(b c))}~$⇒$ \lisp{(A B C)} \newline
    \lisp{(cons 'b nil)}~$⇒$ \lisp{(B)} \newline
    \lisp{(cons 'a 'b)}~$⇒$ \lisp{(A.B)} ⏎

    \rownumber                                        &
    eq/2                                              &
    eager                                             &
    בדיקה אם שני הפרמטרים הם אטומים השווים זה לזה     &
    \lisp{(eq t t)}~$⇒$ \lisp{T} \newline
    \lisp{(eq t nil)}~$⇒$ \lisp{NIL} \newline
    \lisp{(eq nil nil)}~$⇒$ \lisp{T} \newline
    \lisp{(eq 'a 'a)}~$⇒$ \lisp{T} \newline
    \lisp{(eq '(a a) '(a a))}~$⇒$ \lisp{NIL} ⏎

    \rownumber                                        &
    error/*                                   &
    eager                                             &
    הדפסת כל הפרמטרים ועצירת ביצוע התכנית             &
    \lisp{(error)}~$⇒$ ✗ \newline
    \lisp{(error A)}~$⇒$ ✗ \newline
    \lisp{(error 'my-message-on X)}~$⇒$ ✗ ⏎

    \rownumber                                        &
    set/2                                             &
    eager                                             &
    יצירת קישור בין אטום ובין ביטוי~\E|S|. &
    \lisp{(set 'a '(b c))}~$⇒$ \lisp{(b c)}\newline
    \lisp{(set 'b nil)}~$⇒$ \lisp{NIL}
    \label{atomic:count}
 ⏎
    \bottomrule
  \end{tabularx}
\end{table}

§§ הפונקציות המוגדרות מראש
הפונקציות המוגדרות מראש ממומשות באמצעות הפונקציות האטומיות. הצגנו את המימוש הזה
כבר, אך ההצגה הייתה קטעים קטעים, במהלך הדיון. \פנה|תכנית:מראש| אוספת את כל הקטעים האלו. 

\begin{תכנית}[H]
\immediate\closeout \libraryFile
\begin{LTR}
  \lstinputlisting[language=Mini,style=display,
    numbers=left,
    stepnumber=1,
    numbersep=2pt,
    xleftmargin=3ex,
    numberblanklines=false,
    numberstyle=\tiny\bf,
    backgroundcolor=\color{orange!20}
  ]{\jobname.library.lisp} 
\end{LTR}
\כיתוב|המימוש הפונקציות המוגדרות מראש|
\תגית|תכנית:מראש|
\end{תכנית}

\פנה|טבלה:מראש| מסכמת את המפרט של הפונקציות המוגדרות מראש.

\begin{table}[H]
  \כיתוב|הפונקציות המוגדרות מראש במיני-ליספ|
  \תגית|טבלה:מראש|
  \footnotesize
  \def\lisp#1{#1}
  \setlength{\tabcolsep}{1pt}%
  \rowcolors{2}{blue!10}{white}%
  \begin{tabularx}\textwidth{>{\scriptsize}r>{\setLR\scriptsize}c>{\scriptsize}c
    >{\scriptsize\raggedleft\arraybackslash}X
    >{\setLR\scriptsize\raggedright\arraybackslash}X
    >{\setLR\scriptsize\raggedright\arraybackslash}X
    }
    \toprule
   \normalsize \bfseries #                                                &
   \normalsize \bfseries {\text{/arity}}שם                                 &
   \normalsize \bfseries סמנטיקה                                           &
   \normalsize \bfseries תמצית                                             &
   \multicolumn1c{\normalsize \bfseries \RL{הגדרה}}                        &
   \multicolumn1c{\normalsize \bfseries \RL{דוגמאות}} ⏎
   \midrule

   \rownumber                                                              &
   \E|defun/3|                                                             &
   normal                                                                  &
   יצירת קישור בין אטום ובין ביטוי~\E|S| שהוא פונקציה בסמנטיקה \E|eager|. &
   \lstinline{(set 'defun}\newline
   \mbox\quad\lstinline{(nlambda (n p b)} \newline
   \mbox\qquad \lstinline{(set n (lambda p b))))}                               &
   \T|(defun f (x y) (y x))|\newline\quad
⏎

    \rownumber                                                              &
    lambda/2                                                                &
    normal                                                                  &
    יצירת פונקציה אנונימית שהסמנטיקה שלה היא \E|eager|. &
    \lstinline{(ndefun lambda(p b)}\newline
    \mbox\quad\lstinline{('lambda p b))}                                         &
    \lstinline{((lambda (x) (car (cdr x)))}\newline
    \mbox\quad\lstinline{'(a b)}~$⇒$ \lstinline{B} ⏎

    \rownumber                                                              &
    nil/0                                                                   &
    n/a                                                                     &
    האטום הנקבע על ידי~$ε$, סדרת ריקה של תווים, והמציין את הרשימה הריקה. אטום
    זה מציין את עצמו, ונחשב גם לערך הבוליאני של שקר, \E|false|. &
    \lstinline{(set 'nil 'nil)}                                                  &
    \lstinline{nil}~$⇒$ \lstinline{NIL} \newline
    \lstinline{()}~$⇒$ \lstinline{NIL} \newline
    \lstinline{(eq nil t)}~$⇒$ \lstinline{NIL} ⏎

    \rownumber                                                              &
    null/1                                                                  &
    eager                                                                   &
    בדיקה אם הארגומנט הוא האטום \lstinline{nil}                                  &
    \lstinline{(defun null (x)}\newline
    \mbox\quad\lstinline{(eq x nil))}                                            &
    \lstinline{(null t)}~$⇒$ \lstinline{NIL} \newline
    \lstinline{(null nil)}~$⇒$ \lstinline{T}\newline
    \lstinline{(null 'a)}~$⇒$ \lstinline{NIL} \newline
    \lstinline{(null '(a a))}~$⇒$ \lstinline{NIL} ⏎
    \rownumber                                                              &
    ndefun/3                                                                &
    normal                                                                  &
    יצירת קישור בין אטום ובין ביטוי~\E|S| שהוא פונקציה בסמנטיקה \E|normal|. &
    \lstinline{(set 'ndefun}\newline
    \mbox\quad\lstinline{(nlambda (n p b)} \newline
    \mbox\qquad\lstinline{(set n (nlambda p b))))}                               &
    \T|(ndefun f(x y) (y x))|\newline\quad~$⇒$ \T|F|⏎

    \rownumber                                                              &
    nlambda/2                                                               &
    normal                                                                  &
    יצירת פונקציה אנונימית שהסמנטיקה שלה היא \E|normal|. &
    \lstinline{(ndefun nlambda(p b)}\newline\mbox\quad\lstinline{('nlambda p b))}     &
    \T|(nlambda (x y) (y x))|~$⇒$
    \newline\mbox\quad\T|(nlambda (x y) (y x)|
 ⏎
    \rownumber                                                              &
    quote/1                                                                 &
    normal                                                                  &
    החזרת הארגומנט מבלי לשערך אותו. &
    \lstinline{(ndefun quote (x) x)}                                             &
    \lstinline{(quote a)}~$⇒$ \lstinline{A} \newline
    \lstinline{(quote (b c))}~$⇒$ \lstinline{(B C)} \newline
    \lstinline{'a}~$⇒$ \lstinline{A} \newline
    \lstinline{'(b c)}~$⇒$ \lstinline{(B C)} \newline ⏎

    \rownumber                                                              &
    t/0                                                                     &
    n/a                                                                     &
    האטום \E|t|, המציין את \E|t|, כלומר את עצמו. נחשב גם לערך הבוליאני של אמת,
    \E|(true)|. &
    \lstinline{(set 't 't)}                                                      &
    \lstinline{t}~$⇒$ \lstinline{T} \newline \lstinline{(eq nil nil)}~$⇒$ \lstinline{T} ⏎
    \bottomrule
 \end{tabularx}
\end{table}
