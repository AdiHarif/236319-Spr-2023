"אין זה איבר, גלגל, אף לא כפתור! וכי מה יש בו בשם?", את עשויה לתמוה. "וזה אשר
נקרא בשם זנב, האם לרגל יחשב, אם כך נקרא לו סתם פתאום?"

§ ישויות לעומת שמות
לכולנו ברור כי יש להבחין בין ישות ובין שמה.
יש דבר מה חמקמק בהבחנה שאני רוצה
⌘תחילת{ספרור}
• יש ישויות ללא שם - ישויות אנונימיות.
• יש שמות ללא ישויות - שמות שלא נקשרו לישויות.
• יתכן שלישות אחת יהיו מספר שמות.
• יתכן ששם אחד יתייחס למספר ישויות (למשל שם פונקציה ושם משתנה בשפת java, למשל overloading
• יש שמות שאינם חשובים: נקודות בתכנית שבהם יש חובה לתת שם, אבל השם עצמו חסר
חשיבות. למשל_בפרולוג או whatever ב-METAPOST.
⌘סוף{ספרור}

⌘תחילת{תכנית}
\bash
cat << EOF > echo.c
#include <stdio.h>

typedef char *stringsₜ[];

void printₛtrings(stringsₜ ss) {
  for (; *ss; ss++)
    printf("%s ", *ss);
  printf("\n");
}

int main(int argc, char *argv[], char **envp) {
  printₛtrings(argv);
  printₛtrings(envp);
  return 0;
}
EOF
\END
⌘setLTR
⌘lstinputlisting[language=C++,style=Numbered]{₀0/echo.c}

⌘כיתוב{מתן שם לטיפוס בשפת ⌘סי}
⌘תגית{תכנית:טיפוס:עם:שם:סי}
⌘סוף{תכנית}

⌘תחילת{תכנית}
\begin{CPPn}
struct {
  const char *name;
  const char *action;
} commands[]={
};
\end{CPPn}
⌘כיתוב{הגדרת טיפוס אנונימי בשפת ⌘סי}
⌘תגית{תכנית:טיפוס:אנונימי:סי}
⌘סוף{תכנית}

§ הגדרות
הגדרה
§ הכרזות

§ יצירת ישויות אנונימיות

§ מרחבי שמות
