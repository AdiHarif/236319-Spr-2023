גם מי שאינו אורניתולוג יודע כי כל תרנגולת היא אפרוח שהתפתח לאחר שבקע מביצה, וכל ביצה הוטלה על ידי תרנגולת קודמת. ⌘מונח[פרדוקס הביצה והתרנגולת]{פרדוקס הביצה והתרנגולת} המיוחס לאריסטו⌘הערת␣שוליים{384-322 לפנה"ס, יליד סטאגירה ובנו של ניקומכוס, רופא החצר של מלך מוקדון דאז. מגדולי הפילוסופים ביוון העתיקה.}, שואל מה אם הביצה קדמה לתרנגולת, או שמא, התרנגולת קדמה לביצה. 
• הטענה שהתרנגולת קדמה לביצה לא יכולה להיות נכונה, כי הרי התרנגולת הראשונה הייתה אפרוח שבקע מביצה.
• אם תמצי לומר שהביצה קדמה לתרנגולת, מיד יטען כנגדך שהביצה הזו הוטלה על ידי תרנגולת. 
ההנחה הסמויה בפרדוקס זה היא ששרשרת הזמן, כשהיא נמתחת לאחור, היא סופית, כלומר שגילו של היקום הוא  סופי.
בפרק זה, נעסוק במופעים של פרדוקסים אלגנטיים דומים בתחום של שפות התכנות. בדיוק כמו הביצה והתרנגולת, הפרדוקסליות הטמונה בפרדוקסים אלו היא קלה ביותר להבנה. לעיתים יהיו לפרדוקסים היתרם אלגנטיים, קצרים ופשוטים (אך לא נכונים בעליל). אך במרבית המקרים, טיבם של פרדוקסים הוא זה שהיתרם בא רק לאחר חקירה עמקנית, שהיא הרבה פחות אלגנטית מהפרדוקס עצמו.

נדגים זאת במספר היתרים שונים של ⌘מונח{פרדוקס הביצה והתרנגולת}. 
⌘תחילת{ספרור}
היתר דתי: באופן כללי, האל ברא את התרנגולת. על פי גישתן של דתות רבות, האל ברא את העולם, כאשר הוא מכיל בתוכו מינים שונים. בפרט, האל ברא את העולם, ובשלב מסוים נבראו התרנגולות יש מאין. בדת היהודית, למשל, בשלב בריאת בעלי החיים, עולה⌘הערת␣שוליים{"ויברא אלהים, את התנינם הגדולים ואת כל-נפש החיה אשר שרצו המים למינהם, ואת כל-עוף כנף למינהו, וירא אלוהים, כי-טוב", בראשית א' כ"א.} כי החיות נבראו בוגרות. בפרט, נבראה לראשונה תרנגולת בוגרת (או מספר תרנגולות בוגרות), אשר הטילו את הביצים הראשונות, עד לתרנגולותינו אנו. 
• היתר יווני: ⌘מונח[אלוהים]{אלוהים} ברא הן את התרנגולת והן את הביצה. מדובר גם כאן בהיתר דתי, וריאציה להיתר הראשון. בגרסא זו, נבראו ע"י האל התרנגולות והביצים, כרעם ביום בהיר. פתרון זה עוקף את ההנחה שמקורן של תרנגולות בהכרח מביצים.
• היתר מדעי: לדידם של אלו המאמינים בתורת האבולוציה, אין הכרח שכל ביצה ממנה נוצרה תרנגולת היא של תרנגולת. למעשה, מדובר בהנחה פשטנית אשר אינה מתארת נכונה את היווצרות החיים. כל מין התפתח ממין קודם וקדום על ידי שורה של מוטציות גנטיות אקראיות שהתרחשו במרוצת השנים. באמצעות מנגנונים של מוטציות ⌘הערת␣שוליים{למעשה, אין זה מדויק. לקורא המעוניין בהעשרה ביולוגית, נרחיב כי מנגנון בשם "רקומבינציה" (שאינו נחשב ל"מוטציה") יוצר מגוון ביולוגי בין-דורי אף באדם, אשר באמצעותו, באפיק נפרד להורשה "רגילה", מתקבל מגוון גנטי במין מסוים. לפרטים נוספים.} וברירה טבעית, התפתחו לאורך השנים, באופן הדרגתי, יצורים שהם התרנגולות אותם אנחנו מכירים היום. הנ"ל סותר במידת מה את הגדרת הפרדוקס גם כן, וגם לו וריאנטים רבים. באופן כללי, לפי רובן התרנגולות התפתחו מאב קדמון כלשהו בתהליך הדרגתי, כך שלשאלת "מי קדם למי" לא נותרת משמעות רבה. דוגמה לתאוריה ולתשובה.
ההיתר הנכון⌘הערת␣שוליים{לדעת כותב חוברת הקורס. מדובר בשאלה שמעסיקה את בני האדם מקדמת דנא, וכל הפתרונות האפשריים לגיטימיים עד אשר יוכח אחרת.}: התרנגולת והביצה - חד המה. התהליך האבולוציוני שאחראי ליצירת תרנגולות כולל שתי שרשראות שונות, אך מצומדות. השרשראות הללו מתחילות בתא הראשון בו היו חיים, כאשר השרשרת הראשונה היא שהשפיעה על מהלך חייו של יצור התרנגולת-ביצה, כלומר על מהלך חייו כצעיר, ואילו השרשרת השנייה השפיעה על מהלך חייו כבוגר. הוכחה לכך תוצע כדלקמן: נתבונן בשרשרת ההורה-צאצא של תרנגולת ספציפית, נניח, של הדודה סימה. הנ"ל (התרנגולת!) בקעה מביצה מסוימת שהטילה תרנגולת שנרכשה ע"י הדודה סימה ביריד, והשרשרת ממשיכה אחורה עד לאותו חד-תא היולי ממנו נוצרו החיים לראשונה, ואשר גרם ליצירת חד-תא אחר, שהתחיל את השרשרת שמובילה את התרנגולת האהובה של סימה.
נרצה לדון בפועלו של ריצ'רד דוקינס ⌘הערת␣שוליים{ביולוג אבולוציוני בריטי, הידוע בשל ספרי המדע הפופולרי שחיבר. יש יטענו שהוא סטיבן הוקינג בגרסתו הביולוגית.}, בהציעו ניסוי מחשבתי לבירור מקורו של האדם. קחו תמונה של עצמכם, ושימו מעליה תמונה של אביכם. מעליה הניחו תמונה של אביו, והמשיכו כך הלאה מספר בלתי-מוגבל של פעמים. בסופו של דבר תגיעו ככל הנראה לאב הקדמון ביותר של כל המינים. הטענה היא שבמעבר על השרשרת, לא נוכל להבחין בהבדל ממשי בין 2 דורות עוקבים, אך בין 2 דורות שמרחקם ניכר, כמעט ולא יוותר דמיון. בשרשראות המתחילות מהתרנגולת של הדוד סימה, לא הנחנו תמונות, כי אם סרט של כל ההתפתחות של היצורים, בזה על גבי זה, ועולה מכך מסקנה אודות הפרדוקס עצמו.
נשים לב כי הוכחה זו, וכל דרך החשיבה הזו הנוגעת להיתר זה, היא למעשה ⌘מונח {רדוקציה} של אופן יצירת החיים. הסתכלנו על שרשרת בעלי חיים המקושרת ביחסי "התפתח מ-", שמתחילה בנו ונגמרת בתא החי הראשון. הבעיה של יצירת החיים נחשבת פרדוקסלית הרבה פחות, וגם לה היתרים רבים (דתיים, ביולוגיים וכו'), אך הם נחשבים בעיני רבים "מוזרים" הרבה פחות, שכן יש לכל היתר שכזה תימוכין וראיות.
⌘סוף{ספרור}
ראינו אם כן פרדוקס משובב נפש.  פרדוקס שקל מאוד להבין אותו, קל להתירו באמצעים מטאפיזיים.

⌘פרק{נקודת תחילת ביצוע}

נעסוק בתת-פרק זה בסוגיית ⌘מונח[נקודת תחילת ביצוע]{נקודת תחילת ביצוע} של תכנית.
ראינו, בין השאר בתכניות "שלום, עולם!", הוראות בקוד אשר אחראיות להדפסת שורות טקסט. חלק מהוראות מעין אלו מכונות "⌘פקודות", והן בין השאר מרכיבות תכנית. ⌘פקודות עשויות להיות מאוגדות במקטעים, המכונים ⌘מונח{בלוקים}. בלוקים אלו עשויים להיות ⌘{מקוננים} אלו באלו, וניתן לכנותם בשם. בשפות שונות, ובכללן שפת ⌘סי, לבלוקים משוימים אלו ניתן השם "פונקציה" (בְּ⌘שי{Pascal}, למשל, מבחינים בין "פונקציות" ל"פרוצדורות" - שני סוגים שונים של בלוקים שכאלו). 
בהינתן אחת מאותן פונקציות, כפי שתוארו, ניתן ללמוד רבות אודות הפונקציות לה תקרא הפונקציה שבידינו. ניתן גם ללמוד, בתלות מסוימת בכישורינו אנו ובכישוריו של הוגה הפוקנציה, מה תכליתה ומהו האופן בו היא מוציאה תכליתה זו מהכוח אל הפועל. דא עקא, לא ניתן ללמוד מקריאה כזו או אחרת מי יקרא לפוקנציה. הדבר שקול לקריאת ספר בישול. מעיון בספר, עמוק ככל שיהיה, לא יוכל לדעת איש מי יקרא בספר ואילו פשעים קולינריים יחולל בשמו. 
נרצה לקוות שבשפות תכנות בהן נשתמש יוגדר סדר ביצוע בין הבלוקים האלו ⌘הערת␣שוליים{דטרמיניזם ועקביות בשפות תכנות הינה דבר כמעט הכרחי במערכות גדולות. עם זאת, יש לכך חריגות. לפרטים נוספים.}. במילים אחרות, שפות שונות מגדירות בצורה שונה את ⌘מונח{נקודת תחילת הביצוע} של תכנית מחשב, בפרט כאשר התכנית מורכבת מקבצי טקסט שונים המכילים, בין השאר, בלוקים מהסוג שתואר. נתאר לפיכך גישות שונות של שפות תכנות שונות באשר לסוגיה זו.
⌘תחילת{ספרור}
הגישה ה⌘מונח{אוטרקית} - פסקל: על פי הגישה האוטרקית, קיימת בתכנית מילה מוגדרת מראש, שחייבת להופיע פעם אחת ויחידה בתכנית, אשר החל ממנה מתחילה לרוץ התכנית. בשפת ⌘שי{פסקל}, למשל, המילה השמורה ⌘קד{program} מגדירה את מקום תחילת ביצוע התכנית. בשפת התכנות ⌘שי{AWK}⌘הערת␣שוליים{שפת סקריפט שפותחה במעבדות בל בשנת 1977. תוכננה לעיבוד טקסט וזהו עיקר השימוש בה.} התכנית מתחילה לרוץ החל מהמילה השמורה ⌘קד{begin}, אך בשפה זו אין חובה לתת שם לבלוק ממנו מתחיל הביצוע.
1. הגישה ה⌘מונח{מטאפיזית} - שפת ⌘סי: זוהי גישה נפוצה יותר בשפות תכנות, לפיה ריצת התכנית מתחילה בפונקציה בעלת שם מסוים, אלא ששם זה אינו מוגדר מראש, אלא ניתן לשינוי ע"י המשתמש. שם הפונקציה ממנה מתחילים 
בפרט אינו מוגדר על פי סביבת העבודה. בשפת ⌘סי, למשל, ממומשת גישה זו, ולא קיימת מילה המקבילה
ל-⌘קד{program} בְּPascal. ריצת התכנית מתחילה בדר"כ מהפונקציה ⌘קד{main}, ועם זאת אפשר לשנות זאת באמצעות
הגדרות ההידור של תכנית. להלן דוגמה לכך משפת ⌘סי: (המקור- מסמך "צעדים ראשונים", עמוד 20)
⌘תחילת{קוד}

/* Hello, World! in C for MS-Windows */

#include <windows.h>

int PASCAL WinMain(HINSTANCE hInstance,
 HINSTANCE hPrevInstance, LPSTR CmdLine, int Show)
{
 MessageBox(
	GetActiveWindow(),
	“Hello, World!”,
	“Hello Windows World”,
	MB_OK);
  return 0;
}
⌘סוף{קוד}
3. הגישה ההוליסטית - ⌘גאוה⌘הערת␣שוליים{ברוסית: קרפדה!}: גישה שמכלילה את הגישה הקודמת. לפי גישה זו, נקודת התחלת הביצוע עודנה חיצונית לשפה עצמה, אך עם זאת השפה מגדירה קביעות מדויקות באשר לנקודת תחילת הביצוע. לא יתכן במצב זה כי תחילת הביצוע תוגדר ב⌘מונח{סביבת הפיתוח}, ולכן לא יתכן שתהיה נקודת תחילת ביצוע שונה ב-2 סביבות פיתוח שונות. הנ"ל מתקיים בשפת ⌘שי{Eiffel}, שם כותב המתכנת קובץ בתחביר הדומה לזה של השפה, בו מוגדרת בין היתר נקודת תחילת הביצוע. זהו המצב גם בשפת ג'אווה, בה ניתן באמצעות מנגנון בשם ⌘מונח{Reflection}, להתחיל ביצוע מכל מקום, בעזרת פונקציית ⌘קד{main} בכל מחלקה. הנ"ל מציב קושי בתכנון שפה יבילה והוליסטית (מדוע?).
4. הגישה של ביצוע אינטרקטיבי - שפות עם ⌘מונח[מפרשים]{מפרשים}⌘הערת␣שוליים{Interpreter.}: בשפות מֵפוׂרַשוׂת, המצב מעט שונה. תחילת ביצוע התכנית היא בפקודה הראשונה המובאת בפני המפרש, וממשיכה הלאה ככל שחפץ לב המתכנת להמשיך ולתת פקודות בפני מפרשו. הנ"ל מתרחש בלולאה המכונה: ⌘מונח{RIEL - Read, Interpret, Execute Loop}. משמעותה נובעת משמה - עבור כל פקודה מבצע המפרש ארבעה שלבים בעיבוד הפקודה: קריאתה, פרשונה, ביצועה והמשך לפקודה הבאה. נקודת ההתחלה תהיה אפוא הפקודה הראשונה המוקלדת, או הנטענת לפרשון. שפה המממשת עקרון זה היא  ⌘שי{Ocaml}.
סוף{ספרור}

עסקנו עד כה בנקודת תחילת הביצוע של תכניות מחשב, ועם זאת זנחנו את הנושא החשוב מכל: איך המחשב מתחיל לעבוד? הרי ידוע לכל שעל מנת להפעיל תכניות מחשב, על תכניות אחרות להפעילן טרם לכך. נשאלת השאלה מי הפעיל את התכנית הראשונה? ובכן, נסביר: בחומרת המחשב קיים קוד צרוב (קובץ הוראות בסיסיות למחשב) בשם ⌘מונח[ROM: Read Only Memory]{ROM: Read Only Memory} - מדובר בקובץ הוראות בסיסי ביותר הגורם לביצוען של הפעולות ההִיוּלִיוֹת ביותר בהפעלת המחשב. הן מפעילות אחת, בשלב מסוים קוראות מידע  והוראות מזיכרון המחשב, ובעקבותן מופעלת אחרת, בשרשרת עד להדלקת המחשב. למותר לציין שהתהליך כולל שלבים מרובים, בהם ניתן לעשות בדרגות שונות. למעוניינים בהרחבה: הסבר קצר, הסבר מקיף יותר, פקולטה שזהו עניינה.




⌘פרק{הסכנה שבתכניות המשעשעות המדפיסות את עצמן}

מהי תכנית המדפיסה את עצמה? ובכן, הגדרה נאיבית תהיה תכנית אשר הפלט שלה היא היא עצמה, אך נחדד, שהרי גם התכנית הריקה תענה על הגדרה זו, ולא נרצה פתרון פשטני כל כך שייחשב פתרון תקף לבעיה אלגנטית זו. לא נרצה גם שתכנית שמקבלת את עצמה כקלט טקסטואלי ומדפיסה טקסט זה תיחשב, שכן מדובר בפתרון טכני בלבד. 
לפיכך, ההגדרה היא כדלקמן: תכנית המדפיסה את עצמה, ובלעז, ⌘מונח{Quine}, היא תכנית לא ריקה אשר לא מקבלת קלט והפלט היחידי שלה הוא התכנית עצמה (כך נימנע גם מלכלול תכניות אשר מדפיסות לאורך הזמן את כל הפלטים האפשריים, בזה אחר זה, עד אשר תדפיס בעת מסוים את עצמה). תכניות מעין אלו מהוות אבן שואבת במדעי המחשב, וניתן להתייחס אליהן בצורה מתמטית באופן הבא: אם נתייחס לסביבת הביצוע כאל פונקציה (מקבוצת התכניות אל קבוצת הפלטים), נקבל כי תכנית המדפיסה את עצמה היא נקודת שבת⌘הערת␣שוליים{Fixed Point. נציין כי למונח הרחבות רבות בתחום הטופולוגיה המתמטית, המכלילות את המונח למרחבים מטריים שונים. לפרטים נוספים.}. 
בנוסף, מבחינה ספרותית, ניתן לומר שקוד שכזה הוא קוד ארס-פואטי, שכן הוא עוסק בכתיבת קוד בעצמו. 
נראה מספר דוגמאות⌘הערת␣שוליים{ראוי להדגיש נקודה חשובה זו.} לתכניות המדפיסות את עצמן:
⌘גאוה:


Perl:



Python:

ולהלן דוגמה ב⌘שי{שיא חד}, המלווה בהסבר ובהרחבה.



מהו הטריק⌘הערת␣שוליים{סיבה נוספת להתעניינות בנושא זה היא הופעתו כשאלה בשיעורי הבית, בסמסטר בו נכתב סיכום זה.} שבזכותו עובדות תכניות אלו? ניתן לחלק באופן גס את הקוד ל-2 חלקים: 
⌘תחילת{ספרור}
1. מערך מחרוזות ו/או אוסף קבועים, אשר מכילים את קוד הביצוע של התכנית.
2. קוד הביצוע של התכנית, אשר מכיל הוראות להדפסה פעמיים של מערך המחרוזות המדובר, וכן קבועים נוספים הדרושים לשכפול מדויק של הקוד לתוך מה שיודפס. 
סיום{ספרור}
בריצת התכנית יודפס, כאמור, פעמיים המערך והתווים הרלוונטיים - בפעם הראשונה עבור הדפסת החלק הראשון של התכנית (מערך המחרוזות), ובפעם השנייה עבור הדפסת החלק השני - קוד הביצוע ממש. 

עם זאת, טמונה בקודים מעין אלו סכנה של ממש. תכניות כאלו עשויות להוות כלי להחדרה של ⌘מונח{וירוסים}⌘הערת␣שוליים{למעשה, מינוח מדויק יותר הוא סוס טרויאני, מונח שעל משמעותו ניתן וראוי לדון רבות.} בידי זֵדִים⌘הערת␣שוליים{הסבר מלא למילה זֵד  שֵם ז:  בלשון המקרא אדם רע, רשע; "טָפְלוּ עָלַי שֶׁקֶר זֵדִים" (תהלים קיט סט). [מילון רב-מילים]}. כתב על כך ⌘מונח{קן תומפסון}⌘הערת␣שוליים{אבי UNIX! חלוץ אמריקאי בתחום מדעי המחשב, ידוע בשל תרומתו לפיתוח שפות התכנות B,Go והגדרת UTF-8.} במאמרו⌘הערת␣שוליים {להבנה מלאה יותר של אופן הפעולה של סוס טרויאני, ניתן ללחוץ כאן.
 }.
נתחיל בתיאור מנגנון רלוונטי. נניח שנתון לנו קובץ המקור של המהדר של שפת ⌘סי, ונרצה להכניס בו שינוי מסוים - נרצה שבהינתן תו הבקרה ⌘קד{v/} , תודפס מפלצת ⌘קד{ASCII} הזו. כיצד נעשה זאת? ובכן, בהינתן קוד המקור האידיאלי הבא של המהדר, נגלה כי אין זו משימה קשה במיוחד:
מדובר בקוד המקבל תווים בשפה, ומחזיר את התו הרלוונטי עבור כל מקרה. היות שהמהדר של שפת ⌘סי כתוב בעצמו בשפת ⌘סי, המהדר של השפה "מכיר" את התווים המיוחדות האלו, ויודע "מה לעשות". 
על מנת להשלים את הקוד על מנת שיכיל את התוספת שלנו (ידוע שהדפסת מפלצות ⌘קד{ASCII} שכאלו מביאה למורת רוח מרובה בקרב המשתמשים), נבצע את השינוי הבא: 
הוספנו את השורה המתאימה, וננסה עתה להדר את הקובץ החדש של המהדר באמצעות המהדר הישן שברשותנו. אך, אבוי, שוד ושבר, נקבל שגיאה! הרי המהדר הישן לא יודע מפלצות ⌘קד{ASCII} מהן, ולא יודע מהו התו ⌘קד{v/}. 
נרצה לפיכך "לאלף" את המהדר הישן להכיר תוסף זה, על מנת שיוכל להדר אותו, ולהפוך אותו ל⌘מונח{קומפיילר} תקני המכיל את השינוי. ניגש אפוא לקובץ המקור של המהדר הישן, ונוסיף בו את השינוי הבא (אחרון, מבטיחים!):
כעת, המהדר הישן מכיר את התו ⌘קד{v/}. נניח שקיימת מחרוזת תווים כלשהי, המייצגת את מפלצת ה-⌘קד{ASCII} המועדפת עלינו, וכי היא מיוצגת ע"י ה-11 בדוגמה שלהלן. נהדר באמצעותו את המהדר החדש שכתבנו, ונקבל ⌘מונח{קובץ ⌘קד{binary}} שמכיל ⌘מונח{מהדר} חדש, המטמיע את התוספת החדשה. ניתן באמצעות תוצר זה להדר  תכניות שיבצעו תוספת זו.

באופן דומה, אך זהה מבחינה רעיונית, ניתן, לאחר מאמץ מחשבתי ניכר, לתאר נזקים כבירים אף יותר מהופעה של מפלצת (מאיימת ככל שתהיה) על צג המחשב. דוגמה לכך היא פריצה לחשבונות פרטיים במערכות ⌘שי{UNIX}, ע"י הוספת פרצה, לפיה ניתן יהיה להתחבר לכל חשבון באמצעות סיסמא כלשהי (נניח ⌘קד{ iAmHackerHoHoHo }). בשלב הבא, נבצע את התהליך כפי שביצענו בדוגמה הקודמת, ובאמצעותו נקבל את מערכת ⌘שי{UNIX} החדשה, בה תהיה קיימת פרצה זו. הפרצה תהיה מקודדת ומוטמעת במערכת החדשה, ללא יכולת זיהוי. זהו ⌘מונח{סוס טרויאני} עמיד בפני התקפות. 





